\documentclass{ltjsreport}
\RequirePackage{luatex85}
\usepackage[utf8]{inputenc}
\usepackage{enumerate}
\usepackage{amsthm}
\usepackage{amsfonts}
\usepackage{amsmath}
\usepackage{amssymb}
\usepackage{ytableau}
\usepackage{docmute}
\usepackage{mathtools}
\usepackage{xr}
\usepackage[all]{xy}



\theoremstyle{definition}
\newtheorem{defin}{定義}[subsection]
\newtheorem{theo}[defin]{定理}
\newtheorem{cor}[defin]{系}
\newtheorem{prop}[defin]{命題}
\newtheorem{lemm}[defin]{補題}
\newtheorem{notice}[defin]{注意}
\newtheorem{eg}[defin]{例}


\renewcommand{\labelenumi}{(\roman{enumi})}


\newcommand{\invlimit}{\mathop{\lim_{\longleftarrow}}}
\newcommand{\dirlimit}{\mathop{\lim_{\longrightarrow}}}
\newcommand{\ind}{\text{Ind}\:}
\newcommand{\Hom}{\text{Hom}}
\newcommand{\tr}{\text{tr}\:}
\newcommand{\id}[1]{\text{id}_{#1}}
\newcommand{\sgn}{\mathrm{sgn}}
\newcommand{\res}[1]{\text{Res}_{#1}}
\newcommand{\generated}[1]{\langle\:#1\:\rangle}
\newcommand{\im}{\text{Im }}
\newcommand{\rank}{\text{rank }}
\newcommand{\del}[2]{\frac{\partial #1}{\partial #2}}
\newcommand{\delsametwo}[2]{\frac{\partial^2 #1}{\partial #2^2}}
\newcommand{\delothertwo}[3]{\frac{\partial^2#1}{\partial#2\partial#3}}
\newcommand{\ddel}[2]{\frac{\partial}{\partial #2}#1}
\newcommand{\ddelsametwo}[3]{\frac{\partial^2}{\partial #2^2}#1}
\newcommand{\ddelothertwo}[3]{\frac{\partial^2}{\partial#2\partial#3}#1}
\newcommand{\simneq}{\not\simeq}
\newcommand{\transpose}[1]{^t\!#1}
\newcommand{\ie}{\text{i.e.}}
\newcommand{\inv}[1]{#1^{-1}}
\newcommand{\real}{\mathbb{R}}
\newcommand{\complex}{\mathbb{C}}
\newcommand{\integer}{\mathbb{Z}}
\newcommand{\quotient}{\mathbb{Q}}
\newcommand{\natnum}{\mathbb{N}}
\newcommand{\proj}{\mathbb{P}}
\newcommand{\tensor}[3]{#1\otimes_#2#3}
\newcommand{\map}[3]{#1:#2\rightarrow#3}
\newcommand{\aut}[2]{\mathrm{Aut}_{#1} (#2)}
\newcommand{\hommoph}[2]{\mathrm{Hom}_{#1}(#2)}
\newcommand{\gl}[1]{\mathrm{GL}(#1)}
\newcommand{\set}[2]{\left\{\:#1\:\middle|\:#2\:\right\}}
\newcommand{\pmat}[1]{\begin{pmatrix} #1
\end{pmatrix}}
\newcommand{\vmat}[1]{\begin{vmatrix} #1
\end{vmatrix}}
\newcommand{\br}{\vskip\baselineskip}


\begin{document}
\chapter{Schur多項式}
\section{Schur多項式}
\subsection{対称多項式と交代多項式}
\begin{defin}
    $n$変数多項式$f\in\integer[x_1,\cdots,x_n]$が対称多項式であるとは、任意の置換$\sigma\in\mathfrak{S}_n$に対して$\sigma f:=f(x_{\sigma(1)},\cdots,x_{\sigma(n)})=f(x_1,\cdots,x_n)$が成り立つことをいう。対称多項式全体のなす$\integer[x_1,\cdots,x_n]$の部分集合を$\integer[x_1,\cdots,x_n]^{\mathfrak{S}_n}$と書く。
    $f$が交代多項式であるとは、任意の置換$\sigma$に対して$\sigma f:=f(x_{\sigma(1)},\cdots,x_{\sigma(n)})=\sgn(\sigma)f(x_1,\cdots,x_n)$が成り立つことをいう。ただし$\sgn$は置換の符号である。
\end{defin}

\begin{eg}
    $xy, x+y, x^2+y^2$はいずれも$\integer[x,y]$の対称多項式であり、$x-y$は交代多項式である。$xy^2$, $x+2y$などは対称でも交代でもない
\end{eg}

\begin{prop}
    $\integer[x_1,\cdots,x_n]^{\mathfrak{S}_n}$は$\integer[x_1,\cdots,x_n]$の部分環をなす
\end{prop}

\begin{proof}
    $f,g\in\integer[x_1,\cdots,x_n]$, $\sigma\in\mathfrak{S}_n$に対して
    \[
    \sigma(f+g)=\sigma f+\sigma g,\qquad \sigma(f\cdot g)=\sigma f\cdot \sigma g    
    \]
    が成り立つことから従う。
\end{proof}

\begin{eg}[単項対称式]
    整数$n>1$を固定する。非負整数列$\alpha=(\alpha_1,\cdots,\alpha_n)$, $\beta=(\beta_1,\cdots,\beta_n)$に対して、ある置換$\sigma\in\mathfrak{S}_n$が存在して
    \[
    \beta=\sigma\alpha=(\alpha_{\inv{\sigma}(1)},\cdots,\alpha_{\inv{\sigma}(n)})    
    \]
    となるとき、$\beta\sim\alpha$と書く。広義単調減少な$\alpha=(\alpha_1,\cdots,\alpha_n)$に対して
    \[
    m_\alpha=\sum_{\beta\sim\alpha}x_1^{\beta_1}\cdots x_n^{\beta_n}    
    \]
    と定めると、$m_\alpha$は対称式である。
    \begin{align*}
        &m_{2,1}(x,y)=x^2y+xy^2\\
        &m_{2,2,0}(x,y,z)=x^2y^2+y^2z^2+z^2x^2
    \end{align*}
\end{eg}

\begin{eg}[べき和対称式]
    整数$n>1$を固定する。$k=1,2,\cdots,n$に対して$(k)=(k,0,\cdots,0)$とする。
    $p_k$を
    \[
    p_k
    =m_{(k)}
    =x_1^k+\cdots+x_n^k
    \]
    によって定義する。$p_k$はもちろん対称多項式である。
\end{eg}

\begin{eg}[基本対称式・完全対称式]
    整数$n>1$を固定する。$k=1,2,\cdots,n$に対して、$1^k=(1,1,\cdots,1,0,\cdots,0)$を最初の$k$個が$1$で、残りが$0$の数列とする。また
    \[
    \mathcal{Y}_{k,n}=\set{(\alpha_1,\cdots,\alpha_n)\in\integer_{\geq 0}^n}{\alpha_1\geq \cdots\geq\alpha_n,\quad\alpha_1+\cdots+\alpha_n=k}    
    \]
    とする。
    \begin{align*}
        &e_k=m_{1^k}=\sum_{1\leq i_1<i_2<\cdots<i_k\leq n}x_{i_1}x_{i_2}\cdots x_{i_k}\\
        &h_k=\sum_{\alpha\in\mathcal{Y}_{k,n}}m_\alpha=\sum_{1\leq i_1\leq i_2\leq\cdots\leq i_k\leq n}x_{i_1}x_{i_2}\cdots x_{i_k}
    \end{align*}
    として、$e_k$を$k$次基本対称式, $h_k$を$k$次完全対称式という。
    \begin{align*}
        &e_1=x_1+\cdots+x_n,\quad e_2=x_1x_2+x_1x_3+\cdots,\quad e_n=x_1x_2\cdots x_n\\
        &h_1=x_1+\cdots+x_n,\quad h_2=x_1^2+x_1x_2+\cdots, \quad h_n=x_1^n+x_1^{n-1}x_2+\cdots\\
        &h_{n+1}=x_1^{n+1}+x_1^nx_2+\cdots
    \end{align*}
\end{eg}


$n$変数の基本対称式は$e_1,\cdots, e_n$だけだが、完全対称式は無限に存在することに注意。ここで定義したさまざまな対称多項式は、対称多項式環$\integer[x_1,\cdots,x_n]^{\mathfrak{S}_n}$における良い性質をもっている。

\begin{prop}\label{m_is_basis}
    $\set{m_\alpha}{\alpha=(\alpha_1\geq\cdots\geq\alpha_n), \alpha_n\geq 0}$は$\integer[x_1,\cdots,x_n]^{\mathfrak{S}_n}$の基底をなす
\end{prop}

\begin{proof}
    $\set{m_\alpha}{\alpha=(\alpha_1\geq\cdots\geq\alpha_n), \alpha_n\geq 0}$が一次独立であることは、$\alpha\neq\beta$ならば$m_\alpha$, $m_\beta$は異なる単項式を含むことからわかる。よって$\integer[x_1,\cdots,x_n]^{\mathfrak{S}_n}$を生成することを示す。対称多項式
    \[
    f(x_1,\cdots,x_n)=\sum_{i_1,\cdots,i_n}c_{i_1,\cdots,i_n}x_1^{i_1}\cdots x_n^{i_n}
    \]
    について、任意の置換$\sigma\in\mathfrak{S}_n$に対して
    \begin{align*}
        f(x_1,\cdots,x_m)&=f(x_{\sigma(1)},\cdots,x_{\sigma(n)})\\
        &=\sum_{i_1,\cdots,i_n}c_{i_1,\cdots,i_n}x_{\sigma(1)}^{i_1}\cdots x_{\sigma(n)}^{i_n}\\
        &=\sum_{i_1,\cdots,i_n}c_{i_1,\cdots,i_n}x_1^{i_{\inv{\sigma}(1)}}\cdots x_n^{i_{\inv{\sigma}(n)}}\\
        &=\sum_{i_1,\cdots,i_n}c_{i_{\sigma(1)},\cdots,i_{\sigma(n)}}x_1^{i_1}\cdots x_n^{i_n}
    \end{align*}
    よって
    \[
        c_{i_1,\cdots,i_n}=c_{i_{\sigma(1)},\cdots,i_{\sigma(n)}}
    \]
    がなりたつ。したがって
    \[
    f=\sum_{\alpha}c_\alpha m_\alpha   
    \]
    となることがわかる。
\end{proof}

\begin{theo}[対称式の基本定理]\label{FT_of_sym}
    任意の対称多項式は基本対称式の多項式で表される。すなわち
    \[
    \integer[x_1,\cdots,x_n]^{\mathfrak{S}_n}=\integer[e_1,\cdots, e_n]
    \]
    が成り立つ。
\end{theo}

\begin{proof}
    命題\ref{m_is_basis}より、$m_\alpha$が$e_1,\cdots,e_n$の多項式で表されることを示せばよい。$\mathcal{Y}_n=\bigcup_{k=1}^\infty\mathcal{Y}_{k,n}$とおく。$\mathcal{Y}_n$には辞書式順序による全順序を入れておく。$\alpha\in\mathcal{Y}_n$に関する帰納法によって示そう。$\mathcal{Y}_n$の最小元は$(1,0,\cdots,0)$であり、
    \[
    m_{1,0,\cdots,0}=e_1    
    \]
    であるからよい。$\alpha=(\alpha_1\geq\cdots\geq\alpha_n)\in\mathcal{Y}_n$を$\alpha>(1,0,\cdots,0)$であるとする。
    \[
        g(x_1,\cdots,x_n)=m_\alpha-e_n^{\alpha_n}e_{n-1}^{\alpha_{n-1}-\alpha_n}\cdots e_2^{\alpha_2-\alpha_3}e_1^{\alpha_1-\alpha_2} 
    \]
    とおく。$g$は対称多項式だが、
    \[
    g=\sum_{\beta}m_\beta    
    \]
    と表した時、このときすべての$\beta$は$\alpha$より真に小さいことを示そう。$h=e_n^{\alpha_n}e_{n-1}^{\alpha_{n-1}-\alpha_n}\cdots e_2^{\alpha_2-\alpha_3}e_1^{\alpha_1-\alpha_2}$とおく。まず、
    \[
    e_n^{\alpha_n}=x_1^{\alpha_n}\cdots x_n^{\alpha_n}    
    \]
    より$h$を展開したときの単項式の指数はすべて$(\alpha_n,\cdots,\alpha_n)$以上であることがわかる。次に
    \[
    e_{n-1}^{\alpha_{n-1}-\alpha_n}=\left(\sum_{1\leq i_1<\cdots<i_{n-1}\leq n}x_{i_1}\cdots x_{i_{n-1}}\right)^{\alpha_{n-1}-\alpha_n}
    \]
    より$h$の単項式の指数で最も大きいものは
    \[
    (\alpha_{n-1},\cdots,\alpha_{n-1},\alpha_n)    
    \]
    以上であることがわかる。このことを繰り返していけば、$h$の指数最大の単項式は
    \[
    (\alpha_1,\alpha_2,\cdots,\alpha_n)    
    \]
    になることがわかる。またその係数が$1$であることも従う。よって$\beta <\alpha$であるから、帰納法の仮定により主張が成立する。
\end{proof}

\begin{eg}
    完全対称式は対称多項式なので定理\ref{FT_of_sym}より基本対称式の多項式である。実際
    \begin{align*}
        &h_1=e_1\\
        &h_2=e_1^2-e_2\\
        &h_3=e_1^3+e_3-2e_1e_2
    \end{align*}
    一般に
    \[
    h_k=\vmat{e_1&e_2&e_3&\cdots&e_k\\
              1&e_1&e_2&\cdots&e_{k-1}\\
              0&1&e_1&\cdots&e_{k-2}\\
              \vdots&\vdots&\vdots&\ddots&\vdots&\\
              0&0&0&\cdots&e_1}
    \]
    が成り立つことがわかる(第2部参照)。
\end{eg}


次に交代多項式についてみていこう。


\begin{defin}
    $\alpha=(a_1,\cdots,a_n)$, $a_k\in\integer_{\geq 0}$に対して多項式$A_\alpha\in\integer[x_1,\cdots,x_n]$を
    \[
        A_\alpha=\det((x_i^{a_j}))
    \]
    によって定める。行列式の交代性から、$A_\alpha$は交代多項式である。よって、$\alpha$に重複があるなら$A_\alpha=0$となる。
\end{defin}

\begin{eg}
    $\delta=(n-1,n-2,\cdots,1,0)$のとき
    \[
        A_\delta=\vmat{x_1^{n-1}&x_1^{n-2}&\cdots&x_1&1\\
                       x_2^{n-1}&x_2^{n-2}&\cdots&x_2&1\\
                       \vdots&\vdots&\ddots&\vdots&\vdots\\
                       x_n^{n-1}&x_n^{n-2}&\cdots&x_n&1}
    \]
    はVandermonde行列式に他ならない。したがって
    \[
    A_\delta=\prod_{i<j}(x_i-x_j)    
    \]
\end{eg}

\begin{prop}\label{alter_poly}
    任意の交代多項式は$A_\delta$で割り切れる
\end{prop}

\begin{proof}
    $f\in\integer[x_1,\cdots,x_n]$を交代多項式とする。交代性から$i<j$のとき$f$は$x_i$に$x_j$を代入すると$0$になる。よって$f$は$x_i-x_j$で割り切れる。$x_i-x_j$は既約多項式であり、$(i,j)$, $(k,l)$が異なるならば$x_i-x_j$, $x_k-x_l$は互いに素である。$\integer[x_1,\cdots,x_n]$はUFDであるので$f$は$A_\delta$で割り切れる。
\end{proof}


\subsection{Schur多項式}

\begin{defin}[Schur多項式]
    $\alpha=(a_1,\cdots,a_n)$, $a_1>\cdots> a_n\geq 0$に対して
    \[
    s_\alpha=\frac{A_\alpha}{A_\delta}  
    \]
    をSchur多項式という。
\end{defin}


命題\ref{alter_poly}より、$A_\alpha$は$A_\delta$で割り切れるので$s_\alpha$は多項式である。また任意の置換$\sigma\in\mathfrak{S}_n$に対して
\[
    \sigma s_\alpha=\frac{\sigma A_\alpha}{\sigma A_\delta}=\frac{\sgn (\sigma)A_\alpha}{\sgn (\sigma)A_\delta}=s_\alpha    
\]
となるからSchur多項式は対称多項式である。

\begin{eg}
    $\alpha=(4,2,0)$とする。
    \begin{align*}
        s_\alpha=\frac{\vmat{x_1^4&x_1^2&1\\
                             x_2^4&x_2^2&1\\
                             x_3^4&x_3^2&1}}
                      {\vmat{x_1^2&x_1^1&1\\
                             x_2^2&x_2^1&1\\
                             x_3^2&x_3^1&1}}
                &=\frac{(x_1^2-x_2^2)(x_1^2-x_3^2)(x_2^2-x_3^2)}{(x_1-x_2)(x_1-x_3)(x_2-x_3)}\\
                &=(x_1+x_2)(x_1+x_3)(x_2+x_3)\\
                &=(x_1+x_2+x_3)(x_1x_2+x_2x_3+x_1x_3)=e_1e_2
    \end{align*}
\end{eg}


Schur多項式について重要な命題が次の定理である。

\begin{theo}\label{schur}
    $n>0$を整数とする。Schur多項式の集合$\set{s_\alpha}{\alpha=(a_1,\cdots,a_n), a_1> \cdots> a_n\geq 0}$は対称多項式のなす環$\integer[x_1,\cdots,x_n]^{\mathfrak{S}_n}$の基底をなす
\end{theo}

\begin{proof}
    次の補題を示す。
    \begin{lemm}
        $\mathcal{S}=\set{(a_1,\cdots,a_n)}{a_1>\cdots>a_n\geq 0}$とする。交代多項式全体のなす$\integer$加群は$\{A_\alpha\}_{\alpha\in\mathcal{S}}$を基底にもつ
    \end{lemm}
    \begin{proof}
        $f(x_1,\cdots,x_n)$を交代多項式とする。
        \[
        f(x_1,\cdots,x_n)=\sum_{i_1,\cdots,i_n}c_{i_1,\cdots,i_n}x_1^{i_1}\cdots x_n^{i_n}    
        \]
        とおく。任意の置換$\sigma\in\mathfrak{S}_n$に対して
        \begin{align*}
            f(x_1,\cdots,x_n)&=\sgn(\sigma)f(x_{\sigma(1)},\cdots,x_{\sigma(n)})\\
            &=\sum_{i_1,\cdots,i_n}\sgn(\sigma)c_{i_1,\cdots,i_n}x_{\sigma(1)}^{i_1}\cdots x_{\sigma(n)}^{i_n}\\
            &=\sum_{i_1,\cdots,i_n}\sgn(\sigma)c_{i_{\sigma(1)},\cdots,i_{\sigma(n)}}x_1^{i_1}\cdots x_n^{i_n}
        \end{align*}
        がなりたつ。よって
        \begin{equation}
            \sgn(\sigma)c_{i_{\sigma(1)},\cdots,i_{\sigma(n)}}=c_{i_1,\cdots,i_n}
        \end{equation}
        これにより、$(i_1,\cdots,i_n)$に重複がある場合
        \[
        c_{i_1,\cdots,i_n}=0
        \]
        であることがわかる。よって
        \[
        f(x_1,\cdots,x_n)=\sum_{(i_1,\cdots,i_n)\in\mathcal{S}}\sum_{\sigma\in\mathfrak{S}_n}c_{i_{\sigma(1)},\cdots,i_{\sigma(n)}}x_1^{i_{\sigma(1)}}\cdots x_n^{i_\sigma(n)}    
        \]
        と書くことができる。再び(2)より
        \begin{align*}
        f(x_1,\cdots,x_n)
        &=\sum_{(i_1,\cdots,i_n)\in\mathcal{S}}\sum_{\sigma\in\mathfrak{S}_n}c_{i_{\sigma(1)},\cdots,i_{\sigma(n)}}x_1^{i_{\sigma(1)}}\cdots x_n^{i_\sigma(n)}\\
        &=\sum_{(i_1,\cdots,i_n)\in\mathcal{S}}\sum_{\sigma\in\mathfrak{S}_n}\sgn(\sigma)c_{i_1,\cdots,i_n}x_1^{i_{\sigma(1)}}\cdots x_n^{i_\sigma(n)}\\
        &=\sum_{(i_1,\cdots,i_n)\in\mathcal{S}}c_{i_1,\cdots,i_n}\sum_{\sigma\in\mathfrak{S}_n}\sgn(\sigma)x_1^{i_{\sigma(1)}}\cdots x_n^{i_\sigma(n)}\\
        &=\sum_{(i_1,\cdots,i_n)\in\mathcal{S}}c_{i_1,\cdots,i_n}A_{(i_1,\cdots,i_n)}
        \end{align*}
        $\{A_\alpha\}_{\alpha\in\mathcal{S}}$が一次独立であることは$\alpha\neq\beta$ならば$A_\alpha$と$A_\beta$は異なる単項式を含むことからわかる。
    \end{proof}

    定理の証明に戻る。$f$が対称多項式ならば$fA_\delta$は交代多項式であるから、補題により
    \[
    fA_\delta=\sum_{\alpha\in\mathcal{S}}c_\alpha A_\alpha    
    \]
    両辺を$A_\delta$で割って
    \[
    f=\sum_{\alpha\in\mathcal{S}}c_{\alpha}\frac{A_\alpha}{A_\delta}=\sum_{\alpha\in\mathcal{S}}c_{\alpha}s_\alpha    
    \]
    一意的に表せることは$\{A_\alpha\}_{\alpha\in\mathcal{S}}$が一次独立であることからわかる。
\end{proof}


定理\ref{schur}より、2つのSchur多項式の積はSchur多項式の線形結合であることがわかる。次節ではその係数を記述するLittlewood-Richardson規則について解説する。





\section{Littlewood-Richardson規則}
\subsection{Young図形}
\begin{defin}
    非負整数列$\lambda=(\lambda_1,\lambda_2,\cdots)$, $\lambda_1\geq\lambda_2\geq\cdots\geq \lambda_{k}=\lambda_{k+1}=\cdots=0$に対して、$1$行目に$\lambda_1$個の箱を書き、$2$行目に$\lambda_2$個の箱を書き...と続けてできる図形をYoung図形といい、同じく$\lambda$で表す。箱が1つもないYoung図形、すなわち$(0,0,\cdots)$は$\varnothing$で表す。$\lambda_{n+1}=0$のときたんに$\lambda=(\lambda_1,\cdots,\lambda_n)$と書くこともある。
    また$|\lambda|=\lambda_1+\lambda_2+\cdots$とし、これを$\lambda$の大きさという。
\end{defin}

\begin{eg}
    \[
    (2,1)=\ydiagram{2,1},\quad (4,3,3,1)=\ydiagram{4,3,3,1},\quad (3)=\ydiagram{3},\quad (1,1,1,1)=\ydiagram{1,1,1,1}
    \]
\end{eg}

\begin{defin}
    2つのYoung図形$\lambda=(\lambda_1,\cdots,\lambda_n,\cdots)$, $\mu=(\mu_1,\cdots,\mu_n,\cdots)$に対して、
    \[
    \lambda\subset \mu\Leftrightarrow \lambda_1\leq \mu_1,\cdots,\lambda_n\leq\mu_n,\cdots   
    \]
    と定義する。このとき$\lambda$は$\mu$の部分Young図形であるという。
\end{defin}

\begin{defin}
    $n$行からなるYoung図形の全体を第1節と同じ記号$\mathcal{Y}_n$で表す。すなわち
    \[
    \mathcal{Y}_n=\set{\lambda=(\lambda_1,\cdots,\lambda_n)}{\lambda_1\geq \cdots\geq\lambda_n\geq 0}
    \]
    である。
\end{defin}



Young図形とSchur多項式との関係は次の命題で表される
\begin{prop}
    $\mathcal{S}=\set{(a_1,\cdots,a_n)}{a_1>\cdots>a_n\geq 0}$と$\mathcal{Y}_n$には次の全単射が存在する。
    \[
    \mathcal{Y}_n\owns \lambda \mapsto \alpha=\lambda+\delta \in\mathcal{S}   
    \]
    ただし$\delta=(n-1,n-2,\cdots,1,0)$である
\end{prop}

\begin{proof}
    $\lambda\in\mathcal{Y}_n$は単調減少であるから、実際に$\lambda+\delta\in\mathcal{S}$であることはわかる。逆に任意の$\alpha\in\mathcal{S}$に対して、$\delta$が$\mathcal{S}$の辞書式順序に関する最小元であることから$\alpha-\delta\in\mathcal{Y}_n$であることもわかり、全単射であることが従う。
\end{proof}

よってYoung図形$\lambda$に対応するSchur多項式を$s_\lambda=\frac{A_{\lambda+\delta}}{A_\delta}$と書くことにする。

\begin{defin}
    $\lambda\in\mathcal{Y}_n$に対して、$\lambda$の各箱に次の条件が満たされるように数字を書き入れたものを形$\lambda$の半標準タブローという。
    \begin{itemize}
        \item 各数字は1以上n以下
        \item 各行は左から右に広義単調増加
        \item 各列は上から下に狭義単調増加
    \end{itemize}
    形$\lambda$の半標準タブロー全体のなす集合を$\mathcal{T}(\lambda)$と書く。半標準タブロー$T\in\mathcal{T}(\lambda)$について、$T$に数字$k\in\{1,\cdots,n\}$が$t_k$個書かれているとき$\omega_k(T)=t_k$のように書き、
    \[
    \omega(T)=(t_1,\cdots,t_n)
    \]
    とし、これを$T$のウェイトと呼ぶ。
\end{defin}

\begin{eg}\label{tableau_21}
    形$(2,1)=\ydiagram{2,1}\in\mathcal{Y}_3$の半標準タブローは次の通りである
    \begin{align*}
        \mathcal{T}((2,1))=\{\quad
        &\begin{ytableau}
            1&1\\
            2
        \end{ytableau},\qquad
        \begin{ytableau}
            1&2\\
            2
        \end{ytableau},\qquad
        \begin{ytableau}
            1&1\\
            3
        \end{ytableau},\qquad
        \begin{ytableau}
            1&3\\
            3
        \end{ytableau}\\
        &\begin{ytableau}
            1&2\\
            3
        \end{ytableau},\qquad
        \begin{ytableau}
            1&3\\
            2
        \end{ytableau},\qquad
        \begin{ytableau}
            2&2\\
            3
        \end{ytableau},\qquad
        \begin{ytableau}
            2&3\\
            3
        \end{ytableau}
        \quad\}
    \end{align*}
    しかし次などは半標準タブローではない
    \[
    \begin{ytableau}
        1&1\\
        1
    \end{ytableau},\qquad
    \begin{ytableau}
        2&1\\
        3
    \end{ytableau}  
    \]
\end{eg}

\begin{defin}
    Young図形$\lambda\in\mathcal{Y}_n$に対して次で定まる多項式を$\lambda$のタブロー和という。
    \[
        T_\lambda=\sum_{T\in\mathcal{T}(\lambda)}x_1^{\omega_1(T)}\cdots x_n^{\omega_n(T)}
    \]
\end{defin}

\begin{eg}
    例\ref{tableau_21}より、
    \[
    T_{(2,1)}=x_1^2x_2+x_1x_2^2+x_1^2x_3+x_1x_3^2+2x_1x_2x_3+x_2^2x_3+x_2x_3^2=e_1e_2=s_{(2,1)}
    \]
\end{eg}

\begin{eg}\label{1row_schur}
    $\lambda$が1行からなるYoung図形$\lambda=(k)\in\mathcal{Y}_n$の場合、$T_\lambda=h_k$である。なぜなら、形$(k)$の半標準タブローは左端に$1$をいくつか書き($0$個でもよい),続けて$2$をいくつか書き,$3$をいくつか書き...と続けて得られるから、
    \[
    1\leq i_1\leq \cdots\leq i_k\leq n    
    \]
    をみたす$i_1,\cdots,i_k$の組み合わせと1対1に対応するからである。
\end{eg}

\begin{eg}\label{1col_schur}
    $\lambda$が1列からなるYoung図形$\lambda=1^k\in\mathcal{Y}_n$ ($k\leq n$)の場合、
    $T_\lambda=e_k$である。なぜなら、形$1^k$の半標準タブローは
    \[
    1\leq i_1<\cdots<i_k\leq n    
    \]
    をみたす$i_1,\cdots,i_k$の組み合わせと1対1に対応するからである。
\end{eg}



\subsection{Littlewood-Richardson規則}

\begin{theo}[Littlewood-Richardson規則]\label{LR}
    Young図形$\lambda,\mu\in\mathcal{Y}_n$について
    \[
    s_\lambda s_\mu=\sum_{\nu\in\mathcal{Y}_n}\eta^\nu_{\lambda\mu}s_\nu    
    \]
    とおいたとき、
    \[
    \eta^\nu_{\lambda\mu}=\#\!\set{T\in\mathcal{T}(\mu)}{T\text{は$\lambda$-goodであり、}\omega(T)=\nu-\lambda}  
    \]
    が成り立つ。係数$\eta^\nu_{\lambda\mu}$をLittlewood-Richardson数と呼ぶ。
\end{theo}

ここで$T\in\mathcal{T}(\mu)$が$\lambda$-goodであるとは、次の条件を満たすことをいう。$T$に書かれている数字を上から下、右から左へ読んでいったときにできる数字の並びを$c(T)$とする。
\[
T=\begin{ytableau}
    1&1&2&2\\
    3&3&3&4\\
    4&5
\end{ytableau}\qquad \rightarrow \qquad c(T)=2423135134
\]
$c(T)_j$を$c(T)$の左から$j$番目までの部分列とするとき
\[
\lambda+\omega(c(T)_j)\in\mathcal{Y}_n,\qquad\forall j=1,\cdots,|\mu|
\]
が成り立つとき、$T$は$\lambda$-goodであるという。すなわち、「$T$の右上から左下へ数字を読んでいくとき、読まれた数に対応する$\lambda$の行に箱を追加する」という操作を続けて各ステップでYoung図形であることが保たれるということである。
\begin{eg}
$T=$ \begin{ytableau}
    1&1\\
    3
\end{ytableau} は \ydiagram{1,1} -goodである。実際$c(T)=113$であり
\begin{align*}
\ydiagram{1,1}\quad\rightarrow\quad
\begin{ytableau}
    *(white)&*(yellow)\\
    *(white)
\end{ytableau}\quad\rightarrow\quad
\begin{ytableau}
    *(white)&*(white)&*(yellow)\\
    *(white)
\end{ytableau}\quad\rightarrow\quad
\begin{ytableau}
    *(white)&*(white)&*(white)\\
    *(white)\\
    *(yellow)
\end{ytableau}\\
\end{align*}

一方で$T$は \ydiagram{1} -goodではない。
\[
    \ydiagram{1}\quad\rightarrow\quad
    \begin{ytableau}
        *(white)&*(yellow)
    \end{ytableau}\quad\rightarrow\quad
    \begin{ytableau}
        *(white)&*(white)&*(yellow)\\
    \end{ytableau}\quad\rightarrow\quad
    \begin{ytableau}
        *(white)&*(white)&*(white)\\
        \none\\
        *(yellow)
    \end{ytableau}\text{ はYoung図形でない}
\]
\end{eg}

\begin{eg}\label{empty-good}
    $\varnothing$-goodであるような形$\mu$の半標準タブローは$1$行目がすべて1, $2$行目がすべて2, ... というものただ一つである。
    この半標準タブローを$\mu^{st}$と書く。

    $T$が$\varnothing$-goodであるとする。$\varnothing$に箱を$1$つ追加してYoung図形になるためには第$1$行目に追加しなければならない。よって$T$の一番右上には$1$が入っており、半標準タブローの行単調性から$1$行目はすべて$1$である。半標準タブローの列単調性から$2$行目の一番右は$2$以上が入っているはずであり、$3$より大きければYoung図形ができないので$2$である。よって行単調性から$2$行目はすべて$2$である。以下同様にして$k$行目に入っている数字はすべて$k$であることがわかる。
\end{eg}

\begin{eg}
    $\lambda=$ \ydiagram{2,1} $\in\mathcal{Y}_2$ とし、$s_\lambda^2$をSchur多項式の線形結合として表そう。$\lambda$-goodな形$\lambda$の半標準タブローは
    \begin{align*}
        T_1=\begin{ytableau}
            1&1\\
            2
        \end{ytableau},\qquad
        T_2=\begin{ytableau}
            1&2\\
            2
        \end{ytableau}
    \end{align*}
    ですべてである。それぞれのウェイトは
    \begin{align*}
        \omega(T_1)=(2,1),\quad\omega(T_2)=(1,2)
    \end{align*}
    定理\ref{LR}より
    \[
    s_\lambda^2=s_{4,2}+s_{3,3}
    \]
    である。実際、定義より
    \begin{align*}
    &s_\lambda=\frac{\vmat{x^3&x\\y^3&y}}{\vmat{x&1\\y&1}}=\frac{x^3y-xy^3}{x-y}=xy(x+y),\qquad 
    s_\lambda^2=x^4y^2+2x^3y^3+x^2y^4\\
    &s_{4,2}=\frac{\vmat{x^5&x^2\\y^5&y^2}}{\vmat{x&1\\y&1}}=\frac{x^5y^2-x^2y^5}{x-y}=x^2y^2(x^2+xy+y^2)=x^4y^2+x^3y^3+x^2y^4\\
    &s_{3,3}=\frac{\vmat{x^4&x^3\\y^4&y^3}}{\vmat{x&1\\y&1}}=\frac{x^4y^3-x^3y^4}{x-y}=x^3y^3
    \end{align*}
    で確かに正しい。

\end{eg}

定理\ref{LR}の証明のあらすじを述べよう。ポイントになるのが次の等式(補題\ref{equality})である:
\[
A_{\lambda+\delta}T_\mu=\sum_{T\in\mathcal{T}(\mu)}A_{\lambda+\omega(T)+\delta} 
\]
この等式はタブロー和$T_\mu$が対称多項式であること(命題\ref{tableausum})から示される。右辺に関して、$T$が$\lambda$-goodでない項たちは互いにキャンセルされることが示され(補題\ref{cancellation})、結局
\begin{equation}
A_{\lambda+\delta}T_\mu=\sum_{T:\lambda\text{-good}}A_{\lambda+\omega(T)+\delta}   
\end{equation}
ここで、$\lambda=\varnothing$の場合を考えると例\ref{empty-good}
より
\[
A_{\delta}T_\mu=A_{\omega({\mu^{st}})+\delta}
\]
両辺を$A_\delta$で割れば
\[
T_\mu=\frac{A_{\omega(\mu^{st})+\delta}}{A_\delta}=\frac{A_{\mu+\delta}}{A_\delta}=s_\mu
\]
すなわち、タブロー和はSchur多項式と等しいということが導かれる。再び一般の$\lambda$に対し式(3)の両辺を$A_\delta$で割って
\[
s_\lambda s_\mu=\sum_{T:\lambda\text{-good}}s_{\lambda+\omega(T)}
\]
これより主張が従う。

あらすじで用いた命題・等式を示そう。

\begin{prop}\label{tableausum}
    タブロー和$T_\lambda$は対称多項式である。
\end{prop}

\begin{proof}
    対称群は隣り合う数字の互換$\sigma=(k-1,k)$, $k=2,\cdots,n$によって生成されるから、
    \[
    \sigma T_\lambda=T_\lambda    
    \]
    を証明すればよい。ポイントになるのは半標準タブローの集合$\mathcal{T}(\lambda)$上の対合\footnote{集合$X$上の対合とは写像$\map{\iota}{X}{X}$であって$\iota^2=\id{X}$をみたすものをいう}$\iota$であって
    \begin{equation}
    \omega(\iota(T))=\sigma(\omega(T))  
    \end{equation}
    をみたすものの存在である。ここで、
    \[
    \sigma(\omega(T))=(\omega_{\inv{\sigma}(1)}(T),\cdots,\omega_{\inv{\sigma}(n)}(T))  
    \]
    である。このような$\iota$が構成できれば、
    \begin{align*}
    \sigma T_\lambda&=\sum_{T\in\mathcal{T}(\lambda)}x_{\sigma(1)}^{\omega_1(T)}\cdots x_{\sigma(n)}^{\omega_n(T)}\\
    &=\sum_{T\in\mathcal{T}(\lambda)}x_1^{\omega_{{\inv{\sigma}(1)}}(T)}\cdots x_n^{\omega_{{\inv{\sigma}(n)}}(T)}\\
    &=\sum_{T\in\mathcal{T}(\lambda)}x_1^{\omega_1(\iota(T))}\cdots x_n^{\omega_n(\iota(T))}\\
    &=T_\lambda
    \end{align*}
    となり対称性が従う。最後の等式は$\iota$が全単射であることによる。

    このような$\iota$は次のように構成される。まず条件(4)は、半標準タブロー$T$と$\iota(T)$は書かれている$k-1$と$k$の数が逆転した関係にある、ということを意味している。最初に$T$が一行のYoung図形からなる場合を考えよう。半標準タブローの単調性から$k-1$か$k$の書かれている部分はひとつながりの帯領域をなしており、その長さは$\omega_{k-1}(T)+\omega_{k}(T)$である。よってこの帯領域の数字を、左$\omega_{k}(T)$個の箱に$k-1$, 残りの$\omega_{k-1}(T)$個の箱に$k$を入れるように変更したものを$\iota(T)$とすれば、これは条件(4)を満たす半標準タブローになる。
    \[
    T=\quad
    \begin{ytableau}
        \none[\cdots]&
        \scriptstyle k-2&
        *(yellow)\scriptstyle k-1&
        *(yellow)\scriptstyle k-1&
        *(yellow)\scriptstyle k&
        *(yellow)\scriptstyle k&
        *(yellow)\scriptstyle k&
        \scriptstyle k+1&
        \none[\cdots]
    \end{ytableau}\qquad\rightarrow\qquad
    \iota(T)=\quad
    \begin{ytableau}
        \none[\cdots]&
        \scriptstyle k-2&
        *(yellow)\scriptstyle k-1&
        *(yellow)\scriptstyle k-1&
        *(yellow)\scriptstyle k-1&
        *(yellow)\scriptstyle k&
        *(yellow)\scriptstyle k&
        \scriptstyle k+1&
        \none[\cdots]
    \end{ytableau}
    \]
    また、この場合に$\iota^2(T)=T$が成立していることもわかる。

    一般の半標準タブロー$T$に対しては一行の場合の操作を拡張することで得られる。まず、$T$の箱が自由であることを
    \begin{itemize}
        \item 箱に$k$が入っており、上の箱は$k-1$より真に小さい
        \item 箱に$k-1$が入っており、下の箱は$k$より真に大きいか下に箱がない
    \end{itemize}
    のどちらかを満たしていることと定義する。例えば$k=4$において
    \[
    T=\begin{ytableau}
        1&1&1&1&2&2\\
        2&2&*(green)3&*(yellow)3&*(yellow)3&*(yellow)4\\
        *(yellow)3&*(yellow)3&*(green)4&5\\
        5
    \end{ytableau}  
    \]
    黄色の箱は自由であり、緑の箱は自由でない。不自由な箱は数字を入れ替えると単調性が崩れるので、入れ替えることができないという意味で不自由である。したがって数字の入れ替えをするには、自由な箱のみを考えればよい。重要なこととして、
    \begin{quote}
        自由な箱の全体はいくつかの帯領域をなし、さらに帯は各行にたかだか1つである。
    \end{quote}
    実際
    \begin{itemize}
        \item $k-1$が書かれている箱が自由なら、その右にある$k-1$の書かれた箱はすべて自由である。なぜなら半標準タブローの行単調性から、その下にある箱はすべて$k$より真に大きいからである。
        \item $k$が書かれている箱が自由なら、その左にある$k$の書かれた箱はすべて自由である。なぜなら半標準タブローの行単調性から、その上にある箱はすべて$k-1$より真に小さいからである。
    \end{itemize}
    より、各行に帯領域はたかだか一つである。そこで各帯領域に対して、1行の場合の入れ替え操作を行った半標準タブローを$\iota(T)$と置けば、$\iota(T)$は条件(4)を満たす。なぜなら、不自由な箱は$k-1$が書かれているものと$k$が書かれているもので同数あり、1行の場合に条件(4)は満たされているからである。また半標準タブローの列単調性から$\iota(T)$と$T$で箱の自由性は保たれるので$\iota^2(T)=T$であることもわかる。また、もし$T$に自由な箱が存在しない場合は$\iota(T)=T$とする。これで構成できた。
    \[
    T=\quad\begin{ytableau}
        1&1&1&1&2&2\\
        2&2&*(green)3&*(yellow)3&*(yellow)3&*(yellow)4\\
        *(yellow)3&*(yellow)3&*(green)4&5\\
        5
    \end{ytableau}\qquad\rightarrow\qquad
    \iota(T)=\quad\begin{ytableau}
        1&1&1&1&2&2\\
        2&2&*(green)3&*(yellow)3&*(yellow)4&*(yellow)4\\
        *(yellow)4&*(yellow)4&*(green)4&5\\
        5
    \end{ytableau}
    \]
\end{proof}

\begin{lemm}\label{equality}
    $\lambda,\mu\in\mathcal{Y}_n$に対して
    \[
    A_{\lambda+\delta}T_\mu=\sum_{T\in\mathcal{T}(\mu)}A_{\lambda+\omega(T)+\delta}
    \]
    が成り立つ。
\end{lemm}

\begin{proof}
    $\text{Alt}_n=\sum_{\sigma\in\mathfrak{S}_n}\sgn(\sigma)\sigma$とおく(これは交代化作用素と呼ばれる)。交代化作用素と対称多項式をかけることは可換である。実際、$f\in\integer[x_1,\cdots,x_n]^{\mathfrak{S}_n}$, $g\in\integer[x_1,\cdots,x_n]$に対し
    \begin{align*}
        \text{Alt}_n(fg)&=\sum_{\sigma\in\mathfrak{S}_n}\sgn(\sigma)\sigma(fg)\\
        &=\sum_{\sigma\in\mathfrak{S}_n}\sgn(\sigma)\sigma f\cdot \sigma g\\
        &=f\cdot\sum_{\sigma\in\mathfrak{S}_n}\sgn(\sigma)\sigma g\\
        &=f\cdot\text{Alt}_n(g)
    \end{align*}
    である。
    \[
    A_{\lambda+\delta}=\text{Alt}_n(x_1^{\lambda_1+\delta_1}\cdots x_n^{\lambda_n+\delta_n})
    \]
    だから命題\ref{tableausum}より
    \begin{align*}
        A_{\lambda+\delta}T_\mu&=\text{Alt}_n(T_\mu\cdot x_1^{\lambda_1+\delta_1}\cdots x_n^{\lambda_n+\delta_n})\\
        &=\text{Alt}_n\left(
            \sum_{T\in\mathcal{T}(\mu)}x_1^{\lambda_1+\omega_1(T)+\delta_1}\cdots x_n^{\lambda_n+\omega_n(T)+\delta_n}
        \right)\\
        &=\sum_{\sigma\in\mathfrak{S}_n}\sum_{T\in\mathcal{T}(\mu)}\sgn(\sigma)x_{\sigma(1)}^{\lambda_1+\omega_1(T)+\delta_1}\cdots x_{\sigma(n)}^{\lambda_n+\omega_n(T)+\delta_n}\\
        &=\sum_{T\in\mathcal{T}(\mu)}\sum_{\sigma\in\mathfrak(S)_n}\sgn(\sigma)x_{\sigma(1)}^{\lambda_1+\omega_1(T)+\delta_1}\cdots x_{\sigma(n)}^{\lambda_n+\omega_n(T)+\delta_n}\\
        &=\sum_{T\in\mathcal{T}(\mu)}A_{\lambda+\omega(T)+\delta}
    \end{align*}
\end{proof}

\begin{lemm}\label{cancellation}
    $\lambda,\mu\in\mathcal{Y}_n$に対して、形$\mu$の半標準タブローで$\lambda$-goodでないものを$\lambda$-badと呼び、その全体を$\mathcal{T}(\mu)^{\lambda-bad}$とおく。このとき
    \[
    \sum_{T\in\mathcal{T}(\mu)^{\lambda-bad}}A_{\lambda+\omega(T)+\delta}=0    
    \]
    が成り立つ。
\end{lemm}

\begin{proof}
    この証明においてもポイントになるのが$\mathcal{T}(\mu)^{\lambda-bad}$上の対合$\iota$であって
    各$T\in\mathcal{T}(\mu)^{\lambda-bad}$に対してある$k$が存在して$\sigma=(k-1,k)$に対して
    \begin{equation}
    \lambda+\omega(\iota(T))+\delta=\sigma(\lambda+\omega(T)+\delta)
    \end{equation}
    をみたすものの存在である。このような$\iota$が構成されれば、$A_{\lambda+\omega(T)+\delta}$たちはペアごとに打ち消される。実際、
    \begin{align*}
        \sum_{T\in\mathcal{T}(\mu)^{\lambda-bad}}A_{\lambda+\omega(T)+\delta}
        &=\frac{1}{2}\sum_{T\in\mathcal{T}(\mu)^{\lambda-bad}}(A_{\lambda+\omega(T)+\delta}+A_{\lambda+\omega(\iota(T))+\delta})\\
        &=\frac{1}{2}\sum_{T\in\mathcal{T}(\mu)^{\lambda-bad}}(A_{\lambda+\omega(T)+\delta}+A_{\sigma(\lambda+\omega(T)+\delta)})\\
        &=\frac{1}{2}\sum_{T\in\mathcal{T}(\mu)^{\lambda-bad}}(A_{\lambda+\omega(T)+\delta}-A_{\lambda+\omega(T)+\delta})\\
        &=0
    \end{align*}

    (5)をみたす$\iota$を構成するために、条件(5)が成り立つための必要条件から考察していく。(5)が成り立つには
    \[
    \lambda_k+\omega_k(\iota(T))+\delta_k=\lambda_{k-1}+\omega_{k-1}(T)+\delta_{k-1}    
    \]
    したがって
    \begin{equation}
    \omega_k(\iota(T))=\omega_{k-1}(T)+(\lambda_{k-1}-\lambda_k)+1    
    \end{equation}
    となることが必要である。この右辺の値は、$\lambda$の$k$行目にいくつ箱を追加するとYoung図形でなくなるか、ということを表していることに注意する。このような$k$と$\iota(T)$をみつけたいのである。

    そこで、
    \[
    \lambda+\omega(c(T)_j)\notin\mathcal{Y}_n
    \]
    を満たす最小の$j$をとってこよう。これは$\lambda$-badの定義から必ず存在する。そして$j$に対応する箱に入っている数字を$k$とおく。すなわち、$j$ステップ目で$k$行目に箱を追加すると初めてYoung図形でなくなるとする。 またこの箱を悪い箱と呼ぶことにする。このとき
    \begin{equation}
    \omega_{k}(c(T)_j)=\omega_{k-1}(c(T)_j)+(\lambda_{k-1}-\lambda_k)+1  
    \end{equation}
    が成り立つ。ここで、$T$を悪い箱よりも左側にある部分$T_1$と悪い箱を含む右側の部分$T_2$に分割する。
    例えば
    \[
    T=\quad
        \begin{ytableau}
            1&1&1&2&2\\
            2&2&2&3\\
            3&3&*(yellow)4&4\\
            4&5&5
        \end{ytableau}     
    \text{ は }\ydiagram{2}\text{ -bad}
    \]
    においては、黄色い箱が悪い箱で
    \[
    T_1=\quad\begin{ytableau}
        1&1\\
        2&2\\
        3&3\\
        4&5
    \end{ytableau},\qquad 
    T_2=\quad\begin{ytableau}
        1&2&2\\
        2&3\\
        *(yellow)4&4\\
        5
    \end{ytableau}
    \]
    である。すると、半標準タブローの列単調性から悪い箱の下にある箱には$k+1$以上しか存在しないから、
    \[
    \omega_k(c(T)_j)=\omega_k(T_2),\quad \omega_{k-1}(c(T)_j)=\omega_{k-1}(T_2)    
    \]
    よって(7)は
    \[
    \omega_k(T_2)=\omega_{k-1}(T_2)+(\lambda_{k-1}-\lambda_k)+1
    \]
    と書き換えることができる。$\iota(T)$の満たすべき必要条件(6)は
    \begin{align*}
        \omega_k(\iota(T))&=\omega_{k-1}(T)+(\lambda_{k-1}+\lambda_k)+1\\
        \omega_k(\iota(T_1))+\omega_k(\iota(T_2))&=\omega_{k-1}(T_1)+\omega_{k-1}(T_2)+(\lambda_{k-1}+\lambda_k)+1
    \end{align*}
    となるが、$\iota(T_2)=T_2$であると仮定すれば
    \begin{align*}
        \omega_k(\iota(T_1))+\omega_k(T_2)&=\omega_{k-1}(T_1)+\omega_{k-1}(T_2)+(\lambda_{k-1}+\lambda_k)+1\\
        \omega_k(\iota(T_1))&=\omega_{k-1}(T_1)
    \end{align*}
    結局、$\iota(T)$は次のように定義すればよいであろうことがわかる。
    \[
    \text{$\iota(T)$は$T_1$に命題\ref{tableausum}で定義した対合を施し、$T_2$には何もしない}    
    \]
    \[
    T=\quad
    \begin{ytableau}
        1&1&1&2&2\\
        2&2&2&3\\
        3&3&*(yellow)4&4\\
        4&5&5
    \end{ytableau}\qquad\rightarrow\qquad
    \iota(T)=\quad
    \begin{ytableau}
        1&1&1&2&2\\
        2&2&2&3\\
        3&4&*(yellow)4&4\\
        4&5&5
    \end{ytableau} 
    \]



    示すべきことは
    \begin{enumerate}
        \item 実際に$\iota(T)$が$\lambda$-badな半標準タブローであること
        \item $\iota(T)$が(5)をみたすこと
    \end{enumerate}
    である。
    \begin{enumerate}
        \item $\iota$が$T_2$には何もしないことから、$\lambda$-badであることは直ちに従う。よって$\iota(T)$が半標準タブローであることさえ示せばよい。$\iota(T_1)$は命題\ref{tableausum}から半標準タブローであり、$T_2$も半標準タブローだから、問題になるのは$\iota(T_1)$と$T_2$の境界部分である。悪い箱は命題\ref{tableausum}の証明中の意味で自由である。すなわちその上にある箱は$k-1$より真に小さい。
        \begin{quote}
            なぜならもし悪い箱の上に$k-1$があったとすると、$j-1$ステップ目で$k-1$行目に箱を追加してもYoung図形であることは保たれている。よってそのとき$k$行目の箱の数は$k-1$行目の箱の数と同じかそれ以下である。もし同じなら$j-2$ステップの時点では$k-1$行目の箱の数が$k$行目の箱の数より小さいこととなり、これはYoung図形になっていない。$k-1$行目の箱の数以下であるなら$j$ステップ目に$k$行目に箱を追加してもYoung図形であることは保たれるから、悪い箱であることに矛盾する。
        \end{quote}
        よって$T$の悪い箱よりも上部分は考えなくてよい。悪い箱の下部分は半標準タブローの列単調性から$k$より真に大きいのでここも考えなくてよい。したがって問題になるのは悪い箱の左に入っている数が$\iota$によってどうなるかということだけであるが、$\iota$は$k-1$と$k$を適当に入れ替える操作なので単調性は崩れない。


        \item $\iota$は結局のところ$k-1$と$k$を(6)が成り立つように入れ替える操作であるから、
        \begin{align*}
            &\lambda_{k}+\omega_k(\iota(T))+\delta_k=\lambda_{k-1}+\omega_{k-1}(T)+\delta_{k-1}\\
            & l\neq k,\;k-1\implies \lambda_{l}+\omega_l(\iota(T))+\delta_l=\lambda_{l}+\omega_{l}(T)+\delta_{l}\\
        \end{align*}
        が成り立つ。命題\ref{tableausum}の対合を用いているので$\iota$もまた対合であるから
        \begin{align*}
            \lambda_{k-1}+\omega_{k-1}(\iota(T))+\delta_{k-1}
            &=\lambda_{k}+\omega_{k}(\iota^2(T))+\delta_k\\
            &=\lambda_{k}+\omega_{k}(T)+\delta_k
        \end{align*}
        よって$\sigma=(k, k-1)$として
        \[
            \lambda+\omega(\iota(T))+\delta=\sigma(\lambda+\omega(T)+\delta)
        \]
        が成り立つ。
    \end{enumerate}
\end{proof}



Littlewood-Richardson規則の特別な場合として、$\lambda$が一行のYoung図形の場合はPieriの規則と呼ばれ、比較的簡単に計算できる。

\begin{defin}
    Young図形$\mu\subset \nu$, $|\nu|=|\mu|+k$に対して、$\nu/\mu$が水平帯であるとは
    \[
    \text{
        $\nu$に含まれ、$\mu$に含まれない箱が各列にたかだか一つ
        }
    \]
    を満たすことをいう。このことは
    \[
    \nu_l\leq \mu_{l-1}    
    \]
    がすべての$l=2,3,\cdots$について成り立つことと同値である。
\end{defin}

\begin{theo}[Pieriの規則]\label{Pieri}
    $\lambda=(k)$, $\mu\in\mathcal{Y}_n$に対して
    \[
    s_\lambda s_\mu=\sum_{\substack{|\nu|=|\mu|+k\\\nu/\mu\text{は水平帯}}}s_\nu    
    \]
    が成り立つ
\end{theo}

\begin{proof}
    定理\ref{LR}より、$\mu$-goodな形$\lambda$の半標準タブローを考える。$T$が形$\lambda$の$\mu$-goodな半標準タブローであるとする。いま$\lambda$は一行のYoung図形だから、$T$が$\mu$-goodであることは
    \[
    \omega_{l}(T)+\mu_l\leq \mu_{l-1}    
    \]
    がすべての$l=2,3,\cdots, n$に対して成り立つことと同値である。$\nu=\mu+\omega(T)$とすれば、これは
    $\nu/\mu$が水平帯であることに他ならない。
\end{proof}

\begin{eg}
    $\lambda=\quad\ydiagram{2}$, $\mu=\quad\ydiagram{2,1}\quad\in\mathcal{Y}_3$として$s_\lambda s_\mu$を計算する。定理\ref{Pieri}より大きさ$5$のYoung図形で水平帯となっているものを探せばよい。それらは
    \begin{align*}
        \nu_1=\quad\ydiagram[*(yellow)]{2+2}*[*(white)]{2,1},\qquad
        \nu_2=\quad\ydiagram[*(yellow)]{2+1,1+1}*[*(white)]{2,1},\qquad
        \nu_3=\quad\ydiagram[*(yellow)]{0,1+1,1}*[*(white)]{2,1},\qquad
        \nu_4=\quad\ydiagram[*(yellow)]{2+1,0,1}*[*(white)]{2,1}
    \end{align*}
    だから
    \[
    s_\lambda s_\mu=s_{\nu_1}+s_{\nu_2}+s_{\nu_3}+s_{\nu_4}
    \]
\end{eg}

例\ref{1row_schur}より、Pieriの規則は
\[
h_ks_\mu=\sum_{\substack{|\nu|=|\mu|+k\\\nu/\mu\text{は水平帯}}}s_\nu  
\]
と書くこともできる。


\begin{cor}[Youngの規則]\label{Young_rule_for_poly}
    $\lambda\in\mathcal{Y}_n$に対して
    \[
    h_\lambda=h_{\lambda_1}\cdots h_{\lambda_n}=s_{(\lambda_1)}\cdots s_{(\lambda_n)}    
    \]
    とおく。$h_\lambda$は対称多項式であるが、そのSchur多項式への分解について次が成り立つ:
    \[
    h_\lambda=s_{\lambda}+\sum_{\mu>\lambda}k_{\lambda\mu}s_\mu    
    \]
    係数$k_{\lambda\mu}$をKostka数という。
\end{cor}

\begin{proof}
    $\lambda$の行数$n$に関する帰納法で示す。$n=1$のときは$h_\lambda=s_\lambda$ゆえに明らか。$n>1$とする。$\lambda'=(\lambda_2,\cdots,\lambda_n)$として帰納法の仮定より
    \[
    h_{\lambda_2}\cdots h_{\lambda_n}=s_{\lambda'}+\sum_{\mu>\lambda'}k_{\lambda'\mu}s_\mu    
    \]
    と書けるから
    \begin{align*}
        h_{\lambda_1}h_{\lambda_2}\cdots h_{\lambda_n}
        &=s_{(\lambda_1)}\left(s_{\lambda'}+\sum_{\mu>\lambda'}k_{\lambda'\mu}s_\mu \right)
    \end{align*}
    となる。$s_{(\lambda_1)}s_{\lambda'}$について考えると、Pieriの規則(系\ref{Pieri})より、
    \[
        s_{(\lambda_1)}s_{\lambda'}=\sum_{\substack{|\nu|=|\lambda'|+\lambda_1\\\nu/\lambda'\text{は水平帯}}}s_\nu 
    \]
    となるが、$l=2,3,\cdots,n$に対して
    \begin{align*}
        \lambda_l=\lambda'_{l-1}
    \end{align*}
    だから$\lambda/\lambda'$は水平帯である。また、$\nu/\lambda'$が水平帯であるような$|\nu|=|\lambda'|+\lambda_1$をみたす任意の$\nu$について、
    \[
    \nu_l\leq\lambda'_{l-1}=\lambda_l,\qquad l=2,3,\cdots,n
    \]
    だから、
    \[
    \nu_1=|\nu|-(\nu_2+\cdots+\nu_n)\geq |\lambda|-(\lambda_2+\cdots+\lambda_n)=\lambda_1    
    \]
    よって$\lambda\leq \nu$である。したがって
    \[
        s_{(\lambda_1)}s_{\lambda'}=s_{\lambda}+\sum_{\mu>\lambda}s_\mu
    \]
    と書くことができる。一方で、$\mu>\lambda'$に対して$s_{(\lambda_1)}s_\mu$を考えると、再びPieriの規則から
    \begin{equation}\label{tmp}
    s_{(\lambda_1)}s_\mu=\sum_{\substack{|\nu|=|\mu|+\lambda_1 \\ \nu/\mu\text{は水平帯}}}s_\nu
    \end{equation}
    であるが、$\nu>\lambda$となることを示れば証明が完了する。次の補題を示す。
    \begin{lemm}
        Young図形$\lambda,\mu$について、$\mu>\lambda$であるとする。$\mu$に$k$個の箱を、水平帯条件が成り立つように追加してできる最小のYoung図形を$\overline{\mu}$とおき、同様に$\lambda$に$k$個の箱を、水平帯条件が成り立つように追加してできる最小のYoung図形を$\overline{\lambda}$とおく。このとき$\overline{\mu}>\overline{\lambda}$である。
    \end{lemm}

    \begin{proof}
       辞書式順序の定義から、$\lambda$や$\mu$に箱を追加する際、最も小さくなるようにするには各列に左から箱を足していけばよい。
        \begin{gather*}
            k=3, \lambda=\quad\ydiagram{4,3,1}\quad\text{のとき}\\
            \ydiagram[*(yellow)]{0,0,0,1}*[*(white)]{4,3,1}\quad\rightarrow\quad
            \ydiagram[*(yellow)]{0,0,1+1,1}*[*(white)]{4,3,1}\quad\rightarrow\quad
            \ydiagram[*(yellow)]{0,0,1+2,1}*[*(white)]{4,3,1}\quad=\overline{\lambda}
        \end{gather*}
        また、$\lambda<\mu$であることは「$\lambda$の列数と$μ$の列数を右から辞書式順序で比べたとき、$μ$のほうが大きい」ということと同値だから、$λ$, $μ$の列に左から1つずつ箱を足しても大小は保たれる。
    \end{proof}

    式(\ref{tmp})において$\nu\mu$は水平帯であり、$\mu>\lambda'$だったが、$\lambda'$に$\lambda_1$個の箱を水平帯となるように追加した最小のYoung図形は$\lambda$に他ならなかったから、補題により$\nu>\lambda$    
    
\end{proof}


    
\end{document}