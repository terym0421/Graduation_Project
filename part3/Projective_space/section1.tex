\documentclass{ltjsareport}
\input{../../setting.js}

\begin{document}
  \section{射影空間}  

  まずは、数え上げ問題の舞台となる射影空間について解説する。
  \begin{defin}
    $\complex^{n+1}\setminus\{0\}$上の同値関係$\sim$を
    \[
    z\sim w\Leftrightarrow \text{ある$c\in\complex$が存在して}w=cz  
    \]
    と定義する。$\proj^n=(\complex^{n+1}\setminus\{0\})/\sim$を$n$次元(複素)射影空間という。
  \end{defin}

  射影空間を考える理由に次の2つがある。
  \begin{itemize}
    \item $\proj^n$における平行線は1点で交わる
    \item $\proj^n$における$k$次元平面は$\complex^{n+1}$の$k+1$次元線形部分空間と1対1に対応する
  \end{itemize}
  これらのことを説明しよう。
\end{document}