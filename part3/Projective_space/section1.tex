\documentclass{ltjsreport}
\RequirePackage{luatex85}
\usepackage[utf8]{inputenc}
\usepackage{enumerate}
\usepackage{amsthm}
\usepackage{amsfonts}
\usepackage{amsmath}
\usepackage{amssymb}
\usepackage{ytableau}
\usepackage{docmute}
\usepackage{mathtools}
\usepackage{xr}
\usepackage[all]{xy}



\theoremstyle{definition}
\newtheorem{defin}{定義}[subsection]
\newtheorem{theo}[defin]{定理}
\newtheorem{cor}[defin]{系}
\newtheorem{prop}[defin]{命題}
\newtheorem{lemm}[defin]{補題}
\newtheorem{notice}[defin]{注意}
\newtheorem{eg}[defin]{例}


\renewcommand{\labelenumi}{(\roman{enumi})}


\newcommand{\invlimit}{\mathop{\lim_{\longleftarrow}}}
\newcommand{\dirlimit}{\mathop{\lim_{\longrightarrow}}}
\newcommand{\ind}{\text{Ind}\:}
\newcommand{\Hom}{\text{Hom}}
\newcommand{\tr}{\text{tr}\:}
\newcommand{\id}[1]{\text{id}_{#1}}
\newcommand{\sgn}{\mathrm{sgn}}
\newcommand{\res}[1]{\text{Res}_{#1}}
\newcommand{\generated}[1]{\langle\:#1\:\rangle}
\newcommand{\im}{\text{Im }}
\newcommand{\rank}{\text{rank }}
\newcommand{\del}[2]{\frac{\partial #1}{\partial #2}}
\newcommand{\delsametwo}[2]{\frac{\partial^2 #1}{\partial #2^2}}
\newcommand{\delothertwo}[3]{\frac{\partial^2#1}{\partial#2\partial#3}}
\newcommand{\ddel}[2]{\frac{\partial}{\partial #2}#1}
\newcommand{\ddelsametwo}[3]{\frac{\partial^2}{\partial #2^2}#1}
\newcommand{\ddelothertwo}[3]{\frac{\partial^2}{\partial#2\partial#3}#1}
\newcommand{\simneq}{\not\simeq}
\newcommand{\transpose}[1]{^t\!#1}
\newcommand{\ie}{\text{i.e.}}
\newcommand{\inv}[1]{#1^{-1}}
\newcommand{\real}{\mathbb{R}}
\newcommand{\complex}{\mathbb{C}}
\newcommand{\integer}{\mathbb{Z}}
\newcommand{\quotient}{\mathbb{Q}}
\newcommand{\natnum}{\mathbb{N}}
\newcommand{\proj}{\mathbb{P}}
\newcommand{\tensor}[3]{#1\otimes_#2#3}
\newcommand{\map}[3]{#1:#2\rightarrow#3}
\newcommand{\aut}[2]{\mathrm{Aut}_{#1} (#2)}
\newcommand{\hommoph}[2]{\mathrm{Hom}_{#1}(#2)}
\newcommand{\gl}[1]{\mathrm{GL}(#1)}
\newcommand{\set}[2]{\left\{\:#1\:\middle|\:#2\:\right\}}
\newcommand{\pmat}[1]{\begin{pmatrix} #1
\end{pmatrix}}
\newcommand{\vmat}[1]{\begin{vmatrix} #1
\end{vmatrix}}
\newcommand{\br}{\vskip\baselineskip}

\begin{document}
  \section{射影空間}  

  まずは、数え上げ問題の舞台となる射影空間について解説する。
  \begin{defin}
    $\complex^{n+1}\setminus\{0\}$上の同値関係$\sim$を
    \[
    z\sim w\Leftrightarrow \text{ある$c\in\complex$が存在して}w=cz  
    \]
    と定義する。$\proj^n=(\complex^{n+1}\setminus\{0\})/\sim$を$n$次元(複素)射影空間という。$(z_0,z_1,\cdots,z_n)\in\complex^{n+1}$の同値類を$[z_0:z_1:\cdots:z_n]$と書いて、これを斉次座標という。
  \end{defin}

  $\proj^n$の部分集合$U_i$を
  \[
  U_i=\set{[z_0:z_1:\cdots:z_n]\in\proj^n}{z_i\neq 0}=\set{[z_0:\cdots:z_{i-1}:1:z_{i+1}:\cdots:z_n]\in\proj^n}{z_i\neq 0} 
  \]
  によって定める。$[z_0:z_1\cdots:z_i:\cdots:z_n]\in U_i$のとき
  \[
    [z_0:z_1\cdots:z_i:\cdots:z_n]=[\frac{z_0}{z_i}:\frac{z_1}{z_i}:\cdots:1:\cdots:\frac{z_n}{z_i}] 
  \]
  だから、
  \[
    \varphi_i:U_i\owns [z_0:z_1\cdots:z_i:\cdots:z_n]
    \mapsto 
    (\frac{z_0}{z_i},\frac{z_1}{z_i},\cdots,\frac{z_{i-1}}{z_i},\frac{z_{i+1}}{z_i},\cdots:\frac{z_n}{z_i})
    \in\complex^{n}
  \]
  のような写像を考えることができるが、$\varphi_i$は全単射であり、$U_i$と$\complex^n$を同一視することができる\footnote{
    $\proj^n$に$\simeq$により定まる商位相を入れたとき$\varphi_i$は同相となる。
  }。
  \[
  \proj^n=\bigcup_{i=0}^nU_i  
  \]
  が成り立つから、$\proj^n$は$\complex^n$を$n+1$枚貼り合わせた構造を持っている。

  複素数上で定義をしたが実数上でもまったく同様に実射影空間が定義できる。射影空間を考える理由に次の2つがある。
  \begin{itemize}
    \item $\proj^n$における$k$次元平面は$\complex^{n+1}$の$k+1$次元線形部分空間と1対1に対応する。
    \item $\proj^n$における平行線は1点で交わる
  \end{itemize}
  これらのことを説明しよう。

  \begin{defin}
    $A$をランク$n-r$の$(n-r)\times (n+1)$行列とする。
    \[
    \set{[z_0:z_1:\cdots:z_n]\in\proj^n}{A\pmat{z_0\\z_1\\\vdots\\z_n}=\pmat{0\\\vdots\\0}}  
    \]
    の形の$\proj^n$の部分集合を$\proj^n$のr次元平面という。とくに$r=1$のときは直線, $r=n-1$のときは超平面と呼ぶ。
  \end{defin}

  \begin{eg}
    直線
    \[
    L=\set{[z_0:z_1:z_2]\in\proj^2}{z_2-2z_0-z_1=0}  
    \]
    を考えてみる。
    \begin{align*}
    &L\cap U_0=\set{[1:z_1:z_2]\in\proj^2}{z_2-z_1-2=0} \approx \set{(x,y)\in\complex^2}{x-y-2=0}\subset \complex^2\\
    &L\cap U_1=\set{[z_0:1:z_2]\in\proj^2}{z_2-2z_0-1=0}
    \approx \set{(x,y)\in\complex^2}{y-2x-1=0}\subset\complex^2\\
    &L\cap U_2=\set{[z_0:z_1:1]\in\proj^2}{1-2z_0-z_1=0}
    \approx \set{(x,y)\in\complex^2}{1-2x-y=0}\subset\complex^2
    \end{align*}
    
  \end{eg}
\end{document}