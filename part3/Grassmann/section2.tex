\documentclass{ltjsreport}
\input{../../setting.tex}

\begin{document}
\section{Grassmann多様体とSchubert多様体}
\subsection{Grassmann多様体}

前節の準備をもとに数え上げ問題を定式化しよう。以下では係数体はすべて$\complex$で考えているとする。

\begin{defin}
  $n+1$次元ベクトル空間の$d$次元部分空間全体のなす集合を$G(d,n)$と書き、これをGrassmann多様体という。
\end{defin}

第3部冒頭で述べた数え上げ問題においては$\proj^3$中の直線全体を考えたいから、$G(2,4)$を考察していくことになる。重要な考え方として、ある条件をみたす直線の集合を$G(2,4)$の部分多様体としてとらえることで、「複数の条件を満たす直線の数え上げ$\Leftrightarrow$いくつかの$G(2,4)$の部分多様体の交点を数える」という問題の変換を行う。


  

\end{document}