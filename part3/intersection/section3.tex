\documentclass{ltjsreport}
\RequirePackage{luatex85}
\usepackage[utf8]{inputenc}
\usepackage{enumerate}
\usepackage{amsthm}
\usepackage{amsfonts}
\usepackage{amsmath}
\usepackage{amssymb}
\usepackage{ytableau}
\usepackage{docmute}
\usepackage{mathtools}
\usepackage{xr}
\usepackage[all]{xy}



\theoremstyle{definition}
\newtheorem{defin}{定義}[subsection]
\newtheorem{theo}[defin]{定理}
\newtheorem{cor}[defin]{系}
\newtheorem{prop}[defin]{命題}
\newtheorem{lemm}[defin]{補題}
\newtheorem{notice}[defin]{注意}
\newtheorem{eg}[defin]{例}


\renewcommand{\labelenumi}{(\roman{enumi})}


\newcommand{\invlimit}{\mathop{\lim_{\longleftarrow}}}
\newcommand{\dirlimit}{\mathop{\lim_{\longrightarrow}}}
\newcommand{\ind}{\text{Ind}\:}
\newcommand{\Hom}{\text{Hom}}
\newcommand{\tr}{\text{tr}\:}
\newcommand{\id}[1]{\text{id}_{#1}}
\newcommand{\sgn}{\mathrm{sgn}}
\newcommand{\res}[1]{\text{Res}_{#1}}
\newcommand{\generated}[1]{\langle\:#1\:\rangle}
\newcommand{\im}{\text{Im }}
\newcommand{\rank}{\text{rank }}
\newcommand{\del}[2]{\frac{\partial #1}{\partial #2}}
\newcommand{\delsametwo}[2]{\frac{\partial^2 #1}{\partial #2^2}}
\newcommand{\delothertwo}[3]{\frac{\partial^2#1}{\partial#2\partial#3}}
\newcommand{\ddel}[2]{\frac{\partial}{\partial #2}#1}
\newcommand{\ddelsametwo}[3]{\frac{\partial^2}{\partial #2^2}#1}
\newcommand{\ddelothertwo}[3]{\frac{\partial^2}{\partial#2\partial#3}#1}
\newcommand{\simneq}{\not\simeq}
\newcommand{\transpose}[1]{^t\!#1}
\newcommand{\ie}{\text{i.e.}}
\newcommand{\inv}[1]{#1^{-1}}
\newcommand{\real}{\mathbb{R}}
\newcommand{\complex}{\mathbb{C}}
\newcommand{\integer}{\mathbb{Z}}
\newcommand{\quotient}{\mathbb{Q}}
\newcommand{\natnum}{\mathbb{N}}
\newcommand{\proj}{\mathbb{P}}
\newcommand{\tensor}[3]{#1\otimes_#2#3}
\newcommand{\map}[3]{#1:#2\rightarrow#3}
\newcommand{\aut}[2]{\mathrm{Aut}_{#1} (#2)}
\newcommand{\hommoph}[2]{\mathrm{Hom}_{#1}(#2)}
\newcommand{\gl}[1]{\mathrm{GL}(#1)}
\newcommand{\set}[2]{\left\{\:#1\:\middle|\:#2\:\right\}}
\newcommand{\pmat}[1]{\begin{pmatrix} #1
\end{pmatrix}}
\newcommand{\vmat}[1]{\begin{vmatrix} #1
\end{vmatrix}}
\newcommand{\br}{\vskip\baselineskip}

\begin{document}
\section{Grassmann多様体の交叉理論}
\subsection{一般の位置}

前節で導入したように、$\proj^n$の固定された旗に対して、ある特定の位置条件にある線形部分多様体をパラメータづける空間がSchubert多様体であった。したがって今度は複数の旗に対してそれぞれのSchubert多様体がどのように交わるかを記述することを考える。ここで重要になるのが2つの旗が一般の位置にあるという条件である。これが第3章冒頭に述べた「ある程度一般の状況」という言葉の意味である。一般の位置にある2つの旗のSchubert多様体に対してはその次元がうまくふるまうことが知られており、それによって交点の数え上げに整った代数的・組み合わせ的計算が現れる。

\begin{defin}
  $F^\bullet$, $E^\bullet$を$\complex^n$の旗とする。各$k$において
  \[
  F^k\cap E^{n-k}=0  
  \]
  が成り立つとき、$F^\bullet,E^\bullet$は一般の位置にあるという。
\end{defin}

\begin{eg}
  旗$F^\bullet$に対して$F^k=\generated{v_{k+1},\cdots,v_n}$となる基底$v_1,\cdots,v_n$をとる。$F^k_{op}$を
  \[
  F^k_{op}=\generated{v_1,\cdots,v_{n-k}}  
  \]
  とすれば
  \[
  F^{n-k}_{op}\cap F^k=\generated{v_1,\cdots,v_k}\cap\generated{v_{k+1},\cdots,v_n}  =0
  \]
  となるから、$F^\bullet,F^\bullet_{op}$は一般の位置にある。$F^\bullet_{op}$を$F^\bullet$の反対旗という。
\end{eg}

\begin{eg}
  $F^\bullet$を標準旗とする。$g\in \gl_n(\complex)$を$g=(v_1,\cdots,v_n)=(a_{ij})$とすれば
  \[
  gF^k=\generated{v_{k+1},\cdots,v_n}
  \]
  である。$F^\bullet$, $gF^\bullet$が一般の位置にあるための必要十分条件は、各$k$において
  \[
  e_{k+1},\cdots,e_n,v_{n-k+1},\cdots,v_n  
  \]
  が一次独立となることである。すなわち$\det(e_{k+1},\cdots,e_n,v_{n-k+1},\cdots,v_n)\neq 0$である。よって
  $F^\bullet$, $gF^\bullet$が一般の位置にあるような$g$のなす$\gl_n(\complex)$の部分集合はZariski開集合である。Zariski開集合は稠密であるので、ほとんどすべての旗は一般の位置にあるといってよい。
\end{eg}

\begin{prop}
  $F^\bullet$, $E^\bullet$を一般の位置にある旗とする。$\complex^n$の基底$v_1,\cdots,v_n$を適当にとって、
  \[
  F^k=\generated{v_{k+1},\cdots,v_n}, \quad E^{n-k}=\generated{v_1,\cdots,v_k}  
  \]
  となるようにできる。
\end{prop}

\begin{proof}
  $\dim (E^{n-k}\cap F^{k-1})=1$を示す。
\end{proof}




\end{document}