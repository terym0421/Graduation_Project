\chapter*{はじめに}
Schur多項式は数学の様々な分野に現れる対称多項式で、その組み合わせ論的性質がまったく関係のないように思える問題の解を与えることがしばしばある。本稿では、表現論と幾何学でSchur多項式がどのように用いられるかを紹介する。


第1章ではまずSchur多項式の定義と基本性質について述べる。特に重要なのがSchur多項式が対称多項式環の基底をなすという事実である。これにより、2つのSchur多項式の積はSchur多項式の線形結合で表せられることがわかるが、その係数はLittlewood-Richardson規則という組み合わせ論的ルールによって記述される。その系として、完全対称多項式をSchur多項式の線形結合で表すYoungの規則を解説する。


第2章では有限群の表現論の一般論と、その具体例として対称群の表現論、およびSchur-Weyl双対性について紹介する。表現論とは群や多元環などの抽象的な代数系を、ベクトル空間への作用(これを表現という)を通して研究する分野である。第2章では有限群の表現論の一般論を解説し、その後最も基本的な対称群の表現論と一般線形群の表現論について解説する。表現論の目標の一つは代数系が与えられたとき、その既約表現(ある意味で最も小さい表現)を分類することである。特に有限群の複素有限次元表現の場合、既約表現の同値類と群の共役類の個数は等しいことが知られている。一般に共役類と既約表現に標準的な全単射を構成することは期待できないが、対称群の場合はYoung対称子によってパラメータ付けることができる。対称群の表現の同値類をすべてあつめたものは表現環という環の構造をもつ。表現環は対称関数環という、対称多項式の無限変数への一般化にあたるものとまったく同型であることが示される。これによって表現の研究に対称多項式の組み合わせ論的知識を用いることができ、第1章で導入したLittlewood-Richardson規則などが活かされる。一般線形群の表現においては、Weyl表現について解説する。Weyl表現とは、テンソル積空間$V^{\otimes k}$の分解という素朴な問題の解として現れる表現である。その$V^{\otimes k}$における重複度は、対称群の既約表現と双対的な関係になっており、これはSchur-Weyl双対性と呼ばれる。この双対性から、Weyl表現の分析にも対称関数の知識が役立たれる様子をみる。

第3章では数え上げ幾何学を紹介する。数え上げ幾何の古典的な問題として、幾何学的な条件を満たす直線の本数を数えることがある。そのような条件を満たす直線の集合はSchubert多様体と呼ばれる空間をなし、数え上げ問題を解くことはSchubert多様体の交叉を調べることに対応する。交叉を調べる際にまたしてもLittlewood-Richardson規則が現れ、これを用いて数え上げ問題に解答を与えることが目標である。


本稿で用いる記号を整理しておく。$\integer$, $\quotient$, $\real$, $\complex$でそれぞれ、整数, 有理数, 実数, 複素数全体のなす集合を表すものとする。