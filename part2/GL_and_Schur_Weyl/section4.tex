\documentclass{ltjsreport}
\input{../../setting.tex}


\begin{document}

\section{一般線形群の表現とSchur-Weyl双対性}

前節までで対称群の既約表現に関して解説してきたが、次に対称群と表現論的に関係の深い一般線形群の表現について解説する。とくに多項式表現と呼ばれる表現のクラスが、Schur-Weyl双対性を通して対称群の表現と密接にかかわりあっている。


\subsection{Schur-Weyl双対性}

次のような問題を考えることから始める。$V$を$n$次元ベクトル空間としたとき、テンソル空間$V^{\otimes k}$の部分空間として対称テンソル空間$\Sym^k(V)$, 交代テンソル空間$\Alt^k(V)$というものが
\begin{align*}
  &\Sym^k(V)=\set{v_1\otimes\cdots\otimes v_k}{v_{\sigma(1)}\otimes\cdots\otimes v_{\sigma (k)}=v_1\otimes\cdots\otimes v_k,\quad\text{for all $\sigma\in\mathfrak{S}_k$}}\\
  &\Alt^k(V)=\set{v_1\otimes\cdots\otimes v_k}{v_{\sigma(1)}\otimes\cdots\otimes v_{\sigma (k)}=\sgn(\sigma)v_1\otimes\cdots\otimes v_k,\quad\text{for all $\sigma\in\mathfrak{S}_k$}}
\end{align*}
によって定義された。
\[
\dim\Sym^k(V)=\pmat{n+k-1\\k},\qquad\dim\Alt^k(V)=\pmat{n\\k}  
\]
より、$k=2$の場合
\[
\dim\Sym^2(V)+\dim\Alt^2(V)=n^2=\dim V\otimes V  
\]
だから、$\Sym^k(V)\cap\Alt^k(V)=0$に注意すれば
\[
V\otimes V=\Sym^2(V)\oplus\Alt^2(V)  
\]
が成り立つ。$k>2$のときは次元が足りず、対称テンソルと交代テンソル以外の部分がでてくる。その分解を与える規則を考える。特に、表現を含めた分解を考えることが鍵になる。

$\sigma\in \mathfrak{S}_k$に対して、
\[
\sigma(v_1\otimes\cdots\otimes v_k)=v_{\inv{\sigma}(1)}\otimes\cdots\otimes v_{\inv{\sigma}(k)}  
\]
によって$V^{\otimes k}$を$\mathfrak{S}_k$の表現とみなす。あるいは同じことだが
\[
  (v_1\otimes\cdots\otimes v_k)\sigma=v_{\sigma(1)}\otimes\cdots\otimes v_{\sigma(k)}
\]
によって右$\complex[\mathfrak{S}_k]$加群とみなす。

さらに、$g\in \gl(V)$に対して
\[
g(v_1\otimes\cdots\otimes v_k)=gv_1\otimes\cdots\otimes gv_k  
\]
とによって$V^{\otimes k}$は$\gl(V)$の表現とみなすこともできる。$V^{\otimes k}$は$\mathfrak{S}_k$, $\gl(V)$両方の作用を同時に受けている。さらに次が成り立つ。

\begin{prop}\label{comm_of_sym_gl}
  任意の$\sigma\in\mathfrak{S}_k$, $g\in\gl(V)$に対して、
  \[
  \sigma g(v_1\otimes\cdots\otimes v_k)
  =
  g \sigma(v_1\otimes\cdots\otimes v_k)
  \]
  である。
\end{prop}

\begin{proof}
  $u_i=gv_i$とおく。
  \begin{align*}
    \sigma g(v_1\otimes\cdots\otimes v_k)
    &=\sigma(gv_1\otimes\cdots\otimes gv_k)\\
    &=\sigma(u_1\otimes\cdots\otimes u_k)\\
    &=u_{\inv{\sigma}(1)}\otimes\cdots\otimes u_{\inv{\sigma}(k)}\\
    &=gv_{\inv{\sigma}(1)}\otimes\cdots\otimes gv_{\inv{\sigma}(k)}\\
    &=g\sigma (v_1\otimes\cdots\otimes v_k)
  \end{align*}
\end{proof}

このとき、
$c_{(k)}=\sum_{\sigma\in \mathfrak{S}_n}\sigma\in\complex[\mathfrak{S}_n]$, $c_{1^k}=\sum_{\sigma\in \mathfrak{S}_n}\sgn(\sigma)\sigma\in\complex[\mathfrak{S}_n]$とおくと
\[
\Sym^k(V)=c_{(k)}V^{\otimes k},\qquad\Alt^k(V)=c_{1^k}V^{\otimes k}
\]
となることがわかり、命題\ref{comm_of_sym_gl}よりこの2つは$\gl(V)$の表現でもある。よって$k>2$のときにも、$\lambda\in\mathcal{P}_k$に対するYoung対称子$c_\lambda$による像
\[
W_\lambda(V)=c_\lambda V^{\otimes k}  
\]
を考察することは自然である。再び\ref{comm_of_sym_gl}より$W_\lambda(V)$は$\gl(V)$の部分表現になるが、このとき次が成り立つ

\begin{theo}[Schur-Weyl双対性]\label{schur_weyl}
  $W_\lambda(V)$は$\gl(V)$の既約表現であり、$\mathfrak{S}_\lambda\times\gl(V)$の表現として
  \[
  V^{\otimes k}\simeq \bigoplus_{\lambda\in\mathcal{P}_k}S_\lambda\boxtimes W_\lambda(V)  
  \]
  が成り立つ。
\end{theo}


ポイントになるのは次の補題である。

\begin{lemm}\label{DCT}
  $G$を有限群, $A=\complex[G]$とし、$V$を右$A$加群, $B=\End_A(V)$とする。$\{U_i\}_i$を互いに同型でない単純$A$加群とし、$W_i=\Hom_A(U_i,V)\simeq\Hom_A(V,U_i)$とする。$f\in W_i, b\in B$に対して
  \[
  bf=b\circ f  
  \]
  として$W_i$を左$B$加群とみなす。
  このとき$(A,B)$-両側加群として分解
    \[
      V\simeq \bigoplus_iU_i\otimes_\complex W_i  
    \]
  が成り立ち、さらに$W_i$は単純$B$加群である。
\end{lemm}

\begin{proof}
  系\ref{multiplicity}より、右$A$加群として
  \[
  V\simeq \bigoplus_iU_i\otimes_{\complex}W_i  
  \]
  と分解されるが、この同型を与える写像$\phi_i:U_i\otimes_\complex W_i\rightarrow V$について、$a\in A, b\in B$に対して
  \[
  \phi_i(xa\otimes bf)=\phi_i(xa\otimes bf)=bf(xa)=(bf(x))a=b(f(x)a)  
  \]
  となるから$\phi_i$は両側加群の準同型である。よってこの分解は$(A,B)$-両側加群として成り立つ。$W_i$の単純性を示す。任意の$f\in W_i$に対して$Bf=W_i$が成り立つことを示せばよい。$U_i$は単純$A$加群だから、$0$でない$U_i$の元$u$をとって、$U_i=uA$とおく。任意の$f,f'\in W_i$に対して$v=f(u), v'=f(u')$とおく。Maschkeの定理から
  \[
  V=vA\oplus X  
  \]
  を満たす部分$A$加群$X$が存在する。そこで、$b:V\rightarrow V$を
  \[
  b(va)=v'a,\qquad b(x)=0\:(x\in X)  
  \]
  によって定めると、$b\in\End_A(V)=B$であり、$f'=b\circ f$である。よって示せた。
\end{proof}


\begin{lemm}\label{gen_of_symtensor}
  $V$を$n$次元ベクトル空間とする。$\Sym^k(V)$は$\{v\otimes\cdots\otimes v\}_{v\in V}$によって生成される
\end{lemm}

\begin{proof}
  $n$変数$k$次斉次多項式のなすベクトル空間を$S$とおく。主張は$S$が1次斉次多項式の$k$乗で生成されることと同値である。単項式
  \[
  x_1^{i_1}\cdots x_n^{i_n},\quad i_1+\cdots+i_n=k
  \]
  が生成されることを示せば十分である。$f_0(x_1,\cdots,x_n)=(x_1+\cdots +x_n)^k$とおく。
  \[
  f_1(x_1,\cdots,x_n)=f_0(2x_1,x_2,\cdots,x_n)-2^{k}f_0(x_1,\cdots,x_n)  
  \]
  とすれば、$f_1$は$x_1^k$を含む項をもたない。次に
  \[
  f_2(x_1,\cdots,x_n)=f_1(2x_1,x_2,\cdots,x_n)-2^{k-1}f_1(x_1,\cdots,x_n)  
  \]
  とすれば$f_2$は$x_1^{k}$, $x_1^{k-1}$を含む項をもたない。この操作を$i_1$以外に対して行えば、最終的に$x_1^{i_1}$を含む項以外をもたないような多項式$g_0(x_1,\cdots,x_n)$を得る。そして$g_0$は作り方から、一次斉次多項式の$k$乗の線形結合で表される。同様に
  \[
  g_1(x_1,\cdots,x_n)=g_0(x_1,2x_2,x_3,\cdots,x_n)-2^{k}g_0(x_1,\cdots,x_n)  
  \]
  とすれば$g_1$は$x_2^k$を含む項をもたない。再びこの操作を繰り返して$x_1^{i_1}$, $x_2^{i_2}$を含む項以外をもたないような多項式を得る。これを繰り返していけば、有限回のうちに$x_1^{i_1}x_2^{i_2}\cdots x_n^{i_n}$を作ることができる。
\end{proof}


\begin{lemm}\label{lemma}
  $A$を$\End(V^{\otimes k})$における$\complex[\mathfrak{S}_k]$の像とし、$B$を$\End(V^{\otimes k})$における$\complex[\gl(V)]$の像とする。$B=\End_A(V^{\otimes k})$が成り立つ。
\end{lemm}

\begin{proof}
  $\End_A(V^{\otimes k})=(\End(V^{\otimes k}))^{\mathfrak{S}_n}=(\End(V)^{\otimes k})^{\mathfrak{S}_n}=\Sym^k(\End(V))$であるから補題\ref{gen_of_symtensor}より、$\End_A(V^{\otimes k})$は$\{X\otimes\cdots\otimes X\}_{X\in\End(V)}$によって生成される。$\gl(V)\subset\End(V)$なのだから、$B\subset \End_A(V^{\otimes k})$は直ちに従う。$g\in\gl(V)$に対して$g\otimes \cdots\otimes g$が生成する$\End(V^{\otimes k})$の部分代数を$B'$とする。任意の(正則とは限らない)$X\in\End(V)$に対して$X\otimes \cdots\otimes X$が$B'$に含まれることを示せばよい。$X+tE$は有限個の$t$を除いて正則である
  \footnote{
    行列式は$t$の多項式
  }
  から、$X$に収束する$\gl(V)$の点列$(X+t_iE)$が存在する
  \footnote{
    すなわち$\gl(V)$は$\End(V)$の稠密集合
  }。$\End(V^{\otimes k})$は有限次元だから$B'$は閉部分空間であるので、
  \[
  X^{\otimes k}=\lim_{i\rightarrow\infty}(X+t_iE)^{\otimes k}\in B'  
  \]
  が成り立つ\footnote{
    表現は連続だからこのような極限操作が可能である。
  }
\end{proof}

定理\ref{schur_weyl}を示そう。
\begin{proof}
  補題\ref{lemma}と定理\ref{DCT}より$\mathfrak{S}_k\times\gl(V)$の表現(すなわち$(\complex[\mathfrak{S}_k],\complex[\gl(V)])$-両側加群)として
  \[
  V^{\otimes k}=\bigoplus_{\lambda\in\mathcal{P}_k}S_\lambda\boxtimes \Hom_A(V^{\otimes k},S_\lambda,)  
  \]
  と分解される。
  
  ここで対称群の任意の表現は自己双対的である。すなわち、任意の指標$\chi$に対して、
  \[
  \chi^*(g)=\overline{\chi(g)}=\chi(\inv{g})=\chi(g)  
  \]
  が成り立つ。これより表現として$(V^{\otimes k})^*\simeq V^{\otimes k}$となる。

  $A=\complex[\mathfrak{S}_k]$として、$S_\lambda=Ac_\lambda$だから
  \begin{align*}
    \Hom_A(V^{\otimes k},S_\lambda)
    &=(V^{\otimes k})^*\otimes_AS_\lambda\\
    &=V^{\otimes k}\otimes_AAc_\lambda\\
    &=V^{\otimes k}c_\lambda\\
    &=W_\lambda(V)
  \end{align*}

  定理\ref{DCT}より$\Hom_A(S_\lambda,V^{\otimes k})$は既約$\gl(V)$表現だから$W_\lambda$は既約である。
\end{proof}







\begin{cor}
  $\lambda\in\mathcal{P}_k$に対して、$d_\lambda=\dim_\complex S_\lambda$とおく。$\gl(V)$の表現として
  \[
  V^{\otimes k}\simeq \bigoplus_{\lambda\in\mathcal{P}_k} W_\lambda(V)^{\otimes d_\lambda} 
  \]
  が成り立つ。
\end{cor}

\begin{eg}
  $\dim V=n$とする。$\lambda$が$n$行よりも長いYoung図形のとき$W_\lambda(V)=0$となる。$\lambda$の共役Young図形を$\lambda^*=(\lambda^*_1,\cdots,\lambda^*_s)$, $\lambda_1^*>n$とおく。このとき、
  \[
  b_\lambda
  =\left(
    \sum_{\mathfrak{S}_{\lambda^*_s}}\sgn(\sigma_s)\sigma_s
   \right)\cdots 
   \left(
    \sum_{\mathfrak{S}_{\lambda^*_1}}\sgn(\sigma_1)\sigma_1
   \right)  
  \]
  と書くことができる。$\tilde{b}_\lambda=\sum_{\mathfrak{S}_{\lambda^*_1}}\sgn(\sigma_1)\sigma_1$とする。
  \[
  \lambda=\:\begin{ytableau}
    1 & 5\\
    2 & 6\\
    3 \\
    4
  \end{ytableau}
    ,\text{ のとき }
  b_\lambda=
  \left(
    \sum_{\tau\in\mathfrak{S}_2}\sgn(\tau)\tau
  \right)
  \left(
    \sum_{\sigma\in\mathfrak{S}_4}\sgn(\sigma)\sigma
  \right),\:\tilde{b}_\lambda=\sum_{\sigma\in\mathfrak{S}_4}\sgn(\sigma)\sigma
  \]
  $V^{\otimes k}$を$\lambda$の各箱に$V$の元が書かれているものと同一視する。$V$の基底を$e_1,\cdots,e_n$とすれば$V^{\otimes k}$の元は$e_1,\cdots,e_n$を$\lambda$の各箱に配置した元で生成される。
  \[
  \lambda=\ydiagram{2,2,1,1}\text{ のとき }V^{\otimes k}=\generated{\begin{ytableau}
                e_{i_1} & e_{i_5}\\
                e_{i_2} & e_{i_6}\\
                e_{i_3}\\
                e_{i_4}
              \end{ytableau}}
  \]
  このとき、$\lambda^*_1>n$より、$e_{i_1},\cdots,e_{i_{\lambda^*_1}}$には必ず重複がある。よって
  \[
  \tilde{b}_\lambda(e_{i_1}\otimes\cdots\otimes e_{i_{\lambda^*_1}})=0
  \]
  だから、$b_\lambda V^{\otimes k}=0$である。
\end{eg}

\begin{eg}
  $k=3$, $n>k$とする。$\lambda=\:\ydiagram{2,1}\:$として
  \[
  V^{\otimes 3}=\Sym^3(V)\oplus (S_\lambda\boxtimes W_\lambda(V))\oplus \Alt^3(V)  
  \]
  であり、$\dim S_{\lambda}=2$であるので
  \[
  \dim W_\lambda(V)=\frac{1}{2}(n^3-\dim\Sym^3(V)-\dim\Alt^3(V))=\frac{1}{12}(4n^3-3n^2-13n-6)
  \]
\end{eg}


\subsection{一般線形群の表現に対する指標}

前節でWeyl表現という、一般線形群の既約表現を構成したが、これらが互いに同型でないことは示していなかった。一般線形群の表現に対しても指標と呼ばれる関数を定義することができ、有限群の場合ほどではないが強力な道具となる。

\begin{defin}
  $\rho:\gl(V)\rightarrow \gl(W)$を表現とする。$H$を$\gl(V)$の対角行列のなす部分群とする。
  $\ch_W:H\rightarrow \complex$を
  \[
  \ch_W(g)=\tr(\rho(g))  
  \]
  によって定め、$\rho$の指標という。
\end{defin}

\begin{eg}
  $\lambda=(k)\in\mathcal{P}_k$, $\dim_\complex V=n>k$とする。$\ch_\lambda=\ch_{W_\lambda(V)}$とすると、$g\in H$に対して、$g$の固有値を$x_1,\cdots,x_n$とすると、
  \[
  \ch_\lambda(g)=h_k(x_1,\cdots,x_n) 
  \]
  となる。実際$g=\diag(x_1,\cdots,x_n)$とおくと$\Sym^k(V)$の基底$e_{i_1}\cdots e_{i_k}$に対して
  \[
  g^{\otimes k}(e_{i_1}\cdots e_{i_k})=x_{i_1}\cdots x_{i_k}e_{i_1}\cdots e_{i_k}  
  \]
  である。
\end{eg}

\begin{eg}
  同様に$\lambda=(1^k)$ならば、
  \[
  \ch_\lambda(g)=e_k(x_1,\cdots,x_n)  
  \]
  である。
\end{eg}




\end{document}