\documentclass{ltjsreport}
\RequirePackage{luatex85}
\usepackage[utf8]{inputenc}
\usepackage{enumerate}
\usepackage{amsthm}
\usepackage{amsfonts}
\usepackage{amsmath}
\usepackage{amssymb}
\usepackage{ytableau}
\usepackage{docmute}
\usepackage{mathtools}
\usepackage{xr}
\usepackage[all]{xy}



\theoremstyle{definition}
\newtheorem{defin}{定義}[subsection]
\newtheorem{theo}[defin]{定理}
\newtheorem{cor}[defin]{系}
\newtheorem{prop}[defin]{命題}
\newtheorem{lemm}[defin]{補題}
\newtheorem{notice}[defin]{注意}
\newtheorem{eg}[defin]{例}


\renewcommand{\labelenumi}{(\roman{enumi})}


\newcommand{\invlimit}{\mathop{\lim_{\longleftarrow}}}
\newcommand{\dirlimit}{\mathop{\lim_{\longrightarrow}}}
\newcommand{\ind}{\text{Ind}\:}
\newcommand{\Hom}{\text{Hom}}
\newcommand{\tr}{\text{tr}\:}
\newcommand{\id}[1]{\text{id}_{#1}}
\newcommand{\sgn}{\mathrm{sgn}}
\newcommand{\res}[1]{\text{Res}_{#1}}
\newcommand{\generated}[1]{\langle\:#1\:\rangle}
\newcommand{\im}{\text{Im }}
\newcommand{\rank}{\text{rank }}
\newcommand{\del}[2]{\frac{\partial #1}{\partial #2}}
\newcommand{\delsametwo}[2]{\frac{\partial^2 #1}{\partial #2^2}}
\newcommand{\delothertwo}[3]{\frac{\partial^2#1}{\partial#2\partial#3}}
\newcommand{\ddel}[2]{\frac{\partial}{\partial #2}#1}
\newcommand{\ddelsametwo}[3]{\frac{\partial^2}{\partial #2^2}#1}
\newcommand{\ddelothertwo}[3]{\frac{\partial^2}{\partial#2\partial#3}#1}
\newcommand{\simneq}{\not\simeq}
\newcommand{\transpose}[1]{^t\!#1}
\newcommand{\ie}{\text{i.e.}}
\newcommand{\inv}[1]{#1^{-1}}
\newcommand{\real}{\mathbb{R}}
\newcommand{\complex}{\mathbb{C}}
\newcommand{\integer}{\mathbb{Z}}
\newcommand{\quotient}{\mathbb{Q}}
\newcommand{\natnum}{\mathbb{N}}
\newcommand{\proj}{\mathbb{P}}
\newcommand{\tensor}[3]{#1\otimes_#2#3}
\newcommand{\map}[3]{#1:#2\rightarrow#3}
\newcommand{\aut}[2]{\mathrm{Aut}_{#1} (#2)}
\newcommand{\hommoph}[2]{\mathrm{Hom}_{#1}(#2)}
\newcommand{\gl}[1]{\mathrm{GL}(#1)}
\newcommand{\set}[2]{\left\{\:#1\:\middle|\:#2\:\right\}}
\newcommand{\pmat}[1]{\begin{pmatrix} #1
\end{pmatrix}}
\newcommand{\vmat}[1]{\begin{vmatrix} #1
\end{vmatrix}}
\newcommand{\br}{\vskip\baselineskip}


\begin{document}

\section{一般線形群の表現とSchur-Weyl双対性}

前節までで対称群の既約表現に関して解説してきたが、次に対称群と表現論的に関係の深い一般線形群の表現について解説する。とくに多項式表現と呼ばれる表現のクラスが、Schur-Weyl双対性を通して対称群の表現と密接にかかわりあっている。


\subsection{Schur-Weyl双対性}

次のような問題を考えることから始める。$V$を$n$次元ベクトル空間としたとき、テンソル空間$V^{\otimes k}$の部分空間として対称テンソル空間$\Sym^k(V)$, 交代テンソル空間$\Alt^k(V)$というものが
\begin{align*}
  &\Sym^k(V)=\set{v_1\otimes\cdots\otimes v_k}{v_{\sigma(1)}\otimes\cdots\otimes v_{\sigma (k)}=v_1\otimes\cdots\otimes v_k,\quad\text{for all $\sigma\in\mathfrak{S}_k$}}\\
  &\Alt^k(V)=\set{v_1\otimes\cdots\otimes v_k}{v_{\sigma(1)}\otimes\cdots\otimes v_{\sigma (k)}=\sgn(\sigma)v_1\otimes\cdots\otimes v_k,\quad\text{for all $\sigma\in\mathfrak{S}_k$}}
\end{align*}
によって定義された。
\[
\dim\Sym^k(V)=\pmat{n+k-1\\k},\qquad\dim\Alt^k(V)=\pmat{n\\k}  
\]
より、$k=2$の場合
\[
\dim\Sym^2(V)+\dim\Alt^2(V)=n^2=\dim V\otimes V  
\]
だから、$\Sym^k(V)\cap\Alt^k(V)=0$に注意すれば
\[
V\otimes V=\Sym^2(V)\oplus\Alt^2(V)  
\]
が成り立つ。$k>2$のときは次元が足りず、対称テンソルと交代テンソル以外の部分がでてくる。その分解を与える規則を考える。特に、表現を含めた分解を考えることが鍵になる。

$\sigma\in \mathfrak{S}_k$に対して、
\[
\sigma(v_1\otimes\cdots\otimes v_k)=v_{\inv{\sigma}(1)}\otimes\cdots\otimes v_{\inv{\sigma}(k)}  
\]
によって$V^{\otimes k}$を$\mathfrak{S}_k$の表現とみなす。あるいは同じことだが
\[
  (v_1\otimes\cdots\otimes v_k)\sigma=v_{\sigma(1)}\otimes\cdots\otimes v_{\sigma(k)}
\]
によって右$\complex[\mathfrak{S}_k]$加群とみなす。

さらに、$g\in \gl(V)$に対して
\[
g(v_1\otimes\cdots\otimes v_k)=gv_1\otimes\cdots\otimes gv_k  
\]
とによって$V^{\otimes k}$は$\gl(V)$の表現とみなすこともできる。$V^{\otimes k}$は$\mathfrak{S}_k$, $\gl(V)$両方の作用を同時に受けている。さらに次が成り立つ。

\begin{prop}\label{comm_of_sym_gl}
  任意の$\sigma\in\mathfrak{S}_k$, $g\in\gl(V)$に対して、
  \[
  \sigma g(v_1\otimes\cdots\otimes v_k)
  =
  g \sigma(v_1\otimes\cdots\otimes v_k)
  \]
  である。
\end{prop}

\begin{proof}
  $u_i=gv_i$とおく。
  \begin{align*}
    \sigma g(v_1\otimes\cdots\otimes v_k)
    &=\sigma(gv_1\otimes\cdots\otimes gv_k)\\
    &=\sigma(u_1\otimes\cdots\otimes u_k)\\
    &=u_{\inv{\sigma}(1)}\otimes\cdots\otimes u_{\inv{\sigma}(k)}\\
    &=gv_{\inv{\sigma}(1)}\otimes\cdots\otimes gv_{\inv{\sigma}(k)}\\
    &=g\sigma (v_1\otimes\cdots\otimes v_k)
  \end{align*}
\end{proof}

このとき、
$c_{(k)}=\sum_{\sigma\in \mathfrak{S}_n}\sigma\in\complex[\mathfrak{S}_n]$, $c_{1^k}=\sum_{\sigma\in \mathfrak{S}_n}\sgn(\sigma)\sigma\in\complex[\mathfrak{S}_n]$とおくと
\[
\Sym^k(V)=V^{\otimes k}c_{(k)},\qquad\Alt^k(V)=V^{\otimes k}c_{1^k}
\]
となることがわかり、命題\ref{comm_of_sym_gl}よりこの2つは$\gl(V)$の表現でもある。よって$k>2$のときにも、$\lambda\in\mathcal{P}_k$に対するYoung対称子$c_\lambda$による像
\[
W_\lambda(V)=V^{\otimes k}c_\lambda   
\]
を考察することは自然である。再び命題\ref{comm_of_sym_gl}より$W_\lambda(V)$は$\gl(V)$の部分表現になるが、このとき次が成り立つ

\begin{theo}[Schur-Weyl双対性]\label{schur_weyl}
  $W_\lambda(V)$は$\gl(V)$の既約表現であり、$\mathfrak{S}_k\times\gl(V)$の表現として
  \[
  V^{\otimes k}\simeq \bigoplus_{\lambda\in\mathcal{P}_k}S_\lambda\boxtimes W_\lambda(V)  
  \]
  が成り立つ。
\end{theo}


ポイントになるのは次の補題である。

\begin{lemm}\label{DCT}
  $G$を有限群, $A=\complex[G]$, $U$を右$A$加群, $B=\End_A(U)$とおく。
  \begin{enumerate}
    \item $c\in A$に対して、左$B$加群として$U\otimes_AAc\simeq Uc$が成り立つ。
    \item $c\in A$に対して、$W=Ac$が単純左$A$加群ならば$U\otimes_AW=Uc$は単純左$B$加群である。
    が成り立つ。
  \end{enumerate}
\end{lemm}

\begin{proof}
  \begin{enumerate}
    \item 次の可換図式を考える。
    \[
    \xymatrix{
      U\otimes_AA \ar[r]^{\cdot c} \ar[d]  & U\otimes_AAc \ar[r] \ar[d]  & U\otimes_AA \ar[d]\\
      U \ar[r]^{\cdot c} & Uc \ar[r] & U 
    }  
    \]
    ここで$\cdot c$は$c$倍写像(全射)であり、$U\otimes_AAc\rightarrow U\otimes_AA$, $U_c\rightarrow U$は埋め込み(単射), 両端の縦の写像は同型である。Diagram Chasingをすれば中央の縦の写像も同型になることがわかる。

    \item $U$が単純右$A$加群である場合を考える。このときSchurの補題から$B\simeq\complex$であることに注意する。$\dim_\complex U\otimes_AW\leq 1$であることを示せばよい。Wedderburnの構造定理より、
    \[
    A\simeq\prod_{i=1}^rM_{m_i}(\complex)
    \]
    である。$W$は$A$の極小左イデアルなので、$W$は
    \begin{align*}
    &W=\set{(M_1,\cdots,M_i,\cdots,M_r)}{M_k=0,(k\neq i), M_i\in I_\alpha}=0\times\cdots\times I_\alpha\times\cdots\times 0 \\
    &I_\alpha=\set{\pmat{
      0 & \cdots & a_1 & \cdots & 0\\
      0 & \cdots & a_2 & \cdots & 0\\
      \vdots & \cdots & \vdots & \cdots &\vdots\\
      0 & \cdots & a_{m_i} & \cdots & 0
      }}{a_1,a_2\cdots,a_{m_i}\in\complex\text{は$\alpha$列}}
    \end{align*}
    このような形になる。同様に$U$を$A$の極小右イデアルと同一視すれば
    \footnote{
      $G$の正則表現はすべての既約表現を直和因子に含むことに注意
    }、
    \begin{align*}
      &U=\set{(M_1,\cdots,M_j,\cdots,M_r)}{M_k=0,(k\neq j), M_j\in J_\beta}=0\times\cdots\times J_\beta\times\cdots\times 0  \\
      &J_\beta=\set{\pmat{
        0 & 0 & \cdots & 0\\
        \vdots & \vdots & \cdots & \vdots\\
        a_1 & a_2 &\cdots & a_{m_j}\\
        \vdots & \vdots & \cdots & \vdots\\
        0 & 0 & \cdots & 0 
        }}{a_1,a_2\cdots,a_{m_j}\in\complex\text{は$\beta$行}}
      \end{align*}
      となるから、
      \begin{align*}
        U\otimes_AW&=\left\{\begin{array}{cc}
          J_\beta\otimes_{M_{m_i}(\complex)}I_\alpha=\complex E_{\beta\alpha} & \text{if $i=j$}\\
          0 & \text{otherwise}
        \end{array}\right.
      \end{align*}
      よって$\dim_\complex (U\otimes_AW)\leq 1$だから単純である。

      一般の$U$については、
      $A$の半単純性より$U$を単純右$A$加群の直和$U\simeq\bigoplus_iU_k^{\oplus n_k}$に分解しておいて、再びSchurの補題から
      \[
      B=\End_A(U)=\End_A(\bigoplus_kU_k^{\oplus n_k})=\bigoplus_k\End_A(U_k^{\oplus n_k})=\prod_kM_{n_k}(\complex)
      \]
      また、前半に示したことにより
      \[
      U\otimes_AW=\bigoplus_k(U_k\otimes_AW)^{\oplus n_k}\simeq \complex^{\oplus n_i}
      \]
      となる。実際、$U_j\otimes_AW\simeq \complex$ならば$U_j=0\times\cdots\times J_\beta\times\cdots\times 0$と表した時、$J_\beta$は$i$番目の位置になければならないから、$U_j\simeq U_i$となる。$\complex^{\oplus n_i}$は単純左$M_{n_i}(\complex)$加群だから、$U\otimes_AW$は単純左$B$加群である。
  \end{enumerate}
\end{proof}


\begin{lemm}\label{gen_of_symtensor}
  $V$を$n$次元ベクトル空間とする。$\Sym^k(V)$は$\{v\otimes\cdots\otimes v\}_{v\in V}$によって生成される
\end{lemm}

\begin{proof}
  $n$変数$k$次斉次多項式のなすベクトル空間を$S$とおく。主張は$S$が1次斉次多項式の$k$乗で生成されることと同値である。単項式
  \[
  x_1^{i_1}\cdots x_n^{i_n},\quad i_1+\cdots+i_n=k
  \]
  が生成されることを示せば十分である。$f_0(x_1,\cdots,x_n)=(x_1+\cdots +x_n)^k$とおく。
  \[
  f_1(x_1,\cdots,x_n)=f_0(2x_1,x_2,\cdots,x_n)-2^{k}f_0(x_1,\cdots,x_n)  
  \]
  とすれば、$f_1$は$x_1^k$を含む項をもたない。次に
  \[
  f_2(x_1,\cdots,x_n)=f_1(2x_1,x_2,\cdots,x_n)-2^{k-1}f_1(x_1,\cdots,x_n)  
  \]
  とすれば$f_2$は$x_1^{k}$, $x_1^{k-1}$を含む項をもたない。この操作を$i_1$以外に対して行えば、最終的に$x_1^{i_1}$を含む項以外をもたないような多項式$g_0(x_1,\cdots,x_n)$を得る。そして$g_0$は作り方から、一次斉次多項式の$k$乗の線形結合で表される。同様に
  \[
  g_1(x_1,\cdots,x_n)=g_0(x_1,2x_2,x_3,\cdots,x_n)-2^{k}g_0(x_1,\cdots,x_n)  
  \]
  とすれば$g_1$は$x_2^k$を含む項をもたない。再びこの操作を繰り返して$x_1^{i_1}$, $x_2^{i_2}$を含む項以外をもたないような多項式を得る。これを繰り返していけば、有限回のうちに$x_1^{i_1}x_2^{i_2}\cdots x_n^{i_n}$を作ることができる。
\end{proof}


\begin{lemm}\label{lemma}
  $A$を$\End(V^{\otimes k})$における$\complex[\mathfrak{S}_k]$の像とし、$B$を$\End(V^{\otimes k})$における$\complex[\gl(V)]$の像とする。$B=\End_A(V^{\otimes k})$が成り立つ。
\end{lemm}

\begin{proof}
  $\End_A(V^{\otimes k})=(\End(V^{\otimes k}))^{\mathfrak{S}_n}=(\End(V)^{\otimes k})^{\mathfrak{S}_n}=\Sym^k(\End(V))$であるから補題\ref{gen_of_symtensor}より、$\End_A(V^{\otimes k})$は$\{X\otimes\cdots\otimes X\}_{X\in\End(V)}$によって生成される。$\gl(V)\subset\End(V)$なのだから、$B\subset \End_A(V^{\otimes k})$は直ちに従う。$g\in\gl(V)$に対して$g\otimes \cdots\otimes g$が生成する$\End(V^{\otimes k})$の部分代数を$B'$とする。任意の(正則とは限らない)$X\in\End(V)$に対して$X\otimes \cdots\otimes X$が$B'$に含まれることを示せばよい。$X+tE$は有限個の$t$を除いて正則である
  \footnote{
    行列式は$t$の多項式
  }
  から、$X$に収束する$\gl(V)$の点列$(X+t_iE)$が存在する
  \footnote{
    すなわち$\gl(V)$は$\End(V)$の稠密集合
  }。$\End(V^{\otimes k})$は有限次元だから$B'$は閉部分空間であるので、
  \[
  X^{\otimes k}=\lim_{i\rightarrow\infty}(X+t_iE)^{\otimes k}\in B'  
  \]
  が成り立つ\footnote{
    表現は連続だからこのような極限操作が可能である。
  }
\end{proof}

定理\ref{schur_weyl}を証明しよう
\begin{proof}
  補題\ref{DCT}と補題\ref{gen_of_symtensor}, 補題\ref{lemma}より、
  \[
  W_\lambda(V)=V^{\otimes k}c_\lambda  
  \]
  は単純$\gl(V)$加群であることが従う。ここで、$V^{\otimes k}$は左$A=\complex[\mathfrak{S}_k]$加群とみなすこともできたことを思い出す。
  \begin{align*}
    &\sigma\in\mathcal{H}_\lambda\implies\inv{\sigma}\in\mathcal{H}_\lambda\\
    &\tau\in\mathcal{V}_\lambda\implies\inv{\tau}\in\mathcal{V}_\lambda
  \end{align*}
  が成り立つことに注意すると、
  \[
  c_\lambda V^{\otimes k}=V^{\otimes k}c_\lambda  
  \]
  となることがわかる。

  系\ref{multiplicity}より、
  \[
  V^{\otimes k}\simeq \bigoplus_{\lambda\in\mathcal{P}_k}S_\lambda\otimes_\complex\Hom_A(S_\lambda,V^{\otimes k})  
  \]
  となる。$B=\gl(V)$加群として$\Hom_A(S_\lambda,V^{\otimes k})\simeq W_\lambda(V)$となることを示せばよい。$S_\lambda=Ac_\lambda$だから、$\phi:\Hom_A(S_\lambda,V^{\otimes k})\rightarrow W_\lambda(V)$を
  \[
    \phi(f)=f(c_\lambda)=f(c_\lambda^2)=c_\lambda f(c_\lambda)
  \]
  によって定める\footnote{$c_\lambda$は適当にスカラー倍することでべき等元であった}。$b\in\End_A(V^{\otimes k})=\gl(V)$に対して、
  \[
  \phi(bf)=c_\lambda bf(c_\lambda)=b(c_\lambda f(c_\lambda))=b\phi(f)
  \]
  より$\phi$は左$B$加群の準同型である。$S_\lambda=Ac_\lambda$, $W_\lambda(V)=c_\lambda V^{\otimes k}$より、$\phi$は全単射である。
\end{proof}


\begin{cor}
  $\lambda\in\mathcal{P}_k$に対して、$d_\lambda=\dim_\complex S_\lambda$とおく。$\gl(V)$の表現として
  \[
  V^{\otimes k}\simeq \bigoplus_{\lambda\in\mathcal{P}_k} W_\lambda(V)^{\otimes d_\lambda} 
  \]
  が成り立つ。
\end{cor}

\begin{eg}
  $\dim V=n$とする。$\lambda$が$n$行よりも長いYoung図形のとき$W_\lambda(V)=0$となる。$\lambda$の共役Young図形を$\lambda^*=(\lambda^*_1,\cdots,\lambda^*_s)$, $\lambda_1^*>n$とおく。このとき、
  \[
  b_\lambda
  =\left(
    \sum_{\mathfrak{S}_{\lambda^*_s}}\sgn(\sigma_s)\sigma_s
   \right)\cdots 
   \left(
    \sum_{\mathfrak{S}_{\lambda^*_1}}\sgn(\sigma_1)\sigma_1
   \right)  
  \]
  と書くことができる。$\tilde{b}_\lambda=\sum_{\mathfrak{S}_{\lambda^*_1}}\sgn(\sigma_1)\sigma_1$とする。
  \[
  \lambda=\:\begin{ytableau}
    1 & 5\\
    2 & 6\\
    3 \\
    4
  \end{ytableau}
    ,\text{ のとき }
  b_\lambda=
  \left(
    \sum_{\tau\in\mathfrak{S}_2}\sgn(\tau)\tau
  \right)
  \left(
    \sum_{\sigma\in\mathfrak{S}_4}\sgn(\sigma)\sigma
  \right),\:\tilde{b}_\lambda=\sum_{\sigma\in\mathfrak{S}_4}\sgn(\sigma)\sigma
  \]
  $V^{\otimes k}$を$\lambda$の各箱に$V$の元が書かれているものと同一視する。$V$の基底を$e_1,\cdots,e_n$とすれば$V^{\otimes k}$の元は$e_1,\cdots,e_n$を$\lambda$の各箱に配置した元で生成される。
  \[
  \lambda=\ydiagram{2,2,1,1}\text{ のとき }V^{\otimes k}=\generated{\begin{ytableau}
                e_{i_1} & e_{i_5}\\
                e_{i_2} & e_{i_6}\\
                e_{i_3}\\
                e_{i_4}
              \end{ytableau}}
  \]
  このとき、$\lambda^*_1>n$より、$e_{i_1},\cdots,e_{i_{\lambda^*_1}}$には必ず重複がある。よって
  \[
  \tilde{b}_\lambda(e_{i_1}\otimes\cdots\otimes e_{i_{\lambda^*_1}})=0
  \]
  だから、$b_\lambda V^{\otimes k}=0$である。
\end{eg}

\begin{eg}
  $k=3$, $n>k$とする。$\lambda=\:\ydiagram{2,1}\:$として
  \[
  V^{\otimes 3}=\Sym^3(V)\oplus (S_\lambda\boxtimes W_\lambda(V))\oplus \Alt^3(V)  
  \]
  であり、$\dim S_{\lambda}=2$であるので
  \[
  \dim W_\lambda(V)=\frac{1}{2}(n^3-\dim\Sym^3(V)-\dim\Alt^3(V))=\frac{1}{12}(4n^3-3n^2-13n-6)
  \]
\end{eg}


\subsection{一般線形群の表現に対する指標}

前節でWeyl表現という一般線形群の既約表現を構成したが、これらが互いに同型でないことは示していなかった。本節ではこれを示し、さらに表現の指標を導入する。指標は有限群の場合のようにある範囲での表現を特徴づけるものであり、一般線形群においても強力な道具となる。

\begin{defin}
  $\rho:\gl(V)\rightarrow \gl(W)$を表現とする。$H$を$\gl(V)$の対角行列のなす部分群とする。
  $\ch_W:H\rightarrow \complex$を
  \[
  \ch_W(g)=\tr(\rho(g))  
  \]
  によって定め、$\rho$の指標という。
\end{defin}

\begin{eg}
  $\lambda=(k)\in\mathcal{P}_k$, $\dim_\complex V=n>k$とする。$\ch_\lambda=\ch_{W_\lambda(V)}$とすると、$g\in H$に対して、$g$の固有値を$x_1,\cdots,x_n$とすると、
  \[
  \ch_\lambda(g)=h_k(x_1,\cdots,x_n) 
  \]
  となる。実際$g=\diag(x_1,\cdots,x_n)$とおくと$\Sym^k(V)$の基底$e_{i_1}\cdots e_{i_k}$に対して
  \[
  g^{\otimes k}(e_{i_1}\cdots e_{i_k})=x_{i_1}\cdots x_{i_k}e_{i_1}\cdots e_{i_k}  
  \]
  である。
\end{eg}

\begin{eg}
  同様に$\lambda=(1^k)$ならば、
  \[
  \ch_\lambda(g)=e_k(x_1,\cdots,x_n)  
  \]
  である。
\end{eg}

有限群の場合と同様、トレースの性質から次が従う。

\begin{prop}$W,U$を$\gl(V)$の表現とする。
  \begin{enumerate}
    \item $\ch_{W\oplus U}=\ch_W+\ch_U$
    \item $\ch_{W\otimes U}=\ch_W\ch_U$
  \end{enumerate}
  が成り立つ。
\end{prop}

\begin{theo}\label{char_schur}
  $\lambda\in\mathcal{P}_k$, $\dim_\complex V=n$とする。$W_\lambda(V)$の指標を$\ch_\lambda$とおくと、$g=\diag(x_1,\cdots,x_n)$に対して
  \[
  \ch_\lambda(g)=s_\lambda(x_1,\cdots,x_n)
  \]
  が成り立つ。
\end{theo}

\begin{proof}
  $\lambda=(\lambda_1,\cdots,\lambda_s)$とする。補題\ref{DCT}を$M_\lambda=Aa_\lambda$に用いて、
  \[
  V^{\otimes k}a_\lambda\simeq V^{\otimes k}\otimes_AM_\lambda  
  \]
  $a_\lambda$の定義から、
  \[
  V^{\otimes k}a_\lambda
  =\Sym^{\lambda_1}(V)\otimes_{\complex}\cdots\otimes_\complex\Sym^{\lambda_s}(V)
  =W_{(\lambda_1)}(V)\otimes\cdots\otimes W_{(\lambda_s)}(V)
  \]
  となる。Youngの規則(系\ref{young_rule_for_rep})より
  \[
  M_\lambda
  =S_\lambda\oplus\bigoplus_{\mu>\lambda}S_\mu^{\oplus k_{\lambda\mu}}
  =Ac_\lambda\oplus\bigoplus_{\mu>\lambda}Ac_\mu^{\oplus k_{\lambda\mu}}
  \]
  したがって、
  \[
  W_{(\lambda_1)}(V)\otimes\cdots\otimes W_{(\lambda_s)}(V)
  \simeq 
  W_\lambda(V)\oplus\bigoplus_{\mu>\lambda}W_{\mu}(V)^{\oplus k_{\lambda\mu}}
  \]
  両辺の指標をとれば
  \[
  s_\lambda=\ch_\lambda+\sum_{\mu>\lambda}k_{\lambda\mu}\ch_\mu  
  \]
  よって帰納法により従う。
\end{proof}

\begin{cor}
  $W_\lambda(V)\simeq W_\mu(V)$であることと$\lambda=\mu$は同値である。
\end{cor}

\begin{proof}
  定理$\ref{char_schur}$より$\lambda\neq\mu$ならば指標が異なる。
\end{proof}

\begin{cor}
  $W,U$を$\gl(V)$の表現で、ともにWeyl表現の直和で表されるものとする。$W\simeq U$となるための必要十分条件はその指標が等しいことである。
\end{cor}

\begin{proof}
  Schur多項式の一次独立性から従う。
\end{proof}

定理\ref{char_schur}の証明中に現れた式を再掲しておく

\begin{cor}[Youngの規則]
  \[
  W_{(\lambda_1)}(V)\otimes\cdots\otimes W_{(\lambda_s)}(V)
  \simeq 
  W_\lambda(V)\oplus\bigoplus_{\mu>\lambda}W_{\mu}(V)^{\oplus k_{\lambda\mu}}
  \]
  が成り立つ。ただし$k_{\lambda\mu}$はKostka数である。
\end{cor}

\begin{cor}[Littlewood-Richardson規則]
  $2$つのYoung図形$\lambda,\mu$に対して、
  \[
  W_\lambda(V)\otimes_{\complex}W_\mu(V)\simeq \bigoplus_\nu W_\nu(V)^{\oplus \eta^{\nu}_{\lambda\mu}}
  \]
  が成り立つ。ただし$\eta^{\nu}_{\lambda\mu}$はLittlewood-Richardson数である。
\end{cor}

\begin{proof}
  $W_\lambda(V)\otimes_{\complex}W_\mu(V)$がWeyl表現の直和に表されることを示せばよい。$\lambda\in\mathcal{P}_k$, $\mu\in\mathcal{P}_l$とする。
  \begin{align*}
    W_\lambda(V)\otimes_\complex W_\mu(V)
    &\simeq V^{\otimes k}c_\lambda\otimes_{\complex}V^{\otimes l}c_\mu\\
    &\simeq V^{\otimes (k+l)}(c_\lambda\otimes c_\mu)
  \end{align*}
  ここで、$c_\lambda\otimes c_\mu\in\complex[\mathfrak{S}_k]\otimes_\complex\complex[\mathfrak{S}_l]=\complex[\mathfrak{S}_{k}\times\mathfrak{S}_l]\subset\complex[\mathfrak{S}_{k+l}]$である。$c=c_\lambda\otimes c_\mu$とおく。
  \[
  \complex[\mathfrak{S}_{k+l}]c=\bigoplus_{\nu\in\mathcal{P}_{k+l}}S_\nu^{\oplus t_\nu}
  \]
  とおけば\footnote{
    実際は個の表現は$\complex[\mathfrak{S}_{k+l}]c=\ind_{\complex[\mathfrak{S}_k]\times\complex[\mathfrak{S}_l]}^{\complex[\mathfrak{S}_{k+l}]}S_\lambda\boxtimes S_\mu=S_\lambda\circ S_\mu$に他ならない。
  }、補題\ref{DCT}より
  \begin{align*}
    V^{\otimes (k+l)}c
    &\simeq V^{\otimes (k+l)}\otimes_{\complex[\mathfrak{S}_{k+l}]}\complex[\mathfrak{S}_{k+l}]c\\
    &\simeq V^{\otimes (k+l)}\otimes_{\complex[\mathfrak{S}_{k+l}]}\bigoplus_{\nu\in\mathcal{P}_{k+l}}S_\nu^{\oplus t_\nu}\\
    &\simeq \bigoplus_{\nu\in\mathcal{P}_{k+l}}V^{\otimes (k+l)}\otimes_{\complex[\mathfrak{S}_{k+l}]}\complex[\mathfrak{S}_{k+l}]c_\nu^{\oplus t_\nu}\\
    &\simeq \bigoplus_{\nu\in\mathcal{P}_{k+l}}W_\nu(V)^{\oplus t_\nu}\\
  \end{align*}
\end{proof}



\end{document}