\documentclass{ltjsreport}
\RequirePackage{luatex85}
\usepackage[utf8]{inputenc}
\usepackage{enumerate}
\usepackage{amsthm}
\usepackage{amsfonts}
\usepackage{amsmath}
\usepackage{amssymb}
\usepackage{ytableau}
\usepackage{docmute}
\usepackage{mathtools}
\usepackage{xr}
\usepackage[all]{xy}



\theoremstyle{definition}
\newtheorem{defin}{定義}[subsection]
\newtheorem{theo}[defin]{定理}
\newtheorem{cor}[defin]{系}
\newtheorem{prop}[defin]{命題}
\newtheorem{lemm}[defin]{補題}
\newtheorem{notice}[defin]{注意}
\newtheorem{eg}[defin]{例}


\renewcommand{\labelenumi}{(\roman{enumi})}


\newcommand{\invlimit}{\mathop{\lim_{\longleftarrow}}}
\newcommand{\dirlimit}{\mathop{\lim_{\longrightarrow}}}
\newcommand{\ind}{\text{Ind}\:}
\newcommand{\Hom}{\text{Hom}}
\newcommand{\tr}{\text{tr}\:}
\newcommand{\id}[1]{\text{id}_{#1}}
\newcommand{\sgn}{\mathrm{sgn}}
\newcommand{\res}[1]{\text{Res}_{#1}}
\newcommand{\generated}[1]{\langle\:#1\:\rangle}
\newcommand{\im}{\text{Im }}
\newcommand{\rank}{\text{rank }}
\newcommand{\del}[2]{\frac{\partial #1}{\partial #2}}
\newcommand{\delsametwo}[2]{\frac{\partial^2 #1}{\partial #2^2}}
\newcommand{\delothertwo}[3]{\frac{\partial^2#1}{\partial#2\partial#3}}
\newcommand{\ddel}[2]{\frac{\partial}{\partial #2}#1}
\newcommand{\ddelsametwo}[3]{\frac{\partial^2}{\partial #2^2}#1}
\newcommand{\ddelothertwo}[3]{\frac{\partial^2}{\partial#2\partial#3}#1}
\newcommand{\simneq}{\not\simeq}
\newcommand{\transpose}[1]{^t\!#1}
\newcommand{\ie}{\text{i.e.}}
\newcommand{\inv}[1]{#1^{-1}}
\newcommand{\real}{\mathbb{R}}
\newcommand{\complex}{\mathbb{C}}
\newcommand{\integer}{\mathbb{Z}}
\newcommand{\quotient}{\mathbb{Q}}
\newcommand{\natnum}{\mathbb{N}}
\newcommand{\proj}{\mathbb{P}}
\newcommand{\tensor}[3]{#1\otimes_#2#3}
\newcommand{\map}[3]{#1:#2\rightarrow#3}
\newcommand{\aut}[2]{\mathrm{Aut}_{#1} (#2)}
\newcommand{\hommoph}[2]{\mathrm{Hom}_{#1}(#2)}
\newcommand{\gl}[1]{\mathrm{GL}(#1)}
\newcommand{\set}[2]{\left\{\:#1\:\middle|\:#2\:\right\}}
\newcommand{\pmat}[1]{\begin{pmatrix} #1
\end{pmatrix}}
\newcommand{\vmat}[1]{\begin{vmatrix} #1
\end{vmatrix}}
\newcommand{\br}{\vskip\baselineskip}


\begin{document}

\section{一般線形群の表現とSchur-Weyl双対性}

前節までで対称群の既約表現に関して解説してきたが、次に対称群と表現論的に関係の深い一般線形群の表現について解説する。とくに多項式表現と呼ばれる表現のクラスが、Schur-Weyl双対性を通して対称群の表現と密接にかかわりあっている。



\subsection{Lie群とLie代数}
Schur-Weyl双対性のために若干のLie群・Lie代数の知識を用いる。

\begin{defin}[Lie群]
  $G$を群であり複素多様体でもあるとする。$G$の演算$\cdot:G\times G\rightarrow G$, および逆元を取る写像$\inv{}:G\rightarrow G$がともに正則であるとき、$G$を(複素)Lie群という。Lie群の間の写像$f:G\rightarrow H$について、$f$が群準同型かつ正則であるとき$f$をLie群の準同型という。
\end{defin}

\begin{eg}
  $\complex$ベクトル空間$V$に対して一般線形群$\gl(V)$は行列の積に関してLie群である。実際、行列の積は成分の多項式であるし、逆行列は分母が$0$でない有理関数で表されるから正則である。同様に$\text{SL}(V)$, $\text{U}(n)$, $\text{SU}(n)$もLie群である。
\end{eg}

\begin{defin}[Lie代数]
  $\mathfrak{g}$を$\complex$ベクトル空間とする。写像$[\:,\:]:\mathfrak{g}\times \mathfrak{g}\rightarrow \mathfrak{g}$が与えられており
  \begin{enumerate}[(i)]
    \item $[\:,\:]$は双線形
    \item $[x,x]=0$, (交代性)
    \item $[x,[y,z]]+[y,[z,x]]+[z,[x,y]]=0$, (Jacobiの恒等式)
  \end{enumerate}
  をみたすとき、$\mathfrak{g}$を(複素)Lie代数という。Lie代数の積$[,]$を括弧積やLieブラケットと呼ぶ。
  Lie代数の間の写像$f:\mathfrak{g}\rightarrow \mathfrak{h}$について、$f$が線形写像かつ$f([X,Y])=[f(X),f(Y)]$をみたすとき、$f$をLie代数の準同型という。

  とくにこの節ではLie代数はすべて有限次元のものを扱う。
\end{defin}

\begin{eg}
  $\mathfrak{gl}(V)=\End(V)$とし、$X,Y\in\mathfrak{gl}(V)$に対して
  \[
  [X,Y]=XY-YX  
  \]
  とおくと$\mathfrak{gl}(V)$は複素Lie代数である。同様の演算で
  \begin{itemize}
    \item $\mathfrak{sl}(V)=\set{X\in\mathfrak{gl}(V)}{\tr(X)=0}$
    \item $\mathfrak{alt}(V)=\set{X\in\mathfrak{gl}(V)}{\transpose{X}=-X}$
  \end{itemize}
  などもLie代数である。また、一般に$\complex$代数$A$に対して
  \[
  X,Y\in A,\qquad [X,Y]=XY-YX  
  \]
  と定めると$A$はLie代数の構造をもつ。逆にすべてのLie代数がこのように$\complex$代数から誘導されるか、というのは興味深い問題である。
\end{eg}

Lie代数はLie群を調べる際に自然に現れる。$G$をLie群とし、$G$の単位元$e$における接空間$T_eG$に積$[,]$を
\[
\text{正則関数$f:G\rightarrow\complex$に対して }[X,Y](f)=X(f)Y(f)-Y(f)X(f)
\]
によって定める。これによって$T_eG$はLie代数の構造をもつ。これを$G$から定まるLie代数といい、$\text{Lie}(G)$とかく。

\begin{eg}
  $\Lie(\gl(V))=\mathfrak{gl}(V)$である。実際、$\gl(V)$は$M_{n^2}(\complex)$の開集合であり、$T_E(M_{n^2}(\complex))=M_{n^2}(\complex)=\mathfrak{gl}(V)$だから、
  \[
    \Lie(\gl(V))=\mathfrak{gl}(V)
  \]
\end{eg}




\begin{defin}\label{induced_hom}
  Lie群の準同型$\rho:G\rightarrow H$が与えられたとき、その微分$(d\rho)_{e}:\Lie(G)\rightarrow\Lie(H)$はLie代数の準同型となる。すなわち
  \[
    (d\rho)_e([X,Y])=[(d\rho)_e(X),(d\rho)_e(Y)] 
  \]
  を満たす。これを$\rho$が誘導するLie代数の準同型と呼ぶ。
\end{defin}

\begin{defin}
  $V$をベクトル空間,$G$をLie群とし、Lie群の準同型$\rho:G\rightarrow\gl(V)$を$G$の表現という。また、$\mathfrak{g}$をLie代数とし、Lie代数の準同型$\rho:\mathfrak{g}\rightarrow\mathfrak{gl}(V)$を$\mathfrak{g}$の表現という。
\end{defin}

定義\ref{induced_hom}より、Lie群の表現$\rho:G\rightarrow \gl(V)$はLie代数の表現$(d\rho)_e:\Lie(G)\rightarrow \mathfrak{gl}(V)$を誘導する。

\begin{eg}\label{tensor_rep_of_gl}
  $V$をベクトル空間、$G=\gl(V)$とする。$\rho:G\rightarrow\gl({V^{\otimes m}})$を
  \[
  \rho(g)=g\otimes\cdots\otimes g,\quad\ie\quad
  \rho(g)(v_1\otimes\cdots\otimes v_m)=gv_1\otimes\cdots\otimes gv_m  
  \]
  によって定めると$\rho$は$G$の表現になる。$\rho$の誘導するLie代数の表現を求める。$V$の基底$e_1,\cdots,e_n$を固定して$G$を$\gl_n(\complex)$と同一視し、$E_{ij}\in G$を行列単位とする。このとき$g\in G$は
  \[
  g=\sum_{i,j}x_{ij}E_{ij}  
  \]
  と座標表示でき、$g\otimes\cdots\otimes g\in\gl(V^{\otimes m})$は
  \[
  g\otimes\cdots\otimes g=\sum_{i_1,j_1,\cdots,i_m,j_m}x_{i_1j_1}\cdots x_{i_mj_m}E_{i_1j_1}\otimes\cdots\otimes E_{i_mj_m}  
  \]
  と座標表示できる。$y_{i_1,j_1,\cdots,i_m,j_m}=x_{i_1j_1}\cdots x_{i_mj_m}$とおくと、
  \[
  \del{y_{i_1,j_1,\cdots,i_m,j_m}}{x_{k,l}}=
  \left\{\begin{array}{cc}
    x_{i_1j_1}\cdots \hat{x}_{i_sj_s} \cdots x_{i_mj_m} & \text{ある$s$で$(k,l)=(i_s,j_s)$}\\
    0 & \text{otherwise}
  \end{array}\right.
  \]
  となる。ここで$\hat{x}_{i_sj_s}$は$x_{i_sj_s}$を取り除いていることを意味する。これより$X=E$(単位行列)のとき
  \[
  \del{y_{i_1,j_1,\cdots,i_m,j_m}}{x_{k,l}}(E)
  =\left\{\begin{array}{cc}
    1 & \text{ある$s$で$(k,l)=(i_s,j_s)$かつ$i_1=j_1,\cdots,i_m=j_m$, ($i_s=j_s$は除く)}\\
    0 & \text{otherwise}
  \end{array}\right.
  \]
  となる。よって$X\in\Lie(G)=T_EG=M_{n^2}(\complex)$に対して
  \[
  X=\sum_{k,l}a_{kl}E_{{kl}}
  \]
  とおくと
  \begin{align*}
  (d\rho)_E(X)
  &=\sum_{i_1,j_1,\cdots,i_m,j_m}
      \left(
        \sum_{k,l}\del{y_{i_1,j_1,\cdots,i_m,j_m}}{x_{k,l}}(E)a_{kl}
      \right)
      E_{i_1j_1}\otimes\cdots\otimes E_{i_mj_m}\\
  &=\sum_{i_2=j_2,\cdots,i_m=j_m}
      \left(
        \sum_{i_1,j_1}a_{i_1j_1}E_{i_1j_1}
      \right)
    \otimes E_{i_2i_2}\otimes\cdots\otimes E_{i_mi_m}\\
  &\quad +
    \sum_{i_1=j_1,i_3=j_3\cdots,i_m=j_m}
      E_{i_1i_1}\otimes
      \left(
        \sum_{i_2,j_2}a_{i_2j_2}E_{i_2j_2}
      \right)
    \otimes E_{i_3i_3}\otimes\cdots\otimes E_{i_mi_m}\\
  &\quad +\\
  &\quad \vdots\\
  &\quad +
    \sum_{i_1=j_1,\cdots,i_{m-1}=j_{m-1}}
      E_{i_1i_1}\otimes\cdots\otimes E_{i_{m-1}j_{m-1}}\otimes
      \left(
        \sum_{i_m,j_m}a_{i_mj_m}E_{i_mj_m}
      \right)\\
  &=X\otimes E\otimes\cdots\otimes E
   +E\otimes X\otimes E\otimes\cdots\otimes E
   +\cdots 
   +E\otimes E\otimes\cdots\otimes X
  \end{align*}
\end{eg}

すべてのLie代数が$\complex$代数から誘導されるか、という問題について考えよう。$\mathfrak{g}$をLie代数とし、$\mathcal{T}(\mathfrak{g})$を$\mathfrak{g}$のテンソル代数とする。すなわち、
\[
\mathcal{T}(\mathfrak{g})=\bigoplus_{k=0}^\infty\mathfrak{g}^{\otimes k}  
\]
このとき$\mathcal{I}$を
\[
\set{[X,Y]-X\otimes Y-Y\otimes X}{X,Y\in\mathfrak{g}}  
\]
によって生成される$\mathcal{T}(\mathfrak{g})$の両側イデアルとして、
\[
\mathcal{U}(\mathfrak{g})=\mathcal{T}(\mathfrak{g})/\mathcal{I}  
\]
とする。$T,S\in\mathcal{U}(\mathfrak{g})$に対して括弧積$[,]$を
\[
[T,S]=T\otimes S-S\otimes T  
\]
によって定めると$\mathcal{U}(\mathfrak{g})$はLie代数となる。これを$\mathfrak{g}$の普遍包絡代数という。$\mathfrak{g}$から$\mathcal{U}(\mathfrak{g})$に対しては、
\[
\mathfrak{g}\rightarrow \mathcal{T}(\mathfrak{g})\rightarrow \mathcal{U}(\mathfrak{g})  
\]
なる自然なLie代数の準同型$\sigma:\mathfrak{g}\rightarrow\mathcal{U}(\mathfrak{g})$が存在する。
普遍包絡代数については次の定理が知られている。

\begin{theo}[Poincare-Birkhoff-Wittの定理]
  $e_1,\cdots,e_n$を$\mathfrak{g}$の基底とする。このとき\\
  $\set{\sigma(e_{i_1})\otimes\cdots\otimes\sigma(e_{i_k})}{1\leq i_1\leq\cdots\leq i_k\leq n}$は$\mathcal{U}(\mathfrak{g})$の基底となる。
\end{theo}

この定理から、$\sigma$が単射であることが従う。実際、$\{\sigma(e_i)\}$は主張の基底の一部に含まれている。したがって、すべてのLie代数はある$\complex$代数から誘導されるLie代数の部分代数として実現できるのである。普遍包絡代数は表現論的にも重要である。

\begin{prop}[普遍包絡代数の普遍性]
  $\mathcal{U}(\mathfrak{g})$, $\sigma:\mathfrak{g}\rightarrow\mathcal{U}(\mathfrak{g})$は次の性質をもつ。
  \begin{quote}
    任意の$\complex$代数$A$とLie代数の準同型$f:\mathfrak{g}\rightarrow A$が与えられたとき、$\complex$代数の準同型$\overline{f}:\mathcal{U}(\mathfrak{g})\rightarrow A$が一意的に存在して、$f=\overline{f}\circ\sigma$を満たす。
  \end{quote}
  \[
  \xymatrix{
  \mathcal{U}(\mathfrak{g}) \ar@{.>}[rd]^{\overline{f}} \\
  \mathfrak{g} \ar[u]^{\sigma} \ar[r]_f & A
  }
  \]
  また、この性質をもつ$\complex$代数$\mathcal{U}(\mathfrak{g})$とLie代数の準同型$\sigma$の組は同型を除いて一意的である。
\end{prop}

これにより、$\mathfrak{g}$の表現$\rho:\mathfrak{g}\rightarrow \mathfrak{gl}(V)$は$\complex$代数の準同型$\overline{\rho}:\mathcal{U}(\mathfrak{g})\rightarrow \mathfrak{gl}(V)$を一意的に誘導する。すなわち、$\mathfrak{g}$の表現を調べる代わりに$\mathcal{U}(\mathfrak{g})$の表現を調べればよいということになる。


\begin{eg}
  $\mathfrak{gl}(V)$の普遍包絡代数を求めよう。
\end{eg}


\subsection{Double Centralizer Theorem}

もう一つSchur-Weyl双対性の証明で用いる定理を一つ解説しておく。

\begin{theo}
  
\end{theo}

\end{document}