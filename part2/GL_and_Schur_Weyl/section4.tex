\documentclass{ltjsreport}
\input{../../setting.tex}


\begin{document}

\section{一般線形群の表現とSchur-Weyl双対性}

前節までで対称群の既約表現に関して解説してきたが、次に対称群と表現論的に関係の深い一般線形群の表現について解説する。とくに多項式表現と呼ばれる表現のクラスが、Schur-Weyl双対性を通して対称群の表現と密接にかかわりあっている。


\subsection{Double Centralizer Theorem}

Schur-Weyl双対性の証明で用いる定理を一つ解説しておく。

\begin{theo}[Double Centralizer Theorem]\label{DCT}
  $V$を有限次元ベクトル空間、$A$を$\End(V)$の半単純部分環とし、$B=\End_A(V)$とする。このとき、
  \begin{enumerate}
    \item $B$は半単純環である
    \item $A=\End_B(V)$が成り立つ
    \item $A\otimes_\complex B$加群として分解
    \[
      V\simeq \bigoplus_iU_i\otimes_\complex W_i  
    \]
    が成り立つ。ここで$U_i$は単純$A$加群で$W_i=\Hom_A(U_i,V)$は単純$B$加群である。
  \end{enumerate}
\end{theo}

\begin{proof}
  (i)から示す。$A$は有限次元半単純$\complex$代数だから、系\ref{multiplicity}の証明において、「既約表現」を「単純$A$加群」, 「$G$線形」を「$A$準同型」にそのまま変えて$A$加群として
  \[
  V\simeq \bigoplus_{i}U_i\otimes_\complex \Hom_A(U_i,V)  
  \]
  と分解されることがわかる。よって
  \begin{align*}
    B&=\End_A(V)\\
    &=\Hom_A \left(\bigoplus_{i}U_i\otimes_\complex \Hom_A(U_i,V),A\right)\\
    &=\bigoplus_{i}\Hom_A(U_i\otimes_\complex \Hom_A(U_i,V),V)\\
    &=\bigoplus_{i}\Hom_\complex(\Hom_A(U_i,V),\Hom_A(U_i,V)),\qquad\text{($\Hom$と$\otimes$の随伴性)}\\
    &=\bigoplus_i\End_\complex(W_i)
  \end{align*}
  となる。したがって$B$は有限個の全行列環の直積に同型であるからWedderburnの構造定理(付録参照)により$B$は半単純環である。
  
  次に(iii)を示す。まず$W_i$が単純$B$加群であることを示そう。そのためには、$B$が$W_i$に推移的に作用することを示せばよい。
  \footnote{
    一般に$A$加群$M$が単純であることと任意の$0$でない$M$の元$x$に対して$M=Ax$が成り立つことは同値である。
  }
  $f\in W_i=\Hom_A(U_i,V)$, $\phi\in B=\End_A(V)$に対して$\phi$は$f$に写像の合成として作用する。$U_i$は単純$A$加群だから、$0$でない$U_i$の元$u$を1つ固定して$U_i=Au$である。任意の$f,f'\in\Hom_A(U_i,V)$に対して
  \[
  v=f(u),\qquad v'=f'(u)  
  \]
  とおくと、$A$は半単純だから
  \[
  V=Av\oplus M
  \]
  をみたす$V$の部分$A$加群$M$が存在する。そこで、$\phi\in B$を
  \[
  \phi(av) = av',\qquad \phi(w)=0,\qquad(a\in A, w\in W)  
  \]
  によって定めれば、$f'=\phi\circ f$である。

  $V$は少なくとも$A$加群として
  \[
    V\simeq \bigoplus_{i}U_i\otimes_\complex W_i
  \]
  と分解されるが、$U_i\otimes_\complex W_i$は$A\otimes_\complex B$加群でもあるからよい。

  最後に(ii)を示す。
\end{proof}

\begin{notice}
  $A\otimes_{\complex}B$加群は
  \[
  A,\quad B\rightarrow A\otimes 1,\quad 1\otimes B
  \]
  によって$A$加群でもあり$B$加群でもあるから、$B$は$U_i$に自明に作用し、$A$は$W_i$に自明に作用する。
\end{notice}



\subsection{Schur-Weyl双対性}

次のような問題を考えることから始める。$V$を$n$次元ベクトル空間としたとき、テンソル空間$V^{\otimes k}$の部分空間として対称テンソル空間$\Sym^k(V)$, 交代テンソル空間$\Alt^k(V)$というものが
\begin{align*}
  &\Sym^k(V)=\set{v_1\otimes\cdots\otimes v_k}{v_{\sigma(1)}\otimes\cdots\otimes v_{\sigma (k)}=v_1\otimes\cdots\otimes v_k,\quad\text{for all $\sigma\in\mathfrak{S}_k$}}\\
  &\Alt^k(V)=\set{v_1\otimes\cdots\otimes v_k}{v_{\sigma(1)}\otimes\cdots\otimes v_{\sigma (k)}=\sgn(\sigma)v_1\otimes\cdots\otimes v_k,\quad\text{for all $\sigma\in\mathfrak{S}_k$}}
\end{align*}
によって定義された。
\[
\dim\Sym^k(V)=\pmat{n+k-1\\k},\qquad\dim\Alt^k(V)=\pmat{n\\k}  
\]
より、$k=2$の場合
\[
\dim\Sym^2(V)+\dim\Alt^2(V)=n^2=\dim V\otimes V  
\]
だから、$\Sym^k(V)\cap\Alt^k(V)=0$に注意すれば
\[
V\otimes V=\Sym^2(V)\oplus\Alt^2(V)  
\]
が成り立つ。一般の$k$に対してもこのようなテンソル空間の分解を行うことを考える。特に、表現も含めた分解を考えることが鍵になる。

$\sigma\in \mathfrak{S}_k$に対して、
\[
\sigma(v_1\otimes\cdots\otimes v_k)=v_{\inv{\sigma}(1)}\otimes\cdots\otimes v_{\inv{\sigma}(k)}  
\]
によって$V^{\otimes k}$を$\mathfrak{S}_n$の表現とみなす。さらに、$g\in \gl(V)$に対して
\[
g(v_1\otimes\cdots\otimes v_k)=gv_1\otimes\cdots\otimes gv_k  
\]
とによって$V^{\otimes k}$は$\gl(V)$の表現とみなすこともできる。$V^{\otimes k}$は$\mathfrak{S}_k$, $\gl(V)$両方の作用を同時に受けている。さらに次が成り立つ。

\begin{prop}\label{comm_of_sym_gl}
  任意の$\sigma\in\mathfrak{S}_k$, $g\in\gl(V)$に対して、
  \[
  \sigma g(v_1\otimes\cdots\otimes v_k)
  =
  g \sigma(v_1\otimes\cdots\otimes v_k)
  \]
  である。
\end{prop}

\begin{proof}
  $u_i=gv_i$とおく。
  \begin{align*}
    \sigma g(v_1\otimes\cdots\otimes v_k)
    &=\sigma(gv_1\otimes\cdots\otimes gv_k)\\
    &=\sigma(u_1\otimes\cdots\otimes u_k)\\
    &=u_{\inv{\sigma}(1)}\otimes\cdots\otimes u_{\inv{\sigma}(k)}\\
    &=gv_{\inv{\sigma}(1)}\otimes\cdots\otimes gv_{\inv{\sigma}(k)}\\
    &=g\sigma (v_1\otimes\cdots\otimes v_k)
  \end{align*}
\end{proof}

このとき、
$c_{(k)}=\sum_{\sigma\in \mathfrak{S}_n}\sigma\in\complex[\mathfrak{S}_n]$, $c_{1^k}=\sum_{\sigma\in \mathfrak{S}_n}\sgn(\sigma)\sigma\in\complex[\mathfrak{S}_n]$とおくと
\[
\Sym^k(V)=c_{(k)}V^{\otimes k},\qquad\Alt^k(V)=c_{1^k}V^{\otimes k}
\]
となることがわかり、命題\ref{comm_of_sym_gl}よりこの2つは$\gl(V)$の表現でもある。よって$k>2$のときにも、$\lambda\in\mathcal{P}_k$に対するYoung対称子$c_\lambda$による像
\[
W_\lambda=c_\lambda V^{\otimes k}  
\]
を考察することは自然である。再び\ref{comm_of_sym_gl}より$W_\lambda$は$\gl(V)$の部分表現になるが、このとき次が成り立つ

\begin{theo}[Schur-Weyl双対性]\label{schur_weyl}
  $W_\lambda$は$\gl(V)$の既約表現であり、$\mathfrak{S}_\lambda\times\gl(V)$の表現として
  \[
  V^{\otimes k}\simeq \bigoplus_{\lambda\in\mathcal{P}_k}S_\lambda\boxtimes W_\lambda  
  \]
  が成り立つ。
\end{theo}

定理\ref{schur_weyl}を示そう。

\begin{lemm}\label{gen_of_symtensor}
  $V$を$n$次元ベクトル空間とする。$\Sym^k(V)$は$\{v\otimes\cdots\otimes v\}_{v\in V}$によって生成される
\end{lemm}

\begin{proof}
  $n$変数$k$次斉次多項式のなすベクトル空間を$S$とおく。主張は$S$が1次斉次多項式の$k$乗で生成されることと同値である。単項式
  \[
  x_1^{i_1}\cdots x_n^{i_n},\quad i_1+\cdots+i_n=k
  \]
  が生成されることを示せば十分である。$f_0(x_1,\cdots,x_n)=(x_1+\cdots +x_n)^k$とおく。
  \[
  f_1(x_1,\cdots,x_n)=f_0(2x_1,x_2,\cdots,x_n)-2^{k}f_0(x_1,\cdots,x_n)  
  \]
  とすれば、$f_1$は$x_1^k$を含む項をもたない。次に
  \[
  f_2(x_1,\cdots,x_n)=f_1(2x_1,x_2,\cdots,x_n)-2^{k-1}f_1(x_1,\cdots,x_n)  
  \]
  とすれば$f_2$は$x_1^{k}$, $x_1^{k-1}$を含む項をもたない。この操作を$i_1$以外に対して行えば、最終的に$x_1^{i_1}$を含む項以外をもたないような多項式$g_0(x_1,\cdots,x_n)$を得る。そして$g_0$は作り方から、一次斉次多項式の$k$乗の線形結合で表される。同様に
  \[
  g_1(x_1,\cdots,x_n)=g_0(x_1,2x_2,x_3,\cdots,x_n)-2^{k}g_0(x_1,\cdots,x_n)  
  \]
  とすれば$g_1$は$x_2^k$を含む項をもたない。再びこの操作を繰り返して$x_1^{i_1}$, $x_2^{i_2}$を含む項以外をもたないような多項式を得る。これを繰り返していけば、有限回のうちに$x_1^{i_1}x_2^{i_2}\cdots x_n^{i_n}$を作ることができる。
\end{proof}


\begin{lemm}\label{lemma}
  $A$を$\End(V^{\otimes k})$における$\complex[\mathfrak{S}_k]$の像とし、$B$を$\End(V^{\otimes k})$における$\complex[\gl(V)]$の像とする。$A$は半単純環であり、$B=\End_A(V^{\otimes k})$が成り立つ。
\end{lemm}

\begin{proof}
  Maschkeの定理(定理\ref{general_maschke})より$\complex[\mathfrak{S}_n]$は半単純であり、$A$は$\complex[\mathfrak{S}_n]$の剰余環であるが、半単純環の剰余環はまた半単純であるからよい。

  $\End_A(V^{\otimes k})=(\End(V^{\otimes k}))^{\mathfrak{S}_n}=(\End(V)^{\otimes k})^{\mathfrak{S}_n}=\Sym^k(\End(V))$であるから補題\ref{gen_of_symtensor}より、$\End_A(V^{\otimes k})$は$\{X\otimes\cdots\otimes X\}_{X\in\End(V)}$によって生成される。$\gl(V)\subset\End(V)$なのだから、$B\subset \End_A(V^{\otimes k})$は直ちに従う。$g\in\gl(V)$に対して$g\otimes \cdots\otimes g$が生成する$\End(V^{\otimes k})$の部分代数を$B'$とする。任意の(正則とは限らない)$X\in\End(V)$に対して$X\otimes \cdots\otimes X$が$B'$に含まれることを示せばよい。$X+tE$は有限個の$t$を除いて正則である
  \footnote{
    行列式は$t$の多項式
  }
  から、$X$に収束する$\gl(V)$の点列$(X+t_iE)$が存在する
  \footnote{
    すなわち$\gl(V)$は$\End(V)$の稠密集合
  }。$\End(V^{\otimes k})$は有限次元だから$B'$は閉部分空間であるので、
  \[
  X^{\otimes k}=\lim_{i\rightarrow\infty}(X+t_iE)^{\otimes k}\in B'  
  \]
  が成り立つ\footnote{
    表現は連続だからこのような極限操作が可能である。
  }
\end{proof}

定理\ref{schur_weyl}を示そう。
\begin{proof}
  補題\ref{lemma}と定理\ref{DCT}より$\mathfrak{S}_k\times\gl(V)$の表現(すなわち$\complex[\mathfrak{S}_k]\otimes_{\complex}\complex[\gl(V)]$加群)として
  \[
  V^{\otimes k}=\bigoplus_{\lambda\in\mathcal{P}_k}S_\lambda\boxtimes \Hom_A(S_\lambda,V^{\otimes k})  
  \]
  と分解される。
  
  ここで$S_\lambda$の指標を$\chi_\lambda$とすると、$S_\lambda$の反傾表現の指標を$\chi^*_\lambda$とすると、
  \[
  \chi^*_\lambda(g)=\overline{\chi_\lambda(g)}=\chi_\lambda(\inv{g})=\chi_\lambda(g)  
  \]
  である。実際$g$は置換なので$g$と$\inv{g}$は同一の共役類に含まれる。よって表現として$S_\lambda\simeq S_\lambda^*$が成り立つ。

  $S_\lambda=\complex[\mathfrak{S}_k]c_\lambda$だから
  \begin{align*}
    \Hom_A(S_\lambda,V^{\otimes k})
    &=(S_\lambda)^*\otimes_AV^{\otimes k}\\
    &=S_\lambda\otimes_AV^{\otimes k}\\
    &=\complex[\mathfrak{S}_k]c_\lambda\otimes_{\complex[\mathfrak{S}_k]}V^{\otimes k}\\
    &=c_\lambda V^{\otimes k}\\
    &=W_\lambda
  \end{align*}

  定理\ref{DCT}より$\Hom_A(S_\lambda,V^{\otimes k})$は既約$\gl(V)$表現だから$W_\lambda$は既約である。
\end{proof}

$W_\lambda$を$\gl(V)$のWeyl表現という。

\begin{eg}
  $\dim V=n$とする。$\lambda$が$n$行よりも長いYoung図形のとき$W_\lambda=0$となる。$\Lambda$の共役Young図形を$\lambda^*=(\lambda^*_1,\cdots,\lambda^*_s)$, $\lambda_1^*>n$とおく。このとき、
  \[
  b_\lambda
  =\left(
    \sum_{\mathfrak{S}_{\lambda^*_s}}\sgn(\sigma_s)\sigma_s
   \right)\cdots 
   \left(
    \sum_{\mathfrak{S}_{\lambda^*_1}}\sgn(\sigma_1)\sigma_1
   \right)  
  \]
  と書くことができる。$\tilde{b}_\lambda=\sum_{\mathfrak{S}_{\lambda^*_1}}\sgn(\sigma_1)\sigma_1$とする。
  \[
  \lambda=\:\begin{ytableau}
    1 & 5\\
    2 & 6\\
    3 \\
    4
  \end{ytableau}
    ,\text{ のとき }
  b_\lambda=
  \left(
    \sum_{\tau\in\mathfrak{S}_2}\sgn(\tau)\tau
  \right)
  \left(
    \sum_{\sigma\in\mathfrak{S}_4}\sgn(\sigma)\sigma
  \right),\:\tilde{b}_\lambda=\sum_{\sigma\in\mathfrak{S}_4}\sgn(\sigma)\sigma
  \]
  $V^{\otimes k}$を$\lambda$の各箱に$V$の元が書かれているものと同一視する。$V$の基底を$e_1,\cdots,e_n$とすれば$V^{\otimes k}$の元は$e_1,\cdots,e_n$を$\lambda$の各箱に配置した元で生成される。
  \[
  \lambda=\ydiagram{2,2,1,1}\text{ のとき }V^{\otimes k}=\generated{\begin{ytableau}
                e_{i_1} & e_{i_5}\\
                e_{i_2} & e_{i_6}\\
                e_{i_3}\\
                e_{i_4}
              \end{ytableau}}
  \]
  このとき、$\lambda^*_1>n$より、$e_{i_1},\cdots,e_{i_{\lambda^*_1}}$には必ず重複がある。よって
  \[
  \tilde{b}_\lambda(e_{i_1}\otimes\cdots\otimes e_{i_{\lambda^*_1}})=0
  \]
  だから、$b_\lambda V^{\otimes k}=0$である。
\end{eg}

\begin{eg}
  $k=3$, $n>k$とする。$\lambda=\:\ydiagram{2,1}\:$として
  \[
  V^{\otimes 3}=\Sym^3(V)\oplus (S_\lambda\boxtimes W_\lambda)\oplus \Alt^3(V)  
  \]
  であり、$\dim S_{\lambda}=2$であるので
  \[
  \dim W_\lambda=\frac{1}{2}(n^3-\dim\Sym^3(V)-\dim\Alt^3(V))=\frac{1}{12}(4n^3-3n^2-13n-6)
  \]
\end{eg}

\end{document}