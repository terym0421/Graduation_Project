\documentclass{ltjsreport}
\RequirePackage{luatex85}
\usepackage[utf8]{inputenc}
\usepackage{enumerate}
\usepackage{amsthm}
\usepackage{amsfonts}
\usepackage{amsmath}
\usepackage{amssymb}
\usepackage{ytableau}
\usepackage{docmute}
\usepackage{mathtools}
\usepackage{xr}
\usepackage[all]{xy}



\theoremstyle{definition}
\newtheorem{defin}{定義}[subsection]
\newtheorem{theo}[defin]{定理}
\newtheorem{cor}[defin]{系}
\newtheorem{prop}[defin]{命題}
\newtheorem{lemm}[defin]{補題}
\newtheorem{notice}[defin]{注意}
\newtheorem{eg}[defin]{例}


\renewcommand{\labelenumi}{(\roman{enumi})}


\newcommand{\invlimit}{\mathop{\lim_{\longleftarrow}}}
\newcommand{\dirlimit}{\mathop{\lim_{\longrightarrow}}}
\newcommand{\ind}{\text{Ind}\:}
\newcommand{\Hom}{\text{Hom}}
\newcommand{\tr}{\text{tr}\:}
\newcommand{\id}[1]{\text{id}_{#1}}
\newcommand{\sgn}{\mathrm{sgn}}
\newcommand{\res}[1]{\text{Res}_{#1}}
\newcommand{\generated}[1]{\langle\:#1\:\rangle}
\newcommand{\im}{\text{Im }}
\newcommand{\rank}{\text{rank }}
\newcommand{\del}[2]{\frac{\partial #1}{\partial #2}}
\newcommand{\delsametwo}[2]{\frac{\partial^2 #1}{\partial #2^2}}
\newcommand{\delothertwo}[3]{\frac{\partial^2#1}{\partial#2\partial#3}}
\newcommand{\ddel}[2]{\frac{\partial}{\partial #2}#1}
\newcommand{\ddelsametwo}[3]{\frac{\partial^2}{\partial #2^2}#1}
\newcommand{\ddelothertwo}[3]{\frac{\partial^2}{\partial#2\partial#3}#1}
\newcommand{\simneq}{\not\simeq}
\newcommand{\transpose}[1]{^t\!#1}
\newcommand{\ie}{\text{i.e.}}
\newcommand{\inv}[1]{#1^{-1}}
\newcommand{\real}{\mathbb{R}}
\newcommand{\complex}{\mathbb{C}}
\newcommand{\integer}{\mathbb{Z}}
\newcommand{\quotient}{\mathbb{Q}}
\newcommand{\natnum}{\mathbb{N}}
\newcommand{\proj}{\mathbb{P}}
\newcommand{\tensor}[3]{#1\otimes_#2#3}
\newcommand{\map}[3]{#1:#2\rightarrow#3}
\newcommand{\aut}[2]{\mathrm{Aut}_{#1} (#2)}
\newcommand{\hommoph}[2]{\mathrm{Hom}_{#1}(#2)}
\newcommand{\gl}[1]{\mathrm{GL}(#1)}
\newcommand{\set}[2]{\left\{\:#1\:\middle|\:#2\:\right\}}
\newcommand{\pmat}[1]{\begin{pmatrix} #1
\end{pmatrix}}
\newcommand{\vmat}[1]{\begin{vmatrix} #1
\end{vmatrix}}
\newcommand{\br}{\vskip\baselineskip}


\begin{document}

\section{表現環と対称関数環}

\subsection{続・対称多項式}
対称群の表現と対称多項式の間には深い関係がある。次節でそのことを解説するが、そのための準備として対称多項式に関してより詳しく解説する。以下正の整数$n$を固定し、$\Lambda^k_n$を$n$変数の$k$次斉次対称多項式のなす$\integer$加群とする。第1部の記号を復習すると、$n$行のYoung図形$\lambda$に対して
\[
m_\lambda=\sum_{\alpha\sim\lambda}x_1^{\alpha_1}\cdots x_n^{\alpha_n}  
\]
とし、に対して
\begin{align*}
  &e_k=m_{1^k}=\sum_{1\leq i_1<\cdots<i_k\leq n}x_{i_1}\cdots x_{i_k},\qquad(k=1,\cdots,n)\\
  &h_k=\sum_{\lambda\in\mathcal{P}_k}m_\lambda=\sum_{1\leq i_1\leq\cdots\leq i_k\leq n}x_{i_1}\cdots x_{i_k}\\
  &p_k=m_{(k)}=x_1^k+\cdots+x_n^k
\end{align*}
とするのであった。$e_k$や$h_k$に対しては、その母関数を考えることは有用である。すなわち
\begin{align}
  &E(t)=1+e_1t+e_2t^2+\cdots+e_nt^n=\prod_{i=1}^n(1+x_it) \label{gen_func_of_e}\\
  &H(t)=1-h_1t+h_2t^2+\cdots+(-1)^nh_nt^n+\cdots=\prod_{i=1}^n\frac{1}{1+x_it}
\end{align}
である。とくに$E(t)H(t)=1$であるので、$k=1,\cdots,n$のとき
\begin{equation}\label{e_to_h}
  e_k-h_1e_{k-1}+\cdots+(-1)^{k-1}e_1h_{k-1}+(-1)^kh_k=0   
\end{equation}
を得る。

\begin{prop}\label{det_formula}
  $k=1,\cdots,n$に対して
  \begin{align*}
    h_k=\vmat{
      e_1&e_2&e_3&\cdots&e_k\\
      1&e_1&e_2&\cdots&e_{k-1}\\
      0&1&e_1&\cdots&e_{k-2}\\
      \vdots&\vdots&\vdots&\ddots&\vdots&\\
      0&0&0&\cdots&e_1},\qquad
    e_k=\vmat{
      h_1&h_2&h_3&\cdots&h_k\\
      1&h_1&h_2&\cdots&h_{k-1}\\
      0&1&h_1&\cdots&h_{k-2}\\
      \vdots&\vdots&\vdots&\ddots&\vdots&\\
      0&0&0&\cdots&h_1
    }
  \end{align*}
\end{prop}

\begin{proof}
  まったく同様なので$h_k$の場合だけ示す。$e_1=h_1$であり、$k-1$までこの公式が成り立っていたとすると、
  \begin{align*}
    e_k&=h_1e_{k-1}-h_2e_{k-2}+\cdots+(-1)^{k-1}h_k \\
    &=h_1\vmat{
      h_1&h_2&\cdots&h_{k-1}\\
      1&h_1&\cdots&h_{k-2}\\
      \vdots&\vdots&\ddots&\vdots&\\
      0&0&\cdots&h_1
    }-h_2\vmat{
      h_1&\cdots&h_{k-1}\\
      \vdots&\ddots&\vdots&\\
      0&\cdots&h_1
    }+\cdots+(-1)^{k-1}h_k\\
    &=\vmat{
      h_1&h_2&h_3&\cdots&h_k\\
      1&h_1&h_2&\cdots&h_{k-1}\\
      0&1&h_1&\cdots&h_{k-2}\\
      \vdots&\vdots&\vdots&\ddots&\vdots&\\
      0&0&0&\cdots&h_1
    }
  \end{align*}
\end{proof}

次に、$\lambda\in\mathcal{P}_k$に対して
\begin{align*}
  &e_\lambda=e_{\lambda_1}\cdots e_{\lambda_n}\\
  &h_\lambda=h_{\lambda_1}\cdots h_{\lambda_n}\\
  &p_\lambda=p_{\lambda_1}\cdots p_{\lambda_n}
\end{align*}
とする。

\begin{prop}\label{various_basis}
  $\Lambda^k_n$の次の部分集合について
  \begin{enumerate}[(i)]
    \item $\{m_\lambda\}_\lambda$ ただし$\lambda$は大きさが$k$で$n$行
    \item $\{e_\lambda\}_\lambda$ ただし$\lambda$は大きさが$k$で$n$列
    \item $\{h_\lambda\}_\lambda$ ただし$\lambda$は大きさが$k$で$n$列
    \item $\{s_\lambda\}_\lambda$ ただし$\lambda$は大きさが$k$で$n$行
    \item $\{p_\lambda\}_\lambda$ ただし$\lambda$は大きさが$k$で$n$列
  \end{enumerate}
  (i)$\sim$(iv)は$\Lambda^k_n$の$\integer$上の基底をなし、(v)は$\quotient\otimes_{\integer}\Lambda^k_n$の基底をなす。
\end{prop}

\begin{proof}
  (i)は命題\ref{m_is_basis}の証明を斉次部分で考えればまったく同様である。(ii)については定理\ref{FT_of_sym}の証明において、任意の対称多項式$f$が
  \[
  e_1^{a_1}\cdots e_n^{a_n}
  \]
  で生成されていることを示したことからわかる。(iv)は\ref{schur}の証明を斉次部分で行えばよい。(iii)については、命題\ref{det_formula}より$\{h_\lambda\}$が$\{e_\lambda\}$を生成することがわかるが、ともに集合の濃度が等しいことから基底をなすことがわかる。(v)が基底をなすことを示そう。(v)が(iii)を生成することを示せばよい。
  \begin{align*}
    &\quad1+h_1t+h_2t^2+\cdots\\
    &=H(-t)\\
    &=\prod_{i=1}^n\frac{1}{1-x_it}\\
    &=\exp \left(\sum_{i=1}^n-\log(1-x_it)\right)\\
    &=\exp \left(\sum_{i=1}^n\sum_{r=1}^\infty\frac{x_i^rt^r}{r}\right)\\
    &=\exp \left(\sum_{r=1}^\infty\frac{p_r}{r}t^r\right)\\
    &=\prod_{r=1}^\infty\exp \left(\frac{p_r}{r}t^r\right)\\
    &=\prod_{r=1}^\infty \sum_{m_r=0}^\infty \frac{p_r^{m_r}}{m_r!\cdot r^{m_r}}t^{r\cdot m_r}\\
    &=
    \left(\sum_{m_1=0}^\infty \frac{p_1^{m_1}}{m_1!\cdot 1^{m_1}}t^{m_1}\right)\cdot
    \left(\sum_{m_2=0}^\infty \frac{p_2^{m_2}}{m_2!\cdot 2^{m_2}}t^{2 \cdot m_2}\right)\cdot
    \left(\sum_{m_3=0}^\infty \frac{p_3^{m_3}}{m_3!\cdot 3^{m_3}}t^{3 \cdot m_3}\right)\cdots
  \end{align*}
  となるから、Young図形$\lambda=(\lambda_1,\lambda_2,\cdots)$に対して
  \[
  z(\lambda)=\prod_{i}m_i!\cdot i^{m_i},\qquad \text{ただし$m_i$は$\lambda$に現れる$i$の個数}  
  \]
  とおけば、最後の式は
  \begin{align*}
    &\quad\sum_{m_1,m_2,m_3,\cdots}\frac{p_1^{m_1}p_2^{m_2}p_3^{m_3}\cdots}{(m_1!\cdot 1^{m_1})(m_2!\cdot 2^{m_2})(m_3!\cdot 3^{m_3})\cdots}t^{m_1+2\cdot m_2+3\cdot m_3+\cdots}\\
    &=\sum_{\lambda}\frac{p_\lambda}{z(\lambda)}t^{|\lambda|}\\
    &=\sum_{k=0}^\infty \sum_{\lambda\in\mathcal{P}_{k}}\frac{p_\lambda}{z(\lambda)}t^k
  \end{align*}
  となる。よって
  \[
  h_k=\sum_{\lambda\in\mathcal{P}_{k}}\frac{p_\lambda}{z(\lambda)}
  \]
  が成り立つ。$n$列のYoung図形$\lambda$に対して
  \[
  h_\lambda=h_{\lambda_1}\cdots h_{\lambda_s},\qquad \lambda_i\leq n 
  \]
  とおくと、
  \[
  h_{\lambda_i}=\sum_{\mu_i\in\mathcal{P}_{\lambda_i}}\frac{p_{\mu_i}}{z(\mu_i)}  
  \]
  であるから
  \[
  h_\lambda=\sum_{\mu_1,\cdots,\mu_s}\frac{p_{\mu_1}\cdots p_{\mu_s}}{z(\mu_1)\cdots z(\mu_s)}  
  \]
  各$\mu_i$は$\lambda_i\leq n$の分割を与えているから、$p_{\mu_1}\cdots p_{\mu_s}$はたかだか$n$列のYoung図形に対応するべき和対称式である。よって(v)も基底を与える。
\end{proof}

証明中に現れた等式は重要なので再掲しておく。

\begin{prop}\label{p_to_h}
  正の整数$k,n$に対して
  \[
    h_k(x_1,\cdots,x_n)=\sum_{\lambda\in\mathcal{P}_{k}}\frac{p_\lambda(x_1,\cdots,x_n)}{z(\lambda)}
  \]
  が成り立つ。
\end{prop}

後に必要になる公式を用意しておく


\begin{lemm}[Cauchyの等式]\label{formal_power_series_relation}
  形式的べき級数$\prod_{i=1}^m\prod_{j=1}^n\frac{1}{1-x_iy_j}$は次と等しい。
  \begin{align*}
    &\text{(i):}\:\sum_{\lambda,\lambda_{n+1}=0}h_\lambda(x_1,\cdots,x_m)m_\lambda(y_1,\cdots,y_n),\qquad \text{ただし和は$n$行Young図形全体をわたる}\\
    &\text{(ii):}\:\sum_{\lambda}\frac{1}{z(\lambda)}p_\lambda(x_1,\cdots,x_m)p_\lambda(y_1,\cdots,y_n),\qquad \text{ただし和はすべてのYoung図形全体をわたる}\\
    &\text{(iii):}\:\sum_{\lambda}s_\lambda(x_1,\cdots,x_m)s_\lambda(y_1,\cdots,y_n),\qquad \text{ただし和はすべてのYoung図形全体をわたる}
  \end{align*}
\end{lemm}

\begin{proof}
  (i)を示す。
  \begin{align*}
    &\quad\prod_{i=1}^m\prod_{j=1}^n\frac{1}{1-x_iy_j}\\
    &=\prod_{j=1}^n H(y_j)\\
    &=\prod_{j=1}^n \left(
      \sum_{k_j=0}^\infty h_{k_j}(x_1,\cdots,x_m)y_j^{k_j}
      \right)\\
    &=\left(
      \sum_{k_1=0}^\infty h_{k_1}(x_1,\cdots,x_m)y_1^{k_1}
      \right)\cdot
      \left(
      \sum_{k_2=0}^\infty h_{k_2}(x_1,\cdots,x_m)y_2^{k_2}
      \right)
      \cdots
      \left(
      \sum_{k_n=0}^\infty h_{k_n}(x_1,\cdots,x_m)y_n^{k_n}
      \right)\\
    &=\sum_{k_1,\cdots,k_n}h_{k_1}\cdots h_{k_n}y_1^{k_1}  
      \cdots y_n^{k_n}\\
    &=\sum_{\lambda, \lambda_{n+1}=0}h_\lambda(x_1,\cdots,
      x_m)m_\lambda(y_1,\cdots,y_n)
  \end{align*}
  (ii)を示す。命題\ref{p_to_h}より
  \begin{align*}
  \prod_{i,j}\frac{1}{1-x_iy_j}
  &=1+h_1(\{x_iy_j\})+h_2(\{x_iy_j\})+\cdots\\
  &=1+\sum_{\lambda\in\mathcal{P}_1}\frac{1}{z(\lambda)}p_\lambda(\{x_iy_j\})+\sum_{\lambda\in\mathcal{P}_2}\frac{1}{z(\lambda)}p_\lambda(\{x_iy_j\})+\cdots\\
  &=\sum_{\lambda}\frac{1}{z(\lambda)}p_\lambda(\{x_iy_j\})\\
  &=\sum_{\lambda}\frac{1}{z(\lambda)}p_\lambda(x_1,\cdots,x_m)p_\lambda(y_1,\cdots,y_n)
  \end{align*}
  (iii)については、Robinson-Schensted-Knuth対応と呼ばれる対応を用いて証明される。詳細は付録を参照
\end{proof}


\begin{lemm}\label{p_to_m}
  $\mu\in\mathcal{P}_k$とし、変数の数$n$は$n\geq k$とする。
  \[
  p_\mu=\sum_{\lambda\in\mathcal{P}_k}\xi_{\lambda\mu}m_\lambda
  \]
  とおいたとき、
  \[
  \xi_{\lambda\mu}=\sum\prod_{q=1}^n\frac{m_q!}{c_{1q}!\cdots c_{nq}!}=\chi_{M_\lambda}(g_\mu)  
  \]
  ただし、$g_\mu$は$\mu$を置換の型にもつ元であり、$\mu=(1^{m_1},2^{m_2},\cdots)$であり、また和は命題\ref{m_lambda_char_1}と同様
  \begin{itemize}
    \item $c_{p1}+2c_{p2}+3c_{p3}+\cdots+nc_{pn}=\lambda_p$, for all $p=1,\cdots,n$
    \item $c_{1q}+c_{2q}+c_{3q}+\cdots+c_{nq}=m_q$, for all $q=1,\cdots,n$
  \end{itemize}
  をみたす非負行列$\{c_{pq}\}\in M_n(\integer_{\geq 0})$全体をわたる。
\end{lemm}

\begin{proof}
  $p_\mu=p_1^{m_1}p_2^{m_2}\cdots p_n^{m_n}$であるが、
  \[
  p_q^{m_q}=(x_1^q+x_2^q+\cdots+x_n^q)^{m_q}
  =\sum_{
    c_{1q},c_{2q},\cdots,c_{nq}\geq 0
  } \frac{m_q!}{c_{1q}!c_{2q}!\cdots c_{nq}!}x_1^{c_{1q}}x_2^{c_{2q}}\cdots x_n^{c_{nq}}
  \]
  であるので、
  \begin{align*}
  p_\mu
  &=\prod_{q=1}^n\sum_{
    c_{1q},c_{2q},\cdots,c_{nq}\geq 0
  } \frac{m_q!}{c_{1q}!c_{2q}!\cdots c_{nq}!}x_1^{c_{1q}}x_2^{c_{2q}}\cdots x_n^{c_{nq}}\\
  &=\sum\prod_{q=1}^n\frac{m_q!}{c_{1q}!c_{2q}!\cdots c_{nq}!}x_1^{\mu_1}x_2^{\mu_2}\cdots x_n^{\mu_n}
  \end{align*}
\end{proof}










\subsection{表現環と対称関数環}

\begin{defin}
  可算無限個の変数をもつ形式的べき級数環$\integer[[x_1,x_2,\cdots]]$を考える。
  \[
  \mathfrak{S}=\set{\map{\sigma}{\natnum}{\natnum}}{\text{$f$は全単射で$f(n)\neq n$なる$n$が有限個}}  
  \]
  とする
  \footnote{
    $\mathfrak{S}$は対称群$\mathfrak{S}_n$と自然な包含$\iota:\mathfrak{S}_n\rightarrow\mathfrak{S}_{n+1}$のなす帰納系の帰納極限である。
  }。
  \[
  \Lambda=\set{f\in \integer[[x_1,x_2,\cdots]]}{\sigma f=f,\:(\text{for all }\sigma\in\mathfrak{S}),\:\text{$f$の単項式の次数は有界}}  
  \]
  $\Lambda$は$\integer[[x_1,x_2,\cdots]]$の部分環で対称関数環と呼ばれる。$\Lambda^k$を
  \[
  \Lambda^k=\set{f\in\Lambda}{\text{$f$の単項式の次数はすべて$k$}}  
  \]
  で定め、$\Lambda^k$の元を$k$次斉次対称関数という。
  \[
  \Lambda=\bigoplus_{k=0}^\infty\Lambda^k  
  \]
  より$\Lambda$は次数付き環の構造をもつ。
\end{defin}

ここで、$\Lambda$の定義において単項式の次数が有界であることを要請するのは自然である。実際、もし仮定しなければ$\Lambda$は$\Lambda^k$の直和にはならない。

\begin{eg}[単項対称関数]\label{monomial_sym_func}
  任意のYoung図形$\lambda=(\lambda_1,\cdots,\lambda_n)$に対して
  \[
  m_{\lambda}=\sum_{\alpha\sim\lambda}x_1^{\alpha_1}x_2^{\alpha_2}\cdots x_{n}^{\alpha_n}
  \]
  とする。ここで指数$\alpha$は、$\lambda$の置換になっているもの全体をわたる。すなわちある$\sigma\in\mathfrak{S}$が存在して$\alpha=\sigma\lambda$をみたすもの全体である。$m_\lambda$は対称関数である。対称多項式の場合と同様の議論で、$\Lambda^k$は$\{m_\lambda\}_{\lambda\in\mathcal{P}_k}$を基底に持つことがわかる。
\end{eg}

\begin{eg}[基本対称関数・完全対称関数]\label{elementary_func}
  \begin{align*}
  &e_k=m_{1^k}
  =\sum_{1\leq i_1<i_2<\cdots<i_k}x_{i_1}x_{i_2}\cdots x_{i_k}\\
  &h_k=\sum_{\lambda\in\mathcal{P}_k} m_\lambda
  =\sum_{1\leq i_1\leq i_2\leq \cdots\leq i_k}x_{i_1}x_{i_2}\cdots x_{i_k}
  \end{align*}
  をそれぞれ、基本対称関数, 完全対称関数という。また、任意のYoung図形$\lambda=(\lambda_1,\cdots,\lambda_n)$に対して
  \begin{align*}
    &e_\lambda=e_{\lambda_1}\cdots e_{\lambda_n}\\
    &h_\lambda=h_{\lambda_1}\cdots h_{\lambda_n}
  \end{align*}
  とする。
  \begin{align*}
    &e_1=x_1+x_2+x_3+\cdots\\
    &e_2=\sum_{i<j}x_ix_j
  \end{align*}
  である。
\end{eg}

\begin{eg}[べき和対称関数]\label{power_func}
  $(k)=(k,0,\cdots,0)$に対して
  \[
  p_{k}=m_{(k)}=x_1^k+x_2^k+\cdots
  \]
  とする。
\end{eg}

このように、対称関数はいままでみてきた対称多項式を自然に無限変数に拡張した概念であり、対称多項式で成り立っていた関係式が対称関数においても成立することが多い。このことは対称関数の$k$次斉次部分$\Lambda^k$が$k$次斉次対称多項式からの射影極限と考えることができることによる。$\Lambda_n^k$を$n$変数$k$次斉次対称多項式のなす$\integer$加群とする。$m\leq n$に対して線形写像$\map{\rho_{m,n}}{\Lambda^k_n}{\Lambda^k_m}$を
\[
\rho_{m,n}(f(x_1,\cdots,x_m,x_{m+1},\cdots,x_n))=f(x_1,\cdots,x_m,0,\cdots,0)  
\]
によって定める。ここで$\rho_{m,n}(f)$は実際に$m$変数の$k$次斉次対称多項式である\footnote{変数の置換と$0$を代入する操作が可換であることによる}。$l\leq m\leq n$に対して
\[
\rho_{l,m}\circ\rho_{m,n}=\rho_{l,n}  
\]
が成り立つから、$\{\Lambda^k_n,\rho_{m,n}\}$は射影系をなす。

\begin{prop}\label{sym_func_is_inverselimit}
  上の状況において、
  \[
    \Lambda^k=\mathop{\lim_{\longleftarrow}}\Lambda^k_n
  \]
  がなりたつ。
\end{prop}

\begin{proof}
  $\theta_n:\Lambda^k\rightarrow\Lambda^k_n$を$n+1$番目以降の変数を$0$にする写像とすれば、
  \[
  \rho_{m,n}\circ\theta_n=\theta_m  
  \]
  が成り立つから、射影極限の普遍性から線形写像
  \[
  \theta:\Lambda^k\rightarrow\mathop{\lim_{\longleftarrow}}\Lambda^k_n
  \]
  が誘導される。$\mathop{\lim_{\longleftarrow}}\Lambda^k_n$から$\Lambda^k$への写像$\varphi$は次のように定義する。$\mathop{\lim_{\longleftarrow}}\Lambda^k_n$の元$(f_n)_{n\in\integer_{>0}}$, ($f_n\in\Lambda^k_n$)に対して、$k$変数の部分に注目すると
  \[
  f_k=\sum_{\lambda\in\mathcal{P}_k}c_\lambda m_\lambda(x_1,\cdots,x_k)
  \]
  と一意的に表せるので
  \[
  \varphi((f_n)_{n\in\integer_{>0}})
  =\sum_{\lambda\in\mathcal{P}_k}c_\lambda m_\lambda
  \]
  と定める。ただし右辺の$m_\lambda$は例\ref{monomial_sym_func}の対称関数である。$\varphi$が$\theta$の逆写像であることを示そう。
  \begin{equation}\label{proj_of_monomial}
  \theta_n(m_\lambda)=
  \left\{\begin{array}{cc}
    m_\lambda(x_1,\cdots,x_n) & \text{if }\lambda_{n+1}=0\\
    0 & \text{otherwise}
  \end{array}\right.  
\end{equation}
  であるが、$\lambda$は$k$の分割であるので$n\geq k$において$\lambda_{n+1}=0$である。よって$n\geq k$ならば
  \begin{equation}\label{theta_n}
  \theta_n(\varphi((f_n)_{n\in\integer_{>0}}))
  =\sum_{\lambda\in\mathcal{P}_k}c_\lambda m_\lambda(x_1,\cdots,x_n)  
  \end{equation}
  が成り立つ。一方、
  \[
  f_n=\sum_{\lambda\in\mathcal{P}_k(n)}d_\lambda m_\lambda(x_1,\cdots,x_n)    
  \]
  とおくと$n\geq k$より$\mathcal{P}_k(n)=\mathcal{P}_k$だから
  \[
  f_n=\sum_{\lambda\in\mathcal{P}_k}d_\lambda m_\lambda(x_1,\cdots,x_n)    
  \]
  となる。よって$\rho_{k,n}(f_n)=f_k$と(\ref{theta_n})より
  \[
  f_n=\theta_n(\varphi((f_n)_{n\in\integer_{>0}}))
  \]
  次に$n<k$の場合、(\ref{proj_of_monomial})より
  \[
  \theta_n(\varphi((f_n)_{n\in\integer_{>0}}))  
  =\sum_{\lambda\in\mathcal{P}(k)}c_\lambda m_\lambda
  \]
  となるが、
  \[
  f_n=\sum_{\lambda\in\mathcal{P}_k}d_\lambda m_\lambda(x_1,\cdots,x_n)    
  \]
  とおけば$\rho_{n,k}(f_k)=f_n$より
  \[
  f_n=\theta_n(\varphi((f_n)_{n\in\integer_{>0}}))
  \]
  以上より
  \[
  \theta\circ\varphi=\text{id}
  \]
  がわかる。逆に任意の$f\in\Lambda^k$に対して
  \[
  f=\sum_{\lambda\in\mathcal{P}_k}c_\lambda m_\lambda  
  \]
  とおけば(\ref{proj_of_monomial})より
  \[
  \theta_k(f)=\sum_{\lambda\in\mathcal{P}_k}c_\lambda m_\lambda(x_1,\cdots,x_k)
  \]
  だから
  \[
  \varphi(\theta(f))=f  
  \]
  がわかる。
\end{proof}


\begin{notice}
  対称多項式環の射影極限をとっても対称関数環にはならないことに注意せよ。例えば対称多項式の列
  \[
    f=((1+x_1),\:(1+x_1)(1+x_2),\:(1+x_1)(1+x_2)(1+x_3),\:\cdots)
  \]
  を考えると、$f$は対称多項式の射影極限の元であるが、次数の有界性を満たさないので対称関数ではない。

  一方
  \[
  \integer[x_1,\cdots,x_n]^{\mathfrak{S}_n}=\bigoplus_{k=0}^\infty \Lambda^k_n  
  \]
  だから、対称関数環は射影極限が直和と可換でない例を与えている。
\end{notice}


\begin{eg}\label{monomial_sym_fun_as_lim}
  $n$行のYoung図形$\lambda$に対して
  \[
  \rho_{n,n+1}(m_\lambda(x_1,\cdots,x_n,x_{n+1}))=m_\lambda(x_1,\cdots,x_n)  
  \]
  が成り立つ。実際
  \begin{align*}
    \rho_{n,n+1}(m_\lambda(x_1,\cdots,x_n,x_{n+1}))
    &=\rho_{n,n+1}\left(
      \sum_{\alpha\sim\lambda}x_1^{\alpha_1}\cdots x_n^{\alpha_n}x_{n+1}^{\alpha_{n+1}}
      \right)\\
    &=\sum_{\substack{\alpha\sim\lambda \\ \alpha_{n+1}=0}}x_1^{\alpha_1}\cdots x_n^{\alpha_n}\\
    &=m_\lambda(x_1,\cdots,x_n)
  \end{align*}
  よって$k=|\lambda|$次対称多項式の列$(m_\lambda(x_1,\cdots,x_l))_{l\in\integer_{\geq n}}$は一つの対称関数を定めるが、これは単項対称関数$m_\lambda$に他ならない。
\end{eg}

\begin{eg}
  例\ref{monomial_sym_fun_as_lim}と命題\ref{various_basis}より$e_\lambda$, $h_\lambda$, $p_\lambda$,$s_\lambda$もすべて一つの対称関数を定める。$e_\lambda,h_\lambda,p_\lambda$の定める対称関数は、例\ref{elementary_func}と例\ref{power_func}に他ならない。また$s_\lambda$の定める対称関数はSchur関数という。
\end{eg}

命題\ref{various_basis}より、次が成り立つ。

\begin{prop}\label{various_symfunc_basis}
  $\Lambda^k$の次の部分集合について、
  \begin{enumerate}[(i)]
    \item $\{m_\lambda\}_{\lambda\in\mathcal{P}_k}$
    \item $\{e_\lambda\}_{\lambda\in\mathcal{P}_k}$
    \item $\{h_\lambda\}_{\lambda\in\mathcal{P}_k}$
    \item $\{s_\lambda\}_{\lambda\in\mathcal{P}_k}$
    \item $\{p_\lambda\}_{\lambda\in\mathcal{P}_k}$
  \end{enumerate}
  (i)$\sim$(iv)は$\Lambda^k$の$\integer$上の基底をなし、(v)は$\quotient\otimes_\integer{\Lambda^k}$上の基底をなす。特に$\lambda$の範囲をすべてのYoung図形全体に変えれば、これらは$\Lambda$または$\quotient\otimes_\integer{\Lambda}$の基底を与える。
\end{prop}

また、命題\ref{det_formula}や命題\ref{p_to_h}, 補題\ref{formal_power_series_relation}, Littlewood-Richardson規則などの関係式は、そのまま対称関数においても成立することがわかる。(変数の制限$\rho_{m,n}$は和や積と可換である)

\begin{prop}\label{various_relation}
  次の関係式が成り立つ\footnote{
    $e_k$の母関数$E(t)=\prod_{i=1}^\infty(1+x_it)$や$h_k$の母関数$H(t)=\prod_{i=1}^\infty\frac{1}{1-x_it}$を用いて補題\ref{formal_power_series_relation}と同じ計算で直接示してもよい。
  }
  \begin{align*}
    &\text{(i):}\:h_k=\vmat{
      e_1&e_2&e_3&\cdots&e_k\\
      1&e_1&e_2&\cdots&e_{k-1}\\
      0&1&e_1&\cdots&e_{k-2}\\
      \vdots&\vdots&\vdots&\ddots&\vdots&\\
      0&0&0&\cdots&e_1},\qquad
    e_k=\vmat{
      h_1&h_2&h_3&\cdots&h_k\\
      1&h_1&h_2&\cdots&h_{k-1}\\
      0&1&h_1&\cdots&h_{k-2}\\
      \vdots&\vdots&\vdots&\ddots&\vdots&\\
      0&0&0&\cdots&h_1
    }\\
    &\text{(ii):}\:h_k=\sum_{\lambda\in\mathcal{P}_k}\frac{p_\lambda}{z(\lambda)}\\
    &\text{(iii):}\:
    \sum_{\lambda}h_\lambda(x) m_\lambda(y)
    =\sum_{\lambda}\frac{1}{z(\lambda)}p_\lambda(x)p_\lambda(y)
    =\sum_{\lambda}s_\lambda(x)s_\lambda(y)
    =\prod_{i,j}\frac{1}{1-x_iy_j}\\
    &\text{(iv):}\:
    s_\lambda s_\mu=\sum_{\nu}\eta^{\nu}_{\lambda\mu}s_\nu,\qquad\text{ただし$\eta^{\nu}_{\lambda\mu}$はLittlewood-Richardson数}\\
    &\text{(vi):}\:
    p_\mu=\sum_{\lambda\in\mathcal{P}_k}\xi_{\lambda\mu}m_\lambda
    \text{ ここで }
    \xi_{\lambda\mu}=\sum\prod_{q=1}^n\frac{m_q!}{c_{1q}!\cdots c_{nq}!}=\chi_{M_\lambda}(g_\mu)  
  \end{align*}
\end{prop}

以下、$\Lambda$の係数を$\quotient$に拡大して考える。命題\ref{various_symfunc_basis}より、$\{s_\lambda\}$は$\Lambda$の基底をなすので、$\Lambda$に
\[
\generated{s_\lambda,s_\mu}=\delta_{\lambda\mu}  
\]
となるような内積を入れて考える。ただし$\delta_{\lambda\mu}$はKroneckerのデルタである。

\begin{prop}\label{inner_prod}
  次が成り立つ
  \begin{enumerate}[(i)]
    \item $\generated{h_\lambda,m_\mu}=\delta_{\lambda\mu}$
    \item $\generated{p_\lambda,p_\mu}=\delta_{\lambda\mu}z(\lambda)$
  \end{enumerate}
\end{prop}

\begin{proof}
  一般に$\Lambda$の基底$\{u_\lambda\}$, $\{v_\lambda\}$について
  \begin{enumerate}[(a)]
    \item $\generated{u_\lambda,v_\mu}=\delta_{\lambda\mu}$
    \item $\sum_{\lambda}u_\lambda(x) v_\lambda(y)=\prod_{i,j}\frac{1}{1-x_iy_j}$
  \end{enumerate}
  が同値であることを示せば命題\ref{various_relation}より従う。

  $u_\lambda(x)=\sum_{\nu_1}a_{\lambda\nu_1}s_{\nu_1}$, $v_\mu=\sum_{\nu_2}b_{\mu\nu_2}s_{\nu_2}$とおく。(a)は
  \begin{equation}\label{equi_1}
    \sum_{\nu}a_{\lambda\nu}b_{\mu\nu}=\delta_{\lambda\mu}
  \end{equation}
  と同値である。命題\ref{various_relation}より
  \[
  \prod_{i,j}\frac{1}{1-x_iy_j}=\sum_{\lambda}s_\lambda(x)s_\lambda(y)  
  \]
  であり、
  \begin{align*}
    \sum_{\lambda}u_\lambda(x)v_\lambda(y)
    &=\sum_{\lambda}\sum_{\nu_1,\nu_2}a_{\lambda\nu_1}b_{\lambda\nu_2}s_{\nu_1}(x)s_{\nu_2}(y)\\
    &=\sum_{\nu_1,\nu_2}\left(\sum_{\lambda}a_{\lambda\nu_1}b_{\lambda\nu_2}\right)s_{\nu_1}s_{\nu_2}\\
    &=\sum_{\nu}\left(\sum_{\lambda}a_{\lambda\nu}b_{\lambda\nu}\right)s_\nu(x)s_\nu(y)
    +\sum_{\nu_1\neq\nu_2}\left(\sum_{\lambda}a_{\lambda\nu_1}b_{\lambda\nu_2}\right)s_{\nu_1}(x)s_{\nu_2}(y)
  \end{align*}
  だから、$\{s_\lambda(x)s_\mu(y)\}_{\lambda,\mu}$は一次独立であることに注意すれば、(b)は
  \begin{equation}\label{equi_2}
    \sum_{\lambda}a_{\lambda\nu_1}b_{\lambda\nu_2}=\delta_{\nu_1\nu_2}  
  \end{equation}
  と同値である。(\ref{equi_1})と(\ref{equi_2})は同値(行列の行ベクトルが正規直交であることと列ベクトルが正規直交であることは同値)であるので、(a)と(b)も同値である。
\end{proof}


次に対称群の表現全体から作られる環を導入する。

\begin{defin}
  $\mathfrak{S}_n$の表現の同値類全体で生成される自由Abel群を$\tilde{R_n}$とする。$D$を
  \[
  \set{[V\oplus W]-[V]-[W]\in R}{V,W\text{はそれぞれ$\mathfrak{S}_n$の表現}}  
  \]
  で生成される$\tilde{R_n}$の部分加群とし、$R_n=\tilde{R_n}/D$とする。$R_0=\integer$として$R=\bigoplus_{n=0}^\infty R_n$とおく。
  
  $\mathfrak{S}_n$, $\mathfrak{S}_m$の表現$V,W$に対して、
  \[
  [V]\circ [W]=[\ind_{\mathfrak{S}_n\times\mathfrak{S}_m}^{\mathfrak{S}_{n+m}}V\boxtimes W]
  \]
  と定める。ここで、$\mathfrak{S}_n\times\mathfrak{S}_m$は$\mathfrak{S}_n$の元を$n+1$から$n+m$を固定する置換と同一視し、$\mathfrak{S}_m$の元を$1$から$n$を固定する置換と同一視することで$\mathfrak{S}_n\times\mathfrak{S}_m$の部分群とみなしている。$\circ$を双線形に拡張することによって$R$は可換環の構造を持つ。$R$を対称群の表現環という。
\end{defin}

\begin{prop}
  $\circ$は実際に乗法を定め、$R$は次数付き可換環となる。
\end{prop}

\begin{proof}
  テンソル積と直和の可換性から、$\mathfrak{S}_n$の表現$V,V'$に対して
  \[
    \ind_{\mathfrak{S}_n\times\mathfrak{S}_m}^{\mathfrak{S}_{n+m}}(V\oplus V')\boxtimes W
    =(\ind_{\mathfrak{S}_n\times\mathfrak{S}_m}^{\mathfrak{S}_{n+m}}V\boxtimes W)
    \oplus
    (\ind_{\mathfrak{S}_n\times\mathfrak{S}_m}^{\mathfrak{S}_{n+m}}V'\boxtimes W)
  \]
  よって
  \[
  ([V]+[V'])\circ [W]=[V\oplus V']\circ [W]=[V]\circ[W]+[V']\circ[W]  
  \]
  より$\circ$は双線形である。$\circ$が可換であることは$\mathfrak{S}_n$の元と$\mathfrak{S}_m$の元が$\mathfrak{S}_{n+m}$において可換であることからわかる。

  乗法が結合的であることを示そう。すなわち
  \[
  ([V]\circ[W])\circ[U]=[V]\circ([W]\circ [U])  
  \]
  を示す。そのためには$\mathfrak{S}_n$, $\mathfrak{S}_m$, $\mathfrak{S}_l$それぞれの表現$V,W,U$に対して、2つの表現
  \[
  \ind_{\mathfrak{S}_{n+m}\times\mathfrak{S}_l}^{\mathfrak{S}_{n+m+l}}\left\{
    (\ind_{\mathfrak{S}_n\times\mathfrak{S}_m}^{\mathfrak{S_{n+m}}}V\boxtimes W)\boxtimes U
  \right\},\qquad
  \ind_{\mathfrak{S}_{n}\times\mathfrak{S}_{m+l}}^{\mathfrak{S}_{n+m+l}}\left\{
    V\boxtimes (\ind_{\mathfrak{S}_m\times\mathfrak{S}_l}^{\mathfrak{S}_{m+l}}W\boxtimes U)
  \right\}
  \]
  の指標が等しいことをみればよい。
  $
    \ind_{\mathfrak{S}_{n+m}\times\mathfrak{S}_l}^{\mathfrak{S}_{n+m+l}}\left\{
      (\ind_{\mathfrak{S}_n\times\mathfrak{S}_m}^{\mathfrak{S_{n+m}}}V\boxtimes W)\boxtimes U
    \right\} 
  $ 
  の指標を$\chi_A$とおき、$\ind_{\mathfrak{S}_n\times\mathfrak{S}_m}^{\mathfrak{S_{n+m}}}V\boxtimes W$の指標を$\chi_{A'}$とおく。誘導指標の公式(命題\ref{ind_char})より、
  \begin{align*}
    \chi_A(g)
    &=\frac{1}{(n+m)!\:l!}
    \sum_{\substack{\sigma\in\mathfrak{S}_{n+m+l} \\ \inv{\sigma}g\sigma\in\mathfrak{S}_{n+m}\times\mathfrak{S}_l}}
    \chi_{A'}((\inv{\sigma}g\sigma)|_{n+m})\chi_U((\inv{\sigma}g\sigma)|_l)
  \end{align*}
  ここで、$(\inv{\sigma}g\sigma)|_{n+m}$は$\inv{\sigma}g\sigma$の最初の$n+m$文字の置換への制限である。
  $g$と$\inv{\sigma}g\sigma$の置換の型は同じだから、$\inv{\sigma}g\sigma\in\mathfrak{S}_{n+m}\times\mathfrak{S}_l$と、$g\in\mathfrak{S}_{n+m}\times\mathfrak{S}_l$は同値である。よって、
  \begin{align*}
    \chi_A(g)=\left\{\begin{array}{lc}
      \frac{(n+m+l)!}{(n+m)!\:l!}\chi_{A'}(g|_{n+m})\chi_U(g|_l) & \text{if $g\in\mathfrak{S}_{n+m}\times\mathfrak{S}_l$}\\
      0 & \text{otherwise}
    \end{array}\right.
  \end{align*}
  となる。
  同様の議論で、
  \begin{align*}
  \chi_{A'}(g|_{n+m})
  &=\frac{1}{n!\:m!}
  \sum_{ \substack{t\in\mathfrak{S_{n+m}} \\ \inv{t}g|_{n+m}t\in\mathfrak{S}_n\times\mathfrak{S}_m} }
  \chi_V((\inv{t}gt)_n)\chi_W((\inv{t}gt)_m) \\
  &=\left\{\begin{array}{lc}
    \frac{(n+m)!}{n!\:m!}\chi_V(g|_n)\chi_W(g|_m) & \text{if $g\in\mathfrak{S}_n\times\mathfrak{S}_m\times\mathfrak{S}_l$}\\
    0 & \text{otherwise}
    \end{array}\right.
  \end{align*}
  となる。結局、
  \begin{equation}\label{char_A}
  \chi_A(g)=\left\{\begin{array}{lc}
    \frac{(n+m+l)!}{n!\:m!\:l!}\chi_V(g|_n)\chi_W(g|_m)\chi_U(g|_l) & \text{if $g\in\mathfrak{S}_n\times\mathfrak{S}_m\times\mathfrak{S}_l$}\\
    0 & \text{otherwise}
  \end{array}\right.  
\end{equation}
  で、$n,m,l$について対称な形となった。同様に
  $
  \ind_{\mathfrak{S}_{n}\times\mathfrak{S}_{m+l}}^{\mathfrak{S}_{n+m+l}}\left\{
    V\boxtimes (\ind_{\mathfrak{S}_m\times\mathfrak{S}_l}^{\mathfrak{S}_{m+l}}W\boxtimes U)
  \right\}
  $
  の指標を計算すると、(\ref{char_A})となることがわかる。
\end{proof}

$R$の係数を$\quotient$に拡大して考える。$[V],[W]\in R_n$に対して
\[
\generated{[V],[W]}=\generated{\chi_V,\chi_W}  
\]
と定義し、$\generated{,}$を$R$に双線形に拡張し内積を入れる。本節の主定理の一つが次の定理である。

\begin{theo}\label{rep_ring_and_func_ring}
  $\varphi:\Lambda\rightarrow R$を$\varphi(h_\lambda)=[M_\lambda]$によって定めると$\varphi$は内積を保つ環の同型
  であり、
  \[
  \varphi(s_\lambda)=[S_\lambda]
  \]
  が成り立つ。ただし$M_\lambda$は例\ref{ind_from_horizontal_perm}の誘導表現, $S_\lambda$は$\lambda$で定まる対称群の既約表現である。
\end{theo}

\begin{proof}
  $\{h_\lambda\}$は$\Lambda$の基底だから$\varphi$は$\quotient$線形写像として定義できることに注意する。また、Youngの規則(定理\ref{Young_rule_for_rep})より
  \[
  [M_\lambda]=[S_\lambda]+\sum_{\mu>\lambda}k_{\lambda\mu}[S_\mu]
  \]
  となるから、$R_n$の基底$\{[S_\lambda]\}$に関して$\{[M_\lambda]\}$を成分表示して並べたとき、対角成分がすべて1の上三角行列になる。すなわち$\{[M_\lambda]\}$は$R_n$の基底となることがわかる。したがって$\varphi$は少なくとも線形同型であることがわかる。

  $\varphi$が環準同型であることを示す。$h_\lambda=h_{\lambda_1}\cdots h_{\lambda_r}$だから、
  \[
  [M_\lambda]=[M_{(\lambda_1)}]\circ\cdots\circ[M_{(\lambda_r)}]  
  \]
  となることを示せばよい。$n=|\lambda|$とおく。各$M_{(\lambda_i)}$について、$(\lambda_i)$は1行のYoung図形だから$M_{(\lambda_i)}$は$\mathfrak{S}_{\lambda_i}$の自明な表現である。よって$M_{(\lambda_i)}\boxtimes M_{(\lambda_j)}$は$\mathfrak{S}_{\lambda_i}\times\mathfrak{S}_{\lambda_j}$の自明な表現だから、
  \[
  [M_{(\lambda_i)}]\circ[M_{(\lambda_j)}]
  =[\ind_{\mathfrak{S}_{\lambda_i}\times\mathfrak{S}_{\lambda_j}}^{\mathfrak{S}_{\lambda_i+\lambda_j}}\mathbf{1}]
  =[M_{(\lambda_i,\lambda_j)}]
  \]
  積は可換かつ結合的だから
  \[
  [M_\lambda]=[M_{(\lambda_1)}]\circ\cdots\circ[M_{(\lambda_r)}]  
  \]
  
  次に$\varphi$が内積を保つことを示す。$\varphi$は同型なので$\varphi$の逆写像$\psi$が内積を保つことを示せばよい。
  命題\ref{various_relation}と命題\ref{inner_prod}より、$\lambda\in\mathcal{P}_n$に対して$h_\lambda$を$p_\lambda$の線形結合で表すことができる:
  \[
  h_\lambda=\sum_{\mu\in\mathcal{P}_n}\frac{\chi_{M_\lambda}(g_\mu)}{z(\mu)}p_\mu  
  \]
  ここで、$z(\mu)=\prod m_i!\cdot i^{m_i}$, ($m_i$は$\mu$に現れる$i$の個数)について、$g_\mu$の共役類の大きさ$|C(g_\mu)|$を考えると
  \[
  |C(g_\mu)|=\frac{n!}{m_1!1^{m_1}\cdots m_n!n^{m_n}}=\frac{n!}{z(\mu)}
  \]
  である。これは、$n$の置換のうち、長さ$i$の巡回置換は$i$通りの同じ表示をもつ($(1,2,3)=(3,1,2)=(2,3,1)$)こと、$m_i$個の巡回置換の並べ替えの分重複があることを考えればわかる。よって
  \[
  z(\mu)=\frac{n!}{|C(g_\mu)|}
  \]
  これより、
  \begin{align*}
    \generated{\psi([M_{\lambda}]),\psi([M_\mu])}
    &=\generated{\sum_{\nu_1\in\mathcal{P}_k}\frac{\chi_{M_\lambda}(g_{\nu_1})}{z(\nu_1)}p_{\nu_1},\sum_{\nu_2\in\mathcal{P}_k}\frac{\chi_{M_\mu}(g_{\nu_2})}{z(\nu_2)}p_{\nu_2}}\\
    &=\sum_{\nu\in\mathcal{P}_k}\frac{\chi_{M_\lambda}(g_\nu)\chi_{M_\mu}(g_\nu)}{z(\nu)}\\
    &=\frac{1}{n!}\sum_{\nu\in\mathcal{P}_k}
    |C(g_\nu)|\chi_{M_\lambda}(g_\nu)\chi_{M_\mu}(g_\nu)\\
    &=\frac{1}{|\mathfrak{S}_n|}
    \sum_{\sigma\in\mathfrak{S_n}}\chi_{M_\lambda}(\sigma)\chi_{M_\mu}(\sigma)\\
    &=\generated{[M_\lambda],[M_\mu]}
  \end{align*}
  よって示せた。最後に$\varphi(s_\lambda)=[S_\lambda]$であることを示す。Youngの規則(系\ref{Young_rule_for_poly},命題\ref{Young_rule_for_rep})より
  \begin{align*}
  \varphi(s_\lambda)&=\varphi(h_\lambda)-\sum_{\nu>\lambda}k_{\lambda\nu}\varphi(s_\nu)\\
  &=[S_\lambda]+\sum_{\nu'>\lambda}k_{\lambda\nu'}[S_\nu']-\sum_{\nu>\lambda}k_{\lambda\nu}\varphi(s_\nu)\\
  &=[S_\lambda]+\sum_{\nu>\lambda}m_{\lambda\nu}[S_\nu]
  \end{align*}
  となるような整数$m_{\lambda\mu}$が存在する。
  \[
  1=\generated{\varphi(s_\lambda),\varphi(s_\lambda)}=1+\sum_{\nu>\lambda}m_{\lambda\nu}^2
  \]
  ゆえに$m_{\lambda\nu}=0$。よって示せた。
\end{proof}



この定理によって対称群の表現を対称関数の知識を使って調べることができる。

\begin{cor}
  $\mathfrak{S}_n$の既約表現$S_\lambda$, $\mathfrak{S}_m$の既約表現$S_\mu$について
  \[
  \ind_{\mathfrak{S}_n\times\mathfrak{S}_m}^{\mathfrak{S}_{n+m}}S_\lambda\boxtimes S_\mu =\bigoplus_{\nu}S_\nu^{\oplus\eta^{\nu}_{\lambda\mu}} 
  \]
  が成り立つ。ただし$\eta^{nu}_{\lambda\mu}$はLittlewood-Richardson数である。
\end{cor}

\begin{cor}
  $k_{\lambda\mu}$をKostka数とする。
  \[
  M_\lambda=S_\lambda\oplus \left(\bigoplus_{\mu>\lambda}S_\mu^{\oplus k_{\lambda\mu}}\right)  
  \]
  が成り立つ。
\end{cor}

\begin{proof}
  系\ref{Young_rule_for_poly}, 定理\ref{Young_rule_for_rep}よりただちに従う。
\end{proof}



\end{document}