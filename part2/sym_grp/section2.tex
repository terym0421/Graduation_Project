\documentclass{ltjsreport}
\RequirePackage{luatex85}
\usepackage[utf8]{inputenc}
\usepackage{enumerate}
\usepackage{amsthm}
\usepackage{amsfonts}
\usepackage{amsmath}
\usepackage{amssymb}
\usepackage{ytableau}
\usepackage{docmute}
\usepackage{mathtools}
\usepackage{xr}
\usepackage[all]{xy}



\theoremstyle{definition}
\newtheorem{defin}{定義}[subsection]
\newtheorem{theo}[defin]{定理}
\newtheorem{cor}[defin]{系}
\newtheorem{prop}[defin]{命題}
\newtheorem{lemm}[defin]{補題}
\newtheorem{notice}[defin]{注意}
\newtheorem{eg}[defin]{例}


\renewcommand{\labelenumi}{(\roman{enumi})}


\newcommand{\invlimit}{\mathop{\lim_{\longleftarrow}}}
\newcommand{\dirlimit}{\mathop{\lim_{\longrightarrow}}}
\newcommand{\ind}{\text{Ind}\:}
\newcommand{\Hom}{\text{Hom}}
\newcommand{\tr}{\text{tr}\:}
\newcommand{\id}[1]{\text{id}_{#1}}
\newcommand{\sgn}{\mathrm{sgn}}
\newcommand{\res}[1]{\text{Res}_{#1}}
\newcommand{\generated}[1]{\langle\:#1\:\rangle}
\newcommand{\im}{\text{Im }}
\newcommand{\rank}{\text{rank }}
\newcommand{\del}[2]{\frac{\partial #1}{\partial #2}}
\newcommand{\delsametwo}[2]{\frac{\partial^2 #1}{\partial #2^2}}
\newcommand{\delothertwo}[3]{\frac{\partial^2#1}{\partial#2\partial#3}}
\newcommand{\ddel}[2]{\frac{\partial}{\partial #2}#1}
\newcommand{\ddelsametwo}[3]{\frac{\partial^2}{\partial #2^2}#1}
\newcommand{\ddelothertwo}[3]{\frac{\partial^2}{\partial#2\partial#3}#1}
\newcommand{\simneq}{\not\simeq}
\newcommand{\transpose}[1]{^t\!#1}
\newcommand{\ie}{\text{i.e.}}
\newcommand{\inv}[1]{#1^{-1}}
\newcommand{\real}{\mathbb{R}}
\newcommand{\complex}{\mathbb{C}}
\newcommand{\integer}{\mathbb{Z}}
\newcommand{\quotient}{\mathbb{Q}}
\newcommand{\natnum}{\mathbb{N}}
\newcommand{\proj}{\mathbb{P}}
\newcommand{\tensor}[3]{#1\otimes_#2#3}
\newcommand{\map}[3]{#1:#2\rightarrow#3}
\newcommand{\aut}[2]{\mathrm{Aut}_{#1} (#2)}
\newcommand{\hommoph}[2]{\mathrm{Hom}_{#1}(#2)}
\newcommand{\gl}[1]{\mathrm{GL}(#1)}
\newcommand{\set}[2]{\left\{\:#1\:\middle|\:#2\:\right\}}
\newcommand{\pmat}[1]{\begin{pmatrix} #1
\end{pmatrix}}
\newcommand{\vmat}[1]{\begin{vmatrix} #1
\end{vmatrix}}
\newcommand{\br}{\vskip\baselineskip}


\begin{document}

\section{対称群の表現論}
\subsection{対称群の既約表現}

前節までに述べたことは有限群の表現論の一般論であり、具体的な群が与えられたときその表現を求める手法を提供しているわけではない。そこでこの節では対称群を例に取り上げ、既約表現の分類を行う。

$\mathcal{P}_n$を大きさ$n$のYoung図形のなす集合とする。既約表現の種類は共役類の数だけあったが、$G=\mathfrak{S}_n$の共役類と$\mathcal{P}_n$の元は1対1に対応することが知られている。$G$の既約表現は$\mathcal{P}_n$から自然に作ることができる。



\begin{defin}
  $\lambda\in\mathcal{P}_n$の各箱に$1$から$n$の各数字を重複なく書き入れた図を形$\lambda$のタブローという。
  $T$をタブローとし、$T$の$i$行目の箱に書かれている数字の集合を$H_i(T)$, 同様に$T$の$j$列目の箱に書かれている数字の集合を$V_j(T)$とする。
\end{defin}

\begin{defin}
  $T$を形$\lambda=(\lambda_1,\cdots,\lambda_s)$のタブローとする。$\sigma\in\mathfrak{S}_n$に対して、$\sigma T$を各数字を$\sigma$によって置換してできるタブローとする。
  \begin{itemize}
    \item 各$i$に対して$H_i(\sigma T)=H_i(T)$が成り立つなら$\sigma$を$T$の水平置換という。$T$の水平置換の全体は$G$の部分群をなす。これを$\mathcal{H}_T$と書き、$T$の水平置換群という。$\mathcal{H}_T=\mathfrak{S}(H_1(T))\times\cdots\times\mathfrak{S}(H_s(T))$である。
    \item 各$j$に対して$V_j(\sigma T)=V_j(T)$が成り立つなら$\sigma$を$T$の垂直置換という。$T$の垂直置換の全体は$G$の部分群をなす。これを$\mathcal{V}_T$と書き、$T$の垂直置換群という。$\mathcal{V}_T=\mathfrak{S}(V_1(T))\times\cdots\times\mathfrak{S}(V_{\lambda_1}(T))$である。
  \end{itemize}
\end{defin}

\begin{eg}\label{tableau_eg}
  形 \ydiagram{3,2} のタブロー$T=$ 
  \begin{ytableau}
    4&5&1\\
    3&2
  \end{ytableau} に対して、
  \begin{equation*}
    \mathcal{H}_T=\mathfrak{S}(\{1,4,5\})\times\mathfrak{S}(\{2,3\}),\qquad \mathcal{V}_T=\mathfrak{S}(\{3,4\})\times\mathfrak{S}(\{2,5\})
  \end{equation*}
  である。
\end{eg}

\begin{eg}\label{canonical_tableau}
  Young図形$\lambda=(\lambda_1,\cdots,\lambda_s)\in\mathcal{P}_n$に対して、$\lambda$の第1行に$1,2,\cdots,\lambda_1$を、$\lambda$の第2行に$\lambda_1+1,\lambda_1+2,\cdots,\lambda_1+\lambda_2$を、と続けてできるタブローを$\lambda$から定まる自然なタブローという。
  \[
  \text{例\ref{tableau_eg}のYoung図形の自然なタブローは }\begin{ytableau}
    1&2&3\\
    4&5
  \end{ytableau}  
  \]
\end{eg}

水平置換$\sigma$が垂直置換でもあるならば、$\sigma$の引き起こす各$H_i(T)$の置換は恒等置換でなければならない。したがって$\sigma=e$である。よって$\mathcal{H}_T\cap\mathcal{V}_T=\{e\}$が成り立つ。また$\mathcal{H}_{gT}=g\mathcal{H}_T\inv{g}$, $\mathcal{V}_{gT}=g\mathcal{V}_T\inv{g}$が成り立つ。実際
\begin{align*}
  \sigma\in\mathcal{H}_{gT}
  &\Leftrightarrow \sigma gT=gT\\
  &\Leftrightarrow \inv{g}\sigma gT=T\\
  &\Leftrightarrow \sigma\in g\mathcal{H}_T\inv{g} 
\end{align*}

群環$\complex[G]$の元$a_T,b_T,c_T$を
\[
a_T=\sum_{\sigma\in \mathcal{H}_T}\sigma,\qquad
b_T=\sum_{\tau\in \mathcal{V}_T}\sgn(\tau)\tau,\qquad c_T=a_Tb_T=\sum_{\sigma\in\mathcal{H}_T,\tau\in\mathcal{V}_T}\sgn(\tau)\sigma\tau
\]
によって定める。$c_T$をYoung対称子という。ここで$c_T$は$0$でないことに注意しておく。実際$c_T$の和に現れる$\sigma\tau$はすべて異なる元である。なぜならもし$\sigma\tau=\sigma'\tau'$, $\sigma,\sigma'\in\mathcal{H}_T$, $\tau,\tau'\in\mathcal{V}_T$ならば、$\mathcal{H}_T\cap\mathcal{V}_T=e$より$\sigma=\sigma'$, $\tau=\tau'$である。


\begin{theo}\label{sym_irr_rep}
  $\complex[G]$の左イデアル$\complex[G]c_T$は極小である。
\end{theo}

定理\ref{sym_irr_rep}を証明しよう。ポイントになるのは次の補題である。
\begin{lemm}\label{young_symmetrizer}
  $\alpha\in\complex[G]$が
  \begin{itemize}
    \item 任意の$\sigma\in\mathcal{H}_T$に対して$\sigma\alpha=\alpha$
    \item 任意の$\tau\in\mathcal{V}_T$に対して$\alpha\tau=\sgn(\tau)\alpha$
  \end{itemize}
  を満たすならば、$\alpha$は$c_T$のスカラー倍である。
\end{lemm}

\begin{proof}
  $\alpha=\sum_{g\in G}a_gg$を仮定を満たす元とする。仮定より$\sigma\in \mathcal{H}_T$に対して
  \[
  \alpha=\inv{\sigma}\alpha=\sum_{g\in G}a_g\inv{\sigma} g=\sum_{g\in G}a_{\sigma g}g
  \]
  よって
  \begin{equation}
    a_{\sigma g}=a_g  \label{a_sigmag}
  \end{equation}
  が成り立つ。また$\tau\in\mathcal{V}_T$に対しては
  \[
  \alpha=\sgn(\tau)\alpha\inv{\tau}=\sum_{g\in G}\sgn(\tau)a_gg\inv{\tau}=\sum_{g\in G}\sgn(\tau)a_{g\tau}{g}
  \]
  より
  \begin{equation}
    a_{g\tau}=\sgn(\tau)a_g  \label{a_gtau}
  \end{equation}
  が成り立つ。(\ref{a_sigmag}),(\ref{a_gtau})より$\sigma\tau\in\mathcal{H}_T\mathcal{V}_T$に対して
  \[
  a_{\sigma\tau}=\sgn(\tau)a_{e}  
  \]
  であることがわかる。よって
  \begin{equation}
  g\notin\mathcal{H}_T\mathcal{V}_T\implies a_g=0 \label{lemm:coefficient_zero} 
  \end{equation}
  を示せば$\alpha=a_ec_T$となって証明が完了する。$g$に関する条件$g\notin\mathcal{H}_T\mathcal{V}_T$について次の補題を示す。
  \begin{lemm}\label{lamm:117}
    $g\in\mathfrak{S}_n$について、$T$の同じ行にある任意の数字$i,j$(ただし$i\neq j$)が$gT$では異なる列にあるならば$g\in\mathcal{H}_T\mathcal{V}_T$が成り立つ。
  \end{lemm}

  \begin{proof}
    $T$のYoung図形$\lambda=(\lambda_1,\cdots,\lambda_r)$の高さ$r$に関する帰納法で示す。$r=1$ならば$\mathcal{H}_T=\mathfrak{S}_n$なので明らか。$r>1$とする。$T$の第1行にある数字に注目する。仮定から、これらは$gT$でそれぞれ異なる列に入っているので、適当に$gT$に垂直置換$\nu\in\mathcal{V}_{gT}$を施すことで$\nu gT$においても第1行に入っているようにできる。
    \[
    T=\quad\begin{ytableau}
      *(yellow)7&*(yellow)8&*(yellow)2\\
      3&5&4\\
      1&6
    \end{ytableau} \qquad \rightarrow \qquad
    gT=\quad\begin{ytableau}
      5&3&4\\
      *(yellow)2&1&*(yellow)8\\
      6&*(yellow)7 
    \end{ytableau} \qquad \rightarrow \qquad
    \nu gT=\quad\begin{ytableau}
      *(yellow)2&*(yellow)7&*(yellow)8\\
      5&1&4\\
      6&3
    \end{ytableau}
    \]
    すなわち
    \[
    H_1(T)=H_1(\nu gT)  
    \]
    が成り立つようにできる。このとき$\nu g$は$T$の第1行への水平置換$\sigma_1$と、
    $T$の第2行以下を取り出したタブロー$T'$への置換$g'$との積
    \[
    \nu g=\sigma_1g'
    \]
    で表される。$g'$は$T'$への置換とみなせば主張の条件をみたすから、帰納法の仮定により
    \[
    g'\in\mathcal{H}_{T'}\mathcal{V}_{T'}  
    \]
    である。$\mathcal{H}_{T'}\subset\mathcal{H}_T$, $\mathcal{V}_{T'}\subset\mathcal{V}_T$だから
    \[
    g' =\sigma_2\tau_2\in\mathcal{H}_T\mathcal{V}_T
    \]
    と書ける。ここで$\nu\in\mathcal{V}_{gT}=g\mathcal{V}_T\inv{g}$だから
    \[
    \nu=g\tau_3\inv{g},\qquad \tau_3\in\mathcal{V}_{T}  
    \]
    よって
    \[
    g=\sigma_1g'\inv{\tau_3}=\sigma_1\sigma_2\tau_2\inv{\tau_3}
    \]
    となるので示せた。
  \end{proof}

  補題\ref{young_symmetrizer}の証明に戻ろう。(\ref{lemm:coefficient_zero})を示せばよいのであった。$g\notin\mathcal{H}_T\mathcal{V}_T$であるのなら、上記の補題から$T$の同じ行になる異なる数字$i,j$であって$gT$では同じ列にあるものが存在する。よって$\sigma=(i, j)$とすれば$\sigma\in\mathcal{H}_T\cap\mathcal{V}_{gT}$である。
  $\mathcal{V}_{gT}=g\mathcal{V}_T\inv{g}$より$\sigma=g\tau\inv{g}$とおけば(\ref{a_sigmag}), (\ref{a_gtau})より
  \[
  a_{g}=a_{\sigma g}=a_{g\tau}=\sgn(\tau)a_g=-a_g 
  \]
  よって$a_g=0$
\end{proof}

\begin{prop}\label{square_of_young_sym}
  \[
  c_T^2=\frac{n!}{\dim_\complex(\complex[G]c_T)}c_T  
  \]
  が成り立つ。
\end{prop}

\begin{proof}
  $\sigma\in\mathcal{H}_T$, $\tau\in\mathcal{V}_T$に対して
  \[
  \sigma a_T=\sigma\sum_{g\in\mathcal{H}_T}g=\sum_{g\in\mathcal{H}_T}\sigma g=a_T  
  \]
  であり、
  \[
  b_T\tau=\sum_{g\in\mathcal{V}_T}\sgn(g)g\tau=\sgn{\tau}b_T  
  \]
  だから、補題\ref{young_symmetrizer}よりある$n_T\in\complex$で
  \[
  c_T^2=n_T c_T  
  \]
  となることはわかる。$n_T$を求めよう。準同型$\map{\phi}{\complex[G]}{\complex[G]}$を
  \[
  \phi(\alpha)=\alpha c_T  
  \]
  によって定める。任意の$g\in G$に対して、
  \[
  gc_T=g+\sum_{hk\in\mathcal{H}_T\mathcal{V}_T\setminus\{e\}}\sgn(k)ghk  
  \]
  となるから、$\phi$の対角成分はすべて$1$である。よって
  \[
  \tr\phi=\dim_\complex\complex[G]=n!  
  \]
  である。$\complex[G]$は半単純だから、
  \[
  \complex[G]=\complex[G]c_T\oplus W  
  \]
  となる左イデアル$W$をとる。すると
  \[
  \complex[G]c_T=\complex[G]c_T^2\oplus Wc_T=\complex[G]c_T\oplus Wc_T 
  \]
  より$Wc_T=0$である。したがって、
  \begin{align*}
    &\phi(\complex[G]c_T)\subset \complex[G]c_T\\
    &\phi(W)=0
  \end{align*}
  となることがわかる。よって
  \[
  \tr\phi=\tr\phi|_{\complex[G]c_T}  
  \]
  である。$\alpha\in\complex[G]$に対して
  \[
  \phi(\alpha c_T)=\alpha\phi(c_T)=n_T\alpha c_T
  \]
  だから、$\complex[G]c_T$は$\phi$の固有値$n_T$の固有空間の部分空間である。
  \[
  \tr\phi|_{\complex[G]c_T}=n_T\dim_\complex\complex[G]c_T 
  \]
  $c_T\neq 0$だから$\dim_\complex\complex[G]c_T\neq 0$,
  よって
  \[
  n_T=\frac{n!}{\dim_\complex\complex[G]c_T}  
  \]
\end{proof}

定理\ref{sym_irr_rep}の証明を述べる

\begin{proof}
  定理\ref{reverse_schur}より
  \[
  \dim_\complex\Hom(\complex[G]c_T,\complex[G]c_T)=1  
  \]
  を示せばよい。命題\ref{young_symmetrizer}より$c_T$は適当にスカラー倍してべき等元になる。よって命題\ref{hom_of_cyclic_module}より
  \[
  \Hom(\complex[G]c_T,\complex[G]c_T)=c_T\complex[G]c_T  
  \]
  である。任意の$c_T\alpha c_T\in c_T\complex[G]c_T$は補題\ref{young_symmetrizer}の仮定をみたすので
  \[
    c_T\alpha c_T=\mu c_T,\qquad\mu\in\complex
  \]
  と書ける。よって$\dim_\complex c_T\complex[G]c_T=1$である。
\end{proof}


Young対称子の定義において$a_T$, $b_T$の積の順序に本質的な違いはない。

\begin{prop}\label{reverse_young_sym}
  $b_T a_T=\tilde{c_T}$とおくと、$\complex[G]\tilde{c_T}\simeq \complex[G]c_T$が成り立つ。
\end{prop}

\begin{proof}
  $\map{\phi}{\complex[G]a_Tb_T}{\complex[G]b_Ta_T}$を
  \[
  \phi(xa_Tb_T)=xa_Tb_Ta_T  
  \]
  $\map{\psi}{\complex[G]b_Ta_T}{\complex[G]a_Tb_T}$を
  \[
  \psi(xb_Ta_T)=xb_Ta_Tb_T  
  \]
  とすれば
  \[
  \psi(\phi(xa_Tb_T))=\psi(xa_Tb_Ta_T)=xa_Tb_Ta_Tb_T=n_Txa_Tb_T  
  \]
  よって$\psi\circ\phi$は0でないスカラー倍写像なので$\phi$は単射、$\psi$は全射である。命題\ref{square_of_young_sym}とまったく同様に$\tilde{c_T}^2=\tilde{n_T}\tilde{c_T}$となる0でないスカラー$\tilde{n_T}$が存在することがわかる。よって$\phi$は同型である。
\end{proof}


\begin{prop}
  $\lambda\in\mathcal{P}_n$とする。$T,U$を$\lambda$に書かれたタブローとすると$\complex[G]c_T\simeq \complex[G]c_U$である。
\end{prop}

\begin{proof}
  このときある$g\in G$が存在して$U=gT$となるから、
  \[
  \mathcal{H}_U=g\mathcal{H}_T\inv{g},\qquad \mathcal{V}_U=g\mathcal{V}_T\inv{g}  
  \]
  よって
  \[
  c_U=a_Ub_U=ga_T\inv{g}gb_T\inv{g}=gc_T\inv{g}  
  \]
  である。
  \[
  \complex[G]c_U=\complex[G]gc_T\inv{g}=\complex[G]c_T\inv{g}  
  \]
  だから、
  \[
  \complex[G]c_T\simeq \complex[G]c_T\inv{g}  
  \]
  を示せばよい。$\map{\phi}{\complex[G]c_T}{\complex[G]c_T\inv{g}}$を
  \[
  \phi(\alpha c_T)=\alpha c_T\inv{g}  
  \]
  と置けば$\phi$は左$\complex[G]$加群の準同型で、$g$を右から書ける準同型が逆写像を与えるので、同型である。
\end{proof}


したがって、同じYoung図形に対しては$\complex[G]c_T$はタブロー$T$の取り方によらず同型である。そこで$\lambda\in\mathcal{P}_n$に対して、$\lambda$の自然なタブロー(例\ref{canonical_tableau})から定まるYoung対称子を$c_\lambda$とし、$S_\lambda=\complex[G]c_\lambda$とおく。





次の定理を証明することで、既約表現の分類は完成する。

\begin{theo}\label{young_and_irr_rep}
  $\lambda,\mu\in\mathcal{P}_n$とする。
  \[
  S_\lambda\simeq S_\mu 
  \]
  となるための必要十分条件は$\lambda=\mu$である
\end{theo}

\begin{proof}
  十分性は明らか。必要性を示す。$\lambda\neq\mu$であるとする。$S_\lambda, S_\mu$は既約表現なので、Schurの補題(補題\ref{schur_lem})より、
  \[
  \dim_\complex\Hom(S_\lambda,S_\mu)=0  
  \]
  を証明すればよいが、命題\ref{hom_of_cyclic_module}より、
  \[
  \Hom(S_\lambda,S_\mu)=c_\lambda\complex[G]c_\mu
  \]
  ゆえに、すべての$g\in G$に対して
  \[
  c_\lambda gc_\mu  =a_\lambda b_\lambda g a_\mu b_\mu = 0
  \]
  が成り立つことを示す。次の補題を示す。
  \begin{lemm}\label{lexicographical}
    $\mathcal{P}_n$に辞書式順序を入れ、$\lambda < \mu$であるとする。$\lambda, \mu$でその自然なタブローを表すものとする。このとき任意の$g\in G$に対して、$\mu$の同じ行にある数字$i,j$であって$g\lambda$でも同じ列にあるものが存在する。
  \end{lemm}

  \begin{proof}
    $\lambda=(\lambda_1,\cdots,\lambda_s)$, $\mu=(\mu_1,\cdots,\mu_t)$とおく。$t$についての帰納法で示す。

    $t=1$の場合$\lambda_1<\mu_1$となるから、$\lambda$の列数は$\mu_1$より少ない。よって鳩の巣原理を用いれば$1,2,\cdots,\mu_1$のうち、$g\lambda$の同じ列にあるペアが必ず存在することがわかる。

    $t>1$とする。$\lambda_1<\mu_1$である場合はまったく同様に鳩の巣原理から従う。$\lambda_1=\mu_1$かつ、$1,2,\cdots,\mu_1$が$g\lambda$ではすべて異なる列に存在するとする。このとき垂直置換$\tau\in\mathcal{V}_{g\lambda}$を施して
    \[
    H_1(\mu)=H_1(\tau g\lambda)=\{1,2,\cdots,\mu_1\}  
    \]
    が成り立つようにできる。そこで、$\mu$, $\tau g\lambda$の2行目以降をとりだしたタブロー$\mu'$, $(\tau g\lambda)'$を考える。すると$(\tau g\lambda)'<\mu'$であるから帰納法の仮定により$\mu'$の同じ行にある数字$i,j$であって$(\tau g\lambda)'$では同じ列にあるものが存在する。$i,j$が$(\tau g\lambda)'$の第$m$列にあるとする。$\tau$は垂直置換だから
    \[
    V_m(\tau g\lambda)=V_m(g\lambda)
    \]
    よって$i,j$は$g\lambda$の同じ列に存在する。
  \end{proof}

  定理\ref{young_and_irr_rep}の証明に戻る。補題から、$\nu=(i,j)$であって$\nu\in \mathcal{H}_\mu\cap\mathcal{V}_{\inv{g}\lambda}$となるものが存在する。よって
  \[
  \nu=\inv{g}\pi g,\qquad \pi\in \mathcal{V}_{\lambda}
  \]
  とおけば
  \begin{align*}
    c_\lambda g c_\mu
    &=a_\lambda b_\lambda g a_\mu b_\mu\\
    &=a_\lambda b_\lambda \sgn(\pi)\pi g a_\mu b_\mu\\
    &=a_\lambda b_\lambda \sgn(\pi)g\nu a_\mu b_\mu\\
    &=\sgn(\pi)a_\lambda b_\lambda g a_\mu b_\mu\\
    &=-c_\lambda g c_\mu
  \end{align*}
  よって
  \[
    c_\lambda g c_\mu=0
  \]
\end{proof}


\begin{eg}
  $\lambda=(n),\mu=(1,1,\cdots,1)\in\mathcal{P}_n$とする。このとき$\mathcal{H}_\lambda=\mathfrak{S}_n$, $\mathcal{V}_\lambda=e$だから、
  \[
  c_\lambda=\sum_{\sigma\in\mathfrak{S}_n}\sigma  
  \]
  また$\mathcal{H}_\mu=e$, $\mathcal{V}_\mu=\mathfrak{S}_n$だから、
  \[
  c_\mu=\sum_{\sigma\in\mathfrak{S}_n}\sgn(\sigma)\sigma  
  \]
  したがって$\lambda$の定める既約表現は自明な表現$1$であり、$\mu$の定める既約表現は置換の符号$\sgn$であるとわかる。
\end{eg}



\begin{eg}
  $G=\mathfrak{S}_3$とする。
  \[
  \lambda=\quad\ydiagram{2,1}  
  \]
  に対応するYoung対称子は
  \[
  c_\lambda=(e+(1,2))(e-(1,3))=e+(1,2)-(1,3)-(1,3,2)
  \]
  である。$c_\lambda$の定める既約表現が、例\ref{S3}で求めた既約表現$U$と一致することをたしかめる。
  \[
  c_\lambda^2=3c_\lambda  
  \]
  となるから、命題\ref{square_of_young_sym}より
  \[
  \dim_\complex\complex[G]c_\lambda=2  
  \]
  である。
  \[
  v=c_\lambda, \qquad u=(1,2,3)c_\lambda=-e+(1,3)-(2,3)+(1,2,3)
  \]
  とすれば、$\complex[G]c_\lambda=\complex v\oplus\complex v$であり、
  \begin{align*}
    &(1,2)v=v,\qquad (1,2)u=-v-u\\
    &(1,2,3)v=u,\qquad (1,2,3)u=-v-u
  \end{align*}
  だから、
  \begin{align*}
  &\tr e=\dim_\complex\complex[G]c_\lambda=2  \\
  &\tr(1,2)=0\\
  &\tr(1,2,3)=-1 
  \end{align*}
  となり、指標が一致している。
\end{eg}



\begin{lemm}\label{tensor_with_sgn_rep}
  $\map{\phi}{\complex[G]}{\complex[G]}$を
  \[
  \phi(g)=\sgn(g)g  
  \]
  を線形に拡張して定める。$\phi$は環準同型であり、対合である。$\varepsilon\in\complex[G]$に対して
  \[
  \complex[G]\varepsilon\otimes_\complex \complex_{\sgn}\simeq \complex[G]\phi(\varepsilon)
  \]
  が成り立つ。ここで、$\complex_{\sgn}$は$\complex_{\sgn}=\complex$であり、
  \[
  g\cdot\lambda=\sgn(g)\lambda,\qquad g\in G,\lambda\in\complex
  \]
  で定まる$\complex[G]$加群である(すなわち$\sgn$表現)。
\end{lemm}

\begin{proof}
  $\map{f}{\complex[G]\phi(\varepsilon)}{\complex[G]\varepsilon\otimes_\complex \complex_{\sgn}}$を
  \[
  f(x)=\phi(x)\otimes 1  
  \]
  で定めれば$g\in G$として
  \begin{align*}
    gf(x)
    &=g(\phi(x)\otimes 1)\\
    &=g\phi(x)\otimes \sgn(g)\\
    &=\sgn(g)g\phi(x)\otimes 1\\
    &=\phi(gx)\otimes 1\\
    &=f(gx)
  \end{align*}
  より$\complex[G]$加群の準同型である。任意の$y\otimes 1\in \complex[G]\varepsilon\otimes_\complex \complex_{\sgn}$に対して
  \[
  f(\phi(y))=y\otimes 1  
  \]
  となり、$\complex[G]\varepsilon\otimes_\complex \complex_{\sgn}$は$y\otimes 1$の形の元で生成されるから、$f$は全射である。
  \[
  \dim_\complex\complex[G]\varepsilon\otimes_\complex \complex_{\sgn}=\dim_\complex\complex[G]\varepsilon\cdot\dim_\complex\complex_{\sgn}=  \dim_\complex\complex[G]\varepsilon 
  \]
  より$f$は同型。
\end{proof}


\begin{eg}
  $\lambda\in\mathcal{P}_n$に対して、$\lambda$の行と列を反転させたものを双対Young図形といい$\lambda^*$と書く。
  \[
  \lambda=\quad\ydiagram{5,2,2}\quad\rightarrow\quad\lambda^*\quad=\ydiagram{3,3,1,1,1}
  \]
  このとき
  \[
  \complex[G]_{c_\lambda}\otimes_\complex\complex_{\sgn}\simeq \complex[G]_{c_{\lambda^*}}
  \]
  となることを示す。定義より、
  \[
  \mathcal{H}_{\lambda^*}=\mathcal{V}_{\lambda},\qquad
  \mathcal{V}_{\lambda^*}=\mathcal{H}_{\lambda}  
  \]
  であるから、
  \[
  a_{\lambda^*}=\phi(b_{\lambda}),\qquad b_{\lambda^*}=\phi(a_{\lambda})
  \]
  したがって
  \[
  c_{\lambda^*}=\phi(b_\lambda)\phi(a_\lambda)=\phi(b_\lambda a_\lambda)=\phi(\tilde{c_\lambda})  
  \]
  である。命題\ref{reverse_young_sym}より、
  \[
  \complex[G]c_\lambda\simeq \complex[G]\tilde{c_\lambda}  
  \]
  であるから、補題\ref{tensor_with_sgn_rep}より、
  \[
  \complex[G]c_{\lambda^*}\simeq \complex[G]c_\lambda\otimes_\complex\complex_{\sgn}  
  \]
\end{eg}

\begin{eg}\label{ind_from_horizontal_perm}
  $\lambda\in\mathcal{P}_n$に対して
  \[
  M_\lambda=\ind_{\mathcal{H}_\lambda}^{G}(\mathbf{1})  
  \]
  とする。ここで$\mathbf{1}$は$\mathcal{H}_\lambda$の自明な表現である。\ref{ind_from_trivial}より、$M_\lambda$は$G/\mathcal{H}_\lambda$の置換表現に他ならない。
  $M_\lambda=\complex[G]a_\lambda$となることを示す。
  $\phi:M_\lambda=\complex[G]\otimes_{\complex[H_\lambda]} \mathbf{1}\rightarrow \complex[G]a_\lambda$を
  \[
  \phi(g\otimes c)=cga_\lambda  
  \]
  $\complex[\mathcal{H}_\lambda]$双線形に拡張して定める。$h\in\mathcal{H}_\lambda$に対して
  \[
  \phi(gh\otimes c)=cgha_\lambda=cga_\lambda=\phi(g\otimes hc)
  \]
  だから$\phi$はwell-definedであり$\complex[G]$準同型である。逆に$\psi:\complex[G]a_\lambda\rightarrow\complex[G]\otimes_{\complex[H_\lambda]} \mathbf{1}$を
  \[
  \psi(ga_\lambda)=\frac{1}{|\mathcal{H}_\lambda|}ga_\lambda\otimes 1=  g\otimes 1
  \]
  によって定めれば$\phi,\psi$は互いに逆写像であるから、同型である。
  
  $\lambda$が1行のYoung図形の場合、$\mathcal{H}_\lambda=G$であるから、$M_\lambda$は自明な表現にほかならない。一方$\lambda$が1列のYoung図形の場合は$\mathfrak{H}_\lambda=1$であるから$M_\lambda$は正則表現である。
  
  この表現が後に重要になるので、$M_\lambda$に関する性質を一つ示しておく。
\end{eg}

\begin{theo}[Youngの規則]\label{young_rule}
  $\lambda\in\mathcal{P}_n$と、辞書式順序で$\lambda$より大きい$\mu\in\mathcal{P}_n$に対して正の整数$k_{\lambda\mu}$が存在して
  \[
  M_\lambda=S_\lambda\oplus \left(\bigoplus_{\mu>\lambda}S_\mu^{\oplus k_{\lambda\mu}}\right)
  \]
  $k_{\lambda\mu}$をKostka数という。
\end{theo}

\begin{proof}
  定理\ref{char_orthogonality}より、
  \[
  \dim_{\complex}\Hom(M_\lambda,S_\mu)=
  \left\{\begin{array}{cc}
    1 & \text{if $\mu=\lambda$}\\
    0 & \text{if $\mu<\lambda$}
  \end{array}\right.  
  \]
  を証明すればよい。例\ref{ind_from_horizontal_perm}より$M_\lambda=\complex[G]a_\lambda$であるから、\ref{hom_of_cyclic_module}より
  \[
  \Hom(M_\lambda,S_\mu)=\Hom(\complex[G]a_\lambda,\complex[G]c_\mu)=a_\lambda\complex[G]c_\mu 
  \]
  である。$\lambda=\mu$の場合、任意の$\alpha\in\complex[G]$に対して$a_\lambda\alpha c_\lambda$ は補題\ref{young_symmetrizer}の条件をみたすから、
  \[
  a_\lambda\complex[G]c_\lambda=\complex c_\lambda
  \]
  よって
  \[
  \dim_\complex\Hom(M_\lambda,S_\mu)=1  
  \]
  次に$\mu<\lambda$のとき、補題\ref{lexicographical}より、任意の$g\in G$に対して$\lambda$の同じ行にある文字$i,j$であって$g\mu$で同じ列にあるものが存在する。すなわち$\sigma=(i,j)$とおけば$\sigma\in \mathcal{H}_\lambda$かつ$\sigma\in\mathcal{V}_{g\mu}=g\mathcal{V}_{\mu}\inv{g}$である。そこで$\sigma=g\tau\inv{g}$, $\tau\in\mathcal{V}_\mu$とおけば、
  \begin{align*}
    a_\lambda g b_\mu
    &=a_\lambda\sigma g b_\mu\\
    &=a_\lambda g\tau b_\mu\\
    &=-a_\lambda g b_\mu
  \end{align*}
  ゆえに$a_\lambda\complex[G]b_\mu=0$であるから、
  \[
  a_\lambda\complex[G]c_\lambda\subset a_\lambda\complex[G]b_\mu=0
  \]
  よって示せた。
\end{proof}

Kostka数はSchur多項式を単項対称式の線形結合で表すときの係数にも現れる。すなわち
\[
s_\lambda=\sum_{\mu}k_{\lambda\mu}m_\mu  
\]
となることが知られている。また、その具体的な値については、$k_{\lambda\mu}$は形$\lambda$のウェイト$\mu$の半標準タブローの数に等しいことも知られている。

また、対称群の既約指標については、その値はすべて有理数であることが知られている(Frobeniusの指標公式)。







\subsection{続・対称多項式}
対称群の表現と対称多項式の間には深い関係がある。次節でそのことを解説するが、そのための準備として対称多項式に関してより詳しく解説する。以下正の整数$n$を固定し、$\Lambda^k_n$を$n$変数の$k$次斉次対称多項式のなす$\integer$加群とする。第1部の記号を復習すると、$n$行のYoung図形$\lambda$に対して
\[
m_\lambda=\sum_{\alpha\sim\lambda}x_1^{\alpha_1}\cdots x_n^{\alpha_n}  
\]
とし、に対して
\begin{align*}
  &e_k=m_{1^k}=\sum_{1\leq i_1<\cdots<i_k\leq n}x_{i_1}\cdots x_{i_k},\qquad(k=1,\cdots,n)\\
  &h_k=\sum_{\lambda\in\mathcal{P}_k}m_\lambda=\sum_{1\leq i_1\leq\cdots\leq i_k\leq n}x_{i_1}\cdots x_{i_k}\\
  &p_k=m_{(k)}=x_1^k+\cdots+x_n^k
\end{align*}
とするのであった。$e_k$や$h_k$に対しては、その母関数を考えることは有用である。すなわち
\begin{align}
  &E(t)=1+e_1t+e_2t^2+\cdots+e_nt^n=\prod_{i=1}^n(1+x_it) \label{gen_func_of_e}\\
  &H(t)=1-h_1t+h_2t^2+\cdots+(-1)^nh_nt^n+\cdots=\prod_{i=1}^n\frac{1}{1+x_it}
\end{align}
である。とくに$E(t)H(t)=1$であるので、$k=1,\cdots,n$のとき
\begin{equation}\label{e_to_h}
  e_k-h_1e_{k-1}+\cdots+(-1)^{k-1}e_1h_{k-1}+(-1)^kh_k=0   
\end{equation}
を得る。

\begin{prop}\label{det_formula}
  $k=1,\cdots,n$に対して
  \begin{align*}
    h_k=\vmat{
      e_1&e_2&e_3&\cdots&e_k\\
      1&e_1&e_2&\cdots&e_{k-1}\\
      0&1&e_1&\cdots&e_{k-2}\\
      \vdots&\vdots&\vdots&\ddots&\vdots&\\
      0&0&0&\cdots&e_1},\qquad
    e_k=\vmat{
      h_1&h_2&h_3&\cdots&h_k\\
      1&h_1&h_2&\cdots&h_{k-1}\\
      0&1&h_1&\cdots&h_{k-2}\\
      \vdots&\vdots&\vdots&\ddots&\vdots&\\
      0&0&0&\cdots&h_1
    }
  \end{align*}
\end{prop}

\begin{proof}
  まったく同様なので$h_k$の場合だけ示す。$e_1=h_1$であり、$k-1$までこの公式が成り立っていたとすると、
  \begin{align*}
    e_k&=h_1e_{k-1}-h_2e_{k-2}+\cdots+(-1)^{k-1}h_k \\
    &=h_1\vmat{
      h_1&h_2&\cdots&h_{k-1}\\
      1&h_1&\cdots&h_{k-2}\\
      \vdots&\vdots&\ddots&\vdots&\\
      0&0&\cdots&h_1
    }-h_2\vmat{
      h_1&\cdots&h_{k-1}\\
      \vdots&\ddots&\vdots&\\
      0&\cdots&h_1
    }+\cdots+(-1)^{k-1}h_k\\
    &=\vmat{
      h_1&h_2&h_3&\cdots&h_k\\
      1&h_1&h_2&\cdots&h_{k-1}\\
      0&1&h_1&\cdots&h_{k-2}\\
      \vdots&\vdots&\vdots&\ddots&\vdots&\\
      0&0&0&\cdots&h_1
    }
  \end{align*}
\end{proof}

次に、$\lambda\in\mathcal{P}_k$に対して
\begin{align*}
  &e_\lambda=e_{\lambda_1}\cdots e_{\lambda_n}\\
  &h_\lambda=h_{\lambda_1}\cdots h_{\lambda_n}\\
  &p_\lambda=p_{\lambda_1}\cdots p_{\lambda_n}
\end{align*}
とする。

\begin{prop}\label{various_basis}
  $\Lambda^k_n$の次の部分集合について
  \begin{enumerate}[(i)]
    \item $\{m_\lambda\}_\lambda$ ただし$\lambda$は大きさが$k$で$n$行
    \item $\{e_\lambda\}_\lambda$ ただし$\lambda$は大きさが$k$で$n$列
    \item $\{h_\lambda\}_\lambda$ ただし$\lambda$は大きさが$k$で$n$列
    \item $\{s_\lambda\}_\lambda$ ただし$\lambda$は大きさが$k$で$n$行
    \item $\{p_\lambda\}_\lambda$ ただし$\lambda$は大きさが$k$で$n$列
  \end{enumerate}
  (i)$\sim$(iv)は$\Lambda^k_n$の$\integer$上の基底をなし、(v)は$\quotient\otimes_{\integer}\Lambda^k_n$の基底をなす。
\end{prop}

\begin{proof}
  (i)は命題\ref{m_is_basis}の証明を斉次部分で考えればまったく同様である。(ii)については定理\ref{FT_of_sym}の証明において、任意の対称多項式$f$が
  \[
  e_1^{a_1}\cdots e_n^{a_n}
  \]
  で生成されていることを示したことからわかる。(iv)は\ref{schur}の証明を斉次部分で行えばよい。(iii)については、命題\ref{det_formula}より$\{h_\lambda\}$が$\{e_\lambda\}$を生成することがわかるが、ともに集合の濃度が等しいことから基底をなすことがわかる。(v)が基底をなすことを示そう。(v)が(iii)を生成することを示せばよい。
  \begin{align*}
    &\quad1+h_1t+h_2t^2+\cdots\\
    &=H(-t)\\
    &=\prod_{i=1}^n\frac{1}{1-x_it}\\
    &=\exp \left(\sum_{i=1}^n-\log(1-x_it)\right)\\
    &=\exp \left(\sum_{i=1}^n\sum_{r=1}^\infty\frac{x_i^rt^r}{r}\right)\\
    &=\exp \left(\sum_{r=1}^\infty\frac{p_r}{r}t^r\right)\\
    &=\prod_{r=1}^\infty\exp \left(\frac{p_r}{r}t^r\right)\\
    &=\prod_{r=1}^\infty \sum_{m_r=0}^\infty \frac{p_r^{m_r}}{m_r!\cdot r^{m_r}}t^{r\cdot m_r}\\
    &=
    \left(\sum_{m_1=0}^\infty \frac{p_1^{m_1}}{m_1!\cdot 1^{m_1}}t^{m_1}\right)\cdot
    \left(\sum_{m_2=0}^\infty \frac{p_2^{m_2}}{m_2!\cdot 2^{m_2}}t^{2 \cdot m_2}\right)\cdot
    \left(\sum_{m_3=0}^\infty \frac{p_3^{m_3}}{m_3!\cdot 3^{m_3}}t^{3 \cdot m_3}\right)\cdots
  \end{align*}
  となるから、Young図形$\lambda=(\lambda_1,\lambda_2,\cdots)$に対して
  \[
  z(\lambda)=\prod_{i}\lambda_i!\cdot i^{\lambda_i}  
  \]
  とおけば、最後の式は
  \begin{align*}
    &\quad\sum_{m_1,m_2,m_3,\cdots}\frac{p_1^{m_1}p_2^{m_2}p_3^{m_3}\cdots}{(m_1!\cdot 1^{m_1})(m_2!\cdot 2^{m_2})(m_3!\cdot 3^{m_3})\cdots}t^{m_1+2\cdot m_2+3\cdot m_3+\cdots}\\
    &=\sum_{\lambda}\frac{p_\lambda}{z(\lambda)}t^{|\lambda|}\\
    &=\sum_{k=0}^\infty \sum_{\lambda\in\mathcal{P}_{k}}\frac{p_\lambda}{z(\lambda)}t^k
  \end{align*}
  となる。よって
  \[
  h_k=\sum_{\lambda\in\mathcal{P}_{k}}\frac{p_\lambda}{z(\lambda)}
  \]
  が成り立つ。$n$列のYoung図形$\lambda$に対して
  \[
  h_\lambda=h_{\lambda_1}\cdots h_{\lambda_s},\qquad \lambda_i\leq n 
  \]
  とおくと、
  \[
  h_{\lambda_i}=\sum_{\mu_i\in\mathcal{P}_{\lambda_i}}\frac{p_{\mu_i}}{z(\mu_i)}  
  \]
  であるから
  \[
  h_\lambda=\sum_{\mu_1,\cdots,\mu_s}\frac{p_{\mu_1}\cdots p_{\mu_s}}{z(\mu_1)\cdots z(\mu_s)}  
  \]
  各$\mu_i$は$\lambda_i\leq n$の分割を与えているから、$p_{\mu_1}\cdots p_{\mu_s}$はたかだか$n$列のYoung図形に対応するべき和対称式である。よって(v)も基底を与える。
\end{proof}

証明中に現れた等式は重要なので再掲しておく。

\begin{prop}\label{p_to_h}
  正の整数$k,n$に対して
  \[
    h_k(x_1,\cdots,x_n)=\sum_{\lambda\in\mathcal{P}_{k}}\frac{p_\lambda(x_1,\cdots,x_n)}{z(\lambda)}
  \]
  が成り立つ。
\end{prop}

後に必要になる公式を用意しておく


\begin{lemm}[Cauchyの等式]\label{formal_power_series_relation}
  形式的べき級数$\prod_{i=1}^m\prod_{j=1}^n\frac{1}{1-x_iy_j}$は次と等しい。
  \begin{align*}
    &\text{(i):}\:\sum_{\lambda,\lambda_{n+1}=0}h_\lambda(x_1,\cdots,x_m)m_\lambda(y_1,\cdots,y_n),\qquad \text{ただし和は$n$行Young図形全体をわたる}\\
    &\text{(ii):}\:\sum_{\lambda}\frac{1}{z(\lambda)}p_\lambda(x_1,\cdots,x_m)p_\lambda(y_1,\cdots,y_n),\qquad \text{ただし和はすべてのYoung図形全体をわたる}\\
    &\text{(iii):}\:\sum_{\lambda}s_\lambda(x_1,\cdots,x_m)s_\lambda(y_1,\cdots,y_n),\qquad \text{ただし和はすべてのYoung図形全体をわたる}
  \end{align*}
\end{lemm}

\begin{proof}
  (i)を示す。
  \begin{align*}
    &\quad\prod_{i=1}^m\prod_{j=1}^n\frac{1}{1-x_iy_j}\\
    &=\prod_{j=1}^n H(y_j)\\
    &=\prod_{j=1}^n \left(
      \sum_{k_j=0}^\infty h_{k_j}(x_1,\cdots,x_m)y_j^{k_j}
      \right)\\
    &=\left(
      \sum_{k_1=0}^\infty h_{k_1}(x_1,\cdots,x_m)y_1^{k_1}
      \right)\cdot
      \left(
      \sum_{k_2=0}^\infty h_{k_2}(x_1,\cdots,x_m)y_2^{k_2}
      \right)
      \cdots
      \left(
      \sum_{k_n=0}^\infty h_{k_n}(x_1,\cdots,x_m)y_n^{k_n}
      \right)\\
    &=\sum_{k_1,\cdots,k_n}h_{k_1}\cdots h_{k_n}y_1^{k_1}  
      \cdots y_n^{k_n}\\
    &=\sum_{\lambda, \lambda_{n+1}=0}h_\lambda(x_1,\cdots,
      x_m)m_\lambda(y_1,\cdots,y_n)
  \end{align*}
  (ii)を示す。命題\ref{p_to_h}より
  \begin{align*}
  \prod_{i,j}\frac{1}{1-x_iy_j}
  &=1+h_1(\{x_iy_j\})+h_2(\{x_iy_j\})+\cdots\\
  &=1+\sum_{\lambda\in\mathcal{P}_1}\frac{1}{z(\lambda)}p_\lambda(\{x_iy_j\})+\sum_{\lambda\in\mathcal{P}_2}\frac{1}{z(\lambda)}p_\lambda(\{x_iy_j\})+\cdots\\
  &=\sum_{\lambda}\frac{1}{z(\lambda)}p_\lambda(\{x_iy_j\})\\
  &=\sum_{\lambda}\frac{1}{z(\lambda)}p_\lambda(x_1,\cdots,x_m)p_\lambda(y_1,\cdots,y_n)
  \end{align*}
  (iii)については、Robinson-Schensted-Knuth対応と呼ばれる対応を用いて証明される。詳細は付録を参照
\end{proof}











\subsection{表現環と対称関数環}

\begin{defin}
  可算無限個の変数をもつ形式的べき級数環$\integer[[x_1,x_2,\cdots]]$を考える。
  \[
  \mathfrak{S}=\set{\map{\sigma}{\natnum}{\natnum}}{\text{$f$は全単射で$f(n)\neq n$なる$n$が有限個}}  
  \]
  とする
  \footnote{
    $\mathfrak{S}$は対称群$\mathfrak{S}_n$と自然な包含$\iota:\mathfrak{S}_n\rightarrow\mathfrak{S}_{n+1}$のなす帰納系の帰納極限である。
  }。
  \[
  \Lambda=\set{f\in \integer[[x_1,x_2,\cdots]]}{\sigma f=f,\:(\text{for all }\sigma\in\mathfrak{S}),\:\text{$f$の単項式の次数は有界}}  
  \]
  $\Lambda$は$\integer[[x_1,x_2,\cdots]]$の部分環で対称関数環と呼ばれる。$\Lambda^k$を
  \[
  \Lambda^k=\set{f\in\Lambda}{\text{$f$の単項式の次数はすべて$k$}}  
  \]
  で定め、$\Lambda^k$の元を$k$次斉次対称関数という。
  \[
  \Lambda=\bigoplus_{k=0}^\infty\Lambda^k  
  \]
  より$\Lambda$は次数付き環の構造をもつ。
\end{defin}

ここで、$\Lambda$の定義において単項式の次数が有界であることを要請するのは自然である。実際、もし仮定しなければ$\Lambda$は$\Lambda^k$の直和にはならない。

\begin{eg}[単項対称関数]\label{monomial_sym_func}
  任意のYoung図形$\lambda=(\lambda_1,\cdots,\lambda_n)$に対して
  \[
  m_{\lambda}=\sum_{\alpha\sim\lambda}x_1^{\alpha_1}x_2^{\alpha_2}\cdots x_{n}^{\alpha_n}
  \]
  とする。ここで指数$\alpha$は、$\lambda$の置換になっているもの全体をわたる。すなわちある$\sigma\in\mathfrak{S}$が存在して$\alpha=\sigma\lambda$をみたすもの全体である。$m_\lambda$は対称関数である。対称多項式の場合と同様の議論で、$\Lambda^k$は$\{m_\lambda\}_{\lambda\in\mathcal{P}_k}$を基底に持つことがわかる。
\end{eg}

\begin{eg}[基本対称関数・完全対称関数]\label{elementary_func}
  \begin{align*}
  &e_k=m_{1^k}
  =\sum_{1\leq i_1<i_2<\cdots<i_k}x_{i_1}x_{i_2}\cdots x_{i_k}\\
  &h_k=\sum_{\lambda\in\mathcal{P}_k} m_\lambda
  =\sum_{1\leq i_1\leq i_2\leq \cdots\leq i_k}x_{i_1}x_{i_2}\cdots x_{i_k}
  \end{align*}
  をそれぞれ、基本対称関数, 完全対称関数という。また、任意のYoung図形$\lambda=(\lambda_1,\cdots,\lambda_n)$に対して
  \begin{align*}
    &e_\lambda=e_{\lambda_1}\cdots e_{\lambda_n}\\
    &h_\lambda=h_{\lambda_1}\cdots h_{\lambda_n}
  \end{align*}
  とする。
  \begin{align*}
    &e_1=x_1+x_2+x_3+\cdots\\
    &e_2=\sum_{i<j}x_ix_j
  \end{align*}
  である。
\end{eg}

\begin{eg}[べき和対称関数]\label{power_func}
  $(k)=(k,0,\cdots,0)$に対して
  \[
  p_{k}=m_{(k)}=x_1^k+x_2^k+\cdots
  \]
  とする。
\end{eg}

このように、対称関数はいままでみてきた対称多項式を自然に無限変数に拡張した概念であり、対称多項式で成り立っていた関係式が対称関数においても成立することが多い。このことは対称関数の$k$次斉次部分$\Lambda^k$が$k$次斉次対称多項式からの射影極限と考えることができることによる。$\Lambda_n^k$を$n$変数$k$次斉次対称多項式のなす$\integer$加群とする。$m\leq n$に対して線形写像$\map{\rho_{m,n}}{\Lambda^k_n}{\Lambda^k_m}$を
\[
\rho_{m,n}(f(x_1,\cdots,x_m,x_{m+1},\cdots,x_n))=f(x_1,\cdots,x_m,0,\cdots,0)  
\]
によって定める。ここで$\rho_{m,n}(f)$は実際に$m$変数の$k$次斉次対称多項式である\footnote{変数の置換と$0$を代入する操作が可換であることによる}。$l\leq m\leq n$に対して
\[
\rho_{l,m}\circ\rho_{m,n}=\rho_{l,n}  
\]
が成り立つから、$\{\Lambda^k_n,\rho_{m,n}\}$は射影系をなす。

\begin{prop}\label{sym_func_is_inverselimit}
  上の状況において、
  \[
    \Lambda^k=\mathop{\lim_{\longleftarrow}}\Lambda^k_n
  \]
  がなりたつ。
\end{prop}

\begin{proof}
  $\theta_n:\Lambda^k\rightarrow\Lambda^k_n$を$n+1$番目以降の変数を$0$にする写像とすれば、
  \[
  \rho_{m,n}\circ\theta_n=\theta_m  
  \]
  が成り立つから、射影極限の普遍性から線形写像
  \[
  \theta:\Lambda^k\rightarrow\mathop{\lim_{\longleftarrow}}\Lambda^k_n
  \]
  が誘導される。$\mathop{\lim_{\longleftarrow}}\Lambda^k_n$から$\Lambda^k$への写像$\varphi$は次のように定義する。$\mathop{\lim_{\longleftarrow}}\Lambda^k_n$の元$(f_n)_{n\in\integer_{>0}}$, ($f_n\in\Lambda^k_n$)に対して、$k$変数の部分に注目すると
  \[
  f_k=\sum_{\lambda\in\mathcal{P}_k}c_\lambda m_\lambda(x_1,\cdots,x_k)
  \]
  と一意的に表せるので
  \[
  \varphi((f_n)_{n\in\integer_{>0}})
  =\sum_{\lambda\in\mathcal{P}_k}c_\lambda m_\lambda
  \]
  と定める。ただし右辺の$m_\lambda$は例\ref{monomial_sym_func}の対称関数である。$\varphi$が$\theta$の逆写像であることを示そう。
  \begin{equation}\label{proj_of_monomial}
  \theta_n(m_\lambda)=
  \left\{\begin{array}{cc}
    m_\lambda(x_1,\cdots,x_n) & \text{if }\lambda_{n+1}=0\\
    0 & \text{otherwise}
  \end{array}\right.  
\end{equation}
  であるが、$\lambda$は$k$の分割であるので$n\geq k$において$\lambda_{n+1}=0$である。よって$n\geq k$ならば
  \begin{equation}\label{theta_n}
  \theta_n(\varphi((f_n)_{n\in\integer_{>0}}))
  =\sum_{\lambda\in\mathcal{P}_k}c_\lambda m_\lambda(x_1,\cdots,x_n)  
  \end{equation}
  が成り立つ。一方、
  \[
  f_n=\sum_{\lambda\in\mathcal{P}_k(n)}d_\lambda m_\lambda(x_1,\cdots,x_n)    
  \]
  とおくと$n\geq k$より$\mathcal{P}_k(n)=\mathcal{P}_k$だから
  \[
  f_n=\sum_{\lambda\in\mathcal{P}_k}d_\lambda m_\lambda(x_1,\cdots,x_n)    
  \]
  となる。よって$\rho_{k,n}(f_n)=f_k$と(\ref{theta_n})より
  \[
  f_n=\theta_n(\varphi((f_n)_{n\in\integer_{>0}}))
  \]
  次に$n<k$の場合、(\ref{proj_of_monomial})より
  \[
  \theta_n(\varphi((f_n)_{n\in\integer_{>0}}))  
  =\sum_{\lambda\in\mathcal{P}(k)}c_\lambda m_\lambda
  \]
  となるが、
  \[
  f_n=\sum_{\lambda\in\mathcal{P}_k}d_\lambda m_\lambda(x_1,\cdots,x_n)    
  \]
  とおけば$\rho_{n,k}(f_k)=f_n$より
  \[
  f_n=\theta_n(\varphi((f_n)_{n\in\integer_{>0}}))
  \]
  以上より
  \[
  \theta\circ\varphi=\text{id}
  \]
  がわかる。逆に任意の$f\in\Lambda^k$に対して
  \[
  f=\sum_{\lambda\in\mathcal{P}_k}c_\lambda m_\lambda  
  \]
  とおけば(\ref{proj_of_monomial})より
  \[
  \theta_k(f)=\sum_{\lambda\in\mathcal{P}_k}c_\lambda m_\lambda(x_1,\cdots,x_k)
  \]
  だから
  \[
  \varphi(\theta(f))=f  
  \]
  がわかる。
\end{proof}


\begin{notice}
  対称多項式環の射影極限をとっても対称関数環にはならないことに注意せよ。例えば対称多項式の列
  \[
    f=((1+x_1),\:(1+x_1)(1+x_2),\:(1+x_1)(1+x_2)(1+x_3),\:\cdots)
  \]
  を考えると、$f$は対称多項式の射影極限の元であるが、次数の有界性を満たさないので対称関数ではない。

  一方
  \[
  \integer[x_1,\cdots,x_n]^{\mathfrak{S}_n}=\bigoplus_{k=0}^\infty \Lambda^k_n  
  \]
  だから、対称関数環は射影極限が直和と可換でない例を与えている。
\end{notice}


\begin{eg}\label{monomial_sym_fun_as_lim}
  $n$行のYoung図形$\lambda$に対して
  \[
  \rho_{n,n+1}(m_\lambda(x_1,\cdots,x_n,x_{n+1}))=m_\lambda(x_1,\cdots,x_n)  
  \]
  が成り立つ。実際
  \begin{align*}
    \rho_{n,n+1}(m_\lambda(x_1,\cdots,x_n,x_{n+1}))
    &=\rho_{n,n+1}\left(
      \sum_{\alpha\sim\lambda}x_1^{\alpha_1}\cdots x_n^{\alpha_n}x_{n+1}^{\alpha_{n+1}}
      \right)\\
    &=\sum_{\substack{\alpha\sim\lambda \\ \alpha_{n+1}=0}}x_1^{\alpha_1}\cdots x_n^{\alpha_n}\\
    &=m_\lambda(x_1,\cdots,x_n)
  \end{align*}
  よって$k=|\lambda|$次対称多項式の列$(m_\lambda(x_1,\cdots,x_l))_{l\in\integer_{\geq n}}$は一つの対称関数を定めるが、これは単項対称関数$m_\lambda$に他ならない。
\end{eg}

\begin{eg}
  例\ref{monomial_sym_fun_as_lim}と命題\ref{various_basis}より$e_\lambda$, $h_\lambda$, $p_\lambda$,$s_\lambda$もすべて一つの対称関数を定める。$e_\lambda,h_\lambda,p_\lambda$の定める対称関数は、例\ref{elementary_func}と例\ref{power_func}に他ならない。また$s_\lambda$の定める対称関数はSchur関数という。
\end{eg}

命題\ref{various_basis}より、次が成り立つ。

\begin{prop}\label{various_symfunc_basis}
  $\Lambda^k$の次の部分集合について、
  \begin{enumerate}[(i)]
    \item $\{m_\lambda\}_{\lambda\in\mathcal{P}_k}$
    \item $\{e_\lambda\}_{\lambda\in\mathcal{P}_k}$
    \item $\{h_\lambda\}_{\lambda\in\mathcal{P}_k}$
    \item $\{s_\lambda\}_{\lambda\in\mathcal{P}_k}$
    \item $\{p_\lambda\}_{\lambda\in\mathcal{P}_k}$
  \end{enumerate}
  (i)$\sim$(iv)は$\Lambda^k$の$\integer$上の基底をなし、(v)は$\quotient\otimes_\integer{\Lambda}$上の基底をなす。特に$\lambda$の範囲をすべてのYoung図形全体に変えれば、これらは$\Lambda$または$\quotient\otimes_\integer{\Lambda}$の基底を与える。
\end{prop}

また、命題\ref{det_formula}や命題\ref{p_to_h}, 補題\ref{formal_power_series_relation}, Littlewood-Richardson規則などの関係式は、そのまま対称関数においても成立することがわかる。(変数の制限$\rho_{m,n}$は和や積と可換である)

\begin{prop}
  次の関係式が成り立つ\footnote{
    (ii)から(iv)は$h_k$の母関数$H(t)=\prod_{i=1}^\infty\frac{1}{1-x_it}$を用いて補題\ref{formal_power_series_relation}と同じ計算で直接示してもよい。
  }
  \begin{align*}
    &\text{(i):}\:h_k=\vmat{
      e_1&e_2&e_3&\cdots&e_k\\
      1&e_1&e_2&\cdots&e_{k-1}\\
      0&1&e_1&\cdots&e_{k-2}\\
      \vdots&\vdots&\vdots&\ddots&\vdots&\\
      0&0&0&\cdots&e_1},\qquad
    e_k=\vmat{
      h_1&h_2&h_3&\cdots&h_k\\
      1&h_1&h_2&\cdots&h_{k-1}\\
      0&1&h_1&\cdots&h_{k-2}\\
      \vdots&\vdots&\vdots&\ddots&\vdots&\\
      0&0&0&\cdots&h_1
    }\\
    &\text{(ii):}\:h_k=\sum_{\lambda\in\mathcal{P}_k}\frac{p_\lambda}{z(\lambda)}\\
    &\text{(iii):}\:
    \sum_{\lambda}h_\lambda m_\lambda
    =\sum_{\lambda}\frac{1}{z(\lambda)}p_\lambda^2
    =\sum_{\lambda}s_\lambda^2,\qquad\text{ただし和はすべてのYoung図形全体をわたる。}\\
    &\text{(iv):}\:
    s_\lambda s_\mu=\sum_{\nu}\eta^{\nu}_{\lambda\mu}s_\nu,\qquad\text{ただし$\eta^{\nu}_{\lambda\mu}$はLittlewood-Richardson数}
  \end{align*}
\end{prop}

以下、$\Lambda$の係数を$\quotient$に拡大して考える。命題\ref{various_symfunc_basis}より、$\{s_\lambda\}$は$\Lambda$の基底をなすので、$\Lambda$に
\[
\generated{s_\lambda,s_\mu}=\delta_{\lambda\mu}  
\]
となるような内積を入れて考える。ただし$\delta_{\lambda\mu}$はKroneckerのデルタである。

\begin{prop}
  次が成り立つ
  \begin{enumerate}[(i)]
    \item $\generated{h_\lambda,m_\mu}=\delta_{\lambda\mu}$
    \item $\generated{p_\lambda,p_\mu}=\delta_{\lambda\mu}z(\lambda)$
  \end{enumerate}
\end{prop}

\begin{proof}
  \begin{enumerate}
    \item $h_\lambda=\sum_{\nu}a_{\lambda\nu}s_\nu$, $m_\mu=\sum_{\nu}b_{\mu\nu}s_\nu$とおく。このとき
    \begin{align*}
      \sum_{\lambda,\mu}h_\lambda m_\mu
      &=\sum_{\lambda,\mu}\sum_{\nu_1,\nu_2}a_{\lambda\nu_1}b_{\mu\nu_2}s_{\nu_1}s_{\nu_2}\\
      &=\sum_{\nu_1,\nu_2}\left(\sum_{\lambda,\mu}a_{\lambda\nu_1}b_{\mu\nu_2}\right)s_{\nu_1}s_{\nu_2}
    \end{align*}
  \end{enumerate}
\end{proof}



次に対称群の表現全体から作られる環を導入する。

\begin{defin}
  $\mathfrak{S}_n$の表現の同値類全体で生成される自由Abel群を$\tilde{R_n}$とする。$D$を
  \[
  \set{[V\oplus W]-[V]-[W]\in R}{V,W\text{はそれぞれ$\mathfrak{S}_n$の表現}}  
  \]
  で生成される$\tilde{R_n}$の部分加群とし、$R_n=\tilde{R_n}/D$とする。$R_0=\integer$として$R=\bigoplus_{n=0}^\infty R_n$とおく。
  
  $\mathfrak{S}_n$, $\mathfrak{S}_m$の表現$V,W$に対して、
  \[
  [V]\circ [W]=[\ind_{\mathfrak{S}_n\times\mathfrak{S}_m}^{\mathfrak{S}_{n+m}}V\boxtimes W]
  \]
  と定める。ここで、$\mathfrak{S}_n\times\mathfrak{S}_m$は$\mathfrak{S}_n$の元を$n+1$から$n+m$を固定する置換と同一視し、$\mathfrak{S}_m$の元を$1$から$n$を固定する置換と同一視することで$\mathfrak{S}_n\times\mathfrak{S}_m$の部分群とみなしている。$\circ$を双線形に拡張することによって$R$は可換環の構造を持つ。$R$を対称群の表現環という。
\end{defin}

\begin{prop}
  $\circ$は実際にwell-definedな乗法を定め、$R$は次数付き可換環となる。
\end{prop}

\begin{proof}
  テンソル積と直和の可換性から、$\mathfrak{S}_n$の表現$V,V'$に対して
  \[
    \ind_{\mathfrak{S}_n\times\mathfrak{S}_m}^{\mathfrak{S}_{n+m}}(V\oplus V')\boxtimes W
    =(\ind_{\mathfrak{S}_n\times\mathfrak{S}_m}^{\mathfrak{S}_{n+m}}V\boxtimes W)
    \oplus
    (\ind_{\mathfrak{S}_n\times\mathfrak{S}_m}^{\mathfrak{S}_{n+m}}V'\boxtimes W)
  \]
  よって
  \[
  ([V]+[V'])\circ [W]=[V\oplus V']\circ [W]=[V]\circ[W]+[V']\circ[W]  
  \]
  より$\circ$はwell-definedである。$\circ$が可換であることは$\mathfrak{S}_n$の元と$\mathfrak{S}_m$の元が$\mathfrak{S}_{n+m}$において可換であることからわかる。
\end{proof}

$R$の係数を$\quotient$に拡大して考える。$[V],[W]\in R_n$に対して
\[
\generated{[V],[W]}=\generated{\chi_V,\chi_W}  
\]
と定義し、$\generated{}$を$R$に双線形に拡張し内積を入れる。本節の主定理の一つが次の定理である。

\begin{theo}\label{rep_ring_and_func_ring}
  $\varphi:\Lambda\rightarrow R$を$\varphi(h_\lambda)=[M_\lambda]$によって定めると$\varphi$は内積を保つ環の同型
  であり、
  \[
  \varphi(s_\lambda)=[S_\lambda]
  \]
  が成り立つ。ただし$M_\lambda$は例\ref{ind_from_horizontal_perm}の誘導表現, $S_\lambda$は$\lambda$で定まる対称群の既約表現である。
\end{theo}

\begin{proof}
  $\{h_\lambda\}$は$\Lambda$の基底だから$\varphi$は$\quotient$線形写像として定義できることに注意する。また、Youngの規則(定理\ref{young_rule})より
  \[
  [M_\lambda]=[S_\lambda]+\sum_{\mu>\lambda}k_{\lambda\mu}[S_\mu]
  \]
  となるから、$R_n$の基底$\{[S_\lambda]\}$に関して$\{[M_\lambda]\}$を行列表示したとき、対角成分がすべて1の上三角行列になる。すなわち$\{[M_\lambda]\}$は$R_n$の基底となることがわかる。したがって$\varphi$は少なくとも線形同型であることがわかる。

  $\varphi$が環準同型であることを示す。$h_\lambda=h_{\lambda_1}\cdots h_{\lambda_r}$だから、
  \[
  [M_\lambda]=[M_{(\lambda_1)}]\circ\cdots\circ[M_{(\lambda_r)}]  
  \]
  となることを示せばよい。$n=|\lambda|$とおく。各$M_{(\lambda_i)}$について、$(\lambda_i)$は1行のYoung図形だから$M_{(\lambda_i)}$は$\mathfrak{S}_{\lambda_i}$の自明な表現である。よって$M_{(\lambda_i)}\boxtimes M_{(\lambda_j)}$は$\mathfrak{S}_{\lambda_i}\times\mathfrak{S}_{\lambda_j}$の自明な表現だから、
  \[
  [M_{(\lambda_i)}]\circ[M_{(\lambda_j)}]=[M_{(\lambda_i,\lambda_j)}]
  \]
  したがって
  \[
  [M_\lambda]=[M_{(\lambda_1)}]\circ\cdots\circ[M_{(\lambda_r)}]  
  \]
  
  次に$\varphi$が内積を保つことを示す。
\end{proof}



この定理によって対称群の表現に関する性質を、対称関数の知識を使って調べることができる。

\begin{cor}
  $\mathfrak{S}_n$の既約表現$S_\lambda$, $\mathfrak{S}_m$の既約表現$S_\mu$について
  \[
  \ind_{\mathfrak{S}_n\times\mathfrak{S}_m}^{\mathfrak{S}_{n+m}}S_\lambda\boxtimes S_\mu =\bigoplus_{\nu}S_\nu^{\oplus\eta^{\nu}_{\lambda\mu}} 
  \]
  が成り立つ。ただし$\eta^{nu}_{\lambda\mu}$はLittlewood-Richardson数である。
\end{cor}




\end{document}