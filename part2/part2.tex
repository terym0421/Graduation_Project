\documentclass{ltjsreport}
\RequirePackage{luatex85}
\usepackage[utf8]{inputenc}
\usepackage{enumerate}
\usepackage{amsthm}
\usepackage{amsfonts}
\usepackage{amsmath}
\usepackage{amssymb}
\usepackage{ytableau}
\usepackage{docmute}
\usepackage{mathtools}
\usepackage{xr}
\usepackage[all]{xy}



\theoremstyle{definition}
\newtheorem{defin}{定義}[subsection]
\newtheorem{theo}[defin]{定理}
\newtheorem{cor}[defin]{系}
\newtheorem{prop}[defin]{命題}
\newtheorem{lemm}[defin]{補題}
\newtheorem{notice}[defin]{注意}
\newtheorem{eg}[defin]{例}


\renewcommand{\labelenumi}{(\roman{enumi})}


\newcommand{\invlimit}{\mathop{\lim_{\longleftarrow}}}
\newcommand{\dirlimit}{\mathop{\lim_{\longrightarrow}}}
\newcommand{\ind}{\text{Ind}\:}
\newcommand{\Hom}{\text{Hom}}
\newcommand{\tr}{\text{tr}\:}
\newcommand{\id}[1]{\text{id}_{#1}}
\newcommand{\sgn}{\mathrm{sgn}}
\newcommand{\res}[1]{\text{Res}_{#1}}
\newcommand{\generated}[1]{\langle\:#1\:\rangle}
\newcommand{\im}{\text{Im }}
\newcommand{\rank}{\text{rank }}
\newcommand{\del}[2]{\frac{\partial #1}{\partial #2}}
\newcommand{\delsametwo}[2]{\frac{\partial^2 #1}{\partial #2^2}}
\newcommand{\delothertwo}[3]{\frac{\partial^2#1}{\partial#2\partial#3}}
\newcommand{\ddel}[2]{\frac{\partial}{\partial #2}#1}
\newcommand{\ddelsametwo}[3]{\frac{\partial^2}{\partial #2^2}#1}
\newcommand{\ddelothertwo}[3]{\frac{\partial^2}{\partial#2\partial#3}#1}
\newcommand{\simneq}{\not\simeq}
\newcommand{\transpose}[1]{^t\!#1}
\newcommand{\ie}{\text{i.e.}}
\newcommand{\inv}[1]{#1^{-1}}
\newcommand{\real}{\mathbb{R}}
\newcommand{\complex}{\mathbb{C}}
\newcommand{\integer}{\mathbb{Z}}
\newcommand{\quotient}{\mathbb{Q}}
\newcommand{\natnum}{\mathbb{N}}
\newcommand{\proj}{\mathbb{P}}
\newcommand{\tensor}[3]{#1\otimes_#2#3}
\newcommand{\map}[3]{#1:#2\rightarrow#3}
\newcommand{\aut}[2]{\mathrm{Aut}_{#1} (#2)}
\newcommand{\hommoph}[2]{\mathrm{Hom}_{#1}(#2)}
\newcommand{\gl}[1]{\mathrm{GL}(#1)}
\newcommand{\set}[2]{\left\{\:#1\:\middle|\:#2\:\right\}}
\newcommand{\pmat}[1]{\begin{pmatrix} #1
\end{pmatrix}}
\newcommand{\vmat}[1]{\begin{vmatrix} #1
\end{vmatrix}}
\newcommand{\br}{\vskip\baselineskip}

\begin{document}
\chapter{対称群と一般線形群の表現}

表現論とは群や多元環をベクトル空間への作用を通して研究する分野である。第2章では最も古典的な有限群の表現論の一般論(Maschkeの定理と指標理論, 群環)と対称群と一般線形群の表現について述べる。とくに対称群・一般線形群の表現は対称多項式と深いかかわりを持っている。

なお第2章で環は、乗法単位元をもつ必ずしも可換とは限らない環を指すとする。また、単に加群といったら環上の左加群を指しているとする。

本章は\cite{tensor_alg_and_rep},\cite{alg_2},\cite{homological},\cite{young_tableau},\cite{serre},\cite{fulton_harris},\cite{adv_alg},\cite{macdonald},\cite{sym_func}をまとめた内容になっている。

\documentclass{ltjsreport}
\RequirePackage{luatex85}
\usepackage[utf8]{inputenc}
\usepackage{enumerate}
\usepackage{amsthm}
\usepackage{amsfonts}
\usepackage{amsmath}
\usepackage{amssymb}
\usepackage{ytableau}
\usepackage{docmute}
\usepackage{mathtools}
\usepackage{xr}
\usepackage[all]{xy}



\theoremstyle{definition}
\newtheorem{defin}{定義}[subsection]
\newtheorem{theo}[defin]{定理}
\newtheorem{cor}[defin]{系}
\newtheorem{prop}[defin]{命題}
\newtheorem{lemm}[defin]{補題}
\newtheorem{notice}[defin]{注意}
\newtheorem{eg}[defin]{例}


\renewcommand{\labelenumi}{(\roman{enumi})}


\newcommand{\invlimit}{\mathop{\lim_{\longleftarrow}}}
\newcommand{\dirlimit}{\mathop{\lim_{\longrightarrow}}}
\newcommand{\ind}{\text{Ind}\:}
\newcommand{\Hom}{\text{Hom}}
\newcommand{\tr}{\text{tr}\:}
\newcommand{\id}[1]{\text{id}_{#1}}
\newcommand{\sgn}{\mathrm{sgn}}
\newcommand{\res}[1]{\text{Res}_{#1}}
\newcommand{\generated}[1]{\langle\:#1\:\rangle}
\newcommand{\im}{\text{Im }}
\newcommand{\rank}{\text{rank }}
\newcommand{\del}[2]{\frac{\partial #1}{\partial #2}}
\newcommand{\delsametwo}[2]{\frac{\partial^2 #1}{\partial #2^2}}
\newcommand{\delothertwo}[3]{\frac{\partial^2#1}{\partial#2\partial#3}}
\newcommand{\ddel}[2]{\frac{\partial}{\partial #2}#1}
\newcommand{\ddelsametwo}[3]{\frac{\partial^2}{\partial #2^2}#1}
\newcommand{\ddelothertwo}[3]{\frac{\partial^2}{\partial#2\partial#3}#1}
\newcommand{\simneq}{\not\simeq}
\newcommand{\transpose}[1]{^t\!#1}
\newcommand{\ie}{\text{i.e.}}
\newcommand{\inv}[1]{#1^{-1}}
\newcommand{\real}{\mathbb{R}}
\newcommand{\complex}{\mathbb{C}}
\newcommand{\integer}{\mathbb{Z}}
\newcommand{\quotient}{\mathbb{Q}}
\newcommand{\natnum}{\mathbb{N}}
\newcommand{\proj}{\mathbb{P}}
\newcommand{\tensor}[3]{#1\otimes_#2#3}
\newcommand{\map}[3]{#1:#2\rightarrow#3}
\newcommand{\aut}[2]{\mathrm{Aut}_{#1} (#2)}
\newcommand{\hommoph}[2]{\mathrm{Hom}_{#1}(#2)}
\newcommand{\gl}[1]{\mathrm{GL}(#1)}
\newcommand{\set}[2]{\left\{\:#1\:\middle|\:#2\:\right\}}
\newcommand{\pmat}[1]{\begin{pmatrix} #1
\end{pmatrix}}
\newcommand{\vmat}[1]{\begin{vmatrix} #1
\end{vmatrix}}
\newcommand{\br}{\vskip\baselineskip}

\begin{document}



\section{有限群の表現論}
\subsection{既約表現とMaschkeの定理}

\begin{defin}
  $G$を群、$V$をベクトル空間とする。群準同型$\map{\rho}{G}{\gl(V)}$が与えられたとき、$(\rho,V)$を$G$の表現といい$V$を表現空間という。
  $\rho$や$V$のことを表現ということもある。
\end{defin} 

以下、本節ではベクトル空間と言ったら複素数体$\complex$上の有限次元ベクトル空間を指すものとし、群と言ったら有限群を指すものとする。

\begin{eg}
  $G$を群、$V=\complex$とする。$\rho:G\rightarrow \gl(V)=\complex^\times$を、すべての$g\in G$に対して$\rho(g)=1$とすると$\rho$は表現になる。これを自明な表現という
\end{eg}

\begin{eg}\label{natural}
  $G=\mathfrak{S}_n$, $V=\complex^n$とする。$\rho:G\rightarrow \gl(V)$を
  $\sigma\in\mathfrak{S}_n$に対して
  \[
  \rho(\sigma)(a_1,\cdots,a_n)=(a_{\inv{\sigma}(1)},\cdots,a_{\inv{\sigma}(b)})  
  \]
  とすると$\rho$は表現になる。
\end{eg}

\begin{eg}\label{regular}
  $G$を群、$\complex[G]$を$G$の元を基底にもつ自由ベクトル空間とする。$\rho:G\rightarrow\gl(\complex[G])$を$g\in G$に対して
  \[
  \rho(g)\left(\sum_{x\in G}a_xx\right)=\sum_{x\in G}a_xgx  
  \]
  によって定めるとこれは表現になる。これを$G$の正則表現という
\end{eg}

文脈から明らかな場合や特に明示する必要がないとき、$\rho(g)x$のことをたんに$gx$と書く。表現論の基本的な問題は、$G$の考えうるあらゆる作用を分類することである。表現の分類の基準となるのは、次の定義である。

\begin{defin}
  $(\rho_1,V_1)$, $(\rho_2,V_2)$を$G$の表現とする。線形写像$\map{\varphi}{V_1}{V_2}$が
  \[
  \rho_2(g)\circ\varphi=\varphi\circ\rho_1(g),\quad \text{for all $g\in G$}
  \]
  をみたすとき、$\varphi$を$G$線形写像という。$G$線形写像の全体を$\Hom_G(V_1,V_2)$と書く。
\end{defin}

\begin{defin}
  $G$の表現$(\rho_1,V_1)$, $(\rho_2,V_2)$の間に同型な$G$線形写像があるとき、$(\rho_1,V_1)$と$(\rho_2,V_2)$同値な表現であるといい、
  \[
  \rho_1\simeq\rho_2  
  \]
  と書く。
\end{defin}

表現の同値は同値関係になる。したがって表現の分類はその同値類を求めることと言い換えられる。いきなりすべての表現を考えるのは難しいのでまずは「小さい表現」を考えたい。そのために、与えられた表現よりも小さい表現とは何かを定義する。

\begin{defin}
  $(\rho,V)$を$G$の表現とする。$V$の部分空間$W$が$G$不変であるとは
  \[
  \rho(g)W\subset W,\qquad \text{for all $g\in G$}  
  \]
  が成り立つことをいう。このとき$\map{\rho'}{G}{\gl(W)}$を
  \[
  \rho'(g)=\rho(g)|_W  
  \]
  によって定義することができ、表現になる。$(\rho',W)$を$(\rho, V)$の部分表現という。定義より、すべての表現$(\rho, V)$は$0$と$V$を部分表現に持っていることに注意。これらを自明な部分表現という。
\end{defin}

\begin{defin}
  $G$の表現$(\rho, V)$が既約であるとは、$V$が非自明な部分表現を持たないことをいう。
\end{defin}

\begin{eg}\label{ker_im}
  $\map{f}{V}{W}$が$G$線形写像であるなら$\ker f\subset V$, $\im f\subset W$はともに$G$不変部分空間である。
\end{eg}

\begin{eg}\label{1dim_rep}
  すべての1次元表現は既約である。実際1次元のベクトル空間$V$の部分空間は$0$と$V$のみである。
\end{eg}

\begin{eg}
  例\ref{natural}の表現を考える。
  \[
  W=\set{(a_1,\cdots,a_n)\in V}{a_1+\cdots+a_n=0}  
  \]
  とすると、$W$は$G$不変である。
  \[
  v=(1,1,\cdots,1)\in V  
  \]
  とし$U=\generated{v}$とおくと
  \[
  \rho(g)v=v  
  \]
  であるから$U$も$G$不変部分空間で、自明な表現と同値である。例\ref{1dim_rep}より$U$は既約である。
\end{eg}

与えられた表現から新しい表現を作る方法について解説する。
\begin{defin}
  $(\rho_1,V_1)$, $(\rho_2,V_2)$を$G$の表現とする。
  \begin{itemize}
    \item $\map{\rho_1\oplus\rho_2}{G}{\gl(V_1\oplus V_2)}$を
    \[
    (\rho_1\oplus\rho_2)(g)(x,y)=(\rho_1(x),\rho_2(y))
    \]
    で定義する。これを$\rho_1$と$\rho_2$の直和という。

    \item $\map{\rho_1\otimes\rho_2}{G}{\gl(V_1\otimes V_2)}$を
    \[
    (\rho_1\otimes\rho_2)(g)(x\otimes y)=\rho_1(x)\otimes\rho_2(y)
    \]
    で定義する。これを$\rho_1$と$\rho_2$の(内部)テンソル積という。

    \item $\map{\rho_1^*}{G}{\gl(V^*)}$を
    \[
    \rho_1^*(g)(f)=f\circ (\rho_1(\inv{g}))
    \]
    で定義する。これを$\rho_1$の反傾表現という。

    \item $(\rho_G,V)$を群$G$の表現、$(\rho_H,W)$を群$H$の表現とする。このとき$\rho_G\boxtimes\rho_H:G\times H\rightarrow GL(V\otimes W)$を
    \[
      \rho_G\boxtimes\rho_H(g,h)(x\otimes y)=\rho_G(g)(x)\otimes\rho_H(h)(y)
    \]
    で定義する。これを$\rho_G$と$\rho_H$の外部テンソル積という。
  \end{itemize}
\end{defin}

  これらが実際に表現になっていることは容易にわかる。実は、有限群の複素数体上の有限次元表現は既約表現の有限個の直和に同値であることがわかる(系\ref{irr_decompose})。すなわち、表現の分類を考える上では本質的に最も小さい表現、既約表現のみを考えれば良いことがわかる。

  \begin{theo}[Maschkeの定理]\label{maschke}
    $V$を$G$の表現とする。任意の$V$の$G$不変部分空間$W$に対して、$V$の$G$不変部分空間$U$が存在し
    \[
    V= W\oplus U
    \]
    がなりたつ。
  \end{theo}

  \begin{proof}
    証明のポイントは$W$への$G$不変な射影を構成することである。$\map{p}{V}{W}$を$G$不変とは限らない何らかの射影とする。
    \[
    f(x)=\frac{1}{|G|}\sum_{h\in G} hp(\inv{h}x)
    \]
    と定めると、$f$は$G$線形な$W$への射影となる。実際任意の$g\in G$に対して
    \begin{align*}
      f(gx)&=\frac{1}{|G|}\sum_{h\in G} hp(\inv{h}gx)\\
      &=\frac{1}{|G|}\sum_{k\in G} gkp(\inv{k}x)\qquad \text{where $k=\inv{g}h$}\\
      &=gf(x)
    \end{align*}
    より$G$線形性は示された。また
    \begin{align*}
      f^2(x)&=f\left(\frac{1}{|G|}\sum_{g\in G}gp(\inv{g}x)\right)\\
      &=\frac{1}{|G|^2}\sum_{g,h\in G}ghp(\inv{h}p(\inv{g}x))
    \end{align*}
    ここで、$\map{p}{V}{W}$は射影で$W$は$G$不変であるから$p(\inv{h}p(\inv{g}x))=\inv{h}p(\inv{g}x)$ゆえに
    \begin{align*}
      f^2(x)=\frac{1}{|G|^2}\sum_{g,h\in G}gp(\inv{g}x)=f(x)
    \end{align*}
    $f(W)\subset W$であり、任意の$W$の元$x$に対して
    \[
    f(x)=\frac{1}{|G|}\sum_{g\in G}gp(\inv{g}x)=\frac{1}{|G|}\sum_{g\in G}g\inv{g}x=x  
    \]
    であるから$f$は$W$への射影である。したがって
    \[
    V=\im f\oplus \ker f=W\oplus \ker f  
    \]
    が成り立つが、$f$は$G$線形なので$\ker f$は$G$不変部分空間である(例\ref{ker_im})。
  \end{proof}


\begin{cor}\label{irr_decompose}
  $V$を$G$の表現とすると、既約表現$W_1,\cdots,W_r$が存在して
  \[
  V\simeq W_1\oplus\cdots\oplus W_r  
  \]
  が表現の同値として成り立つ。このことを$G$の表現の完全可約性という。
\end{cor}

\begin{proof}
  $\dim_\complex V$に関する帰納法で示す。$\dim_\complex V=1$なら$V$は既約であるからよい。$\dim_\complex V>1$で$V$は可約であるとする。このとき
  $V$は非自明な部分表現$V_1$をもつが、定理\ref{maschke}より部分表現$U_1$で
  \[
  V=V_1\oplus U_1  
  \]
  となるものが存在する。$\dim_\complex V_1$, $\dim_\complex U_1<\dim_\complex V$であるから帰納法の仮定により、
  \begin{align*}
    &V_1=W_1\oplus\cdots\oplus W_{s_1},\\
    &U_1=W_{s+1}\oplus\cdots\oplus W_{r},\qquad\text{各$W_i$は既約}
  \end{align*}
  と既約分解できる。したがって$V$も既約分解される。
\end{proof}

\begin{notice}
  定理\ref{maschke}は標数が群の位数と互いに素な任意の体上で成立する。実際証明中で$|G|$で割る操作があるが、それ以外体に依存する議論はしていない。しかし$G$の位数が無限の場合は成り立たない。例えば無限巡回群$\integer$の表現として
  \[
  n\quad\mapsto\quad\pmat{1 & n\\0 & 1}
  \]
  を考える。$\pmat{1 & n\\0 & 1}$の固有空間$V(1)$は$\integer$不変だが、$\integer$不変な補空間をもたない。
  
  ただし定理\ref{maschke}の証明は$V$が無限次元であっても通用する\footnote{
    選択公理により、無限次元ベクトル空間においても任意の部分空間に対する補空間が存在し、それにより射影が得られる。
  }。しかし系\ref{irr_decompose}の証明は次元に関する帰納法を用いているので無限次元では通用しない。「有限個」の既約表現に分解できるということがポイントである。
  
\end{notice}





\subsection{有限群の表現に対する指標}

次に既約表現の分類をする上で鍵となる指標の概念を導入する。

\begin{defin}
  $(\rho,V)$を$G$の表現とする。$\map{\chi_V}{G}{\complex}$を
  \[
  \chi_V(g)=\tr \rho(g)
  \]
  で定め、これを$V$の指標という。
\end{defin}

本節では
指標の直交関係(定理\ref{char_orthogonality})
を示すことが目標である。指標は類関数と呼ばれる群上の関数になっており、類関数のなすベクトル空間に特別な内積を入れるとこの内積に関して指標が正規直交基底をなす、というのが主張である。この系として、
\begin{itemize}
  \item 既約表現の個数は共役類の個数に等しい
  \item 既約表現の分類は既約指標の分類に帰着される
  \item 既約表現の次元に関する公式
\end{itemize}
といったさまざまな有用な事実が導かれる。

表現の各種の演算と指標との関係を見ておく

\begin{prop}\label{char_property}
  $V_1,V_2$を$G$の表現とする。
  \begin{enumerate}
    \item $\chi_{V_1\oplus V_2}=\chi_1+\chi_2$
    \item $\chi_{V_1\otimes V_2}=\chi_1\chi_2$
    \item $\chi_{V_1^*}=\overline{\chi_1}$
  \end{enumerate}
  が成り立つ
\end{prop}

\begin{proof}
  \begin{enumerate}
    \item $\tr(A\oplus B)=\tr(A)+\tr(B)$より従う
    \item $\tr(A\otimes B)=\tr(A)\tr(B)$より従う
    \item $\rho^*(g)=\transpose\rho(\inv{g})$であるから、$\tr(\rho^*(g))=\tr(\transpose\rho(\inv{g}))=\tr(\rho(\inv{g}))$となる。ここで、$G$は有限群であるから$\rho(g)$は有限位数、したがってユニタリ行列である。よって$\rho(g)$の固有値$\lambda_1,\cdots,\lambda_n$はすべて絶対値が$1$なので
    \[
    \tr(\rho(\inv{g}))=\frac{1}{\lambda_1}+\cdots+\frac{1}{\lambda_n}=\overline{\lambda_1}+\cdots+\overline{\lambda_n}=\overline{\tr(\rho(g))}
    \]
  \end{enumerate}
\end{proof}

\begin{eg}
  自明な表現の指標は$1$である。
\end{eg}

\begin{eg}\label{char_st_rep}
  $G=\mathfrak{S}_n$として$V=\complex^n$を
  \[
  \sigma\pmat{x_1\\\vdots\\x_n}=\pmat{x_{\inv{\sigma}(1)}\\\vdots\\x_{\inv{\sigma}(n)}}  
  \]
  によって表現とする(例\ref{natural})。すなわち$e_1,\cdots,e_n$を標準基底として
  \[
  \sigma(e_i)=e_{\sigma(i)}
  \]
  である。よって指標を$\chi_V$とすれば
  \[
  \chi_V(\sigma)=\sharp\set{i\in\{1,\cdots,n\}}{\sigma(i)=i}  
  \]
  となる。$V$は
  \[
  U=\set{\pmat{x_1\\\vdots\\x_n}}{x_1+\cdot+x_n=0},\quad W=\generated{\pmat{1\\\vdots\\1}}  
  \]
  として$V=U\oplus W$と分解されたから、
  \[
  \chi_U=\chi_V-1  
  \]
\end{eg}

\begin{eg}\label{char_reg_rep}
  有限群$G$に対してその正則表現(例\ref{regular})$\complex[G]$を考える。任意の$g\in G$に対して、ある$x\in G$があって$gx=x$ならば$g=e$であるから、$g$を基底$G$に関して行列表示したとき$g\neq e$ならば対角成分はすべて0である。よって指標を$R$とすると
  \[
  R(g)=\left\{\begin{array}{cc}
    0 & \text{if }g\neq e\\
    |G| & \text{if }g=e
  \end{array}\right. 
  \]
\end{eg}

\begin{eg}
  $V$を$G$の表現, $\chi$を$V$の指標とするとき
  \[
  \dim V=\chi(e)  
  \]
  である。実際、$e$の作用を行列表示すれば単位行列になるから、そのトレースは次元に等しい。
\end{eg}

指標の直交関係を示そう。まず、いくつか必要な補題を示す。
\begin{lemm}[Schurの補題]\label{schur_lem}
  $V$, $W$を$G$の既約表現とする。このとき
  \[
  \dim_\complex\Hom_G(V,W)=\left\{\begin{array}{cl}
    1 & \text{if $V\simeq W$ as $G$-representation}\\
    0 & \text{otherwise}
  \end{array}\right.
  \]
  が成り立つ。とくに$V=W$なら$f\in\Hom_G(V,V)$はスカラー写像である。
\end{lemm}

\begin{proof}
  先に後半の主張を示す。$\map{f}{V}{V}$を$G$線形写像とする。$f$の固有空間を$V(\lambda)$とすると、$V(\lambda)$は$G$不変である。実際、$x\in V(\lambda)$, $g\in G$に対して
  \[
  f(gx)=gf(x)=g(\lambda x)=\lambda gx  
  \]
  である。$V$は既約であり$V(\lambda)\neq 0$なので$V(\lambda)=V$よって
  \[
  f=\lambda\id{V}
  \]
  である。

  前半を示そう。$V\simeq W$とし$\varphi\in\Hom_G(V,W)$を$G$同型として固定する。任意の$f\in\Hom_G(V,W)$について、$\inv{\varphi}\circ f:V\rightarrow V$は$G$線形写像であるから前半の結果より
  \[
  \inv{\varphi}\circ f=\lambda\id{V}
  \]
  と表される。すなわち
  \[
  f=\lambda\varphi  
  \]
  である。したがって$\Hom_G(V,W)=\generated{\varphi}$となる。

  $V\simneq W$の場合、$f\in\Hom_G(V,W)$について$V,W$の既約性から
  \[
  \ker f=0\text{または}V,\quad \im f=0\text{または}W   
  \]
  を得るが、$V\neq W$より$\ker f=V$, $\im f=0$すなわち$f=0$である。これで示せた。
\end{proof}

\begin{notice}
  補題\ref{schur_lem}の証明より$f\in\Hom_G(V,W)$は$V\simeq W$なら$0$または同型、$V\simneq W$なら$f=0$であることがわかる。こちらをSchurの補題と呼ぶ場合もある。
\end{notice}


\begin{lemm}\label{hom_representation}
  $(\rho,V),(\theta,W)$を$G$の表現とする。$\Hom(V,W)=V^*\otimes W$より、$\psi:G\rightarrow \gl(\Hom(V,W))$を
  \[
  \psi(g)(f)=\theta(g)\circ f\circ \rho(\inv{g})
  \]
  とするとこれは表現となり、
  \[
  \chi_{\Hom(V,W)}=\overline{\chi_V}\chi_W  
  \]
  が成り立つ
\end{lemm}

\begin{proof}
  命題\ref{char_property}より従う。
\end{proof}

\begin{lemm}\label{fix_dim}
  $V$を$G$の表現とし、$V^G$を$G$の固定点の集合とする。すなわち
  \[
  V^G=\set{v\in V}{\forall g\in G,\quad gv=v}  
  \]
  とする。このとき$V^G$は$V$の部分表現であり
  \[
  \dim V^G=\frac{1}{|G|}\sum_{g\in G}\chi_V(g)
  \]
  が成り立つ。とくに$V^G$は$G$の自明な表現の直和である。
\end{lemm}

\begin{proof}
  $\map{f}{V}{V}$を
  \[
  f(x)=\frac{1}{|G|}\sum_{g\in G}gx
  \]
  で定義すると$f$は射影になる。実際、
  \begin{align*}
    f^2(x)&=\frac{1}{|G|}\sum_{g,h\in G}ghx\\
    &=\frac{1}{|G|}\sum_{k\in G}kx\\
    &=f(x)
  \end{align*}
  である。$h\in G$に対して
  \[
  hf(x)=\frac{1}{|G|}\sum_{g\in G}hgx=\frac{1}{|G|}\sum_{k\in G}kx=f(x)
  \]
  より$\im f\subset V^G$である。逆に任意の$x\in V^G$に対して
  \[
  f(x)=\frac{1}{|G|}\sum_{g\in G}gx=\frac{1}{|G|}\sum_{g\in G}x=x  
  \]
  ゆえに$\im f\subset V^G$. 射影のトレースは像の次元に等しい\footnote{
    射影は$f^2=f$をみたすので固有値は$0$か$1$のどちらかであることから従う。
  }ので
  \[
  \tr(f)=\dim \im f=V^G  
  \]
  だが、
  \[
  \tr (f)=\frac{1}{|G|}\sum_{g\in G}\chi_V(g)  
  \]
  よって示せた。$V^G$が$G$の自明な表現の直和であることは、$V^G$の定義そのものである。
\end{proof}

\begin{eg}
  $\Hom(V,W)^G=\Hom_G(V,W)$である。実際$f\in \Hom(V,W)$に対して
  \[
  \theta(g)\circ f \circ \rho(\inv{g})=f\Leftrightarrow \theta(g)\circ f=f\circ \rho(g)
  \]
  である。
\end{eg}




\begin{defin}
  関数$\map{f}{G}{\complex}$が
  \[
  f(\inv{g}xg)=f(x),\qquad\text{for all $g\in G$}  
  \]
  を満たすとき、$f$を類関数という。類関数全体を$C(G)$と置くと$C(G)$には点ごとに和とスカラー倍を定めて$\complex$ベクトル空間の構造が入る
\end{defin}

\begin{eg}
  トレースの性質$\tr(AB)=\tr(BA)$より表現の指標は類関数である。
\end{eg}

\begin{eg}
  $G$の共役類を$C_1,\cdots,C_s$とし、$G$上の関数$\omega_i$を
  \[
  \omega_i(x)=\left\{\begin{array}{cl}
    1 & \text{if $x\in C_i$}\\
    0 & \text{otherwise}
  \end{array}\right.  
  \]
  で定めると$\omega_i$は類関数であり$\omega_1,\cdots,\omega_s$は$C(G)$の基底である。よって$\dim C(G)=s$である。
\end{eg}

\begin{defin}
  $\phi,\psi\in C(G)$に対して
  \[
  \generated{\phi,\psi}=\frac{1}{|G|}\sum_{g\in G}\overline{\phi(g)}\psi(g)  
  \]
  によって$\generated{\cdot,\cdot}:C(G)\times C(G)\rightarrow \complex$を定めると、これは$C(G)$上のHermite内積となる。$C(G)$にはいつもこの内積が入っているものとする。
\end{defin}

\begin{theo}[指標の直交関係]\label{char_orthogonality}
  $V,W$を$G$の既約表現とする。このとき
  \[
  \generated{\chi_V,\chi_W}=\left\{\begin{array}{cl}
    1 & \text{if $V\simeq W$ as $G$-representation}\\
    0 & \text{otherwise}
  \end{array}\right.  
  \]
  が成り立つ。
\end{theo}

\begin{proof}
  補題\ref{hom_representation}と補題\ref{fix_dim}およびSchurの補題(補題\ref{schur_lem})から
  \begin{align*}
    \generated{\chi_V,\chi_W}&=\frac{1}{|G|}\sum_{g\in G}\overline{\chi_V(g)}\chi_W(g)\\
    &=\frac{1}{|G|}\sum_{g\in G}\chi_{\Hom(V,W)}(g)\\
    &=\dim \Hom(V,W)^G\\
    &=\dim \Hom_G(V,W)\\
    &=\left\{\begin{array}{cl}
      1 & \text{if $V\simeq W$ as $G$-representation}\\
      0 & \text{otherwise}
    \end{array}\right.  
  \end{align*}
\end{proof}

この定理から直ちに、$G$の既約指標は有限個であることがわかる。とくに次が成り立つ。

\begin{cor}\label{irr_char_is_basis}
  $G$の既約指標$\chi_1,\cdots,\chi_r$は$C(G)$の正規直交基底をなす。したがって$r$は$G$の共役類の数に等しい。
\end{cor}

\begin{proof}
  正規直交であることは定理\ref{char_orthogonality}で示されたので、基底であること、すなわち次を示せばよい:
  \begin{quote}
    $f\in C(G)$が$\generated{\chi_i,f}=0$を各$i=1,\cdots, r$に対して満たせば$f=0$である
    \footnote{
    一般に内積空間$V$の正規直交系$v_1,\cdots,v_n$が、性質「$w\in V$がすべての$i$に対して$\generated{w,v_i}=0$をみたすならば$w=0$」をもてば$v_1,\cdots,v_n$は$V$の基底になる。実際、任意の$x\in V$に対して$w=x-(\generated{x,v_1}v_1+\cdots+\generated{x,v_n}v_n)$と置けば$\generated{w,v_i}=0$をみたすから$w=0$
    }
  \end{quote}
  $f$が仮定をみたす類関数であるとする。各$i$について$\chi_i$を指標に持つ既約表現を$(\rho_i,V_i)$とおく。
  \[
  0=\generated{f,\chi_i}=\frac{1}{|G|}\sum_{g\in G}\overline{f(g)}\tr(\rho_i(g))  
  \]
  より、写像$\map{F_i}{V_i}{V_i}$を
  \[
  F_i=\sum_{g\in G}\overline{f(g)}\rho_i(g)  
  \]
  とおけば$\tr(F_i)=0$である。$F_i$は$G$線形写像である。実際$h\in G, x\in V_i$として
  \begin{align*}
  F_i(\rho_i(h)x)&=\sum_{g\in G}\overline{f(g)}\rho_i(gh)x\\
  &=\sum_{k\in G}\overline{f(k\inv{h})}\rho_i(k)x,\qquad\text{where $k=gh$}\\
  &=\sum_{k\in G}\overline{f(\inv{h}k)}\rho_i(k)x,\qquad\text{($f$は類関数)}\\
  &=\sum_{l\in G}\overline{f(l)}\rho_i(hl)x,\qquad\text{where $l=\inv{h}k$}\\
  &=\rho_i(h)F_i(x)
  \end{align*}
  よってSchurの補題(補題\ref{schur_lem})よりある$\lambda\in\complex$で
  \[
  F_i=\lambda\id{V}  
  \]
  となるが、$\tr(F_i)=0$だったから$\lambda=0$でなければならない。よって$F_i=0$であることがわかる。

  次に、$\theta:G\rightarrow \complex[G]$を$G$の正則表現とする。ただし$\complex[G]$は$G$を基底に持つ自由ベクトル空間である。定理\ref{maschke}より$\theta$はいくつかの既約表現の直和に同値である。よって
  \begin{equation}\label{theta}
  \theta=\rho_1^{\oplus m_1}\oplus\cdots\oplus\rho_r^{\oplus m_2}  
  \end{equation}
  とおく。写像$F:\complex[G]\rightarrow\complex[G]$を
  \[
  F=\sum_{g\in G}\overline{f(g)}\theta(g)  
  \]
  とすれば(\ref{theta})より
  \[
  F=\left(\sum_{g\in G}\overline{f(g)}\rho_1(g)\right)^{\oplus m_1}\oplus\cdots\oplus  \left(\sum_{g\in G}\overline{f(g)}\rho_r(g)\right)^{\oplus m_r}=F_1^{\oplus m_1}\oplus\cdots\oplus F_r^{\oplus m_r}=0
  \]
  よって$e$を$G$の単位元として
  \[
  0=F(e)=\sum_{g\in G}\overline{f(g)}g
  \]
  $G$は一次独立であるからすべての$g$について$f(g)=0$
\end{proof}

\begin{cor}\label{rep_and_char}
  $(\rho_1,V_1),(\rho_2,V_2)$を$G$の表現、対応する指標を$\chi_1,\chi_2$とする。$\rho_1\simeq \rho_2$であるための必要十分条件は$\chi_1=\chi_2$が成り立つことである。とくに既約表現の同値類も、共役類と同じ数だけ存在する。
\end{cor}

\begin{proof}
  必要性は明らか。十分性を示す。$V_1,V_2$が既約である場合だけを考えれば、既約指標の一次独立性から従う。もし$\chi_1=\chi_2$かつ$V_1\simneq V_2$であったとする。Schurの補題より
  \[
  \generated{\chi_1,\chi_2}=\dim\Hom_G(V_1,V_2)=0  
  \]
  となるが$\chi_1,\chi_2\neq 0$なので矛盾である。
\end{proof}


\begin{cor}\label{multiplicity}
  $V$を$G$の表現, $W_1,\cdots,W_r$を$G$の既約表現の同値類の完全代表系とし、それぞれの対応する指標を$\chi,\chi_1,\cdots,\chi_r$とおく。
  \[
  V\simeq W_1^{\oplus m_1}\oplus\cdots\oplus W_r^{\oplus m_r}  
  \]
  とすると、
  \[
  m_i=\generated{\chi,\chi_i}=\dim\Hom_G(W_i,V)
  \]
  が成り立つ。$m_i$を$V$の$W_i$に関する重複度という。とくに表現の既約表現への分解は同値の違いを除いて一意的である。
\end{cor}

\begin{cor}
  指標$\chi$が既約指標であるための必要十分条件は$\generated{\chi,\chi}=1$が成り立つことである
\end{cor}

\begin{proof}
  必要性は明らか。十分性を示す。$\chi=m_1\chi_1+\cdots+m_r\chi_r$とおくと
  \[
  \generated{\chi,\chi}=1  
  \]
  であるならば
  \[
  m_1^2+\cdots+m_r^2=1  
  \]
  ゆえにある$i$で$\chi=\chi_i$である。
\end{proof}

\begin{cor}[Schurの補題の逆]\label{reverse_schur}
  $V$を$G$の表現とする。$\dim\Hom_G(V,V)=1$であるならば$V$は既約表現である。
\end{cor}

\begin{proof}
  $\chi$を$V$の指標とするとき、補題\ref{hom_representation}, 補題\ref{fix_dim}より条件は
  \[
  \frac{1}{|G|}\sum_{g\in G}|\chi(g)|^2=1
  \]
  すなわち$\generated{\chi,\chi}=1$に他ならない。
\end{proof}


\begin{prop}\label{dim_formula}
  $W_1,\cdots,W_r$を$G$の既約表現の同値類の完全代表系とする。
  \[
  |G|=(\dim W_1)^2+\cdots +(\dim W_r)^2  
  \]
  が成り立つ
\end{prop}

\begin{proof}
  $\theta$を$G$の正則表現とする。$\theta$の指標を$R$, $W_i$の指標を$\chi_i$とおく。系\ref{multiplicity}より$\theta$の$W_i$に関する重複度を$m_i$とおくと
  \[
  m_i=\generated{R,\chi_i}=\frac{1}{|G|}\sum_{g\in G}\overline{R(g)}\chi_i(g)
  \]
  ここで、
  \[
  R(g)=\tr(\theta(g))=\left\{\begin{array}{cl}
    0 & \text{if $g\neq e$}\\
    |G| & \text{if $g=e$}
  \end{array}\right.  
  \]
  であるから
  \[
  m_i=\chi_i(e)=\dim W_i  
  \]
  よって
  \[
  R=(\dim W_1)\chi_1+\cdots +(\dim W_r)\chi_r  
  \]
  であるから
  \[
  |G|=R(e)=(\dim W_1)^2+\cdots +(\dim W_r)^2  
  \]
\end{proof}


\begin{prop}\label{multi_space}
  $V$を$G$の表現とする。
  \[
  V\simeq \bigoplus_{i=1}^rW_i\otimes_\complex \Hom_G(W_i,V)  
  \]
  が成り立つ。
\end{prop}

\begin{proof}
  $\phi_i:W_i\otimes_\complex\Hom_G(W_i,V)\rightarrow V$を
  \[
  \phi_i(x\otimes f)=f(x)  
  \]
  を双線形に拡張して定める。
  \[
  \phi(gx\otimes f)=f(gx)=g(f(x))=\phi(x\otimes gf)  
  \]
  より$\phi$は$G$線形である。$\phi=\bigoplus_{i}\phi_i$とする。
  
  $V$の既約分解
  \[
    V=\bigoplus_i^r (U_1^{(i)}\oplus\cdots\oplus U_{m_i}^{(i)}),\qquad U^{(i)}_j\simeq W_i
  \]
  を固定し、$\theta^{(i)}_j:U^{(i)}_j\rightarrow W_i$を$G$同型とする。$\psi^{(i)}_j:U^{(i)}_j\rightarrow W_i\otimes_\complex \Hom_G(W_i,V) $を
  \[
  \psi^{(i)}_j(x)=\theta^{(i)}_j(x)\otimes \theta^{(i)-1}_j
  \]
  として$\psi=\bigoplus_{i,j}\psi^{(i)}_j$とする。$\psi$も$G$線形で$\phi\circ\psi=\text{id}$がわかるから、次元を比べて同型であることがわかる。
\end{proof}


\begin{eg}\label{S3}
  $G=\mathfrak{S}_3$の既約指標を全て求めよう。$G$の共役類は
  \[
  e,\:(1,2),\:(1,2,3)  
  \]
  で代表される3つであるから既約表現も3つある。またそれぞれの共役類の濃度は順に
  \[
  1,3,2  
  \]
  である。

  $1$を自明な表現とし、$\sgn$を置換の符号とすると、$\sgn$は1次元の既約表現である。例\ref{char_st_rep}の指標を考えよう。
  \[
  \chi_U(g)=\chi_V(g)-1=|\set{x\in\{1,2,3\}}{gx=x}|-1  
  \]
  であるから
  \[
  \generated{\chi_U,\chi_U}=\frac{1}{6}(1\cdot 2^2+3\cdot 0^2+2\cdot (-1)^2)=\frac{6}{6}=1 
  \]
  よって既約である。まとめると$G$の既約指標は次の3つである
  \[
  \begin{array}{c|c|c|c}
        &\quad e\quad & (1,2) & (1,2,3) \\ \hline
    1   &      1      &   1   &    1    \\ \hline
  \sgn  &      1      &   -1  &    1    \\ \hline
  \chi_U&      2      &   0   &    -1
  \end{array}
  \]
\end{eg}



\begin{eg}
  $G=\mathfrak{S}_4$の既約指標を全て求めよう。$G$の共役類は
  \[
  e,\:(1,2),\:(1,2,3),\:(1,2,3,4),\:(1,2)(3,4)  
  \]
  で代表される5つであるから既約指標も5つある。またそれぞれの共役類の濃度は順に
  \[
  1,6,8,6,3  
  \]
  である。
  
  $\mathfrak{S}_3$と同様、1次元の既約表現として$1$と$\sgn$がある。再び例\ref{char_st_rep}の指標を考える。
  \[
  \chi_U(g)=|\set{x\in\{1,2,3,4\}}{gx=x}|-1  
  \]
  であるから
  \[
  \generated{\chi_U,\chi_U}=\frac{1}{24}(1\cdot 3^2+6\cdot 1^2+8\cdot 0^2+6\cdot (-1)^2+3\cdot(-1)^2)=\frac{24}{24}=1 
  \]
  よって既約である。さらに$\sgn^2=1$より
  \[
  \generated{\chi_U\sgn,\chi_U\sgn}=1 
  \]
  であることがわかるので$\chi_U\sgn$も既約指標である。ここまでをまとめると次の表を得る。
  \[
    \begin{array}{c|c|c|c|c|c}
  
           & \quad e\quad & (1,2) & (1,2,3) & (1,2,3,4) & (1,2)(3,4)\\
      \hline
      1    & 1 &   1   &    1    &     1     &      1    \\
      \hline
      \sgn & 1 &   -1  &    1    &     -1    &      1   \\
      \hline
      \chi_U & 3 &   1   &    0    &     -1    &      -1  \\
      \hline
      \chi_U\sgn & 3 & -1 &   0    &     1    &     -1  \\
      \hline
      \psi       &  x_1  & x_2 & x_3 &   x_4    &     x_5     
    \end{array}
  \]
  あと1つの指標$\psi$は直交関係や次元公式を用いることで具体的な作用の考察なしに求めることができる。次元公式より
  \[
  \psi(e)=24-(1^2+1^2+3^2+3^2)=4  
  \]
  ゆえに$\psi(e)=2$である。直交関係より
  \[
  \left\{\begin{array}{ccc}
    6x_2+8x_3+6x_4+3x_5 & = & -2\\
    -6x_2+8x_3-6x_4+3x_5 & = & -2\\
    6x_2-6x_4-3x_5 & = & -6\\
    4+6x_2^2+8x_3^2+6x_4^2+3x_5^2 &= &24
  \end{array}\right.  
  \]
  これを解くと
  \[
    \begin{array}{c|c|c|c|c|c}
  
           & \quad e\quad & (1,2) & (1,2,3) & (1,2,3,4) & (1,2)(3,4)\\
      \hline
      1    & 1 &   1   &    1    &     1     &      1    \\
      \hline
      \sgn & 1 &   -1  &    1    &     -1    &      1   \\
      \hline
      \chi_U & 3 &   1   &    0    &     -1    &      -1  \\
      \hline
      \chi_U\sgn & 3 & -1 &   0    &     1    &     -1  \\
      \hline
      \psi       &  2  & 0 & -1 &   0 &     2     
    \end{array}
  \]

  $\psi$を指標に持つ既約表現は$\mathfrak{S}_3$の既約表現からつくることができる。$\mathfrak{S}_3$の既約表現
  \[
  U'=\set{\pmat{x_1\\x_2\\x_3}}{x_1+x_2+x_3=0}  
  \]
  を考える。$\mathfrak{S}_4$の正規部分群$N=\{e,(1\:2)(3\:4),(1\:3)(2\:4),(1\:4)(2\:3)\}$を考えると、
  \[
  \mathfrak{S}_4/N\simeq \mathfrak{S}_3  
  \]
  となる\footnote{
    $\mathfrak{S}_4$の$X=\{(1\:2)(3\:4),(1\:3)(2\:4),(1\:4)(2\:3)\}$への共役作用を考えればよい。
  }。
  \[
  \mathfrak{S}_4\rightarrow \mathfrak{S}_4/N\simeq \mathfrak{S}_3\rightarrow \gl(U')
  \]
  は既約表現になる\footnote{
    一般に全射群準同型と既約表現の合成は、再び既約表現になる。
  }が、その指標を$\psi'$とおくと$\psi'=\psi$となることが確認できる。
\end{eg}



\subsection{群環}

本節では群環という代数を導入し、環上の加群論を用いた表現論に関するいくつかの命題を証明する。

\begin{defin}
  $G$を群, $K$を体とする。$K[G]$を$G$を基底にもつ$K$上の自由ベクトル空間とし、$G$の積から自然に定まる演算で$K[G]$に積を入れる。すなわち
  \[
  \left(\sum_{g\in G}a_gg\right)\cdot\left(\sum_{h\in G}b_hh\right)  =\sum_{k \in G}\left(\sum_{gh=k}a_gb_h\right)k
  \]
  である。これによって$K[G]$は$K$上の多元環の構造をもつ。これを$G$の$K$上の群環という。
\end{defin}

$V$を$G$の体$K$上の表現とする。$V$は自然に$K[G]$加群の構造が入り、逆に$K[G]$加群は自然に$G$の表現とみなすことができる。

このとき、
\begin{itemize}
  \item 部分表現は部分加群
  \item 表現の直和は加群の直和
  \item 既約表現は単純加群
  \item $G$線形写像は$K[G]$加群の準同型
  \item 表現の同値は加群の同型
\end{itemize}
にそれぞれ対応することがわかる。ここで、$A$加群$M$が単純であるとは、$M$が非自明な部分加群をもたないことをいう。また、表現のテンソル積は$K[G]$加群としてのテンソル積ではないことに注意。$V,W$を$K[G]$加群とするとき、$V$と$W$の表現のテンソル積は$V\otimes_{K}W$に$g(x\otimes y)=gx\otimes gy$による作用を入れたものである。

である。環$A$を$A$加群とみなしたとき、$A$の部分加群とは$A$の左イデアルにほかならず、$A$に含まれる単純$A$加群は$A$の極小左イデアルである。したがって、$G$の有限次元既約表現を求めることは環$\complex[G]$の極小左イデアルを求めることと同等である。単純性に関連して次の定義をする。

\begin{defin}
  $A$加群$M$が半単純であるとは、任意の$M$の部分加群が$M$の直和因子であることをいう。また、任意の$A$加群が半単純であるとき、$A$を半単純環という。
\end{defin}

\begin{theo}[Maschkeの定理]\label{general_maschke}
  $K[G]$が半単純環であるための必要十分条件は、$|G|$が$p=\text{ch}\:K$で割り切れないことである。
\end{theo}

\begin{proof}
  十分性は定理\ref{maschke}の証明とまったく同様である。必要性を示す。$|G|$が$p$がの倍数であるとする。Wedderburnの構造定理(付録参照)より$K[G]$のJacobson根基が0でないことを示せばよい。$K[G]$の元$m$を
  \[
  m=\sum_{g\in G}g  
  \]
  とおくと、任意の$x\in K[G]$に対して$xm=mx$であり、さらに
  \[
  m^2=\sum_{g,h\in G}gh=|G|m=0  
  \]
  であるから
  \[
  (1-xm)(1+xm)=1-x^2m^2=1  
  \]
  よって$1-xm$は単元であるから$m\in\text{Jac}(K[G])$である。
\end{proof}


$\complex$の標数は0であるから定理\ref{maschke}は定理\ref{general_maschke}の特別な場合である。しかし系\ref{irr_decompose}は一般には成り立たない。考えている表現が有限次元の場合において成り立つことに注意せよ。

以下、$K=\complex$の場合を考える。命題\ref{dim_formula}の証明より、$G$の正則表現は$G$のすべての有限次元既約表現をその次元の数だけ直和因子にもっている。このことを群環のことばで述べると、$\complex[G]$は$\complex[G]$加群として極小左イデアルの直和
\begin{align*}
\complex[G]=&L_1\oplus\cdots\oplus L_s,\quad s=m_1+\cdots+m_r\\
&L_1,\cdots,L_{m_1}\simeq W_1\\
&L_{m_1+1},\cdots,L_{m_1+m_2}\simeq W_2\\
&\qquad \vdots\\
&L_{m_1+\cdots+m_{r-1}+1},\cdots,L_{m_1+\cdots+m_{r-1}+m_r}\simeq W_r
\end{align*}
と分解できるということである。ここで$W_1,\cdots,W_r$は$G$の既約表現から定まる$\complex[G]$加群であり、$m_i=\dim_{\complex}W_i$である。

\begin{defin}
  $A$を環とする。べき等元$e\in A$ $(e^2=e)$が原始的であるとは、
  \[
  e=e_1+e_2,\quad e_1^2=e_1,\quad e_2^2=e_2,\quad e_1e_2=0\implies e_1=0\text{または}e_2=0 
  \]
  を満たすことをいう。
\end{defin}

\begin{prop}
  $A$を半単純環, $e\in A$を単元でないとする。$Ae$が極小左イデアルとなるための必要十分条件は$e$が原始的べき等元となることである。
\end{prop}

\begin{proof}
  $Ae$が極小左イデアルであるとする。
  \[
    e=e_1+e_2,\quad e_1^2=e_1,\quad e_2^2=e_2,\quad e_1e_2=0
  \]
  となる$e_1,e_2\in A$が存在したとすると、
  \[
  e_1=e_1^2=e_1^2+e_1e_2=e_1e\in Ae  
  \]
  同様に$e_2\in Ae$である。よって$Ae$の極小性から$Ae_1=Ae\text{ or }0$, $Ae_2=Ae\text{ or }0$である。$Ae_1=Ae$であったとしよう。このとき
  \[
  e=ce_1,\quad c\in A  
  \]
  とおくことができるから
  \[
  e_2=e-e_1=(c-1)e_1
  \]
  よって
  \[
  e_2=e_2^2=(c-1)e_1e_2=0
  \]
  $Ae_2=A_e$ならば同様の議論で$e_1=0$となる。
  
  逆に$e$が原始的べき等元であるとする。$I\subsetneq Ae$を左イデアルとする。$A$は半単純であるから
  \[
  Ae=I\oplus J  
  \]
  となる左イデアル$J$が存在する。よって
  \[
  e=x+y
  \]
  となる$x\in I, y\in J$をとることができる。$x\in Ae$より
  \[
  x=ce,\quad c\in A  
  \]
  とおくと$xe=ce^2=ce=x$。よって
  \[
  x=xe=x^2+xy  
  \]
  だが、$xy\in J$かつ$I\cap J=0$より$xy=0$。同様に$yx=0$である。したがって$x^2=x$, $y^2=y$も導かれる。$e$は原始的なので$x=0$または$y=0$が成り立つが、これより$I=0$または$J=0$が従う。

  実際、$x=0$であったとして$m\in I$を$m=ae$とおけば
  \[
  m=a(x+y)=ay\in J  
  \]
  $I\cap J=0$より$m=0$
\end{proof}

したがって$G$の既約表現を求める問題は$\complex[G]$の原始的べき等元を求める問題に帰着された。具体的に原始的べき等元を見つけるのは難しいが、対称群の場合はYoung図形とのきれいな対応により構成することができる。次節にそのことを解説する。

最後にべき等元$e$を用いて$Ae$の形に書ける加群の間の準同型について考察する。

\begin{prop}\label{hom_of_cyclic_module}
  $A$を環とする。$e,f\in A$をべき等元とするとき、Abel群の同型として
  \[
  \Hom_A(Ae,Af)\simeq eAf
  \]
  が成り立つ。
\end{prop}

\begin{proof}
  $\phi\in\Hom_A(Ae,Af)$に対して、
  \[
  \phi(e)=af  
  \]
  とおくと、$e$はべき等元であるから
  \[
  \phi(e)=\phi(e^2)=e\phi(e)=eaf  
  \]
  よって$\phi\mapsto eaf$を考えればこれが同型を与える。
\end{proof}

\begin{notice}
  証明からわかる通り、$A$が体$K$上の多元環である場合$e$はべき等元である必要はなく、スカラー倍のずれが許容される。すなわち
  \[
  e^2=\lambda e,\qquad \lambda\in K
  \]
  となる$e$に対しても同様のことが成り立つ。
\end{notice}




\subsection{誘導表現}

部分群の表現が与えられたとき、それを元の群に拡張する方法について解説する。

\begin{defin}\label{ind_rep}
  $G$を群、$H$を$G$の部分群とする。$W$を$H$の表現とするとき、
  \[
  V=\complex[G]\otimes_{\complex[H]}W 
  \]
  は左$\complex[G]$加群の構造をもつ\footnote{
    ここで$\complex[G]$は右からの積で右$\complex[H]$加群とみなしていることに注意。}。$V$を$W$が誘導する$G$の表現といい
  \[
  V=\ind_H^GW  
  \]
  と書く。
\end{defin}

誘導表現は次の普遍性で特徴づけることができる。

\begin{theo}[誘導表現の普遍性]\label{univ_ind_rep}
  $H$を群$G$の部分群、$W$を$H$の表現とする。このとき$G$の表現$V$と$H$線形写像$\map{\iota}{W}{V}$が一意的に存在して、次の性質をもつ:
  \begin{quote}
    任意の$G$の表現$U$と$H$線形写像$\map{f}{W}{U}$が与えられたとき、$G$線形写像$\map{\overline{f}}{V}{U}$が一意的に存在して
    \[
    f=\overline{f}\circ\iota  
    \]
    が成り立つ
  \end{quote}
\end{theo}

\begin{proof}
  定義\ref{ind_rep}の$V$がこの性質を持つことを示す。$\map{\iota}{W}{V}$を
  \[
  \iota(x)=1\otimes x  
  \]
  で定めれば、($\complex[H]$上のテンソル積なので)$\iota$は$H$線形写像である。$\map{f}{W}{U}$を$H$線形写像とする。$\map{\overline{f}}{V}{U}$を
  \[
  \overline{f}(g\otimes x)=gf(x)
  \]
  を双線形に拡張して得られる写像とすれば、
  より、
  \[
  \overline{f}(g(\alpha\otimes x))=\overline{f}(g\alpha\otimes x)=g\alpha f(x)  =g\overline{f}(\alpha\otimes x)
  \]
  $\overline{f}$は$G$線形写像であり$f=\overline{f}\circ \iota$を満たす。 $\overline{f}$の一意性を示す。$G$線形写像$f':W\rightarrow U$も$f=f'\circ \iota$を満たしたとする。$G$線形性から
  \[
  f'(g\otimes x)=gf'(1\otimes x)=gf'(\iota(x))=gf(x)=\overline{f}(g\otimes x)  
  \]
  である。

  最後にこの性質をもつ$V$が一意的であることを示す。$G$の表現$V'$と$H$線形写像$\iota':W\rightarrow V'$がこの性質を満たしたとする。$\iota$の普遍性を用いれば、$\overline{\iota}:V\rightarrow V'$が存在して$\iota'=\overline{\iota}\circ\iota$が成り立つ。また、$\iota'$の普遍性を用いれば$\overline{\iota'}:V'\rightarrow V$が存在して$\iota=\overline{\iota'}\circ \iota'$が成り立つ。
\[
\xymatrix{
  W \ar[r]^{\iota} \ar[d]_{\iota'}& 
  V \ar@<0.5ex>@{.>}[ld]_{\overline{\iota}} \\
  V' \ar@<0.5ex>@{.>}[ru]_{\overline{\iota'}}
  }
\]
  $\overline{\iota}$, $\overline{\iota'}$が互いに逆の写像であることを示そう。$\overline{\iota'}\circ\overline{\iota}:V\rightarrow V$は$G$線形写像であり、
  \[
  (\overline{\iota'}\circ\overline{\iota})\circ\iota=
  \overline{\iota'}\circ(\overline{\iota}\circ\iota)=
  \overline{\iota'}\circ \iota'=\iota
  \]
  を満たす。しかし、$\id{V}\circ \iota=\iota$であるから、$\iota$の普遍性から
  \[
  \overline{\iota'}\circ\overline{\iota}=\id{V}
  \]
  でなければならない。同様に$\overline{\iota}\circ\overline{\iota'}=\id{V'}$である。
\end{proof}

定義\ref{ind_rep}は加群論的で簡単だが、どのような作用を考えているのかがわかりにくい。そこで線形代数的な誘導表現の定義もみておく。

\begin{defin}\label{ind_rep_2}
  $V$を$G$の表現、$W$を$V$の部分空間で$H$の表現であるとする。$R$を$G/H$の完全代表系とする。このとき$V$が$W$から誘導されているとは
  \[
  V=\bigoplus_{\sigma\in R}\sigma W  
  \]
  が成り立つことをいう。
\end{defin}

$\inv{\sigma}\tau\in H$ならば$\sigma W=\tau W$が成り立つのでこの定義は$R$の取り方によらない。$W$の基底を$e_1,\cdots,e_n$として、$R=\{\sigma_1,\cdots,\sigma_r\}$とすれば
\[
V=
\complex\sigma_1 e_1\oplus\cdots\oplus\complex\sigma_1 e_n
\oplus\cdots\oplus
\complex\sigma_r e_1\oplus\cdots\oplus\complex\sigma_r e_n
\]
$G$の元$g$は$G/H$に左からの積で置換作用する。すなわち、各$\sigma_i$に対して、
\[
g\sigma_i=\sigma_j h  
\]
となる$\sigma_j\in R$と$h\in H$が存在する。よってこのとき、$g\sigma_iW=\sigma_jW$、とくに
\begin{equation}\label{ind_action}
  g\sigma_ie_k=\sigma_j he_k  \quad\in\sigma_jW,\qquad \text{for all $k=1,\cdots,n$}  
\end{equation}

となる。$g$の$V$への作用は$G/H$への置換作用と$H$の$W$への作用を組み合わせたようなものである。また式(\ref{ind_action})から次が従う。

\begin{prop}[誘導表現の指標]\label{ind_char}
  $V=\ind_H^GW$のとき、
  \begin{align}
  \chi_V(g)&=\sum_{\substack{\sigma_i\in R \\ \inv{\sigma_i}g\sigma_i\in H}}\chi_W(\inv{\sigma_i}g\sigma_i)
  =\frac{1}{H}\sum_{\substack{\sigma\in G \\ \inv{\sigma}g\sigma\in H}}\chi_W(\inv{\sigma}g\sigma)
\end{align}
が成り立つ。
\end{prop}

\begin{proof}
  式(\ref{ind_action})より、$g$の作用の対角成分を考えるうえで$\sigma_i=\sigma_j$すなわち$i=j$となる部分だけ考えればよいことがわかる。このとき$\inv{\sigma_i}g\sigma_i\in H$であり、
  \[
  h=\inv{\sigma_i}g\sigma_i 
  \]
  であるから、あとは$h$の$W$への作用の対角成分の和をとればよいので前半の式がでる。後半については指標が類関数であること、$|\sigma H|=|H|$であることから従う。
\end{proof}

定義\ref{ind_rep_2}が定義\ref{ind_rep}と一致することを確かめておく。定義\ref{ind_rep_2}の$V$が普遍性(定理\ref{univ_ind_rep})を満たすことを示せば、一意性から従う。

$V=\bigoplus_{\sigma\in R}\sigma W$において、$R$として単位元$e$を含むものをとり、$W$を$eW$と同一視する。この同一視を$\iota$とすれば$\iota$は$H$線形写像である。$U$を任意の$G$の表現とし、$f:W\rightarrow U$を$H$線形写像とする。このとき$\overline{f}:V\rightarrow U$を
\[
\overline{f}(\sigma_i x)=\sigma_i f(x)
\]
によって定義すると、$\overline{f}$は$G$線形である。実際$g\in G$に対して、$g\sigma_i =\tau_j h$となる$\sigma_j\in R, h\in H$をとれば
\[
\overline{f}(g\sigma_i x)=\overline{f}(\sigma_j hx)=\sigma_jf(hx)=\sigma_jhf(x)=g\sigma_if(x)  
\]
また、
\[
\overline{f}\circ\iota(x)=\overline{f}(ex)=ef(x)=f(x)  
\]

%\ref{multiplicity}

\begin{eg}\label{ind_from_trivial}
  $H$を$G$の部分群とする。$X=G/H=\{H,g_1H,\cdots,g_nH\}$とする。$G$の$X$への左からの積による置換表現を考える。すなわち$V$を$X$を基底に持つ自由ベクトル空間として、$G$の左からの積を線形に拡張して$V$を$G$の表現とみなす。$V$の部分空間$W$を
  \[
  W=\complex H\subset V  
  \]
  とすれば$W$は$H$の自明な表現であり、$R=\{e,g_1,\cdots,e_n\}$とすれば
  \begin{align*}
  V&=\complex H\oplus \complex g_1H\oplus\cdots\oplus\complex g_nH\\
  &=\bigoplus_{\sigma\in R}\sigma\complex H\\
  &=\bigoplus_{\sigma\in R}\sigma W
  \end{align*}
  であるから$V=\ind_H^GW$である。すなわち、$H$の自明な表現から誘導される$G$の表現は$G/H$への置換表現である。
\end{eg}
\end{document}

\newpage
\documentclass{ltjsreport}
\RequirePackage{luatex85}
\usepackage[utf8]{inputenc}
\usepackage{enumerate}
\usepackage{amsthm}
\usepackage{amsfonts}
\usepackage{amsmath}
\usepackage{amssymb}
\usepackage{ytableau}
\usepackage{docmute}
\usepackage{mathtools}
\usepackage{xr}
\usepackage[all]{xy}



\theoremstyle{definition}
\newtheorem{defin}{定義}[subsection]
\newtheorem{theo}[defin]{定理}
\newtheorem{cor}[defin]{系}
\newtheorem{prop}[defin]{命題}
\newtheorem{lemm}[defin]{補題}
\newtheorem{notice}[defin]{注意}
\newtheorem{eg}[defin]{例}


\renewcommand{\labelenumi}{(\roman{enumi})}


\newcommand{\invlimit}{\mathop{\lim_{\longleftarrow}}}
\newcommand{\dirlimit}{\mathop{\lim_{\longrightarrow}}}
\newcommand{\ind}{\text{Ind}\:}
\newcommand{\Hom}{\text{Hom}}
\newcommand{\tr}{\text{tr}\:}
\newcommand{\id}[1]{\text{id}_{#1}}
\newcommand{\sgn}{\mathrm{sgn}}
\newcommand{\res}[1]{\text{Res}_{#1}}
\newcommand{\generated}[1]{\langle\:#1\:\rangle}
\newcommand{\im}{\text{Im }}
\newcommand{\rank}{\text{rank }}
\newcommand{\del}[2]{\frac{\partial #1}{\partial #2}}
\newcommand{\delsametwo}[2]{\frac{\partial^2 #1}{\partial #2^2}}
\newcommand{\delothertwo}[3]{\frac{\partial^2#1}{\partial#2\partial#3}}
\newcommand{\ddel}[2]{\frac{\partial}{\partial #2}#1}
\newcommand{\ddelsametwo}[3]{\frac{\partial^2}{\partial #2^2}#1}
\newcommand{\ddelothertwo}[3]{\frac{\partial^2}{\partial#2\partial#3}#1}
\newcommand{\simneq}{\not\simeq}
\newcommand{\transpose}[1]{^t\!#1}
\newcommand{\ie}{\text{i.e.}}
\newcommand{\inv}[1]{#1^{-1}}
\newcommand{\real}{\mathbb{R}}
\newcommand{\complex}{\mathbb{C}}
\newcommand{\integer}{\mathbb{Z}}
\newcommand{\quotient}{\mathbb{Q}}
\newcommand{\natnum}{\mathbb{N}}
\newcommand{\proj}{\mathbb{P}}
\newcommand{\tensor}[3]{#1\otimes_#2#3}
\newcommand{\map}[3]{#1:#2\rightarrow#3}
\newcommand{\aut}[2]{\mathrm{Aut}_{#1} (#2)}
\newcommand{\hommoph}[2]{\mathrm{Hom}_{#1}(#2)}
\newcommand{\gl}[1]{\mathrm{GL}(#1)}
\newcommand{\set}[2]{\left\{\:#1\:\middle|\:#2\:\right\}}
\newcommand{\pmat}[1]{\begin{pmatrix} #1
\end{pmatrix}}
\newcommand{\vmat}[1]{\begin{vmatrix} #1
\end{vmatrix}}
\newcommand{\br}{\vskip\baselineskip}


\begin{document}

\section{対称群の表現論}
\subsection{対称群の既約表現}

前節までに述べたことは有限群の表現論の一般論であり、具体的な群が与えられたときその表現を求める手法を提供しているわけではない。そこでこの節では対称群を例に取り上げ、既約表現の分類を行う。

$\mathcal{P}_n$を大きさ$n$のYoung図形のなす集合とする。既約表現の種類は共役類の数だけあったが、$G=\mathfrak{S}_n$の共役類と$\mathcal{P}_n$の元は1対1に対応することが知られている。$G$の既約表現は$\mathcal{P}_n$から自然に作ることができる。



\begin{defin}
  $\lambda\in\mathcal{P}_n$の各箱に$1$から$n$の各数字を重複なく書き入れた図を形$\lambda$のタブローという。
  $T$をタブローとし、$T$の$i$行目の箱に書かれている数字の集合を$H_i(T)$, 同様に$T$の$j$列目の箱に書かれている数字の集合を$V_j(T)$とする。
\end{defin}

\begin{defin}
  $T$を形$\lambda=(\lambda_1,\cdots,\lambda_s)$のタブローとする。$\sigma\in\mathfrak{S}_n$に対して、$\sigma T$を各数字を$\sigma$によって置換してできるタブローとする。
  \begin{itemize}
    \item 各$i$に対して$H_i(\sigma T)=H_i(T)$が成り立つなら$\sigma$を$T$の水平置換という。$T$の水平置換の全体は$G$の部分群をなす。これを$\mathcal{H}_T$と書き、$T$の水平置換群という。$\mathcal{H}_T=\mathfrak{S}(H_1(T))\times\cdots\times\mathfrak{S}(H_s(T))$である。
    \item 各$j$に対して$V_j(\sigma T)=V_j(T)$が成り立つなら$\sigma$を$T$の垂直置換という。$T$の垂直置換の全体は$G$の部分群をなす。これを$\mathcal{V}_T$と書き、$T$の垂直置換群という。$\mathcal{V}_T=\mathfrak{S}(V_1(T))\times\cdots\times\mathfrak{S}(V_{\lambda_1}(T))$である。
  \end{itemize}
\end{defin}

\begin{eg}\label{tableau_eg}
  形 \ydiagram{3,2} のタブロー$T=$ 
  \begin{ytableau}
    4&5&1\\
    3&2
  \end{ytableau} に対して、
  \begin{equation*}
    \mathcal{H}_T=\mathfrak{S}(\{1,4,5\})\times\mathfrak{S}(\{2,3\}),\qquad \mathcal{V}_T=\mathfrak{S}(\{3,4\})\times\mathfrak{S}(\{2,5\})
  \end{equation*}
  である。
\end{eg}

\begin{eg}\label{canonical_tableau}
  Young図形$\lambda=(\lambda_1,\cdots,\lambda_s)\in\mathcal{P}_n$に対して、$\lambda$の第1行に$1,2,\cdots,\lambda_1$を、$\lambda$の第2行に$\lambda_1+1,\lambda_1+2,\cdots,\lambda_1+\lambda_2$を、と続けてできるタブローを$\lambda$から定まる自然なタブローという。
  \[
  \text{例\ref{tableau_eg}のYoung図形の自然なタブローは }\begin{ytableau}
    1&2&3\\
    4&5
  \end{ytableau}  
  \]
\end{eg}

水平置換$\sigma$が垂直置換でもあるならば、$\sigma$の引き起こす各$H_i(T)$の置換は恒等置換でなければならない。したがって$\sigma=e$である。よって$\mathcal{H}_T\cap\mathcal{V}_T=\{e\}$が成り立つ。また$\mathcal{H}_{gT}=g\mathcal{H}_T\inv{g}$, $\mathcal{V}_{gT}=g\mathcal{V}_T\inv{g}$が成り立つ。実際
\begin{align*}
  \sigma\in\mathcal{H}_{gT}
  &\Leftrightarrow \sigma gT=gT\\
  &\Leftrightarrow \inv{g}\sigma gT=T\\
  &\Leftrightarrow \sigma\in g\mathcal{H}_T\inv{g} 
\end{align*}

群環$\complex[G]$の元$a_T,b_T,c_T$を
\[
a_T=\sum_{\sigma\in \mathcal{H}_T}\sigma,\qquad
b_T=\sum_{\tau\in \mathcal{V}_T}\sgn(\tau)\tau,\qquad c_T=a_Tb_T=\sum_{\sigma\in\mathcal{H}_T,\tau\in\mathcal{V}_T}\sgn(\tau)\sigma\tau
\]
によって定める。$c_T$をYoung対称子という。ここで$c_T$は$0$でないことに注意しておく。実際$c_T$の和に現れる$\sigma\tau$はすべて異なる元である。なぜならもし$\sigma\tau=\sigma'\tau'$, $\sigma,\sigma'\in\mathcal{H}_T$, $\tau,\tau'\in\mathcal{V}_T$ならば、$\mathcal{H}_T\cap\mathcal{V}_T=e$より$\sigma=\sigma'$, $\tau=\tau'$である。


\begin{theo}\label{sym_irr_rep}
  $\complex[G]$の左イデアル$\complex[G]c_T$は極小である。
\end{theo}

定理\ref{sym_irr_rep}を証明しよう。ポイントになるのは次の補題である。
\begin{lemm}\label{young_symmetrizer}
  $\alpha\in\complex[G]$が
  \begin{itemize}
    \item 任意の$\sigma\in\mathcal{H}_T$に対して$\sigma\alpha=\alpha$
    \item 任意の$\tau\in\mathcal{V}_T$に対して$\alpha\tau=\sgn(\tau)\alpha$
  \end{itemize}
  を満たすならば、$\alpha$は$c_T$のスカラー倍である。
\end{lemm}

\begin{proof}
  $\alpha=\sum_{g\in G}a_gg$を仮定を満たす元とする。仮定より$\sigma\in \mathcal{H}_T$に対して
  \[
  \alpha=\inv{\sigma}\alpha=\sum_{g\in G}a_g\inv{\sigma} g=\sum_{g\in G}a_{\sigma g}g
  \]
  よって
  \begin{equation}
    a_{\sigma g}=a_g  \label{a_sigmag}
  \end{equation}
  が成り立つ。また$\tau\in\mathcal{V}_T$に対しては
  \[
  \alpha=\sgn(\tau)\alpha\inv{\tau}=\sum_{g\in G}\sgn(\tau)a_gg\inv{\tau}=\sum_{g\in G}\sgn(\tau)a_{g\tau}{g}
  \]
  より
  \begin{equation}
    a_{g\tau}=\sgn(\tau)a_g  \label{a_gtau}
  \end{equation}
  が成り立つ。(\ref{a_sigmag}),(\ref{a_gtau})より$\sigma\tau\in\mathcal{H}_T\mathcal{V}_T$に対して
  \[
  a_{\sigma\tau}=\sgn(\tau)a_{e}  
  \]
  であることがわかる。よって
  \begin{equation}
  g\notin\mathcal{H}_T\mathcal{V}_T\implies a_g=0 \label{lemm:coefficient_zero} 
  \end{equation}
  を示せば$\alpha=a_ec_T$となって証明が完了する。$g$に関する条件$g\notin\mathcal{H}_T\mathcal{V}_T$について次の補題を示す。
  \begin{lemm}\label{lamm:117}
    $g\in\mathfrak{S}_n$について、$T$の同じ行にある任意の数字$i,j$(ただし$i\neq j$)が$gT$では異なる列にあるならば$g\in\mathcal{H}_T\mathcal{V}_T$が成り立つ。
  \end{lemm}

  \begin{proof}
    $T$のYoung図形$\lambda=(\lambda_1,\cdots,\lambda_r)$の高さ$r$に関する帰納法で示す。$r=1$ならば$\mathcal{H}_T=\mathfrak{S}_n$なので明らか。$r>1$とする。$T$の第1行にある数字に注目する。仮定から、これらは$gT$でそれぞれ異なる列に入っているので、適当に$gT$に垂直置換$\nu\in\mathcal{V}_{gT}$を施すことで$\nu gT$においても第1行に入っているようにできる。
    \[
    T=\quad\begin{ytableau}
      *(yellow)7&*(yellow)8&*(yellow)2\\
      3&5&4\\
      1&6
    \end{ytableau} \qquad \rightarrow \qquad
    gT=\quad\begin{ytableau}
      5&3&4\\
      *(yellow)2&1&*(yellow)8\\
      6&*(yellow)7 
    \end{ytableau} \qquad \rightarrow \qquad
    \nu gT=\quad\begin{ytableau}
      *(yellow)2&*(yellow)7&*(yellow)8\\
      5&1&4\\
      6&3
    \end{ytableau}
    \]
    すなわち
    \[
    H_1(T)=H_1(\nu gT)  
    \]
    が成り立つようにできる。このとき$\nu g$は$T$の第1行への水平置換$\sigma_1$と、
    $T$の第2行以下を取り出したタブロー$T'$への置換$g'$との積
    \[
    \nu g=\sigma_1g'
    \]
    で表される。$g'$は$T'$への置換とみなせば主張の条件をみたすから、帰納法の仮定により
    \[
    g'\in\mathcal{H}_{T'}\mathcal{V}_{T'}  
    \]
    である。$\mathcal{H}_{T'}\subset\mathcal{H}_T$, $\mathcal{V}_{T'}\subset\mathcal{V}_T$だから
    \[
    g' =\sigma_2\tau_2\in\mathcal{H}_T\mathcal{V}_T
    \]
    と書ける。ここで$\nu\in\mathcal{V}_{gT}=g\mathcal{V}_T\inv{g}$だから
    \[
    \nu=g\tau_3\inv{g},\qquad \tau_3\in\mathcal{V}_{T}  
    \]
    よって
    \[
    g=\sigma_1g'\inv{\tau_3}=\sigma_1\sigma_2\tau_2\inv{\tau_3}
    \]
    となるので示せた。
  \end{proof}

  補題\ref{young_symmetrizer}の証明に戻ろう。(\ref{lemm:coefficient_zero})を示せばよいのであった。$g\notin\mathcal{H}_T\mathcal{V}_T$であるのなら、上記の補題から$T$の同じ行になる異なる数字$i,j$であって$gT$では同じ列にあるものが存在する。よって$\sigma=(i, j)$とすれば$\sigma\in\mathcal{H}_T\cap\mathcal{V}_{gT}$である。
  $\mathcal{V}_{gT}=g\mathcal{V}_T\inv{g}$より$\sigma=g\tau\inv{g}$とおけば(\ref{a_sigmag}), (\ref{a_gtau})より
  \[
  a_{g}=a_{\sigma g}=a_{g\tau}=\sgn(\tau)a_g=-a_g 
  \]
  よって$a_g=0$
\end{proof}

\begin{prop}\label{square_of_young_sym}
  \[
  c_T^2=\frac{n!}{\dim_\complex(\complex[G]c_T)}c_T  
  \]
  が成り立つ。
\end{prop}

\begin{proof}
  $\sigma\in\mathcal{H}_T$, $\tau\in\mathcal{V}_T$に対して
  \[
  \sigma a_T=\sigma\sum_{g\in\mathcal{H}_T}g=\sum_{g\in\mathcal{H}_T}\sigma g=a_T  
  \]
  であり、
  \[
  b_T\tau=\sum_{g\in\mathcal{V}_T}\sgn(g)g\tau=\sgn{\tau}b_T  
  \]
  だから、補題\ref{young_symmetrizer}よりある$n_T\in\complex$で
  \[
  c_T^2=n_T c_T  
  \]
  となることはわかる。$n_T$を求めよう。準同型$\map{\phi}{\complex[G]}{\complex[G]}$を
  \[
  \phi(\alpha)=\alpha c_T  
  \]
  によって定める。任意の$g\in G$に対して、
  \[
  gc_T=g+\sum_{hk\in\mathcal{H}_T\mathcal{V}_T\setminus\{e\}}\sgn(k)ghk  
  \]
  となるから、$\phi$の対角成分はすべて$1$である。よって
  \[
  \tr\phi=\dim_\complex\complex[G]=n!  
  \]
  である。$\complex[G]$は半単純だから、
  \[
  \complex[G]=\complex[G]c_T\oplus W  
  \]
  となる左イデアル$W$をとる。すると
  \[
  \complex[G]c_T=\complex[G]c_T^2\oplus Wc_T=\complex[G]c_T\oplus Wc_T 
  \]
  より$Wc_T=0$である。したがって、
  \begin{align*}
    &\phi(\complex[G]c_T)\subset \complex[G]c_T\\
    &\phi(W)=0
  \end{align*}
  となることがわかる。よって
  \[
  \tr\phi=\tr\phi|_{\complex[G]c_T}  
  \]
  である。$\alpha\in\complex[G]$に対して
  \[
  \phi(\alpha c_T)=\alpha\phi(c_T)=n_T\alpha c_T
  \]
  だから、$\complex[G]c_T$は$\phi$の固有値$n_T$の固有空間の部分空間である。
  \[
  \tr\phi|_{\complex[G]c_T}=n_T\dim_\complex\complex[G]c_T 
  \]
  $c_T\neq 0$だから$\dim_\complex\complex[G]c_T\neq 0$,
  よって
  \[
  n_T=\frac{n!}{\dim_\complex\complex[G]c_T}  
  \]
\end{proof}

定理\ref{sym_irr_rep}の証明を述べる

\begin{proof}
  定理\ref{reverse_schur}より
  \[
  \dim_\complex\Hom(\complex[G]c_T,\complex[G]c_T)=1  
  \]
  を示せばよい。命題\ref{young_symmetrizer}より$c_T$は適当にスカラー倍してべき等元になる。よって命題\ref{hom_of_cyclic_module}より
  \[
  \Hom(\complex[G]c_T,\complex[G]c_T)=c_T\complex[G]c_T  
  \]
  である。任意の$c_T\alpha c_T\in c_T\complex[G]c_T$は補題\ref{young_symmetrizer}の仮定をみたすので
  \[
    c_T\alpha c_T=\mu c_T,\qquad\mu\in\complex
  \]
  と書ける。よって$\dim_\complex c_T\complex[G]c_T=1$である。
\end{proof}


Young対称子の定義において$a_T$, $b_T$の積の順序に本質的な違いはない。

\begin{prop}\label{reverse_young_sym}
  $b_T a_T=\tilde{c_T}$とおくと、$\complex[G]\tilde{c_T}\simeq \complex[G]c_T$が成り立つ。
\end{prop}

\begin{proof}
  $\map{\phi}{\complex[G]a_Tb_T}{\complex[G]b_Ta_T}$を
  \[
  \phi(xa_Tb_T)=xa_Tb_Ta_T  
  \]
  $\map{\psi}{\complex[G]b_Ta_T}{\complex[G]a_Tb_T}$を
  \[
  \psi(xb_Ta_T)=xb_Ta_Tb_T  
  \]
  とすれば
  \[
  \psi(\phi(xa_Tb_T))=\psi(xa_Tb_Ta_T)=xa_Tb_Ta_Tb_T=n_Txa_Tb_T  
  \]
  よって$\psi\circ\phi$は0でないスカラー倍写像なので$\phi$は単射、$\psi$は全射である。命題\ref{square_of_young_sym}とまったく同様に$\tilde{c_T}^2=\tilde{n_T}\tilde{c_T}$となる0でないスカラー$\tilde{n_T}$が存在することがわかる。よって$\phi$は同型である。
\end{proof}


\begin{prop}
  $\lambda\in\mathcal{P}_n$とする。$T,U$を$\lambda$に書かれたタブローとすると$\complex[G]c_T\simeq \complex[G]c_U$である。
\end{prop}

\begin{proof}
  このときある$g\in G$が存在して$U=gT$となるから、
  \[
  \mathcal{H}_U=g\mathcal{H}_T\inv{g},\qquad \mathcal{V}_U=g\mathcal{V}_T\inv{g}  
  \]
  よって
  \[
  c_U=a_Ub_U=ga_T\inv{g}gb_T\inv{g}=gc_T\inv{g}  
  \]
  である。
  \[
  \complex[G]c_U=\complex[G]gc_T\inv{g}=\complex[G]c_T\inv{g}  
  \]
  だから、
  \[
  \complex[G]c_T\simeq \complex[G]c_T\inv{g}  
  \]
  を示せばよい。$\map{\phi}{\complex[G]c_T}{\complex[G]c_T\inv{g}}$を
  \[
  \phi(\alpha c_T)=\alpha c_T\inv{g}  
  \]
  と置けば$\phi$は左$\complex[G]$加群の準同型で、$g$を右から書ける準同型が逆写像を与えるので、同型である。
\end{proof}


したがって、同じYoung図形に対しては$\complex[G]c_T$はタブロー$T$の取り方によらず同型である。そこで$\lambda\in\mathcal{P}_n$に対して、$\lambda$の自然なタブロー(例\ref{canonical_tableau})から定まるYoung対称子を$c_\lambda$とし、$S_\lambda=\complex[G]c_\lambda$とおく。





次の定理を証明することで、既約表現の分類は完成する。

\begin{theo}\label{young_and_irr_rep}
  $\lambda,\mu\in\mathcal{P}_n$とする。
  \[
  S_\lambda\simeq S_\mu 
  \]
  となるための必要十分条件は$\lambda=\mu$である
\end{theo}

\begin{proof}
  十分性は明らか。必要性を示す。$\lambda\neq\mu$であるとする。$S_\lambda, S_\mu$は既約表現なので、Schurの補題(補題\ref{schur_lem})より、
  \[
  \dim_\complex\Hom(S_\lambda,S_\mu)=0  
  \]
  を証明すればよいが、命題\ref{hom_of_cyclic_module}より、
  \[
  \Hom(S_\lambda,S_\mu)=c_\lambda\complex[G]c_\mu
  \]
  ゆえに、すべての$g\in G$に対して
  \[
  c_\lambda gc_\mu  =a_\lambda b_\lambda g a_\mu b_\mu = 0
  \]
  が成り立つことを示す。次の補題を示す。
  \begin{lemm}\label{lexicographical}
    $\mathcal{P}_n$に辞書式順序を入れ、$\lambda < \mu$であるとする。$\lambda, \mu$でその自然なタブローを表すものとする。このとき任意の$g\in G$に対して、$\mu$の同じ行にある数字$i,j$であって$g\lambda$でも同じ列にあるものが存在する。
  \end{lemm}

  \begin{proof}
    $\lambda=(\lambda_1,\cdots,\lambda_s)$, $\mu=(\mu_1,\cdots,\mu_t)$とおく。$t$についての帰納法で示す。

    $t=1$の場合$\lambda_1<\mu_1$となるから、$\lambda$の列数は$\mu_1$より少ない。よって鳩の巣原理を用いれば$1,2,\cdots,\mu_1$のうち、$g\lambda$の同じ列にあるペアが必ず存在することがわかる。

    $t>1$とする。$\lambda_1<\mu_1$である場合はまったく同様に鳩の巣原理から従う。$\lambda_1=\mu_1$かつ、$1,2,\cdots,\mu_1$が$g\lambda$ではすべて異なる列に存在するとする。このとき垂直置換$\tau\in\mathcal{V}_{g\lambda}$を施して
    \[
    H_1(\mu)=H_1(\tau g\lambda)=\{1,2,\cdots,\mu_1\}  
    \]
    が成り立つようにできる。そこで、$\mu$, $\tau g\lambda$の2行目以降をとりだしたタブロー$\mu'$, $(\tau g\lambda)'$を考える。すると$(\tau g\lambda)'<\mu'$であるから帰納法の仮定により$\mu'$の同じ行にある数字$i,j$であって$(\tau g\lambda)'$では同じ列にあるものが存在する。$i,j$が$(\tau g\lambda)'$の第$m$列にあるとする。$\tau$は垂直置換だから
    \[
    V_m(\tau g\lambda)=V_m(g\lambda)
    \]
    よって$i,j$は$g\lambda$の同じ列に存在する。
  \end{proof}

  定理\ref{young_and_irr_rep}の証明に戻る。補題から、$\nu=(i,j)$であって$\nu\in \mathcal{H}_\mu\cap\mathcal{V}_{\inv{g}\lambda}$となるものが存在する。よって
  \[
  \nu=\inv{g}\pi g,\qquad \pi\in \mathcal{V}_{\lambda}
  \]
  とおけば
  \begin{align*}
    c_\lambda g c_\mu
    &=a_\lambda b_\lambda g a_\mu b_\mu\\
    &=a_\lambda b_\lambda \sgn(\pi)\pi g a_\mu b_\mu\\
    &=a_\lambda b_\lambda \sgn(\pi)g\nu a_\mu b_\mu\\
    &=\sgn(\pi)a_\lambda b_\lambda g a_\mu b_\mu\\
    &=-c_\lambda g c_\mu
  \end{align*}
  よって
  \[
    c_\lambda g c_\mu=0
  \]
\end{proof}


\begin{eg}
  $\lambda=(n),\mu=(1,1,\cdots,1)\in\mathcal{P}_n$とする。このとき$\mathcal{H}_\lambda=\mathfrak{S}_n$, $\mathcal{V}_\lambda=e$だから、
  \[
  c_\lambda=\sum_{\sigma\in\mathfrak{S}_n}\sigma  
  \]
  また$\mathcal{H}_\mu=e$, $\mathcal{V}_\mu=\mathfrak{S}_n$だから、
  \[
  c_\mu=\sum_{\sigma\in\mathfrak{S}_n}\sgn(\sigma)\sigma  
  \]
  したがって$\lambda$の定める既約表現は自明な表現$1$であり、$\mu$の定める既約表現は置換の符号$\sgn$であるとわかる。
\end{eg}



\begin{eg}
  $G=\mathfrak{S}_3$とする。
  \[
  \lambda=\quad\ydiagram{2,1}  
  \]
  に対応するYoung対称子は
  \[
  c_\lambda=(e+(1,2))(e-(1,3))=e+(1,2)-(1,3)-(1,3,2)
  \]
  である。$c_\lambda$の定める既約表現が、例\ref{S3}で求めた既約表現$U$と一致することをたしかめる。
  \[
  c_\lambda^2=3c_\lambda  
  \]
  となるから、命題\ref{square_of_young_sym}より
  \[
  \dim_\complex\complex[G]c_\lambda=2  
  \]
  である。
  \[
  v=c_\lambda, \qquad u=(1,2,3)c_\lambda=-e+(1,3)-(2,3)+(1,2,3)
  \]
  とすれば、$\complex[G]c_\lambda=\complex v\oplus\complex v$であり、
  \begin{align*}
    &(1,2)v=v,\qquad (1,2)u=-v-u\\
    &(1,2,3)v=u,\qquad (1,2,3)u=-v-u
  \end{align*}
  だから、
  \begin{align*}
  &\tr e=\dim_\complex\complex[G]c_\lambda=2  \\
  &\tr(1,2)=0\\
  &\tr(1,2,3)=-1 
  \end{align*}
  となり、指標が一致している。
\end{eg}



\begin{lemm}\label{tensor_with_sgn_rep}
  $\map{\phi}{\complex[G]}{\complex[G]}$を
  \[
  \phi(g)=\sgn(g)g  
  \]
  を線形に拡張して定める。$\phi$は環準同型であり、対合である。$\varepsilon\in\complex[G]$に対して
  \[
  \complex[G]\varepsilon\otimes_\complex \complex_{\sgn}\simeq \complex[G]\phi(\varepsilon)
  \]
  が成り立つ。ここで、$\complex_{\sgn}$は$\complex_{\sgn}=\complex$であり、
  \[
  g\cdot\lambda=\sgn(g)\lambda,\qquad g\in G,\lambda\in\complex
  \]
  で定まる$\complex[G]$加群である(すなわち$\sgn$表現)。
\end{lemm}

\begin{proof}
  $\map{f}{\complex[G]\phi(\varepsilon)}{\complex[G]\varepsilon\otimes_\complex \complex_{\sgn}}$を
  \[
  f(x)=\phi(x)\otimes 1  
  \]
  で定めれば$g\in G$として
  \begin{align*}
    gf(x)
    &=g(\phi(x)\otimes 1)\\
    &=g\phi(x)\otimes \sgn(g)\\
    &=\sgn(g)g\phi(x)\otimes 1\\
    &=\phi(gx)\otimes 1\\
    &=f(gx)
  \end{align*}
  より$\complex[G]$加群の準同型である。任意の$y\otimes 1\in \complex[G]\varepsilon\otimes_\complex \complex_{\sgn}$に対して
  \[
  f(\phi(y))=y\otimes 1  
  \]
  となり、$\complex[G]\varepsilon\otimes_\complex \complex_{\sgn}$は$y\otimes 1$の形の元で生成されるから、$f$は全射である。
  \[
  \dim_\complex\complex[G]\varepsilon\otimes_\complex \complex_{\sgn}=\dim_\complex\complex[G]\varepsilon\cdot\dim_\complex\complex_{\sgn}=  \dim_\complex\complex[G]\varepsilon 
  \]
  より$f$は同型。
\end{proof}


\begin{eg}
  $\lambda\in\mathcal{P}_n$に対して、$\lambda$の行と列を反転させたものを双対Young図形といい$\lambda^*$と書く。
  \[
  \lambda=\quad\ydiagram{5,2,2}\quad\rightarrow\quad\lambda^*\quad=\ydiagram{3,3,1,1,1}
  \]
  このとき
  \[
  \complex[G]_{c_\lambda}\otimes_\complex\complex_{\sgn}\simeq \complex[G]_{c_{\lambda^*}}
  \]
  となることを示す。定義より、
  \[
  \mathcal{H}_{\lambda^*}=\mathcal{V}_{\lambda},\qquad
  \mathcal{V}_{\lambda^*}=\mathcal{H}_{\lambda}  
  \]
  であるから、
  \[
  a_{\lambda^*}=\phi(b_{\lambda}),\qquad b_{\lambda^*}=\phi(a_{\lambda})
  \]
  したがって
  \[
  c_{\lambda^*}=\phi(b_\lambda)\phi(a_\lambda)=\phi(b_\lambda a_\lambda)=\phi(\tilde{c_\lambda})  
  \]
  である。命題\ref{reverse_young_sym}より、
  \[
  \complex[G]c_\lambda\simeq \complex[G]\tilde{c_\lambda}  
  \]
  であるから、補題\ref{tensor_with_sgn_rep}より、
  \[
  \complex[G]c_{\lambda^*}\simeq \complex[G]c_\lambda\otimes_\complex\complex_{\sgn}  
  \]
\end{eg}

\begin{eg}\label{ind_from_horizontal_perm}
  $\lambda\in\mathcal{P}_n$に対して
  \[
  M_\lambda=\ind_{\mathcal{H}_\lambda}^{G}(\mathbf{1})  
  \]
  とする。ここで$\mathbf{1}$は$\mathcal{H}_\lambda$の自明な表現である。\ref{ind_from_trivial}より、$M_\lambda$は$G/\mathcal{H}_\lambda$の置換表現に他ならない。
  $M_\lambda=\complex[G]a_\lambda$となることを示す。
  $\phi:M_\lambda=\complex[G]\otimes_{\complex[H_\lambda]} \mathbf{1}\rightarrow \complex[G]a_\lambda$を
  \[
  \phi(g\otimes c)=cga_\lambda  
  \]
  $\complex[\mathcal{H}_\lambda]$双線形に拡張して定める。$h\in\mathcal{H}_\lambda$に対して
  \[
  \phi(gh\otimes c)=cgha_\lambda=cga_\lambda=\phi(g\otimes hc)
  \]
  だから$\phi$はwell-definedであり$\complex[G]$準同型である。逆に$\psi:\complex[G]a_\lambda\rightarrow\complex[G]\otimes_{\complex[H_\lambda]} \mathbf{1}$を
  \[
  \psi(ga_\lambda)=\frac{1}{|\mathcal{H}_\lambda|}ga_\lambda\otimes 1=  g\otimes 1
  \]
  によって定めれば$\phi,\psi$は互いに逆写像であるから、同型である。
  
  $\lambda$が1行のYoung図形の場合、$\mathcal{H}_\lambda=G$であるから、$M_\lambda$は自明な表現にほかならない。一方$\lambda$が1列のYoung図形の場合は$\mathfrak{H}_\lambda=1$であるから$M_\lambda$は正則表現である。
  
  この表現が後に重要になるので、$M_\lambda$に関する性質を一つ示しておく。
\end{eg}

\begin{theo}[Youngの規則]\label{young_rule}
  $\lambda\in\mathcal{P}_n$と、辞書式順序で$\lambda$より大きい$\mu\in\mathcal{P}_n$に対して正の整数$k_{\lambda\mu}$が存在して
  \[
  M_\lambda=S_\lambda\oplus \left(\bigoplus_{\mu>\lambda}S_\mu^{\oplus k_{\lambda\mu}}\right)
  \]
  $k_{\lambda\mu}$をKostka数という。
\end{theo}

\begin{proof}
  定理\ref{char_orthogonality}より、
  \[
  \dim_{\complex}\Hom(M_\lambda,S_\mu)=
  \left\{\begin{array}{cc}
    1 & \text{if $\mu=\lambda$}\\
    0 & \text{if $\mu<\lambda$}
  \end{array}\right.  
  \]
  を証明すればよい。例\ref{ind_from_horizontal_perm}より$M_\lambda=\complex[G]a_\lambda$であるから、\ref{hom_of_cyclic_module}より
  \[
  \Hom(M_\lambda,S_\mu)=\Hom(\complex[G]a_\lambda,\complex[G]c_\mu)=a_\lambda\complex[G]c_\mu 
  \]
  である。$\lambda=\mu$の場合、任意の$\alpha\in\complex[G]$に対して$a_\lambda\alpha c_\lambda$ は補題\ref{young_symmetrizer}の条件をみたすから、
  \[
  a_\lambda\complex[G]c_\lambda=\complex c_\lambda
  \]
  よって
  \[
  \dim_\complex\Hom(M_\lambda,S_\mu)=1  
  \]
  次に$\mu<\lambda$のとき、補題\ref{lexicographical}より、任意の$g\in G$に対して$\lambda$の同じ行にある文字$i,j$であって$g\mu$で同じ列にあるものが存在する。すなわち$\sigma=(i,j)$とおけば$\sigma\in \mathcal{H}_\lambda$かつ$\sigma\in\mathcal{V}_{g\mu}=g\mathcal{V}_{\mu}\inv{g}$である。そこで$\sigma=g\tau\inv{g}$, $\tau\in\mathcal{V}_\mu$とおけば、
  \begin{align*}
    a_\lambda g b_\mu
    &=a_\lambda\sigma g b_\mu\\
    &=a_\lambda g\tau b_\mu\\
    &=-a_\lambda g b_\mu
  \end{align*}
  ゆえに$a_\lambda\complex[G]b_\mu=0$であるから、
  \[
  a_\lambda\complex[G]c_\lambda\subset a_\lambda\complex[G]b_\mu=0
  \]
  よって示せた。
\end{proof}

Kostka数はSchur多項式を単項対称式の線形結合で表すときの係数にも現れる。すなわち
\[
s_\lambda=\sum_{\mu}k_{\lambda\mu}m_\mu  
\]
となることが知られている。また、その具体的な値については、$k_{\lambda\mu}$は形$\lambda$のウェイト$\mu$の半標準タブローの数に等しいことも知られている。

また、対称群の既約指標については、その値はすべて有理数であることが知られている(Frobeniusの指標公式)。


\end{document}


\newpage
\documentclass{ltjsreport}
\RequirePackage{luatex85}
\usepackage[utf8]{inputenc}
\usepackage{enumerate}
\usepackage{amsthm}
\usepackage{amsfonts}
\usepackage{amsmath}
\usepackage{amssymb}
\usepackage{ytableau}
\usepackage{docmute}
\usepackage{mathtools}
\usepackage{xr}
\usepackage[all]{xy}



\theoremstyle{definition}
\newtheorem{defin}{定義}[subsection]
\newtheorem{theo}[defin]{定理}
\newtheorem{cor}[defin]{系}
\newtheorem{prop}[defin]{命題}
\newtheorem{lemm}[defin]{補題}
\newtheorem{notice}[defin]{注意}
\newtheorem{eg}[defin]{例}


\renewcommand{\labelenumi}{(\roman{enumi})}


\newcommand{\invlimit}{\mathop{\lim_{\longleftarrow}}}
\newcommand{\dirlimit}{\mathop{\lim_{\longrightarrow}}}
\newcommand{\ind}{\text{Ind}\:}
\newcommand{\Hom}{\text{Hom}}
\newcommand{\tr}{\text{tr}\:}
\newcommand{\id}[1]{\text{id}_{#1}}
\newcommand{\sgn}{\mathrm{sgn}}
\newcommand{\res}[1]{\text{Res}_{#1}}
\newcommand{\generated}[1]{\langle\:#1\:\rangle}
\newcommand{\im}{\text{Im }}
\newcommand{\rank}{\text{rank }}
\newcommand{\del}[2]{\frac{\partial #1}{\partial #2}}
\newcommand{\delsametwo}[2]{\frac{\partial^2 #1}{\partial #2^2}}
\newcommand{\delothertwo}[3]{\frac{\partial^2#1}{\partial#2\partial#3}}
\newcommand{\ddel}[2]{\frac{\partial}{\partial #2}#1}
\newcommand{\ddelsametwo}[3]{\frac{\partial^2}{\partial #2^2}#1}
\newcommand{\ddelothertwo}[3]{\frac{\partial^2}{\partial#2\partial#3}#1}
\newcommand{\simneq}{\not\simeq}
\newcommand{\transpose}[1]{^t\!#1}
\newcommand{\ie}{\text{i.e.}}
\newcommand{\inv}[1]{#1^{-1}}
\newcommand{\real}{\mathbb{R}}
\newcommand{\complex}{\mathbb{C}}
\newcommand{\integer}{\mathbb{Z}}
\newcommand{\quotient}{\mathbb{Q}}
\newcommand{\natnum}{\mathbb{N}}
\newcommand{\proj}{\mathbb{P}}
\newcommand{\tensor}[3]{#1\otimes_#2#3}
\newcommand{\map}[3]{#1:#2\rightarrow#3}
\newcommand{\aut}[2]{\mathrm{Aut}_{#1} (#2)}
\newcommand{\hommoph}[2]{\mathrm{Hom}_{#1}(#2)}
\newcommand{\gl}[1]{\mathrm{GL}(#1)}
\newcommand{\set}[2]{\left\{\:#1\:\middle|\:#2\:\right\}}
\newcommand{\pmat}[1]{\begin{pmatrix} #1
\end{pmatrix}}
\newcommand{\vmat}[1]{\begin{vmatrix} #1
\end{vmatrix}}
\newcommand{\br}{\vskip\baselineskip}


\begin{document}

\section{表現環と対称関数環}

\subsection{続・対称多項式}
対称群の表現と対称多項式の間には深い関係がある。次節でそのことを解説するが、そのための準備として対称多項式に関してより詳しく解説する。以下正の整数$n$を固定し、$\Lambda^k_n$を$n$変数の$k$次斉次対称多項式のなす$\integer$加群とする。第1部の記号を復習すると、$n$行のYoung図形$\lambda$に対して
\[
m_\lambda=\sum_{\alpha\sim\lambda}x_1^{\alpha_1}\cdots x_n^{\alpha_n}  
\]
とし、に対して
\begin{align*}
  &e_k=m_{1^k}=\sum_{1\leq i_1<\cdots<i_k\leq n}x_{i_1}\cdots x_{i_k},\qquad(k=1,\cdots,n)\\
  &h_k=\sum_{\lambda\in\mathcal{P}_k}m_\lambda=\sum_{1\leq i_1\leq\cdots\leq i_k\leq n}x_{i_1}\cdots x_{i_k}\\
  &p_k=m_{(k)}=x_1^k+\cdots+x_n^k
\end{align*}
とするのであった。$e_k$や$h_k$に対しては、その母関数を考えることは有用である。すなわち
\begin{align}
  &E(t)=1+e_1t+e_2t^2+\cdots+e_nt^n=\prod_{i=1}^n(1+x_it) \label{gen_func_of_e}\\
  &H(t)=1-h_1t+h_2t^2+\cdots+(-1)^nh_nt^n+\cdots=\prod_{i=1}^n\frac{1}{1+x_it}
\end{align}
である。とくに$E(t)H(t)=1$であるので、$k=1,\cdots,n$のとき
\begin{equation}\label{e_to_h}
  e_k-h_1e_{k-1}+\cdots+(-1)^{k-1}e_1h_{k-1}+(-1)^kh_k=0   
\end{equation}
を得る。

\begin{prop}\label{det_formula}
  $k=1,\cdots,n$に対して
  \begin{align*}
    h_k=\vmat{
      e_1&e_2&e_3&\cdots&e_k\\
      1&e_1&e_2&\cdots&e_{k-1}\\
      0&1&e_1&\cdots&e_{k-2}\\
      \vdots&\vdots&\vdots&\ddots&\vdots&\\
      0&0&0&\cdots&e_1},\qquad
    e_k=\vmat{
      h_1&h_2&h_3&\cdots&h_k\\
      1&h_1&h_2&\cdots&h_{k-1}\\
      0&1&h_1&\cdots&h_{k-2}\\
      \vdots&\vdots&\vdots&\ddots&\vdots&\\
      0&0&0&\cdots&h_1
    }
  \end{align*}
\end{prop}

\begin{proof}
  まったく同様なので$h_k$の場合だけ示す。$e_1=h_1$であり、$k-1$までこの公式が成り立っていたとすると、
  \begin{align*}
    e_k&=h_1e_{k-1}-h_2e_{k-2}+\cdots+(-1)^{k-1}h_k \\
    &=h_1\vmat{
      h_1&h_2&\cdots&h_{k-1}\\
      1&h_1&\cdots&h_{k-2}\\
      \vdots&\vdots&\ddots&\vdots&\\
      0&0&\cdots&h_1
    }-h_2\vmat{
      h_1&\cdots&h_{k-1}\\
      \vdots&\ddots&\vdots&\\
      0&\cdots&h_1
    }+\cdots+(-1)^{k-1}h_k\\
    &=\vmat{
      h_1&h_2&h_3&\cdots&h_k\\
      1&h_1&h_2&\cdots&h_{k-1}\\
      0&1&h_1&\cdots&h_{k-2}\\
      \vdots&\vdots&\vdots&\ddots&\vdots&\\
      0&0&0&\cdots&h_1
    }
  \end{align*}
\end{proof}

次に、$\lambda\in\mathcal{P}_k$に対して
\begin{align*}
  &e_\lambda=e_{\lambda_1}\cdots e_{\lambda_n}\\
  &h_\lambda=h_{\lambda_1}\cdots h_{\lambda_n}\\
  &p_\lambda=p_{\lambda_1}\cdots p_{\lambda_n}
\end{align*}
とする。

\begin{prop}\label{various_basis}
  $\Lambda^k_n$の次の部分集合について
  \begin{enumerate}[(i)]
    \item $\{m_\lambda\}_\lambda$ ただし$\lambda$は大きさが$k$で$n$行
    \item $\{e_\lambda\}_\lambda$ ただし$\lambda$は大きさが$k$で$n$列
    \item $\{h_\lambda\}_\lambda$ ただし$\lambda$は大きさが$k$で$n$列
    \item $\{s_\lambda\}_\lambda$ ただし$\lambda$は大きさが$k$で$n$行
    \item $\{p_\lambda\}_\lambda$ ただし$\lambda$は大きさが$k$で$n$列
  \end{enumerate}
  (i)$\sim$(iv)は$\Lambda^k_n$の$\integer$上の基底をなし、(v)は$\quotient\otimes_{\integer}\Lambda^k_n$の基底をなす。
\end{prop}

\begin{proof}
  (i)は命題\ref{m_is_basis}の証明を斉次部分で考えればまったく同様である。(ii)については定理\ref{FT_of_sym}の証明において、任意の対称多項式$f$が
  \[
  e_1^{a_1}\cdots e_n^{a_n}
  \]
  で生成されていることを示したことからわかる。(iv)は\ref{schur}の証明を斉次部分で行えばよい。(iii)については、命題\ref{det_formula}より$\{h_\lambda\}$が$\{e_\lambda\}$を生成することがわかるが、ともに集合の濃度が等しいことから基底をなすことがわかる。(v)が基底をなすことを示そう。(v)が(iii)を生成することを示せばよい。
  \begin{align*}
    &\quad1+h_1t+h_2t^2+\cdots\\
    &=H(-t)\\
    &=\prod_{i=1}^n\frac{1}{1-x_it}\\
    &=\exp \left(\sum_{i=1}^n-\log(1-x_it)\right)\\
    &=\exp \left(\sum_{i=1}^n\sum_{r=1}^\infty\frac{x_i^rt^r}{r}\right)\\
    &=\exp \left(\sum_{r=1}^\infty\frac{p_r}{r}t^r\right)\\
    &=\prod_{r=1}^\infty\exp \left(\frac{p_r}{r}t^r\right)\\
    &=\prod_{r=1}^\infty \sum_{m_r=0}^\infty \frac{p_r^{m_r}}{m_r!\cdot r^{m_r}}t^{r\cdot m_r}\\
    &=
    \left(\sum_{m_1=0}^\infty \frac{p_1^{m_1}}{m_1!\cdot 1^{m_1}}t^{m_1}\right)\cdot
    \left(\sum_{m_2=0}^\infty \frac{p_2^{m_2}}{m_2!\cdot 2^{m_2}}t^{2 \cdot m_2}\right)\cdot
    \left(\sum_{m_3=0}^\infty \frac{p_3^{m_3}}{m_3!\cdot 3^{m_3}}t^{3 \cdot m_3}\right)\cdots
  \end{align*}
  となるから、Young図形$\lambda=(\lambda_1,\lambda_2,\cdots)$に対して
  \[
  z(\lambda)=\prod_{i}\lambda_i!\cdot i^{\lambda_i}  
  \]
  とおけば、最後の式は
  \begin{align*}
    &\quad\sum_{m_1,m_2,m_3,\cdots}\frac{p_1^{m_1}p_2^{m_2}p_3^{m_3}\cdots}{(m_1!\cdot 1^{m_1})(m_2!\cdot 2^{m_2})(m_3!\cdot 3^{m_3})\cdots}t^{m_1+2\cdot m_2+3\cdot m_3+\cdots}\\
    &=\sum_{\lambda}\frac{p_\lambda}{z(\lambda)}t^{|\lambda|}\\
    &=\sum_{k=0}^\infty \sum_{\lambda\in\mathcal{P}_{k}}\frac{p_\lambda}{z(\lambda)}t^k
  \end{align*}
  となる。よって
  \[
  h_k=\sum_{\lambda\in\mathcal{P}_{k}}\frac{p_\lambda}{z(\lambda)}
  \]
  が成り立つ。$n$列のYoung図形$\lambda$に対して
  \[
  h_\lambda=h_{\lambda_1}\cdots h_{\lambda_s},\qquad \lambda_i\leq n 
  \]
  とおくと、
  \[
  h_{\lambda_i}=\sum_{\mu_i\in\mathcal{P}_{\lambda_i}}\frac{p_{\mu_i}}{z(\mu_i)}  
  \]
  であるから
  \[
  h_\lambda=\sum_{\mu_1,\cdots,\mu_s}\frac{p_{\mu_1}\cdots p_{\mu_s}}{z(\mu_1)\cdots z(\mu_s)}  
  \]
  各$\mu_i$は$\lambda_i\leq n$の分割を与えているから、$p_{\mu_1}\cdots p_{\mu_s}$はたかだか$n$列のYoung図形に対応するべき和対称式である。よって(v)も基底を与える。
\end{proof}

証明中に現れた等式は重要なので再掲しておく。

\begin{prop}\label{p_to_h}
  正の整数$k,n$に対して
  \[
    h_k(x_1,\cdots,x_n)=\sum_{\lambda\in\mathcal{P}_{k}}\frac{p_\lambda(x_1,\cdots,x_n)}{z(\lambda)}
  \]
  が成り立つ。
\end{prop}

後に必要になる公式を用意しておく


\begin{lemm}[Cauchyの等式]\label{formal_power_series_relation}
  形式的べき級数$\prod_{i=1}^m\prod_{j=1}^n\frac{1}{1-x_iy_j}$は次と等しい。
  \begin{align*}
    &\text{(i):}\:\sum_{\lambda,\lambda_{n+1}=0}h_\lambda(x_1,\cdots,x_m)m_\lambda(y_1,\cdots,y_n),\qquad \text{ただし和は$n$行Young図形全体をわたる}\\
    &\text{(ii):}\:\sum_{\lambda}\frac{1}{z(\lambda)}p_\lambda(x_1,\cdots,x_m)p_\lambda(y_1,\cdots,y_n),\qquad \text{ただし和はすべてのYoung図形全体をわたる}\\
    &\text{(iii):}\:\sum_{\lambda}s_\lambda(x_1,\cdots,x_m)s_\lambda(y_1,\cdots,y_n),\qquad \text{ただし和はすべてのYoung図形全体をわたる}
  \end{align*}
\end{lemm}

\begin{proof}
  (i)を示す。
  \begin{align*}
    &\quad\prod_{i=1}^m\prod_{j=1}^n\frac{1}{1-x_iy_j}\\
    &=\prod_{j=1}^n H(y_j)\\
    &=\prod_{j=1}^n \left(
      \sum_{k_j=0}^\infty h_{k_j}(x_1,\cdots,x_m)y_j^{k_j}
      \right)\\
    &=\left(
      \sum_{k_1=0}^\infty h_{k_1}(x_1,\cdots,x_m)y_1^{k_1}
      \right)\cdot
      \left(
      \sum_{k_2=0}^\infty h_{k_2}(x_1,\cdots,x_m)y_2^{k_2}
      \right)
      \cdots
      \left(
      \sum_{k_n=0}^\infty h_{k_n}(x_1,\cdots,x_m)y_n^{k_n}
      \right)\\
    &=\sum_{k_1,\cdots,k_n}h_{k_1}\cdots h_{k_n}y_1^{k_1}  
      \cdots y_n^{k_n}\\
    &=\sum_{\lambda, \lambda_{n+1}=0}h_\lambda(x_1,\cdots,
      x_m)m_\lambda(y_1,\cdots,y_n)
  \end{align*}
  (ii)を示す。命題\ref{p_to_h}より
  \begin{align*}
  \prod_{i,j}\frac{1}{1-x_iy_j}
  &=1+h_1(\{x_iy_j\})+h_2(\{x_iy_j\})+\cdots\\
  &=1+\sum_{\lambda\in\mathcal{P}_1}\frac{1}{z(\lambda)}p_\lambda(\{x_iy_j\})+\sum_{\lambda\in\mathcal{P}_2}\frac{1}{z(\lambda)}p_\lambda(\{x_iy_j\})+\cdots\\
  &=\sum_{\lambda}\frac{1}{z(\lambda)}p_\lambda(\{x_iy_j\})\\
  &=\sum_{\lambda}\frac{1}{z(\lambda)}p_\lambda(x_1,\cdots,x_m)p_\lambda(y_1,\cdots,y_n)
  \end{align*}
  (iii)については、Robinson-Schensted-Knuth対応と呼ばれる対応を用いて証明される。詳細は付録を参照
\end{proof}











\subsection{表現環と対称関数環}

\begin{defin}
  可算無限個の変数をもつ形式的べき級数環$\integer[[x_1,x_2,\cdots]]$を考える。
  \[
  \mathfrak{S}=\set{\map{\sigma}{\natnum}{\natnum}}{\text{$f$は全単射で$f(n)\neq n$なる$n$が有限個}}  
  \]
  とする
  \footnote{
    $\mathfrak{S}$は対称群$\mathfrak{S}_n$と自然な包含$\iota:\mathfrak{S}_n\rightarrow\mathfrak{S}_{n+1}$のなす帰納系の帰納極限である。
  }。
  \[
  \Lambda=\set{f\in \integer[[x_1,x_2,\cdots]]}{\sigma f=f,\:(\text{for all }\sigma\in\mathfrak{S}),\:\text{$f$の単項式の次数は有界}}  
  \]
  $\Lambda$は$\integer[[x_1,x_2,\cdots]]$の部分環で対称関数環と呼ばれる。$\Lambda^k$を
  \[
  \Lambda^k=\set{f\in\Lambda}{\text{$f$の単項式の次数はすべて$k$}}  
  \]
  で定め、$\Lambda^k$の元を$k$次斉次対称関数という。
  \[
  \Lambda=\bigoplus_{k=0}^\infty\Lambda^k  
  \]
  より$\Lambda$は次数付き環の構造をもつ。
\end{defin}

ここで、$\Lambda$の定義において単項式の次数が有界であることを要請するのは自然である。実際、もし仮定しなければ$\Lambda$は$\Lambda^k$の直和にはならない。

\begin{eg}[単項対称関数]\label{monomial_sym_func}
  任意のYoung図形$\lambda=(\lambda_1,\cdots,\lambda_n)$に対して
  \[
  m_{\lambda}=\sum_{\alpha\sim\lambda}x_1^{\alpha_1}x_2^{\alpha_2}\cdots x_{n}^{\alpha_n}
  \]
  とする。ここで指数$\alpha$は、$\lambda$の置換になっているもの全体をわたる。すなわちある$\sigma\in\mathfrak{S}$が存在して$\alpha=\sigma\lambda$をみたすもの全体である。$m_\lambda$は対称関数である。対称多項式の場合と同様の議論で、$\Lambda^k$は$\{m_\lambda\}_{\lambda\in\mathcal{P}_k}$を基底に持つことがわかる。
\end{eg}

\begin{eg}[基本対称関数・完全対称関数]\label{elementary_func}
  \begin{align*}
  &e_k=m_{1^k}
  =\sum_{1\leq i_1<i_2<\cdots<i_k}x_{i_1}x_{i_2}\cdots x_{i_k}\\
  &h_k=\sum_{\lambda\in\mathcal{P}_k} m_\lambda
  =\sum_{1\leq i_1\leq i_2\leq \cdots\leq i_k}x_{i_1}x_{i_2}\cdots x_{i_k}
  \end{align*}
  をそれぞれ、基本対称関数, 完全対称関数という。また、任意のYoung図形$\lambda=(\lambda_1,\cdots,\lambda_n)$に対して
  \begin{align*}
    &e_\lambda=e_{\lambda_1}\cdots e_{\lambda_n}\\
    &h_\lambda=h_{\lambda_1}\cdots h_{\lambda_n}
  \end{align*}
  とする。
  \begin{align*}
    &e_1=x_1+x_2+x_3+\cdots\\
    &e_2=\sum_{i<j}x_ix_j
  \end{align*}
  である。
\end{eg}

\begin{eg}[べき和対称関数]\label{power_func}
  $(k)=(k,0,\cdots,0)$に対して
  \[
  p_{k}=m_{(k)}=x_1^k+x_2^k+\cdots
  \]
  とする。
\end{eg}

このように、対称関数はいままでみてきた対称多項式を自然に無限変数に拡張した概念であり、対称多項式で成り立っていた関係式が対称関数においても成立することが多い。このことは対称関数の$k$次斉次部分$\Lambda^k$が$k$次斉次対称多項式からの射影極限と考えることができることによる。$\Lambda_n^k$を$n$変数$k$次斉次対称多項式のなす$\integer$加群とする。$m\leq n$に対して線形写像$\map{\rho_{m,n}}{\Lambda^k_n}{\Lambda^k_m}$を
\[
\rho_{m,n}(f(x_1,\cdots,x_m,x_{m+1},\cdots,x_n))=f(x_1,\cdots,x_m,0,\cdots,0)  
\]
によって定める。ここで$\rho_{m,n}(f)$は実際に$m$変数の$k$次斉次対称多項式である\footnote{変数の置換と$0$を代入する操作が可換であることによる}。$l\leq m\leq n$に対して
\[
\rho_{l,m}\circ\rho_{m,n}=\rho_{l,n}  
\]
が成り立つから、$\{\Lambda^k_n,\rho_{m,n}\}$は射影系をなす。

\begin{prop}\label{sym_func_is_inverselimit}
  上の状況において、
  \[
    \Lambda^k=\mathop{\lim_{\longleftarrow}}\Lambda^k_n
  \]
  がなりたつ。
\end{prop}

\begin{proof}
  $\theta_n:\Lambda^k\rightarrow\Lambda^k_n$を$n+1$番目以降の変数を$0$にする写像とすれば、
  \[
  \rho_{m,n}\circ\theta_n=\theta_m  
  \]
  が成り立つから、射影極限の普遍性から線形写像
  \[
  \theta:\Lambda^k\rightarrow\mathop{\lim_{\longleftarrow}}\Lambda^k_n
  \]
  が誘導される。$\mathop{\lim_{\longleftarrow}}\Lambda^k_n$から$\Lambda^k$への写像$\varphi$は次のように定義する。$\mathop{\lim_{\longleftarrow}}\Lambda^k_n$の元$(f_n)_{n\in\integer_{>0}}$, ($f_n\in\Lambda^k_n$)に対して、$k$変数の部分に注目すると
  \[
  f_k=\sum_{\lambda\in\mathcal{P}_k}c_\lambda m_\lambda(x_1,\cdots,x_k)
  \]
  と一意的に表せるので
  \[
  \varphi((f_n)_{n\in\integer_{>0}})
  =\sum_{\lambda\in\mathcal{P}_k}c_\lambda m_\lambda
  \]
  と定める。ただし右辺の$m_\lambda$は例\ref{monomial_sym_func}の対称関数である。$\varphi$が$\theta$の逆写像であることを示そう。
  \begin{equation}\label{proj_of_monomial}
  \theta_n(m_\lambda)=
  \left\{\begin{array}{cc}
    m_\lambda(x_1,\cdots,x_n) & \text{if }\lambda_{n+1}=0\\
    0 & \text{otherwise}
  \end{array}\right.  
\end{equation}
  であるが、$\lambda$は$k$の分割であるので$n\geq k$において$\lambda_{n+1}=0$である。よって$n\geq k$ならば
  \begin{equation}\label{theta_n}
  \theta_n(\varphi((f_n)_{n\in\integer_{>0}}))
  =\sum_{\lambda\in\mathcal{P}_k}c_\lambda m_\lambda(x_1,\cdots,x_n)  
  \end{equation}
  が成り立つ。一方、
  \[
  f_n=\sum_{\lambda\in\mathcal{P}_k(n)}d_\lambda m_\lambda(x_1,\cdots,x_n)    
  \]
  とおくと$n\geq k$より$\mathcal{P}_k(n)=\mathcal{P}_k$だから
  \[
  f_n=\sum_{\lambda\in\mathcal{P}_k}d_\lambda m_\lambda(x_1,\cdots,x_n)    
  \]
  となる。よって$\rho_{k,n}(f_n)=f_k$と(\ref{theta_n})より
  \[
  f_n=\theta_n(\varphi((f_n)_{n\in\integer_{>0}}))
  \]
  次に$n<k$の場合、(\ref{proj_of_monomial})より
  \[
  \theta_n(\varphi((f_n)_{n\in\integer_{>0}}))  
  =\sum_{\lambda\in\mathcal{P}(k)}c_\lambda m_\lambda
  \]
  となるが、
  \[
  f_n=\sum_{\lambda\in\mathcal{P}_k}d_\lambda m_\lambda(x_1,\cdots,x_n)    
  \]
  とおけば$\rho_{n,k}(f_k)=f_n$より
  \[
  f_n=\theta_n(\varphi((f_n)_{n\in\integer_{>0}}))
  \]
  以上より
  \[
  \theta\circ\varphi=\text{id}
  \]
  がわかる。逆に任意の$f\in\Lambda^k$に対して
  \[
  f=\sum_{\lambda\in\mathcal{P}_k}c_\lambda m_\lambda  
  \]
  とおけば(\ref{proj_of_monomial})より
  \[
  \theta_k(f)=\sum_{\lambda\in\mathcal{P}_k}c_\lambda m_\lambda(x_1,\cdots,x_k)
  \]
  だから
  \[
  \varphi(\theta(f))=f  
  \]
  がわかる。
\end{proof}


\begin{notice}
  対称多項式環の射影極限をとっても対称関数環にはならないことに注意せよ。例えば対称多項式の列
  \[
    f=((1+x_1),\:(1+x_1)(1+x_2),\:(1+x_1)(1+x_2)(1+x_3),\:\cdots)
  \]
  を考えると、$f$は対称多項式の射影極限の元であるが、次数の有界性を満たさないので対称関数ではない。

  一方
  \[
  \integer[x_1,\cdots,x_n]^{\mathfrak{S}_n}=\bigoplus_{k=0}^\infty \Lambda^k_n  
  \]
  だから、対称関数環は射影極限が直和と可換でない例を与えている。
\end{notice}


\begin{eg}\label{monomial_sym_fun_as_lim}
  $n$行のYoung図形$\lambda$に対して
  \[
  \rho_{n,n+1}(m_\lambda(x_1,\cdots,x_n,x_{n+1}))=m_\lambda(x_1,\cdots,x_n)  
  \]
  が成り立つ。実際
  \begin{align*}
    \rho_{n,n+1}(m_\lambda(x_1,\cdots,x_n,x_{n+1}))
    &=\rho_{n,n+1}\left(
      \sum_{\alpha\sim\lambda}x_1^{\alpha_1}\cdots x_n^{\alpha_n}x_{n+1}^{\alpha_{n+1}}
      \right)\\
    &=\sum_{\substack{\alpha\sim\lambda \\ \alpha_{n+1}=0}}x_1^{\alpha_1}\cdots x_n^{\alpha_n}\\
    &=m_\lambda(x_1,\cdots,x_n)
  \end{align*}
  よって$k=|\lambda|$次対称多項式の列$(m_\lambda(x_1,\cdots,x_l))_{l\in\integer_{\geq n}}$は一つの対称関数を定めるが、これは単項対称関数$m_\lambda$に他ならない。
\end{eg}

\begin{eg}
  例\ref{monomial_sym_fun_as_lim}と命題\ref{various_basis}より$e_\lambda$, $h_\lambda$, $p_\lambda$,$s_\lambda$もすべて一つの対称関数を定める。$e_\lambda,h_\lambda,p_\lambda$の定める対称関数は、例\ref{elementary_func}と例\ref{power_func}に他ならない。また$s_\lambda$の定める対称関数はSchur関数という。
\end{eg}

命題\ref{various_basis}より、次が成り立つ。

\begin{prop}\label{various_symfunc_basis}
  $\Lambda^k$の次の部分集合について、
  \begin{enumerate}[(i)]
    \item $\{m_\lambda\}_{\lambda\in\mathcal{P}_k}$
    \item $\{e_\lambda\}_{\lambda\in\mathcal{P}_k}$
    \item $\{h_\lambda\}_{\lambda\in\mathcal{P}_k}$
    \item $\{s_\lambda\}_{\lambda\in\mathcal{P}_k}$
    \item $\{p_\lambda\}_{\lambda\in\mathcal{P}_k}$
  \end{enumerate}
  (i)$\sim$(iv)は$\Lambda^k$の$\integer$上の基底をなし、(v)は$\quotient\otimes_\integer{\Lambda^k}$上の基底をなす。特に$\lambda$の範囲をすべてのYoung図形全体に変えれば、これらは$\Lambda$または$\quotient\otimes_\integer{\Lambda}$の基底を与える。
\end{prop}

また、命題\ref{det_formula}や命題\ref{p_to_h}, 補題\ref{formal_power_series_relation}, Littlewood-Richardson規則などの関係式は、そのまま対称関数においても成立することがわかる。(変数の制限$\rho_{m,n}$は和や積と可換である)

\begin{prop}\label{various_relation}
  次の関係式が成り立つ\footnote{
    $e_k$の母関数$E(t)=\prod_{i=1}^\infty(1+x_it)$や$h_k$の母関数$H(t)=\prod_{i=1}^\infty\frac{1}{1-x_it}$を用いて補題\ref{formal_power_series_relation}と同じ計算で直接示してもよい。
  }
  \begin{align*}
    &\text{(i):}\:h_k=\vmat{
      e_1&e_2&e_3&\cdots&e_k\\
      1&e_1&e_2&\cdots&e_{k-1}\\
      0&1&e_1&\cdots&e_{k-2}\\
      \vdots&\vdots&\vdots&\ddots&\vdots&\\
      0&0&0&\cdots&e_1},\qquad
    e_k=\vmat{
      h_1&h_2&h_3&\cdots&h_k\\
      1&h_1&h_2&\cdots&h_{k-1}\\
      0&1&h_1&\cdots&h_{k-2}\\
      \vdots&\vdots&\vdots&\ddots&\vdots&\\
      0&0&0&\cdots&h_1
    }\\
    &\text{(ii):}\:h_k=\sum_{\lambda\in\mathcal{P}_k}\frac{p_\lambda}{z(\lambda)}\\
    &\text{(iii):}\:
    \sum_{\lambda}h_\lambda(x) m_\lambda(y)
    =\sum_{\lambda}\frac{1}{z(\lambda)}p_\lambda(x)p_\lambda(y)
    =\sum_{\lambda}s_\lambda(x)s_\lambda(y)
    =\prod_{i,j}\frac{1}{1-x_iy_j}\\
    &\text{(iv):}\:
    s_\lambda s_\mu=\sum_{\nu}\eta^{\nu}_{\lambda\mu}s_\nu,\qquad\text{ただし$\eta^{\nu}_{\lambda\mu}$はLittlewood-Richardson数}
  \end{align*}
\end{prop}

以下、$\Lambda$の係数を$\quotient$に拡大して考える。命題\ref{various_symfunc_basis}より、$\{s_\lambda\}$は$\Lambda$の基底をなすので、$\Lambda$に
\[
\generated{s_\lambda,s_\mu}=\delta_{\lambda\mu}  
\]
となるような内積を入れて考える。ただし$\delta_{\lambda\mu}$はKroneckerのデルタである。

\begin{prop}
  次が成り立つ
  \begin{enumerate}[(i)]
    \item $\generated{h_\lambda,m_\mu}=\delta_{\lambda\mu}$
    \item $\generated{p_\lambda,p_\mu}=\delta_{\lambda\mu}z(\lambda)$
  \end{enumerate}
\end{prop}

\begin{proof}
  $h_\lambda(x)=\sum_{\nu_1}a_{\lambda\nu_1}s_{\nu_1}(x)$, $m_\mu(y)=\sum_{\nu_2}b_{\mu\nu_2}s_{\nu_2}(y)$とおく。
  \begin{align*}
    h_\lambda(x)m_\mu(y)=\sum_{\nu_1,\nu_2}a_{\lambda\nu_1}b_{\mu\nu_2}s_{\nu_1}(x)s_{\nu_2}(y)
  \end{align*}
\end{proof}



次に対称群の表現全体から作られる環を導入する。

\begin{defin}
  $\mathfrak{S}_n$の表現の同値類全体で生成される自由Abel群を$\tilde{R_n}$とする。$D$を
  \[
  \set{[V\oplus W]-[V]-[W]\in R}{V,W\text{はそれぞれ$\mathfrak{S}_n$の表現}}  
  \]
  で生成される$\tilde{R_n}$の部分加群とし、$R_n=\tilde{R_n}/D$とする。$R_0=\integer$として$R=\bigoplus_{n=0}^\infty R_n$とおく。
  
  $\mathfrak{S}_n$, $\mathfrak{S}_m$の表現$V,W$に対して、
  \[
  [V]\circ [W]=[\ind_{\mathfrak{S}_n\times\mathfrak{S}_m}^{\mathfrak{S}_{n+m}}V\boxtimes W]
  \]
  と定める。ここで、$\mathfrak{S}_n\times\mathfrak{S}_m$は$\mathfrak{S}_n$の元を$n+1$から$n+m$を固定する置換と同一視し、$\mathfrak{S}_m$の元を$1$から$n$を固定する置換と同一視することで$\mathfrak{S}_n\times\mathfrak{S}_m$の部分群とみなしている。$\circ$を双線形に拡張することによって$R$は可換環の構造を持つ。$R$を対称群の表現環という。
\end{defin}

\begin{prop}
  $\circ$は実際に乗法を定め、$R$は次数付き可換環となる。
\end{prop}

\begin{proof}
  テンソル積と直和の可換性から、$\mathfrak{S}_n$の表現$V,V'$に対して
  \[
    \ind_{\mathfrak{S}_n\times\mathfrak{S}_m}^{\mathfrak{S}_{n+m}}(V\oplus V')\boxtimes W
    =(\ind_{\mathfrak{S}_n\times\mathfrak{S}_m}^{\mathfrak{S}_{n+m}}V\boxtimes W)
    \oplus
    (\ind_{\mathfrak{S}_n\times\mathfrak{S}_m}^{\mathfrak{S}_{n+m}}V'\boxtimes W)
  \]
  よって
  \[
  ([V]+[V'])\circ [W]=[V\oplus V']\circ [W]=[V]\circ[W]+[V']\circ[W]  
  \]
  より$\circ$は双線形である。$\circ$が可換であることは$\mathfrak{S}_n$の元と$\mathfrak{S}_m$の元が$\mathfrak{S}_{n+m}$において可換であることからわかる。

  乗法が結合的であることを示そう。すなわち
  \[
  ([V]\circ[W])\circ[U]=[V]\circ([W]\circ [U])  
  \]
  を示す。そのためには$\mathfrak{S}_n$, $\mathfrak{m}$, $\mathfrak{S}_l$それぞれの表現$V,W,U$に対して、2つの表現
  \[
  \ind_{\mathfrak{S}_{n+m}\times\mathfrak{S}_l}^{\mathfrak{S}_{n+m+l}}\left\{
    (\ind_{\mathfrak{S}_n\times\mathfrak{S}_m}^{\mathfrak{S_{n+m}}}V\boxtimes W)\boxtimes U
  \right\},\qquad
  \ind_{\mathfrak{S}_{n}\times\mathfrak{S}_{m+l}}^{\mathfrak{S}_{n+m+l}}\left\{
    V\boxtimes (\ind_{\mathfrak{S}_m\times\mathfrak{S}_l}^{\mathfrak{S}_{m+l}}W\boxtimes U)
  \right\}
  \]
  の指標が等しいことをみればよい。
  $
    \ind_{\mathfrak{S}_{n+m}\times\mathfrak{S}_l}^{\mathfrak{S}_{n+m+l}}\left\{
      (\ind_{\mathfrak{S}_n\times\mathfrak{S}_m}^{\mathfrak{S_{n+m}}}V\boxtimes W)\boxtimes U
    \right\} 
  $ 
  の指標を$\chi_A$とおき、$\ind_{\mathfrak{S}_n\times\mathfrak{S}_m}^{\mathfrak{S_{n+m}}}V\boxtimes W$の指標を$\chi_{A'}$とおく。誘導指標の公式(命題\ref{ind_char})より、
  \begin{align*}
    \chi_A(g)
    &=\frac{1}{(n+m)!\:l!}
    \sum_{\substack{\sigma\in\mathfrak{S}_{n+m+l} \\ \inv{\sigma}g\sigma\in\mathfrak{S}_{n+m}\times\mathfrak{S}_l}}
    \chi_{A'}((\inv{\sigma}g\sigma)|_{n+m})\chi_U((\inv{\sigma}g\sigma)|_l)
  \end{align*}
  ここで、$(\inv{\sigma}g\sigma)|_{n+m}$は$\inv{\sigma}g\sigma$の最初の$n+m$文字の置換への制限である。
  $g$と$\inv{\sigma}g\sigma$の置換の型は同じだから、$\inv{\sigma}g\sigma\in\mathfrak{S}_{n+m}\times\mathfrak{S}_l$と、$g\in\mathfrak{S}_{n+m}\times\mathfrak{S}_l$は同値である。よって、
  \begin{align*}
    \chi_A(g)=\left\{\begin{array}{lc}
      \frac{(n+m+l)!}{(n+m)!\:l!}\chi_{A'}(g|_{n+m})\chi_U(g|_l) & \text{if $g\in\mathfrak{S}_{n+m}\times\mathfrak{S}_l$}\\
      0 & \text{otherwise}
    \end{array}\right.
  \end{align*}
  となる。
  同様の議論で、
  \begin{align*}
  \chi_{A'}(g|_{n+m})
  &=\frac{1}{n!\:m!}
  \sum_{ \substack{t\in\mathfrak{S_{n+m}} \\ \inv{t}g|_{n+m}t\in\mathfrak{S}_n\times\mathfrak{S}_m} }
  \chi_V((\inv{t}gt)_n)\chi_W((\inv{t}gt)_m) \\
  &=\left\{\begin{array}{lc}
    \frac{(n+m)!}{n!\:m!}\chi_V(g|_n)\chi_W(g|_m) & \text{if $g\in\mathfrak{S}_n\times\mathfrak{S}_m\times\mathfrak{S}_l$}\\
    0 & \text{otherwise}
    \end{array}\right.
  \end{align*}
  となる。結局、
  \begin{equation}\label{char_A}
  \chi_A(g)=\left\{\begin{array}{lc}
    \frac{(n+m+l)!}{n!\:m!\:l!}\chi_V(g|_n)\chi_W(g|_m)\chi_U(g|_l) & \text{if $g\in\mathfrak{S}_n\times\mathfrak{S}_m\times\mathfrak{S}_l$}\\
    0 & \text{otherwise}
  \end{array}\right.  
\end{equation}
  で、$n,m,l$について対称な形となった。同様に
  $
  \ind_{\mathfrak{S}_{n}\times\mathfrak{S}_{m+l}}^{\mathfrak{S}_{n+m+l}}\left\{
    V\boxtimes (\ind_{\mathfrak{S}_m\times\mathfrak{S}_l}^{\mathfrak{S}_{m+l}}W\boxtimes U)
  \right\}
  $
  の指標を計算すると、(\ref{char_A})となることがわかる。
\end{proof}

$R$の係数を$\quotient$に拡大して考える。$[V],[W]\in R_n$に対して
\[
\generated{[V],[W]}=\generated{\chi_V,\chi_W}  
\]
と定義し、$\generated{,}$を$R$に双線形に拡張し内積を入れる。本節の主定理の一つが次の定理である。

\begin{theo}\label{rep_ring_and_func_ring}
  $\varphi:\Lambda\rightarrow R$を$\varphi(h_\lambda)=[M_\lambda]$によって定めると$\varphi$は内積を保つ環の同型
  であり、
  \[
  \varphi(s_\lambda)=[S_\lambda]
  \]
  が成り立つ。ただし$M_\lambda$は例\ref{ind_from_horizontal_perm}の誘導表現, $S_\lambda$は$\lambda$で定まる対称群の既約表現である。
\end{theo}

\begin{proof}
  $\{h_\lambda\}$は$\Lambda$の基底だから$\varphi$は$\quotient$線形写像として定義できることに注意する。また、Youngの規則(定理\ref{Young_rule_for_rep})より
  \[
  [M_\lambda]=[S_\lambda]+\sum_{\mu>\lambda}k_{\lambda\mu}[S_\mu]
  \]
  となるから、$R_n$の基底$\{[S_\lambda]\}$に関して$\{[M_\lambda]\}$を行列表示したとき、対角成分がすべて1の上三角行列になる。すなわち$\{[M_\lambda]\}$は$R_n$の基底となることがわかる。したがって$\varphi$は少なくとも線形同型であることがわかる。

  $\varphi$が環準同型であることを示す。$h_\lambda=h_{\lambda_1}\cdots h_{\lambda_r}$だから、
  \[
  [M_\lambda]=[M_{(\lambda_1)}]\circ\cdots\circ[M_{(\lambda_r)}]  
  \]
  となることを示せばよい。$n=|\lambda|$とおく。各$M_{(\lambda_i)}$について、$(\lambda_i)$は1行のYoung図形だから$M_{(\lambda_i)}$は$\mathfrak{S}_{\lambda_i}$の自明な表現である。よって$M_{(\lambda_i)}\boxtimes M_{(\lambda_j)}$は$\mathfrak{S}_{\lambda_i}\times\mathfrak{S}_{\lambda_j}$の自明な表現だから、
  \[
  [M_{(\lambda_i)}]\circ[M_{(\lambda_j)}]
  =[\ind_{\mathfrak{S}_{\lambda_i}\times\mathfrak{S}_{\lambda_j}}^{\mathfrak{S}_{\lambda_i+\lambda_j}}\mathbf{1}]
  =[M_{(\lambda_i,\lambda_j)}]
  \]
  積は可換かつ結合的だから
  \[
  [M_\lambda]=[M_{(\lambda_1)}]\circ\cdots\circ[M_{(\lambda_r)}]  
  \]
  
  次に$\varphi$が内積を保つことを示す。
\end{proof}



この定理によって対称群の表現に関する性質を、対称関数の知識を使って調べることができる。

\begin{cor}
  $\mathfrak{S}_n$の既約表現$S_\lambda$, $\mathfrak{S}_m$の既約表現$S_\mu$について
  \[
  \ind_{\mathfrak{S}_n\times\mathfrak{S}_m}^{\mathfrak{S}_{n+m}}S_\lambda\boxtimes S_\mu =\bigoplus_{\nu}S_\nu^{\oplus\eta^{\nu}_{\lambda\mu}} 
  \]
  が成り立つ。ただし$\eta^{nu}_{\lambda\mu}$はLittlewood-Richardson数である。
\end{cor}

\begin{cor}
  $k_{\lambda\mu}$をKostka数とする。
  \[
  M_\lambda=S_\lambda\oplus \left(\bigoplus_{\mu>\lambda}S_\mu^{\oplus k_{\lambda\mu}}\right)  
  \]
  が成り立つ。
\end{cor}

\begin{proof}
  系\ref{Young_rule_for_poly}, 定理\ref{Young_rule_for_rep}より
\end{proof}

\end{document}

\newpage
\documentclass{ltjsreport}
\RequirePackage{luatex85}
\usepackage[utf8]{inputenc}
\usepackage{enumerate}
\usepackage{amsthm}
\usepackage{amsfonts}
\usepackage{amsmath}
\usepackage{amssymb}
\usepackage{ytableau}
\usepackage{docmute}
\usepackage{mathtools}
\usepackage{xr}
\usepackage[all]{xy}



\theoremstyle{definition}
\newtheorem{defin}{定義}[subsection]
\newtheorem{theo}[defin]{定理}
\newtheorem{cor}[defin]{系}
\newtheorem{prop}[defin]{命題}
\newtheorem{lemm}[defin]{補題}
\newtheorem{notice}[defin]{注意}
\newtheorem{eg}[defin]{例}


\renewcommand{\labelenumi}{(\roman{enumi})}


\newcommand{\invlimit}{\mathop{\lim_{\longleftarrow}}}
\newcommand{\dirlimit}{\mathop{\lim_{\longrightarrow}}}
\newcommand{\ind}{\text{Ind}\:}
\newcommand{\Hom}{\text{Hom}}
\newcommand{\tr}{\text{tr}\:}
\newcommand{\id}[1]{\text{id}_{#1}}
\newcommand{\sgn}{\mathrm{sgn}}
\newcommand{\res}[1]{\text{Res}_{#1}}
\newcommand{\generated}[1]{\langle\:#1\:\rangle}
\newcommand{\im}{\text{Im }}
\newcommand{\rank}{\text{rank }}
\newcommand{\del}[2]{\frac{\partial #1}{\partial #2}}
\newcommand{\delsametwo}[2]{\frac{\partial^2 #1}{\partial #2^2}}
\newcommand{\delothertwo}[3]{\frac{\partial^2#1}{\partial#2\partial#3}}
\newcommand{\ddel}[2]{\frac{\partial}{\partial #2}#1}
\newcommand{\ddelsametwo}[3]{\frac{\partial^2}{\partial #2^2}#1}
\newcommand{\ddelothertwo}[3]{\frac{\partial^2}{\partial#2\partial#3}#1}
\newcommand{\simneq}{\not\simeq}
\newcommand{\transpose}[1]{^t\!#1}
\newcommand{\ie}{\text{i.e.}}
\newcommand{\inv}[1]{#1^{-1}}
\newcommand{\real}{\mathbb{R}}
\newcommand{\complex}{\mathbb{C}}
\newcommand{\integer}{\mathbb{Z}}
\newcommand{\quotient}{\mathbb{Q}}
\newcommand{\natnum}{\mathbb{N}}
\newcommand{\proj}{\mathbb{P}}
\newcommand{\tensor}[3]{#1\otimes_#2#3}
\newcommand{\map}[3]{#1:#2\rightarrow#3}
\newcommand{\aut}[2]{\mathrm{Aut}_{#1} (#2)}
\newcommand{\hommoph}[2]{\mathrm{Hom}_{#1}(#2)}
\newcommand{\gl}[1]{\mathrm{GL}(#1)}
\newcommand{\set}[2]{\left\{\:#1\:\middle|\:#2\:\right\}}
\newcommand{\pmat}[1]{\begin{pmatrix} #1
\end{pmatrix}}
\newcommand{\vmat}[1]{\begin{vmatrix} #1
\end{vmatrix}}
\newcommand{\br}{\vskip\baselineskip}


\begin{document}

\section{一般線形群の表現とSchur-Weyl双対性}

前節までで対称群の既約表現に関して解説してきたが、次に対称群と表現論的に関係の深い一般線形群の表現について解説する。とくに多項式表現と呼ばれる表現のクラスが、Schur-Weyl双対性を通して対称群の表現と密接にかかわりあっている。



\subsection{Lie群とLie代数}
Schur-Weyl双対性のために若干のLie群・Lie代数の知識を用いる。

\begin{defin}[Lie群]
  $G$を群であり複素多様体でもあるとする。$G$の演算$\cdot:G\times G\rightarrow G$, および逆元を取る写像$\inv{}:G\rightarrow G$がともに正則であるとき、$G$を(複素)Lie群という。Lie群の間の写像$f:G\rightarrow H$について、$f$が群準同型かつ正則であるとき$f$をLie群の準同型という。
\end{defin}

\begin{eg}
  $\complex$ベクトル空間$V$に対して一般線形群$\gl(V)$は行列の積に関してLie群である。実際、行列の積は成分の多項式であるし、逆行列は分母が$0$でない有理関数で表される。同様に$\text{SL}(V)$, $\text{U}(n)$, $\text{SU}(n)$もLie群である。
\end{eg}

\begin{defin}[Lie代数]
  $\mathfrak{g}$を$\complex$ベクトル空間とする。写像$[\:,\:]:\mathfrak{g}\times \mathfrak{g}\rightarrow \mathfrak{g}$が与えられており
  \begin{enumerate}[(i)]
    \item $[\:,\:]$は双線形
    \item $[x,x]=0$, (交代性)
    \item $[x,[y,z]]+[y,[z,x]]+[z,[x,y]]=0$, (Jacobiの恒等式)
  \end{enumerate}
  をみたすとき、$\mathfrak{g}$を(複素)Lie代数という。Lie代数の間の写像$f:\mathfrak{g}\rightarrow \mathfrak{h}$について、$f$が線形写像かつ$f([X,Y])=[f(X),f(Y)]$をみたすとき、$f$をLie代数の準同型という。
\end{defin}

\begin{eg}
  $\mathfrak{gl}(V)=\End(V)$とし、$X,Y\in\mathfrak{gl}(V)$に対して
  \[
  [X,Y]=XY-YX  
  \]
  とおくと$\mathfrak{gl}(V)$は複素Lie代数である。同様の演算で
  \begin{itemize}
    \item $\mathfrak{sl}(V)=\set{X\in\mathfrak{gl}(V)}{\tr(X)=0}$
    \item $\mathfrak{alt}(V)=\set{X\in\mathfrak{gl}(V)}{\transpose{X}=-X}$
  \end{itemize}
  などもLie代数である。
\end{eg}

\end{document}

\end{document}