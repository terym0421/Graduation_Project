\documentclass{ltjsreport}
\input{../setting.tex}

\begin{document}
\chapter{対称群と一般線形群の表現}

表現論とは群や多元環をベクトル空間への作用を通して研究する分野である。第2章では最も古典的な有限群の表現論の一般論(Maschkeの定理と指標理論, 群環)と対称群と一般線形群の表現について述べる。とくに対称群・一般線形群の表現は対称多項式と深いかかわりを持っている。

なお第2章で環は、乗法単位元をもつ必ずしも可換とは限らない環を指すとする。また、単に加群といったら環上の左加群を指しているとする。


\documentclass{ltjsreport}
\input{../../setting.tex}

\begin{document}



\section{有限群の表現論}
\subsection{既約表現とMaschkeの定理}

\begin{defin}
  $G$を群、$V$をベクトル空間とする。群準同型$\map{\rho}{G}{\gl{G}}$が与えられたとき、$(\rho,V)$を$G$の表現といい$V$を表現空間という。
  $\rho$や$V$のことを表現ということもある。
\end{defin} 

以下、本節ではベクトル空間と言ったら複素数体$\complex$上の有限次元ベクトル空間を指すものとし、群と言ったら有限群を指すものとする。

\begin{eg}
  $G$を群、$V=\complex$とする。すべての$g\in G$に対して$\rho(g)=\id{V}$とするとこれは表現になる。これを自明な表現という
\end{eg}

\begin{eg}\label{natural}
  $G=\mathfrak{S}_n$, $V=\complex^n$として$\sigma\in\mathfrak{S}_n$に対して
  \[
  \rho(\sigma)(a_1,\cdots,a_n)=(a_{\inv{\sigma}(1)},\cdots,a_{\inv{\sigma}(b)})  
  \]
  とするとこれは表現になる。
\end{eg}

\begin{eg}\label{regular}
  $G$を群、$V$を$G$の元を基底にもつ自由ベクトル空間とする。$g\in G$に対して
  \[
  \rho(g)\sum_{x\in G}a_xx=\sum_{x\in G}a_xgx  
  \]
  によって定めるとこれは表現になる。これを$G$の正則表現という
\end{eg}

文脈から明らかな場合や特に明示する必要がないとき、$\rho(g)x$のことをたんに$gx$と書く。表現論の基本的な問題は、$G$の考えうるあらゆる作用を分類することである。表現の分類の基準となるのは、次の定義である。

\begin{defin}
  $(\rho_1,V_1)$, $(\rho_2,V_2)$を$G$の表現とする。線形写像$\map{\varphi}{V_1}{V_2}$が
  \[
  \rho_2(g)\circ\varphi=\varphi\circ\rho_1(g),\quad \text{for all $g\in G$}
  \]
  をみたすとき、$\varphi$を$G$線形写像という。$G$線形写像の全体を$\Hom_G(V_1,V_2)$と書く。
\end{defin}

\begin{defin}
  $G$の表現$(\rho_1,V_1)$, $(\rho_2,V_2)$の間に同型な$G$線形写像があるとき、$(\rho_1,V_1)$と$(\rho_2,V_2)$同値な表現であるといい、
  \[
  \rho_1\simeq\rho_2  
  \]
  と書く。
\end{defin}

表現の同値は同値関係になる。したがって表現の分類はその同値類を求めることと言い換えられる。いきなりすべての表現を考えるのは難しいのでまずは「小さい表現」を考えたい。そのために、与えられた表現よりも小さい表現とは何かを定義する。

\begin{defin}
  $(\rho,V)$を$G$の表現とする。$V$の部分空間$W$が$G$不変であるとは
  \[
  \rho(g)W\subset W,\qquad \text{for all $g\in G$}  
  \]
  が成り立つことをいう。このとき$\map{\rho'}{G}{GL(W)}$を
  \[
  \rho'(g)=\rho(g)|_W  
  \]
  によって定義することができ、表現になる。$(\rho',W)$を$(\rho, V)$の部分表現という。定義より、すべての表現$(\rho, V)$は$0$と$V$を部分表現に持っていることに注意。これらを自明な部分表現という。
\end{defin}

\begin{defin}
  $G$の表現$(\rho, V)$が既約であるとは、$V$が非自明な部分表現を持たないことをいう。
\end{defin}

\begin{eg}\label{ker_im}
  $\map{f}{V}{W}$が$G$線形写像であるなら$\ker f\subset V$, $\im f\subset W$はともに$G$不変部分空間である。
\end{eg}

\begin{eg}\label{1dim_rep}
  すべての1次元表現は既約である。実際1次元のベクトル空間$V$の部分空間は$0$と$V$のみである。
\end{eg}

\begin{eg}
  例\ref{natural}の表現を考える。
  \[
  W=\set{(a_1,\cdots,a_n)\in V}{a_1+\cdots+a_n=0}  
  \]
  とすると、$W$は$G$不変である。
  \[
  v=(1,1,\cdots,1)\in V  
  \]
  とし$U=\generated{v}$とおくと
  \[
  \rho(g)v=v  
  \]
  だから$U$も$G$不変部分空間で、自明な表現と同値である。例\ref{1dim_rep}より$U$は既約である。
\end{eg}

与えられた表現から新しい表現を作る方法を導入しておく。
\begin{defin}
  $(\rho_1,V_1)$, $(\rho_2,V_2)$を$G$の表現とする。
  \begin{itemize}
    \item $\map{\rho_1\oplus\rho_2}{G}{GL(V_1\oplus V_2)}$を
    \[
    (\rho_1\oplus\rho_2)(g)(x,y)=(\rho_1(x),\rho_2(y))
    \]
    で定義する。これを$\rho_1$と$\rho_2$の直和という。

    \item $\map{\rho_1\otimes\rho_2}{G}{GL(V_1\otimes V_2)}$を
    \[
    (\rho_1\otimes\rho_2)(g)(x\otimes y)=\rho_1(x)\otimes\rho_2(y)
    \]
    で定義する。これを$\rho_1$と$\rho_2$の(内部)テンソル積という。

    \item $\map{\rho_1^*}{G}{GL(V^*)}$を
    \[
    \rho_1^*(g)(f)=f\circ (\rho_1(\inv{g}))
    \]
    で定義する。これを$\rho_1$の反傾表現という。
  \end{itemize}

  $(\rho_G,V)$を群$G$の表現、$(\rho_H,W)$を群$H$の表現とする。このとき$\rho_G\boxtimes\rho_H:G\times H\rightarrow GL(V\otimes W)$を
  \[
    \rho_G\boxtimes\rho_H(g,h)(x\otimes y)=\rho_G(g)(x)\otimes\rho_H(h)(y)
  \]
  で定義する。これを$\rho_G$と$\rho_H$の外部テンソル積という。
\end{defin}

  これらが実際に表現になっていることは容易にわかる。実は、有限群の複素数体上の有限次元表現は既約表現の直和に同値であることがわかる(系\ref{irr_decompose})。すなわち、表現の分類を考える上では本質的に最も小さい表現、既約表現のみを考えれば良いことがわかる。

  \begin{theo}[Maschkeの定理]\label{maschke}
    $V$を$G$の表現とする。任意の$V$の$G$不変部分空間$W$に対して、$V$の$G$不変部分空間$U$が存在し
    \[
    V= W\oplus U
    \]
    がなりたつ。
  \end{theo}

  \begin{proof}
    証明のポイントは$W$への$G$不変な射影を構成することである。$\map{p}{V}{W}$を$G$不変とは限らない何らかの射影とする。
    \[
    f(x)=\frac{1}{|G|}\sum_{h\in G} hp(\inv{h}x)
    \]
    と定めると、$f$は$G$線形な$W$への射影となる。実際任意の$g\in G$に対して
    \begin{align*}
      f(gx)&=\frac{1}{|G|}\sum_{h\in G} hp(\inv{h}gx)\\
      &=\frac{1}{|G|}\sum_{k\in G} gkp(\inv{k}x)\qquad \text{where $k=\inv{g}h$}\\
      &=gf(x)
    \end{align*}
    より$G$線形性は示された。また
    \begin{align*}
      f^2(x)&=f\left(\frac{1}{|G|}\sum_{g\in G}gp(\inv{g}x)\right)\\
      &=\frac{1}{|G|^2}\sum_{g,h\in G}ghp(\inv{h}p(\inv{g}x))
    \end{align*}
    ここで、$\map{p}{V}{W}$は射影で$W$は$G$不変だから$p(\inv{h}p(\inv{g}x))=\inv{h}p(\inv{g}x)$ゆえに
    \begin{align*}
      f^2(x)=\frac{1}{|G|^2}\sum_{g,h\in G}gp(\inv{g}x)=f(x)
    \end{align*}
    $f(W)\subset W$であり、任意の$W$の元$x$に対して
    \[
    f(x)=\frac{1}{|G|}\sum_{g\in G}gp(\inv{g}x)=\frac{1}{|G|}\sum_{g\in G}g\inv{g}x=x  
    \]
    だから$f$は$W$への射影である。したがって
    \[
    V=\im f\oplus \ker f=W\oplus \ker f  
    \]
    が成り立つが、$f$は$G$線形なので$\ker f$は$G$不変部分空間である(例\ref{ker_im})。
  \end{proof}


\begin{cor}\label{irr_decompose}
  $V$を$G$の表現とすると、既約表現$W_1,\cdots,W_r$が存在して
  \[
  V\simeq W_1\oplus\cdots\oplus W_r  
  \]
  が表現の同値として成り立つ。このことを$G$の表現の完全可約性という。
\end{cor}

\begin{proof}
  $\dim_\complex V$に関する帰納法で示す。$\dim_\complex V=1$なら$V$は既約だからよい。$\dim_\complex V>1$で$V$は可約であるとする。このとき
  $V$は非自明な部分表現$V_1$をもつが、定理\ref{maschke}より部分表現$U_1$で
  \[
  V=V_1\oplus U_1  
  \]
  となるものが存在する。$\dim_\complex V_1$, $\dim_\complex U_1<\dim_\complex V$だから帰納法の仮定により、
  \begin{align*}
    &V_1=W_1\oplus\cdots\oplus W_{s_1},\\
    &U_1=W_{s+1}\oplus\cdots\oplus W_{r},\qquad\text{各$W_i$は既約}
  \end{align*}
  と既約分解できる。したがって$V$も既約分解される。
\end{proof}

\begin{notice}
  定理\ref{maschke}は標数が群の位数と互いに素な任意の体上で成立する。実際証明中で$|G|$で割るシーンがあるが、それ以外体に依存する議論はしていない。しかし$G$の位数が無限の場合は成り立たない。例えば無限巡回群$\integer$の表現として
  \[
  n\quad\mapsto\quad\pmat{1 & n\\0 & 1}
  \]
  を考える。$\pmat{1 & n\\0 & 1}$の固有空間$V(1)$は$\integer$不変だが、$\integer$不変な補空間をもたない。
  
  \ref{maschke}の証明は$V$が無限次元であっても通用する\footnote{
    選択公理により、無限次元ベクトル空間においても任意の部分空間に対する補空間が存在し、それにより射影が得られる。
  }。しかし系\ref{irr_decompose}の証明は次元に関する帰納法を用いているので無限次元では通用しない。"有限個"の既約表現に分解できるということがポイントである。
  
\end{notice}





\subsection{指標理論}

次に既約表現の分類をする上で鍵となる指標の概念を導入する。

\begin{defin}
  $(\rho,V)$を$G$の表現とする。$\map{\chi_V}{G}{\complex}$を
  \[
  \chi_V(g)=\tr \rho(g)
  \]
  で定め、これを$V$の指標という。
\end{defin}

本節では
指標の直交関係(定理\ref{char_orthogonality})
を示すことが目標である。指標は類関数と呼ばれる群上の関数になっており、類関数のなすベクトル空間に特別な内積を入れるとこの内積に関して指標が正規直交基底をなす、というのが主張である。この系として、
\begin{itemize}
  \item 既約表現の個数は共役類の個数に等しい
  \item 既約表現の分類は既約指標の分類に帰着される
  \item 既約表現の次元に関する公式
\end{itemize}
といったさまざまな有用な事実が導かれる。

表現の各種の演算と指標との関係を見ておく

\begin{prop}\label{char_property}
  $V_1,V_2$を$G$の表現、対応する指標を$\chi_1,\chi_2$とする。
  \begin{enumerate}
    \item $\chi_{1\oplus 2}=\chi_1+\chi_2$
    \item $\chi_{1\otimes 2}=\chi_1\chi_2$
    \item $\chi_{1^*}=\overline{\chi_1}$
  \end{enumerate}
  が成り立つ
\end{prop}

\begin{proof}
  \begin{enumerate}
    \item $\tr(A\oplus B)=\tr(A)+\tr(B)$より従う
    \item $\tr(A\otimes B)=\tr(A)\tr(B)$より従う
    \item $\Hom(V,V)= V^*\otimes V$であり
    \footnote{$f_1\otimes v_1+\cdots+f_n\otimes v_n\in V^*\otimes V$に対して
    \[
    \phi(x)=f_1(x)v_1+\cdots+f_n(x)v_n
    \]
    によって$\phi\in\Hom(V,V)$を定めればこれが同型を与える。
    }
    $\tr:\Hom(V,V)\rightarrow \complex$は$f\otimes v\in V^*\otimes V$に対して
    \[
    \tr(f\otimes v)=f(v)
    \]
    で与えられることに注意する。$e_1,\cdots,e_n$を$V$の基底として$e_1^*,\cdots,e_n^*$をその双対基底とする。このとき$\rho^*(g)\in\Hom(V^*,V^*)=V\otimes V^*$は
    \begin{align*}
    \rho^*(g)
    &=e_1\otimes(\rho^*(g)e_1^*)+\cdots+e_n\otimes(\rho^*(g)e_n^*)  \\
    &=e_1\otimes(e_1^*\circ\rho(\inv{g}))+\cdots+e_n\otimes(e_n^*\circ\rho(\inv{g}))
    \end{align*}
    と表されるから
    \[
    \tr(\rho^*(g))=(e_1^*\circ\rho(\inv{g}))(e_1)+\cdots (e_n^*\circ\rho(\inv{g}))(e_n)=\tr(\rho(\inv{g}))=\tr(\rho(g)^{-1})
    \]
    である。$g$は有限位数だから$\rho(g)$はユニタリ行列である。よってその固有値$\lambda_1,\cdots,\lambda_n$はすべて絶対値が1なので
    \[
    \tr(\inv{\rho(g)})=\frac{1}{\lambda_1}+\cdots+\frac{1}{\lambda_n}=\overline{\lambda_1}+\cdots+\overline{\lambda_n}=\overline{\tr(\rho(g))}
    \]
    これで示せた
  \end{enumerate}
\end{proof}




指標の直交関係を示そう。まず、いくつか必要な補題を示す。
\begin{lemm}[Schurの補題]\label{schur_lem}
  $V$, $W$を$G$の既約表現とする。このとき
  \[
  \dim_\complex\Hom_G(V,W)=\left\{\begin{array}{cl}
    1 & \text{if $V\simeq W$ as $G$-representation}\\
    0 & \text{otherwise}
  \end{array}\right.
  \]
  が成り立つ。とくに$V=W$なら$f\in\Hom_G(V,V)$はスカラー写像である。
\end{lemm}

\begin{proof}
  先に後半の主張を示す。$\map{f}{V}{V}$を$G$線形写像とする。$f$の固有空間を$V(\lambda)$とすると、$V(\lambda)$は$G$不変である。実際、$x\in V(\lambda)$, $g\in G$に対して
  \[
  f(gx)=gf(x)=g(\lambda x)=\lambda gx  
  \]
  である。$V$は既約であり$V(\lambda)\neq 0$なので$V(\lambda)=V$よって
  \[
  f=\lambda\id{V}
  \]
  である。

  前半を示そう。$V\simeq W$とし$\varphi\in\Hom_G(V,W)$を$G$同型として固定する。任意の$f\in\Hom_G(V,W)$について、$\inv{\varphi}\circ f$は$V\rightarrow V$の$G$線形写像だから前半の結果より
  \[
  \inv{\varphi}\circ f=\lambda\id{V}
  \]
  と表される。すなわち
  \[
  f=\lambda\varphi  
  \]
  である。したがって$\Hom_G(V,W)=\generated{\varphi}$となる。

  $V\simneq W$の場合、$f\in\Hom_G(V,W)$について$V,W$の既約性から
  \[
  \ker f=0\text{または}V,\quad \im f=0\text{または}W   
  \]
  を得るが、$V\neq W$より$\ker f=V$, $\im f=0$すなわち$f=0$である。これで示せた。
\end{proof}

\begin{notice}
  定理\ref{schur_lem}の証明より$f\in\Hom_G(V,W)$は$V\simeq W$なら$0$または同型、$V\simneq W$なら$f=0$であることがわかる。こちらをSchurの補題と呼ぶ場合もある。
\end{notice}


\begin{lemm}\label{hom_representation}
  $(\rho,V),(\theta,W)$を$G$の表現とする。$\rho:G\rightarrow \gl{\Hom(V,W)}$を
  \[
  \rho(g)(f)=\theta(g)\circ f\circ \rho(\inv{g})
  \]
  とするとこれは表現となり、
  \[
  \chi_{\Hom(V,W)}=\overline{\chi_V}\chi_W  
  \]
  が成り立つ
\end{lemm}

\begin{proof}
  $\Hom(V,W)=V^*\otimes W$であることから
  $\rho^*(g)\otimes\theta(g)=\rho(g)$が成り立つこと
  を示せばよい。$f=v^*\otimes w\in V^*\otimes W$に対して、反傾表現およびテンソル表現の定義から
  \begin{align*}
    g(v^*\otimes w)=[v^*\circ (\rho(\inv{g}))]\otimes [\theta(g)w]
  \end{align*}
  となるが、これの定める線形写像は、$x\in V$として
  \begin{align*}
    ([v^*\circ (\rho(\inv{g}))]\otimes [\theta(g)w])x
    &=[v^*\circ (\rho(\inv{g}))]x\cdot \theta(g)w
  \end{align*}
  である。ここで$[v^*\circ (\rho(\inv{g}))]x$はスカラーなので
  \begin{align*}
    [v^*\circ (\rho(\inv{g}))]x\cdot \theta(g)w&=\theta(g)(v^*\circ (\rho(\inv{g}))x)w\\
    &=\theta(g)(v^*\otimes w(\rho(\inv{g})))\\
    &=(\theta(g)\circ f\circ \rho(\inv{g}))x\\
    &=\rho(g)x
  \end{align*}
  指標の公式は命題\ref{char_property}より従う
\end{proof}

\begin{lemm}\label{fix_dim}
  $V$を$G$の表現とし、$V^G$を$G$の固定点の集合とする。すなわち
  \[
  V^G=\set{v\in V}{\forall g\in G,\quad gv=v}  
  \]
  とする。このとき$V^G$は$V$の部分表現であり
  \[
  \dim V^G=\frac{1}{|G|}\sum_{g\in G}\chi_V(g)
  \]
  が成り立つ。とくに$V^G$は$G$の自明な表現の直和である。
\end{lemm}

\begin{proof}
  $\map{f}{V}{V}$を
  \[
  f(x)=\frac{1}{|G|}\sum_{g\in G}gx
  \]
  で定義すると$f$は射影になる。実際、$h\in G$として
  \begin{align*}
    f^2(x)&=\frac{1}{|G|}\sum_{g,h\in G}ghx\\
    &=\frac{1}{|G|}\sum_{k\in G}kx\\
    &=f(x)
  \end{align*}
  である。また
  \[
  hf(x)=\frac{1}{|G|}\sum_{g\in G}hgx=\frac{1}{|G|}\sum_{k\in G}kx=f(x)
  \]
  より$\im f=V^G$である。射影のトレースは像の次元に等しいので
  \[
  \tr(f)=\dim \im f=V^G  
  \]
  だが、
  \[
  \tr (f)=\frac{1}{|G|}\sum_{g\in G}\chi_V(g)  
  \]
  よって示せた。$V^G$が$G$の自明な表現の直和であることは、$V^G$の定義そのものである。
\end{proof}

\begin{eg}
  $\Hom(V,W)^G=\Hom_G(V,W)$である。実際$f\in \Hom(V,W)$の条件について
  \[
  \theta(g)\circ f \circ \rho(\inv{g})=f\Leftrightarrow f\circ \rho(g)=\theta(g)\circ f
  \]
  である。
\end{eg}




\begin{defin}
  関数$\map{f}{G}{\complex}$が
  \[
  f(\inv{g}xg)=f(x),\qquad\text{for all $g\in G$}  
  \]
  を満たすとき、$f$を類関数という。類関数全体を$C(G)$と置くと$C(G)$には自然に$\complex$ベクトル空間の構造が入る
\end{defin}

\begin{eg}
  一般に$\tr(AB)=\tr(BA)$が成り立つので、表現の指標は類関数である。
\end{eg}

\begin{eg}
  $G$の共役類を$C_1,\cdots,C_s$とし、$G$上の関数$\omega_i$を
  \[
  \omega_i(x)=\left\{\begin{array}{cl}
    1 & \text{if $x\in C_i$}\\
    0 & \text{otherwise}
  \end{array}\right.  
  \]
  で定めると$\omega_i$は類関数であり$\omega_1,\cdots,\omega_s$は$C(G)$の基底である。よって$\dim C(G)=s$である。
\end{eg}

\begin{defin}
  $\phi,\psi\in C(G)$に対して
  \[
  \generated{\phi,\psi}=\frac{1}{|G|}\sum_{g\in G}\overline{\phi(g)}\psi(g)  
  \]
  によって$\generated{\cdot,\cdot}:C(G)\times C(G)\rightarrow \complex$を定めると、これは$C(G)$上のHermite内積となる。$C(G)$にはいつもこの内積が入っているものとする。
\end{defin}

\begin{theo}[指標の直交関係]\label{char_orthogonality}
  $V,W$を$G$の既約表現とする。このとき
  \[
  \generated{\chi_V,\chi_W}=\left\{\begin{array}{cl}
    1 & \text{if $V\simeq W$ as $G$-representation}\\
    0 & \text{otherwise}
  \end{array}\right.  
  \]
  が成り立つ。
\end{theo}

\begin{proof}
  補題\ref{hom_representation}と補題\ref{fix_dim}および補題\ref{schur_lem}から
  \begin{align*}
    \generated{\chi_v,\chi_W}&=\frac{1}{|G|}\sum_{g\in G}\overline{\chi_V(g)}\chi_W(g)\\
    &=\frac{1}{|G|}\sum_{g\in G}\chi_{\Hom(V,W)}(g)\\
    &=\dim \Hom(V,W)^G\\
    &=\dim \Hom_G(V,W)\\
    &=\left\{\begin{array}{cl}
      1 & \text{if $V\simeq W$ as $G$-representation}\\
      0 & \text{otherwise}
    \end{array}\right.  
  \end{align*}
\end{proof}


\begin{cor}
  $G$の既約指標は有限個である。したがって$G$の既約表現は同値の違いを除いて有限個である。
\end{cor}

\begin{proof}
  定理\ref{char_orthogonality}より既約指標の集合は$C(G)$で一次独立である。$C(G)$は有限次元だから、既約指標は有限でなければならない
\end{proof}

\begin{cor}\label{irr_char_is_basis}
  $G$の既約指標$\chi_1,\cdots,\chi_r$は$C(G)$の正規直交基底をなす。したがって$r$は$G$の共役類の数に等しい。
\end{cor}

\begin{proof}
  正規直交であることは定理\ref{char_orthogonality}で示されたので、基底であること、すなわち次を示せばよい:
  \begin{quote}
    $f\in C(G)$が$\generated{\chi_i,f}=0$を各$i=1,\cdots, r$に対して満たせば$f=0$である
    \footnote{
    一般に内積空間$V$の正規直交系$v_1,\cdots,v_n$が、「$w\in V$が$\generated{w,v_i}=0$ならば$w=0$」を満たせば$v_1,\cdots,v_n$は$V$の基底になる。実際、任意の$x\in V$に対して$w=x-(\generated{x,v_1}v_1+\cdots+\generated{x,v_n}v_n)$と置けば$\generated{w,v_i}=0$をみたすから$w=0$
    }
  \end{quote}
  $f$が仮定をみたす類関数であるとする。各$i$について$\chi_i$を指標に持つ既約表現を$(\rho_i,V_i)$とおく。
  \[
  0=\generated{f,\chi_i}=\frac{1}{|G|}\sum_{g\in G}\overline{f(g)}\tr(\rho_i(g))  
  \]
  より、写像$\map{F_i}{V_i}{V_i}$を
  \[
  F_i=\sum_{g\in G}\overline{f(g)}\rho_i(g)  
  \]
  とおけば$\tr(F_i)=0$である。$F_i$は$G$線形写像である。実際$h\in G, x\in V_i$として
  \begin{align*}
  F_i(\rho_i(h)x)&=\sum_{g\in G}\overline{f(g)}\rho_i(gh)x\\
  &=\sum_{k\in G}\overline{f(k\inv{h})}\rho_i(k)x,\qquad\text{where $k=gh$}\\
  &=\sum_{k\in G}\overline{f(\inv{h}k)}\rho_i(k)x,\qquad\text{$f$は類関数}\\
  &=\sum_{g\in G}\overline{f(g)}\rho_i(hg)x,\qquad\text{where $g=\inv{h}k$}\\
  &=\rho_i(h)F(x)
  \end{align*}
  よって補題\ref{schur_lem}よりある$\lambda\in\complex$で
  \[
  F_i=\lambda\id{V}  
  \]
  となるが、$\tr(F_i)=0$だったから$\lambda=0$でなければならない。よって$F_i=0$であることがわかる。

  次に、$\theta:G\rightarrow \complex[G]$を$G$の正則表現とする。ただし$\complex[G]$は$G$を基底に持つ自由ベクトル空間である。定理\ref{maschke}より$\theta$はいくつかの既約表現の直和に同値である。よって
  \begin{equation}\label{theta}
  \theta=\rho_1^{\oplus m_1}\oplus\cdots\oplus\rho_r^{\oplus m_2}  
  \end{equation}
  とおく。写像$F:\complex[G]\rightarrow\complex[G]$を
  \[
  F=\sum_{g\in G}\overline{f(g)}\theta(g)  
  \]
  とすれば(\ref{theta})より
  \[
  F=\left(\sum_{g\in G}\overline{f(g)}\rho_1(g)\right)^{\oplus m_1}\oplus\cdots\oplus  \left(\sum_{g\in G}\overline{f(g)}\rho_r(g)\right)^{\oplus m_r}=F_1^{\oplus m_1}\oplus\cdots\oplus F_r^{\oplus m_r}=0
  \]
  よって$e$を$G$の単位元として
  \[
  0=Fe=\sum_{g\in G}\overline{f(g)}g
  \]
  $G$は一次独立だからすべての$g$について$f(g)=0$
\end{proof}

\begin{cor}\label{multiplicity}
  $(\rho,V)$を$G$の表現, $(\rho_1,W_1),\cdots,(\rho_r,W_r)$を$G$の既約表現の同値類の完全代表系とし、それぞれの対応する指標を$\chi,\chi_1,\cdots,\chi_r$とおく。
  \[
  \rho\simeq \rho_1^{\oplus m_1}\oplus\cdots\oplus\rho_r^{\oplus m_r}  
  \]
  とすると、
  \[
  m_i=\generated{\chi,\chi_i}=\dim_\complex\Hom_G(W_i,V)
  \]
  が成り立つ。$m_i$を$\rho$の$\rho_i$に関する重複度という。また
  \[
  V\simeq \bigoplus_{i=1}^rW_i\otimes \Hom_G(W_i,V)  
  \]
  がなりたつ。
\end{cor}

\begin{proof}
  系\ref{irr_char_is_basis}より前半は直ちに従う。後半を示す。
  まず、$\psi_i:W_i\otimes \Hom_G(W_i,V) \rightarrow V$を
  \[
  \psi_i(x\otimes f)=f(x)  
  \]
  を双線形に拡張して定め、$\psi:\bigoplus_{i=1}^rW_i\otimes \Hom_G(W_i,V)\rightarrow V$を$\psi=\oplus_{i=1}^r\psi_i$とする。
  
  $V$の既約表現への直和分解
  \[
  V=\bigoplus_{i=1}^rU^{(i)}_{1}\oplus\cdots\oplus U^{(i)}_{m_i},\qquad U^{(i)}_j\simeq W_i
  \]
  を1つ固定し、$G$同型$\phi^{(i)}_j:W_i\rightarrow U^{(i)}_j$を取る。$\phi^{(i)}_j\in\Hom_G(W_i,V)$であるから、\\$\phi^{(i)}:U^{(i)}_{1}\oplus\cdots\oplus U^{(i)}_{m_i}\rightarrow W_i\otimes\Hom_G(W_i,V)$を
  \[
  \phi^{(i)}(x_1,\cdots,x_{m_i})
  =\phi^{(i)-1}_1(x_1)\otimes \phi^{(i)}_1+\cdots+\phi^{(i)-1}_{m_i}(x_{m_i})\otimes \phi^{(i)}_{m_i}
  \]
  によって定め、$\phi:V\rightarrow \bigoplus_{i=1}^rW_i\otimes\Hom_G(W_i,V)$を$\phi=\oplus_{i=1}^r\phi^{(i)}$とおけば、これらは$G$線形であり
  \[
  \psi\circ\phi=\id{V}  
  \]
  となることがわかる。よって$\phi$は単射であり、
  \[
  dim_{\complex}V=\dim_{\complex}  \bigoplus_{i=1}^rW_i\otimes \Hom_G(W_i,V)
  \]
  だから、同型である。
\end{proof}

\begin{cor}
  表現の既約表現への直和分解は、同値の違いを除いて一意的である。
\end{cor}

\begin{notice}
  既約表現への分解は同値の違いを除いて一意的であるが、与えられた表現を既約表現の直和に分解する標準的な方法があるわけではない。
\end{notice}


\begin{cor}
  指標$\chi$が既約指標であるための必要十分条件は$\generated{\chi,\chi}=1$が成り立つことである
\end{cor}

\begin{proof}
  必要性は明らか。十分性を示す。$\chi=m_1\chi_1+\cdots+m_r\chi_r$とおくと
  \[
  \generated{\chi,\chi}=1  
  \]
  であるならば
  \[
  m_1^2+\cdots+m_r^2=1  
  \]
  ゆえにある$i$で$\chi=\chi_i$である。
\end{proof}

\begin{cor}[Schurの補題の逆]\label{reverse_schur}
  $V$を$G$の表現とする。$\dim_\complex\Hom_G(V,V)=1$であるならば$V$は既約表現である。
\end{cor}

\begin{proof}
  $\chi$を$V$の指標とするとき、補題\ref{hom_representation}, 補題\ref{fix_dim}より条件は
  \[
  \frac{1}{|G|}\sum_{g\in G}|\chi(g)|^2=1
  \]
  すなわち$\generated{\chi,\chi}=1$に他ならない。
\end{proof}



\begin{cor}\label{rep_and_char}
  $\rho_1,\rho_2$を$G$の表現、対応する指標を$\chi_1,\chi_2$とする。$\rho_1\simeq \rho_2$であるための必要十分条件は$\chi_1=\chi_2$が成り立つことである。
\end{cor}

\begin{proof}
  必要性は明らか。十分性を示す。$\theta_1,\cdots, \theta_r$を$G$の既約表現の同値類の完全代表系とし、対応する指標を$\psi_1,\cdots,\psi_r$とおく。定理\ref{maschke}より
  \begin{align*}
    &\rho_1=\theta_1^{\oplus m^{(1)}_1}\oplus\cdots\oplus\theta_r^{\oplus m^{(1)}_r}\\
    &\rho_2=\theta_1^{\oplus m^{(2)}_1}\oplus\cdots\oplus\theta_r^{\oplus m^{(2)}_r}
  \end{align*}
  と分解すれば
  \begin{align*}
    &\chi_1=m^{(1)}_1\psi_1+\cdots+m^{(1)}_r\psi_r\\
    &\chi_2=m^{(2)}_1\psi_1+\cdots+m^{(2)}_r\psi_r
  \end{align*}
  仮定から$\chi_1=\chi_2$であり、系\ref{irr_char_is_basis}より
  \[
  m^{(1)}_1=m^{(2)}_1,\cdots,m^{(1)}_r=m^{(2)}_r
  \]
  である。よって
  \[
  \rho_1\simeq\rho_2  
  \]
\end{proof}


\begin{prop}\label{dim_formula}
  $W_1,\cdots,W_r$を$G$の既約表現の同値類の完全代表系とする。
  \[
  |G|=\dim W_1^2+\cdots +\dim W_r^2  
  \]
  が成り立つ
\end{prop}

\begin{proof}
  $\theta$を$G$の正則表現とする。$\theta$の指標を$R$, $W_i$の指標を$\chi_i$とおく。系\ref{multiplicity}より$\theta$の$W_i$に関する重複度を$m_i$とおくと
  \[
  m_i=\generated{R,\chi_i}=\frac{1}{|G|}\sum_{g\in G}\overline{R(g)}\chi_i(g)
  \]
  ここで、
  \[
  R(g)=\tr(\theta(g))=\left\{\begin{array}{cl}
    0 & \text{if $g\neq e$}\\
    |G| & \text{if $g=e$}
  \end{array}\right.  
  \]
  だから
  \[
  m_i=\chi_i(e)=\dim W_i  
  \]
  よって
  \[
  R=\dim W_1\chi_1+\cdots +\dim W_r\chi_r  
  \]
  だから
  \[
  |G|=R(e)=\dim W_1^2+\cdots +\dim W_r^2  
  \]
\end{proof}

\begin{eg}\label{S3}
  $G=\mathfrak{S}_3$の既約指標を全て求めよう。$G$の共役類は
  \[
  e,\:(1,2),\:(1,2,3)  
  \]
  で代表される3つだから既約表現も3つある。またそれぞれの共役類の濃度は順に
  \[
  1,3,2  
  \]
  である。

  $1$を自明な表現とし、$\sgn:G\rightarrow \complex$を置換の符号とすると、$\sgn$は1次元の既約表現である。例\ref{natural}の直和因子として現れた表現を考える。すなわち$V=\complex^3$として$\map{\rho}{G}{\gl V}$を自然な置換による作用とする。このとき
  \[
  U=\set{(a_1,a_1,a_3)}{a_1+a_2+a_3=0}  
  \]
  が$V$の不変部分空間となり、
  \[
  \rho\simeq\rho_U\oplus 1  
  \]
  となるのであった。$\rho_U$が既約であることを示そう。$\rho_U$の指標を$\chi_U$とすると
  \[
  \chi_U(g)=\chi_V(g)-1=|\set{x\in\{1,2,3\}}{gx=x}|-1  
  \]
  であるから
  \[
  \generated{\chi_U,\chi_U}=\frac{1}{6}(1\cdot 2^2+3\cdot 0^2+2\cdot (-1)^2)=\frac{6}{6}=1 
  \]
  よって既約である。まとめると$G$の既約指標は次の3つである
  \[
  \begin{array}{c|c|c|c}
        &\quad e\quad & (1,2) & (1,2,3) \\ \hline
    1   &      1      &   1   &    1    \\ \hline
  \sgn  &      1      &   -1  &    1    \\ \hline
  \chi_U&      2      &   0   &    -1
  \end{array}
  \]
\end{eg}



\begin{eg}
  $G=\mathfrak{S}_4$の既約指標を全て求めよう。$G$の共役類は
  \[
  e,\:(1,2),\:(1,2,3),\:(1,2,3,4),\:(1,2)(3,4)  
  \]
  で代表される5つだから既約指標も5つある。またそれぞれの共役類の濃度は順に
  \[
  1,6,8,6,3  
  \]
  である。
  
  $\mathfrak{S}_3$と同様、1次元の既約表現として$1$と$\sgn$がある。再び
  例\ref{natural}の直和因子として現れた表現を考える。すなわち
  \[
  U=\set{(a_1,a_2,a_3,a_4)}{a_1+a_2+a_3+a_4=0}  
  \]
  への置換による作用$\rho_U$を考える。
  $\rho_U$が既約であることを示そう。$\rho_U$の指標を$\chi_U$とすると
  \[
  \chi_U(g)=|\set{x\in\{1,2,3,4\}}{gx=x}|-1  
  \]
  であるから
  \[
  \generated{\chi_U,\chi_U}=\frac{1}{24}(1\cdot 3^2+6\cdot 1^2+8\cdot 0^2+6\cdot (-1)^2+3\cdot(-1)^2)=\frac{24}{24}=1 
  \]
  よって既約である。さらに$\sgn^2=1$より
  \[
  \generated{\chi_U\sgn,\chi_U\sgn}=1 
  \]
  であることがわかるので$\chi_U\sgn$も既約指標である。ここまでをまとめると次の表を得る。
  \[
    \begin{array}{c|c|c|c|c|c}
  
           & \quad e\quad & (1,2) & (1,2,3) & (1,2,3,4) & (1,2)(3,4)\\
      \hline
      1    & 1 &   1   &    1    &     1     &      1    \\
      \hline
      \sgn & 1 &   -1  &    1    &     -1    &      1   \\
      \hline
      \chi_U & 3 &   1   &    0    &     -1    &      -1  \\
      \hline
      \chi_U\sgn & 3 & -1 &   0    &     1    &     -1  \\
      \hline
      \psi       &  x_1  & x_2 & x_3 &   x_4    &     x_5     
    \end{array}
  \]
  あと1つの指標$\psi$は直交関係や次元公式を用いることで具体的な作用の考察なしに求めることができる。次元公式より
  \[
  \psi(e)=24-(1^2+1^2+3^2+3^2)=4  
  \]
  ゆえに$\psi(e)=2$である。直交関係より
  \[
  \left\{\begin{array}{ccc}
    6x_2+8x_3+6x_4+3x_5 & = & -2\\
    -6x_2+8x_3-6x_4+3x_5 & = & -2\\
    6x_2-6x_4-3x_5 & = & -6\\
    4+6x_2^2+8x_3^2+6x_4^2+3x_5^2 &= &24
  \end{array}\right.  
  \]
  これを解くと
  \[
    \begin{array}{c|c|c|c|c|c}
  
           & \quad e\quad & (1,2) & (1,2,3) & (1,2,3,4) & (1,2)(3,4)\\
      \hline
      1    & 1 &   1   &    1    &     1     &      1    \\
      \hline
      \sgn & 1 &   -1  &    1    &     -1    &      1   \\
      \hline
      \chi_U & 3 &   1   &    0    &     -1    &      -1  \\
      \hline
      \chi_U\sgn & 3 & -1 &   0    &     1    &     -1  \\
      \hline
      \psi       &  2  & 0 & -1 &   0 &     2     
    \end{array}
  \]
\end{eg}



\subsection{群環}

本節では群環という代数を導入し、環上の加群論を用いた表現論に関するいくつかの命題を証明する。

なおここで環は、乗法単位元をもつ必ずしも可換とは限らない環を指すとする。また加群といったら考えている環上の左加群を指しているとする。

\begin{defin}
  $G$を群, $K$を体とする。$K[G]$を$G$を基底にもつ$K$上の自由ベクトル空間とし、$G$の積から自然に定まる演算で$K[G]$に積を入れる。すなわち
  \[
  \left(\sum_{g\in G}a_gg\right)\cdot\left(\sum_{h\in G}b_hh\right)  =\sum_{k \in G}\left(\sum_{gh=k}a_gb_h\right)k
  \]
  である。これによって$K[G]$は$K$上の多元環の構造をもつ。これを$G$の$K$上の群環という。
\end{defin}

$V$を$G$の体$K$上の表現とする。$V$は自然に$K[G]$加群の構造が入り、逆に$K[G]$加群は自然に$G$の表現とみなすことができる。以下、$K[G]$加群といったらすべて$K$上有限次元のものを考えることにする。

このとき、
\begin{itemize}
  \item 部分表現は部分加群
  \item 表現の直和は加群の直和
  \item 既約表現は単純加群
  \item $G$線形写像は加群の準同型
  \item 表現の同値は加群の同型
\end{itemize}
にそれぞれ対応することがわかる。ただし表現のテンソル積は$K[G]$加群としてのテンソル積ではないことに注意。$V,W$を$K[G]$加群とするとき、$V$と$W$の表現のテンソル積は$V\otimes_{K}W$に$g(x\otimes y)=gx\otimes gy$による作用を入れたものである。

ここで、

\begin{defin}
  $A$を環とする。$A$加群$M$が単純であるとは、$M$が非自明な部分加群をもたないことをいう。
\end{defin}

である。環$A$を$A$加群とみなしたとき、$A$の部分加群とは$A$の左イデアルにほかならず、$A$に含まれる単純$A$加群は$A$の極小左イデアルである。単純性に関連して次の定義をする。

\begin{defin}
  $A$加群$M$が半単純であるとは、任意の$M$の部分加群が$M$の直和因子であることをいう。また、任意の$A$加群が半単純であるとき、$A$を半単純環という。
\end{defin}

\begin{theo}[Maschkeの定理]\label{general_maschke}
  $K[G]$が半単純環であるための必要十分条件は、$|G|$が$p=\text{ch}\:K$で割り切れないことである。
\end{theo}

\begin{proof}
  十分性は定理\ref{maschke}の証明とまったく同様である。必要性を示す。$|G|$が$p$がの倍数であるとする。Wedderburnの構造定理(付録参照)より$K[G]$のJacobson根基が0でないことを示せばよい。$K[G]$の元$m$を
  \[
  m=\sum_{g\in G}g  
  \]
  とおくと、任意の$x\in K[G]$に対して$xm=mx$であり、さらに
  \[
  m^2=\sum_{g,h\in G}gh=|G|m=0  
  \]
  だから
  \[
  (1-xm)(1+xm)=1-x^2m^2=1  
  \]
  よって$1-xm$は単元であるから$m\in\text{Jac}(K[G])$である。
\end{proof}


$\complex$の標数は0だから定理\ref{maschke}は定理\ref{general_maschke}の特別な場合である。しかし系\ref{irr_decompose}は一般には成り立たない。考えている表現が有限次元の場合において成り立つことに注意せよ。

以下、$K=\complex$の場合を考える。命題\ref{dim_formula}の証明より、$G$の正則表現は$G$のすべての有限次元既約表現をその次元の数だけ直和因子にもっている。このことを群環のことばで述べると、$\complex[G]$は$\complex[G]$加群として極小左イデアルの直和
\[
\complex[G]=L_1\oplus\cdots\oplus L_s,\quad s=m_1+\cdots+m_r
\]
に分解でき、適当に$L_1,\cdots,L_s$を並べ替えて
\begin{align*}
&L_1,\cdots,L_{m_1}\simeq W_1\\
&L_{m_1+1},\cdots,L_{m_1+m_2}\simeq W_2\\
&\qquad \vdots\\
&L_{m_1+\cdots+m_{r-1}+1},\cdots,L_{m_1+\cdots+m_{r-1}+m_r}\simeq W_r
\end{align*}
とできるということである。ここで$W_1,\cdots,W_r$は$G$の既約表現から定まる$\complex[G]$加群であり、$m_i=\dim_{\complex}W_i$である。したがって、$G$の有限次元既約表現を求めることは環$\complex[G]$の極小左イデアルを求めることと同等である。

\begin{defin}
  $A$を環とする。べき等元$e\in A$ $(e^2=e)$が原始的であるとは、
  \[
  e=e_1+e_2,\quad e_1^2=e_1,\quad e_2^2=e_2,\quad e_1e_2=0\implies e_1=0\text{または}e_2=0 
  \]
  を満たすことをいう。
\end{defin}

\begin{prop}
  $A$を半単純環, $e\in A$を単元でないとする。$Ae$が極小左イデアルとなるための必要十分条件は$e$が原始的べき等元であることである。
\end{prop}

\begin{proof}
  $Ae$が極小左イデアルとする。
  \[
    e=e_1+e_2,\quad e_1^2=e_1,\quad e_2^2=e_2,\quad e_1e_2=0
  \]
  となる$e_1,e_2\in A$が存在したとすると、
  \[
  e_1=e_1^2=e_1^2+e_1e_2=e_1e\in Ae  
  \]
  同様に$e_2\in Ae$である。よって$Ae$の極小性から$Ae_1=Ae\text{ or }0$, $Ae_2=Ae\text{ or }0$である。$Ae_1=Ae$であったとしよう。このとき
  \[
  e=ce_1,\quad c\in A  
  \]
  とおくことができるから
  \[
  e_2=e-e_1=(c-1)e_1
  \]
  よって
  \[
  e_2=e_2^2=(c-1)e_1e_2=0
  \]
  また$e$はべき等元なので
  \[
  e_1+e_2=e=(e_1+e_2)^2=e_1+-e_2e_1+e_2
  \]
  ゆえに$e_2e_1=0$である。したがって$Ae_2=Ae$ならば同様の議論で$e_1=0$となる。

  逆に$e$が原始的べき等元であるとする。$I\subsetneq Ae$を左イデアルとする。$A$は半単純だから
  \[
  Ae=I\oplus J  
  \]
  となる左イデアル$J$が存在する。よって
  \[
  e=x+y
  \]
  となる$x\in I, y\in J$をとることができる。$x\in Ae$より
  \[
  x=ce,\quad c\in A  
  \]
  とおくと$xe=ce^2=ce=x$。これより、
  \[
  x=xe=x^2+xy  
  \]
  だが、$xy\in J$かつ$I\cap J=0$より$xy=0$。同様に$yx=0$である。したがって$x^2=x$, $y^2=y$も導かれる。$e$は原始的なので$x=0$または$y=0$が成り立つが、これより$I=0$または$J=0$が従う。

  実際、$x=0$であったとして$m\in I$を$m=ae$とおけば
  \[
  m=a(x+y)=ay\in J  
  \]
  $I\cap J=0$より$m=0$
\end{proof}

したがって$G$の既約表現を求める問題は$\complex[G]$の原始的べき等元を求める問題に帰着された。具体的に原始的べき等元を見つけるのは難しいが、対称群の場合はYoung図形とのきれいな対応により構成することができる。次節にそのことを解説する。

最後にべき等元$e$で$Ae$の形の加群の間の準同型について考察する。

\begin{prop}\label{hom_of_cyclic_module}
  $A$を環とする。$e,f\in A$をべき等元とするとき、Abel群の同型として
  \[
  \Hom_A(Ae,Af)\simeq eAf
  \]
  が成り立つ。
\end{prop}

\begin{proof}
  $\phi\in\Hom_A(Ae,Af)$に対して、
  \[
  \phi(e)=af  
  \]
  とおくと、$e$はべき等元だから
  \[
  \phi(e)=\phi(e^2)=e\phi(e)=eaf  
  \]
  よって$\phi\mapsto eaf$を考えればこれが同型を与える。
\end{proof}

\begin{notice}
  証明からわかる通り、$A$が体$K$上の多元環である場合$e$はべき等元である必要はなく、スカラー倍のずれが許容される。すなわち
  \[
  e^2=\lambda e,\qquad \lambda\in K
  \]
  となる$e$に対しても同様のことが成り立つ。
\end{notice}




\subsection{誘導表現}

部分群の表現が与えられたとき、それを元の群に拡張する方法について解説する。

\begin{defin}\label{ind_rep}
  $G$を群、$H$を$G$の部分群とする。$W$を$H$の表現とするとき、
  \[
  V=\complex[G]\otimes_{\complex[H]}W 
  \]
  は左$\complex[G]$加群の構造をもつ\footnote{
    ここで$\complex[G]$は右からの積で右$\complex[H]$加群とみなしていることに注意。}。$V$を$W$が誘導する$G$の表現といい
  \[
  V=\ind_H^GW  
  \]
  と書く。
\end{defin}

誘導表現は次の普遍性で特徴づけることができる。

\begin{theo}[誘導表現の普遍性]\label{univ_ind_rep}
  $H$を群$G$の部分群、$W$を$H$の表現とする。このとき$G$の表現$V$と$H$線形写像$\map{\iota}{W}{V}$が一意的に存在して、次の性質をもつ:
  \begin{quote}
    任意の$G$の表現$U$と$H$線形写像$\map{f}{W}{U}$が与えられたとき、$G$線形写像$\map{\overline{f}}{V}{U}$が一意的に存在して
    \[
    f=\overline{f}\circ\iota  
    \]
    が成り立つ
  \end{quote}
\end{theo}

\begin{proof}
  定義\ref{ind_rep}の$V$がこの性質を持つことを示す。$\map{\iota}{W}{V}$を
  \[
  \iota(x)=1\otimes x  
  \]
  で定めれば、($\complex[H]$上のテンソル積なので)$\iota$は$H$線形写像である。$\map{f}{W}{U}$を$H$線形写像とする。$\map{\overline{f}}{V}{U}$を
  \[
  \overline{f}(g\otimes x)=gf(x)
  \]
  を双線形に拡張して得られる写像とすれば、
  より、
  \[
  \overline{f}(g(\alpha\otimes x))=\overline{f}(g\alpha\otimes x)=g\alpha f(x)  =g\overline{f}(\alpha\otimes x)
  \]
  $\overline{f}$は$G$線形写像であり$f=\overline{f}\circ \iota$を満たす。 $\overline{f}$の一意性を示す。$G$線形写像$f':W\rightarrow U$も$f=f'\circ \iota$を満たしたとする。$G$線形性から
  \[
  f'(g\otimes x)=gf'(1\otimes x)=gf'(\iota(x))=gf(x)=\overline{f}(g\otimes x)  
  \]
  である。

  最後にこの性質をもつ$V$が一意的であることを示す。$G$の表現$V'$と$H$線形写像$\iota':W\rightarrow V'$がこの性質を満たしたとする。$\iota$の普遍性を用いれば、$\overline{\iota}:V\rightarrow V'$が存在して$\iota'=\overline{\iota}\circ\iota$が成り立つ。また、$\iota'$の普遍性を用いれば$\overline{\iota'}:V'\rightarrow V$が存在して$\iota=\overline{\iota'}\circ \iota'$が成り立つ。
\[
\xymatrix{
  W \ar[r]^{\iota} \ar[d]_{\iota'}& 
  V \ar@<0.5ex>@{.>}[ld]_{\overline{\iota}} \\
  V' \ar@<0.5ex>@{.>}[ru]_{\overline{\iota'}}
  }
\]
  $\overline{\iota}$, $\overline{\iota'}$が互いに逆の写像であることを示そう。$\overline{\iota'}\circ\overline{\iota}:V\rightarrow V$は$G$線形写像であり、
  \[
  (\overline{\iota'}\circ\overline{\iota})\circ\iota=
  \overline{\iota'}\circ(\overline{\iota}\circ\iota)=
  \overline{\iota'}\circ \iota'=\iota
  \]
  を満たす。しかし、$\id{V}\circ \iota=\iota$であるから、$\iota$の普遍性から
  \[
  \overline{\iota'}\circ\overline{\iota}=\id{V}
  \]
  でなければならない。同様に$\overline{\iota}\circ\overline{\iota'}=\id{V'}$である。
\end{proof}

定義\ref{ind_rep}は加群論的で簡単だが、どのような作用を考えているのかがわかりにくい。そこで線形代数的な誘導表現の定義もみておく。

\begin{defin}\label{ind_rep_2}
  $V$を$G$の表現、$W$を$V$の部分空間で$H$の表現であるとする。$R$を$G/H$の完全代表系とする。このとき$V$が$W$から誘導されているとは
  \[
  V=\bigoplus_{\sigma\in R}\sigma W  
  \]
  が成り立つことをいう。
\end{defin}

$\inv{\sigma}\tau\in H$ならば$\sigma W=\tau W$が成り立つのでこの定義は$R$の取り方によらない。$W$の基底を$e_1,\cdots,e_n$として、$R=\{\sigma_1,\cdots,\sigma_r\}$とすれば
\[
V=
\complex\sigma_1 e_1\oplus\cdots\oplus\complex\sigma_1 e_n
\oplus\cdots\oplus
\complex\sigma_r e_1\oplus\cdots\oplus\complex\sigma_r e_n
\]
$G$の元$g$は$G/H$に左からの積で置換作用する。すなわち、各$\sigma_i$に対して、
\[
g\sigma_i=\sigma_j h  
\]
となる$\sigma_j\in R$と$h\in H$が存在する。よってこのとき、$g\sigma_iW=\sigma_jW$、とくに
\begin{equation}\label{ind_action}
  g\sigma_ie_k=\sigma_j he_k  \quad\in\sigma_jW,\qquad \text{for all $k=1,\cdots,n$}  
\end{equation}

となる。$g$の$V$への作用は$G/H$への置換作用と$H$の$W$への作用を組み合わせたようなものである。また式(\ref{ind_action})から次がでる。

\begin{prop}[誘導表現の指標]\label{ind_char}
  $V=\ind_H^GW$のとき、
  \begin{align}
  \chi_V(g)&=\sum_{\substack{\sigma_i\in R \\ \inv{\sigma_i}g\sigma_i\in H}}\chi_W(\inv{\sigma_i}g\sigma_i)
  =\frac{1}{H}\sum_{\substack{\sigma\in G \\ \inv{\sigma}g\sigma\in H}}\chi_W(\inv{\sigma}g\sigma)
\end{align}
が成り立つ。
\end{prop}

\begin{proof}
  式(\ref{ind_action})より、$g$の作用の対角成分を考えるうえで$\sigma_i=\sigma_j$すなわち$i=j$となる部分だけ考えればよいことがわかる。このとき$\inv{\sigma_i}g\sigma_i\in H$であり、
  \[
  h=\inv{\sigma_i}g\sigma_i 
  \]
  であるから、あとは$h$の$W$への作用の対角成分の和をとればよいので前半の式がでる。後半については指標が類関数であること、$|\sigma H|=|H|$であることから従う。
\end{proof}

定義\ref{ind_rep_2}が定義\ref{ind_rep}と一致することを確かめておく。定義\ref{ind_rep_2}の$V$が普遍性(定理\ref{univ_ind_rep})を満たすことを示せば、一意性から従う。

$V=\bigoplus_{\sigma\in R}\sigma W$において、$R$として単位元$e$を含むものをとり、$W$を$eW$と同一視する。この同一視を$\iota$とすれば$\iota$は$H$線形写像である。$U$を任意の$G$の表現とし、$f:W\rightarrow U$を$H$線形写像とする。このとき$\overline{f}:V\rightarrow U$を
\[
\overline{f}(\sigma_i x)=\sigma_i f(x)
\]
によって定義すると、$\overline{f}$は$G$線形である。実際$g\in G$に対して、$g\sigma_i =\tau_j h$となる$\sigma_j\in R, h\in H$をとれば
\[
\overline{f}(g\sigma_i x)=\overline{f}(\sigma_j hx)=\sigma_jf(hx)=\sigma_jhf(x)=g\sigma_if(x)  
\]
また、
\[
\overline{f}\circ\iota(x)=\overline{f}(ex)=ef(x)=f(x)  
\]



\begin{eg}\label{ind_from_trivial}
  $H$を$G$の部分群とする。$X=G/H=\{H,g_1H,\cdots,g_nH\}$とし、$V$を$X$を基底に持つ自由ベクトル空間とする。$G$は$V$に置換によって作用する。
  \[
  W=\complex H\subset V  
  \]
  とすれば$W$は$H$の自明な表現であり、$R=\{e,g_1,\cdots,e_n\}$
  \begin{align*}
  V&=\complex H\oplus \complex g_1H\oplus\cdots\oplus\complex g_nH\\
  &=\bigoplus_{\sigma\in R}\sigma\complex H\\
  &=\bigoplus_{\sigma\in R}\sigma W
  \end{align*}
  だから$V=\ind_H^GW$である。すなわち、$H$の自明な表現から誘導される$G$の表現は$G/H$への置換表現である。
\end{eg}
\end{document}

\newpage
\documentclass{ltjsreport}
\input{../../setting.tex}

\begin{document}
\section{Grassmann多様体とSchubert多様体}
\subsection{Grassmann多様体}

前節の準備をもとに数え上げ問題を定式化しよう。以下では係数体はすべて$\complex$で考えているとする。

\begin{defin}
  $\complex^{n}$の$d$次元部分空間全体のなす集合を$\mathcal{G}(d,n)$と書き、これをGrassmann多様体という。
\end{defin}

$d=1$のときGrassmann多様体は射影空間に他ならない。この意味でGrassmann多様体は射影空間の一般化である。
第3章冒頭で述べた数え上げ問題においては$\mathcal{G}(2,4)$を考えることになる。


Grassmann多様体が代数多様体の構造をもつことを示しておく。


まず、$\mathcal{M}(d,n)$をランク$d$の$n\times d$行列全体のなす集合とする。$\mathcal{M}(d,n)$は$\gl_d(\complex)$が右からの積で作用するが、この商$\mathcal{M}(d,n)/\gl_d(\complex)$は$\mathcal{G}(d,n)$と同一視される。実際、$\complex^n$の$d$次元部分空間に対して、その基底を並べた行列を考えればそれは$\gl_d(\complex)$軌道の違いを除いて一意的である。逆に$A\in\mathcal{M}(d,n)$に対して、$[A]$を$A$の列ベクトル(それは1次独立)が生成する部分空間とすれば$[A]\in\mathcal{G}(d,n)$である。
また、$\mathcal{M}(d,n)$は$\affine^{nd}$のZariski開集合であった(例\ref{M_d,n})が、$\mathcal{G}(d,n)$には$\mathcal{M}(d,n)$から誘導される商位相を入れておく。


次に$\complex^n$の$d$階交代テンソル空間$\bigwedge^d\complex^n$を考える。$\bigwedge^d\complex^n$は$\comb{n}{d}$次元ベクトル空間であるから、その射影化$\proj(\bigwedge^d\complex^n)$は$\proj^{\comb{n}{d}-1}$と同一視することができる。また、$e_1,\cdots,e_n$を$\complex^n$の標準基底とすれば$\omega\in\bigwedge^d\complex^n$は
\[
\omega=\sum_{1\leq i_1<\cdots <i_d\leq n}x_{i_1,\cdots, i_d}e_{i_1}\wedge\cdots\wedge e_{i_d}  
\]
と表せるので、$p(\omega)$の斉次座標は
\[
p(\omega)=[x_{i_1,\cdots,i_d}]_{1\leq i_1<\cdots<i_d\leq n}  
\]
のように書くことができる。ただし$p$は射影化$p:\bigwedge^d\complex^n\rightarrow \proj(\bigwedge^d\complex^n)$である。

$A\in\mathcal{M}(d,n)$に対して、$A$の列ベクトルを$v_1,\cdots,v_d\in\complex^n$とし写像$\tilde{\pi}:\mathcal{M}(d,n)\rightarrow \proj(\bigwedge^d\complex^n)$を
\[
\tilde{\pi}(A)=p(v_1\wedge\cdots\wedge v_d)=[\det(A_{i_1,\cdots,i_d})]_{1\leq i_1<\cdots<i_d\leq n}  
\]
とする。$\tilde{\pi}$は多項式写像なので連続である。また、$P=(a_{ij})\in\gl_d(\complex)$に対して、
\begin{align*}
  \tilde{\pi}(AP)
  &=p\left(
    (a_{11}v_1+\cdots+a_{d1}v_d)\wedge\cdots\wedge
    (a_{d1}v_1+\cdots+a_{dd}v_d)
  \right)\\
  &=p(\det P(v_1\wedge\cdots\wedge v_d))\\
  &=p(v_1\wedge\cdots\wedge v_d)\\
  &=\tilde{\pi}(A)
\end{align*}
となるから、$\tilde{\pi}$は連続写像$\pi:\mathcal{G}(d,n)\rightarrow \proj(\bigwedge^d\complex^n)$を誘導する。


\begin{prop}[Plucker埋め込み]\label{plucker}
  $\pi:\mathcal{G}(d,n)\rightarrow \proj(\bigwedge^d\complex^n)=\proj^{\comb{n}{d}-1}$は単射である。
\end{prop}

\begin{proof}
  次の補題を用いる。
  \begin{lemm}\label{ker_wedge}
    $V\in\mathcal{G}(d,n)$に対してその基底$v_1,\cdots,v_d$を固定して、$\omega=v_1\wedge\cdots\wedge v_d\in\bigwedge^d\complex^n$とする。$\Gamma_\omega:\complex^n\rightarrow \bigwedge^{d+1}\complex^n$を
    \[
    \Gamma_\omega(u)=\omega\wedge u  
    \]
    によって定めると、
    \begin{equation*}
      \ker\Gamma_\omega =V
    \end{equation*}
    が成り立つ。
  \end{lemm}
  
  \begin{proof}
    $V$の元が$\ker\Gamma_\omega$に含まれることは明らか。$u\in\ker\Gamma_\omega$であるとする。$v_1,\cdots,v_d$を延長して$\complex^n$の基底$v_1,\cdots,v_d,v_{d+1},\cdots,v_n$をとる。
    \[
    u=a_{1}v_1+\cdots+a_{d}v_d+a_{d+1}v_{d+1}+\cdots+a_nv_n  
    \]
    とおく。
    \begin{align*}
    0=\omega\wedge u
    &=v_1\wedge\cdots\wedge v_d\wedge(a_{1}v_1+\cdots+a_{d}v_d+a_{d+1}v_{d+1}+\cdots+a_nv_n )  \\
    &=a_{d+1}(v_1\wedge\cdots\wedge v_d\wedge v_{d+1})
        +\cdots+
      a_{n}(v_1\wedge\cdots\wedge v_d\wedge v_{n})
    \end{align*}
    となるが、$\{v_{i_1}\wedge\cdots\wedge v_{i_{d+1}}\}_{i_1<\cdots<i_{d+1}}$は1次独立であるので、$a_{d+1}=\cdots=a_n=0$. よって$u\in V$
  \end{proof}

  命題の証明に戻る。$\pi(V)=\pi(U)$であるとする。$U$の基底を$u_1,\cdots,u_d$とすると仮定より
  \[
  cu_1\wedge\cdots\wedge u_d=v_1\wedge\cdots\wedge v_d=\omega  
  \]
  となる定数$c$が存在する。したがって$\Gamma_\omega(u_i)=\omega\wedge u_i=0$であるから補題により、$U=\ker\Gamma_\omega= V$
\end{proof}
  
$\pi(\mathcal{G}(d,n))\subset\proj(\bigwedge^d\complex^n)$が代数的集合であることを示す。

\begin{defin}
  $\omega\in\bigwedge^d \complex^n$がtotally decomposableであるとは、1次独立な$v_1,\cdots,v_d\in V$が存在して$\omega=v_1\wedge\cdots\wedge v_d$となることをいう。
\end{defin}

\begin{lemm}\label{totally_decomposable}
  $\omega\in\bigwedge^d \complex^n$がtotally decomposableであることと$\Gamma_\omega:\complex^n\rightarrow\bigwedge^{d+1}\complex^n$のランクが$n-d$となることは同値である。
\end{lemm}

\begin{proof}
  $\omega=v_1\wedge\cdots\wedge v_d$とおく。このとき補題\ref{ker_wedge}の証明より$\dim\ker\Gamma_\omega=\dim \generated{v_1,\cdots,v_d}=d$であるから$\rank\Gamma_\omega=n-d$である。逆に$\rank\Gamma_\omega=n-d$であるとする。$\dim\ker\Gamma_\omega=d$であるから$\ker\Gamma_\omega$の基底$v_1,\cdots,v_d$をとり、これを延長して$\complex^n$の基底$v_1,\cdots,v_d,v_{d+1},\cdots,v_n$をとって
  \[
  \omega=\sum_{1\leq i_1<\cdots<i_d\leq n}c_{i_1,\cdots,i_d}v_{i_1}\wedge\cdots\wedge v_{i_d}  
  \]
  とおく。すると$\Gamma_\omega(v_j)=0$, $j=1,\cdots,d$より
  \begin{align*}
    &v_1\wedge\omega=0\text{ すなわち } c_{i_1,\cdots,i_d}=0\text{ for } i_1 > 1\\
    &v_2\wedge\omega=0\text{ すなわち } c_{i_1,\cdots,i_d}=0\text{ for }i_2 > 2\\
    &\qquad\vdots\\
    &v_d\wedge\omega=0\text{ すなわち } c_{i_1,\cdots,i_d}=0\text{ for }i_d > d
  \end{align*}
  よって$\omega=c_{1,2,\cdots,d}v_1\wedge\cdots\wedge v_d$
\end{proof}


$T\subset\bigwedge^d\complex^n$をtotally decomposableな元の集合とする。
$\pi(\mathcal{G}(d,n))=\proj(T)$である。$e_1,\cdots,e_n\in \complex^n$を標準基底とし、$\omega\in\bigwedge^d\complex^n$を
\[
\omega=\sum_{1\leq i_1<\cdots<i_d\leq n}x_{i_1,\cdots,i_d}e_{i_1}\wedge\cdots\wedge e_{i_d}  
\]
とおく。補題より、$p(\omega)\in\pi(\mathcal{G}(d,n))$であるための必要十分条件は$\rank\Gamma_\omega=n-d$となることである。この条件は$\Gamma_\omega:\complex^n\rightarrow \bigwedge^d\complex^n$を行列表示したとき、その$(n-d+1)\times(n-d+1)$小行列式がすべて$0$になることと同値である
\footnote{
  $\Gamma_\omega$のランクは必ず$n-d$以上であることに注意。実際、もし$\dim\ker\Gamma_{\omega}\geq d+1$であるなら、補題\ref{totally_decomposable}の証明と同様の議論をすると、$\omega=0$となってしまう。
}。
そして$\Gamma_\omega$の小行列式は$x_{i_1,\cdots,i_d}$の多項式で表されるから、$\pi(\mathcal{G}(d,n))$は$\proj({\bigwedge^d\complex^n})$の代数的集合である。

Grassmann多様体が既約であることを示そう。

\begin{lemm}\label{image_of_irrspace}
  $X$, $Y$を位相空間, $f:X\rightarrow Y$を連続写像とする。$A\subset X$が既約であるならば$f(A)$も既約である。
\end{lemm}

\begin{proof}
  $f(A)$が可約であったとして$f(A)=Z_1\cup Z_2$, $\varnothing\subsetneq Z_1,Z_2\subsetneq f(A)$となる閉集合$Z_1,Z_2$をとる。
  \[
  A\subset \inv{f}(f(A))=\inv{f}(Z_1\cup Z_2)=\inv{f}(Z_1)\cup\inv{f}(Z_2)  
  \]
  $f$は連続であるから$\inv{f}(Z_1), \inv{f}(Z_2)$は閉集合である。
  \[
  A=(A\cap\inv{f}(Z_1))\cup(A\cap\inv{f}(Z_2))
  \]
  より$A$は可約である。
\end{proof}

\begin{prop}
  $\mathcal{G}(d,n)$は既約である。
\end{prop}

\begin{proof}
  $V\in\mathcal{G}(d,n)$を固定して、$\alpha:\gl_n(\complex)\rightarrow\mathcal{G}(d,n)$を
  \[
  \alpha(P)=PV  
  \]
  によって定める。ただし$PV$は$V$の基底を$v_1,\cdots,v_d$とするとき$Pv_1,\cdots,Pv_d$によって生成される$d$次元部分空間を表す。$\alpha$は全射である。実際任意の$d$次元部分空間$W=\generated{w_1,\cdots,w_d}$に対して、各$v_i$を$w_i$に写すような$n$次正則行列$P$をとればよい。また$\alpha$は多項式写像であるから連続である。$\gl_n(\complex)$は既約であるから、補題\ref{image_of_irrspace}より$\mathcal{G}(d,n)$も既約である。
\end{proof}

なお、Plucker埋め込み$\pi$は実際に埋め込み、すなわち像への同相であることが知られている。

最後にGrassmann多様体の次元について考える。$1\leq i_1<\cdots <i_d\leq n$とする。
\[
\tilde{U}_{i_1,\cdots,i_d}=\set{A\in\mathcal{M}(d,n)}{\det A_{i_1,\cdots,i_d}\neq 0}  
\]
とする。ただし$A_{i_1,\cdots,i_d}$は$A$の第$i_1,\cdots,i_d$行をとりだした小正方行列であり、$E_d$は$d$次単位行列である。$\tilde{U}_{i_1,\cdots i_d}$はZariski開集合であり
\[
\mathcal{M}(d,n)=\bigcup_{i_1,\cdots,i_d}\tilde{U}_{i_1,\cdots,i_d}  
\]
である。$\phi:\mathcal{M}(d,n)\rightarrow \mathcal{M}(d,n)/\gl_d(\complex)=\mathcal{G}(d,n)$を自然な写像とすると、$\phi$は開写像である\footnote{
  一般に位相空間$X$に群$G$が作用しているとき、その軌道空間$X/G$への自然な写像$\phi:X\rightarrow X/G$は開写像である。実際、開集合$U\subset X$に対して
  \[
  \inv{\phi}(\phi(U))=\bigcup_{g\in G}gU
  \]
  であり$g$は$X$上の同相であるから右辺の各$gU$は開集合である。
}。したがって$U_{i_1,\cdots,i_d}=\phi(\tilde{U}_{i_1,\cdots,i_d})$は$\mathcal{G}(d,n)$の開集合であり$\mathcal{G}(d,n)=\bigcup_{i_1,\cdots,i_d}U_{i_1,\cdots,i_d}$である。$A\in\tilde{U}_{i_1,\cdots,i_d}$は適当に右から$\gl_d(\complex)$を書けることによって$A_{i_1,\cdots,i_d}=E_{d}$となるようにできるから
\[
U_{i_1,\cdots,i_d}=\set{[A]\in\mathcal{G}(d,n)}{A_{i_1,\cdots,i_d}=E_d}
\]
したがって$U_{i_1,\cdots,i_d}$は$\affine^{d(n-d)}$と同相であるから、事実\ref{dim_property}より
\[
\dim\mathcal{G}(d,n)=\dim U_{i_1,\cdots,i_d}=d(n-d)  
\]






\subsection{Shubert胞体とSchubert多様体}

第3章冒頭で述べた数え上げ問題においては$\proj^3$中の直線全体を考えたいから、$\mathcal{G}(2,4)$を考察していくことになる。重要な考え方として、ある条件をみたす直線の集合を$\mathcal{G}(2,4)$の部分多様体としてとらえることで、「複数の条件を満たす直線の数え上げ$\Leftrightarrow$いくつかの$\mathcal{G}(2,4)$の部分多様体の交点を数える」という問題の変換を行う。このように幾何学的な条件をみたす線形部分多様体をパラメトライズする空間をSchubert多様体という。




正則行列を右からかけることはいくつかの列基本変形を施すことと同値であるから、$[A]\in\mathcal{G}(d,n)$に対して、$A$の第$1$列から順に列基本変形を行えば、$[A]=[(a_{ij})]$はある$1\leq i_1<i_2<\cdots<i_d\leq n$があって
$A$の第$i_1,\cdots,i_d$行を取り出した小正方行列が単位行列であり、$(i_1,1),(i_2,2),\cdots,(i_d,d)$成分よりも右上の成分がすべて$0$になる。式で表すと
\begin{equation}\label{schubertcell}
a_{ij}=\left\{\begin{array}{cl}
  0 & \text{ if } i \leq i_{j}-1 \text{ or } i=i_k,\: j\leq k-1 \text{ for some $1\leq k\leq d$}\\
  1 & \text{ if } i = i_k,\: j = k \text{ for some $1\leq k\leq d$}
\end{array}\right.  
\end{equation}
をみたすということである。

\begin{eg}
  $[A]\in\mathcal{G}(2,4)$は次のいずれかの形になる。
  \begin{align*}
    &\bmat{
      1 & 0\\
      0 & 1\\
      * & *\\
      * & *
    },\qquad 
    \bmat{
      1 & 0\\
      * & 0\\
      0 & 1\\
      * & *
    },\qquad
    \bmat{
      1 & 0\\
      * & 0\\
      * & 0\\
      0 & 1
    }\\
    &\bmat{
      0 & 0\\
      1 & 0\\
      0 & 1\\
      * & *
    },\qquad
    \bmat{
      0 & 0\\
      1 & 0\\
      * & 0\\
      0 & 1
    },\qquad
    \bmat{
      0 & 0\\
      0 & 0\\
      1 & 0\\
      0 & 1
    }
  \end{align*}
  ただし$*$の部分には任意の複素数が入る。
\end{eg}

\begin{defin}
  $1\leq i_1<i_2<\cdots<i_d\leq n$に対して
  \[
  \Omega^\circ_{i_1,\cdots,i_d}=\set{[(a_{ij})]\in\mathcal{G}(d,n)}{a_{ij}\text{ satisfies (\ref{schubertcell})}}  
  \]
  をSchubert胞体という。
\end{defin}

上の議論から、
\[
\mathcal{G}(d,n)=\bigcup_{1\leq i_1<\cdots<i_d\leq n}\Omega^\circ_{i_1,\cdots,i_d}  
\]
が成り立つ。

\begin{eg}
$\mathcal{G}(2,4)$の場合
  \begin{align*}
    &\Omega^\circ_{1,2}=\bmat{
      1 & 0\\
      0 & 1\\
      * & *\\
      * & *
    },\qquad 
    \Omega^\circ_{1,3}=\bmat{
      1 & 0\\
      * & 0\\
      0 & 1\\
      * & *
    },\qquad
    \Omega^\circ_{1,4}=\bmat{
      1 & 0\\
      * & 0\\
      * & 0\\
      0 & 1
    }\\
    &\Omega^\circ_{2,3}=\bmat{
      0 & 0\\
      1 & 0\\
      0 & 1\\
      * & *
    },\qquad
    \Omega^\circ_{2,4}=\bmat{
      0 & 0\\
      1 & 0\\
      * & 0\\
      0 & 1
    },\qquad
    \Omega^\circ_{3,4}=\bmat{
      0 & 0\\
      0 & 0\\
      1 & 0\\
      0 & 1
    }
  \end{align*}
\end{eg}

$\Omega^\circ_{i_1,\cdots,i_d}$に含まれる$*$の数を$k$とすれば$\Omega^\circ_{i_1,\cdots,i_d}$は$\affine^k$に同相であるから$\dim\Omega^\circ_{i_1,\cdots,i_d}=k$である。

Schubert胞体とYoung図形の関係について述べておく。$1\leq i_1<\cdots<i_d\leq n$に対して、
\begin{equation}\label{young_and_schubert}
\lambda_{k}=i_{d+1-k}-d-1+k  
\end{equation}
とおくと、$(\lambda_1,\cdots,\lambda_d)$は$d\times (n-d)$の部分Young図形になる。逆に式(\ref{young_and_schubert})によって$d\times (n-d)$の部分Young図形からSchubert胞体を得ることができる。そこで$\mathcal{Y}_d(n)$を$d\times (n-d)$の部分Young図形全体のなす集合とし、$\lambda\in\mathcal{Y}_d(n)$に対応するSchubert胞体を$\Omega^\circ_{\lambda}$と表すことにする。

式(\ref{young_and_schubert})の対応関係をもう少し詳しく説明する。$d\times (n-d)$のYoung図形を用意し、各辺に沿って一番左下の頂点から一番右上の頂点に行く最短経路を考える。そのパターンは、各ステップごとに上に行くか右に行くかを選べば決まる。右上にたどり着くためには$d$回上に行く選択をしなければならないから、$i_1,\cdots,i_d$ステップ目で上に行き、それ以外では右に行くとすれば、1つ最短経路が定まる。この最短経路によって分けられる$d\times(n-d)$Young図形の左上の部分を、対応するYoung図形$\lambda$とするのである。
\begin{figure}[H]
  \centering
  \includegraphics*[scale=0.6]{C:/Users/Xsaku/OneDrive/Projects/graduate_project/src/young_and_schur.jpg}
\end{figure}
またこのとき対応する$\lambda$は、$A$から$i_1,\dots,i_d$行と$*$の入っている部分を取り除いてできる形(を$90^\circ$左に開店したもの)と同じである。



\begin{eg}
\begin{align*}
  \ytableausetup{boxsize=0.5em}
  &\Omega^\circ_{\varnothing}=\bmat{
    1 & 0\\
    0 & 1\\
    * & *\\
    * & *
  },\qquad 
  \Omega^\circ_{\ydiagram{1}}=\bmat{
    1 & 0\\
    * & \fbox{0}\\
    0 & 1\\
    * & *
  },\qquad
  \Omega^\circ_{\ydiagram{2}}=\bmat{
    1 & 0\\
    * & \fbox{0}\\
    * & \fbox{0}\\
    0 & 1
  }\\
  &\Omega^\circ_{\ydiagram{1,1}}=\bmat{
    \fbox{0} & \fbox{0}\\
    1 & 0\\
    0 & 1\\
    * & *
  },\qquad
  \Omega^\circ_{\ydiagram{2,1}}=\bmat{
    \fbox{0} & \fbox{0}\\
    1 & 0\\
    * & \fbox{0}\\
    0 & 1
  },\qquad
  \Omega^\circ_{\ydiagram{2,2}}=\bmat{
    \fbox{0} & \fbox{0}\\
    \fbox{0} & \fbox{0}\\
    1 & 0\\
    0 & 1
  }
  \ytableausetup{boxsize=normal}
\end{align*}
$\Box$で囲った成分のなす形とYoung図形とが対応している。
\end{eg}

\begin{prop}
  $I=(1\leq i_1<\cdots<i_d\leq n)$, $J=(1\leq j_1<\cdots<\leq n)$に対応するYoung図形をそれぞれ$\lambda,\mu$とする。このとき
  \[
  \lambda\subset\mu\Leftrightarrow i_k\leq j_k\text{ for all k}  
  \]
  である。
\end{prop}

\begin{proof}
  $\lambda\subset\mu$の定義は$\lambda_k\leq\mu_k$, for all $k$であるから、式(\ref{young_and_schubert})より直ちに従う。
\end{proof}



\begin{defin}[標準的な旗に付随するShubert多様体]
  $\lambda\in\mathcal{Y}_d(n)$に対して、
  \[
  \Omega_\lambda=\bigcup_{\mu\supset\lambda}\Omega^\circ_\mu
  \]
  をSchubert多様体という。
\end{defin}

ここで定義したSchubert多様体は、正確には標準的な旗に付随するSchubert多様体と呼ぶべきものである。旗の概念は後で定義するが、一般のSchubert多様体はこの標準的なSchubert多様体のもつ、次の幾何学的な性質を一般化する形で定める。

\begin{prop}\label{schubert_var}
  $\lambda\in\mathcal{Y}_d(n)$とする。$e_1,\cdots,e_n$を$\complex^n$の標準基底とし$F^k=\generated{e_{k+1},\cdots,e_n}$とおく。
  \[
  \Omega_{\lambda}=\set{V\in\mathcal{G}(d,n)}{\dim(V\cap F^{\lambda_k+d-k})\geq k}  
  \]
  が成り立つ。
\end{prop}

\begin{proof}
  $\mu\supset\lambda$とする。式(\ref{young_and_schubert})によって$\lambda,\mu$に対応する整数列をそれぞれ$(i_1,\cdots,i_d)$, $(j_1,\cdots,j_d)$とする。すなわち
  \[
  i_k=\lambda_{d+1-k}+k,\qquad j_k=\mu_{d+1-k}+k  
  \]
  である。$\mu\supset\lambda$より$j_k\geq i_k$である。
  
  $V\in\Omega_{\mu}^\circ$はベクトル
  \begin{align*}
    &v_1=e_{j_1}+f_1,\quad f_1\in\generated{e_{j_1+1},\cdots,e_n}=F^{j_1}\\
    &v_2=e_{j_2}+f_2,\quad f_2\in\generated{e_{j_2+1},\cdots,e_n}=F^{j_2}\\
    &\quad\vdots\\
    &v_d=e_{j_d}+f_d,\quad f_d\in\generated{e_{j_d+1},\cdots,e_n}=F^{j_d}
  \end{align*}
  を適当にとって$V=\generated{v_1,v_2,\cdots,v_d}$とできる。$\lambda_k+d-k=(i_{d+1-k}-d-1+k)+d-k=i_{d+1-k}-1$であるが、$\dim(V\cap F^{i_{d+1-k}-1})\geq k$を示そう。$v_{d+1-k},v_{d+2-k},\cdots,v_d\in F^{j_{d+1-k}-1}$である。$j_k\geq i_k$であるから$F^{j_{d+1-k}-1}\subset F^{i_{d+1-k}-1}$. したがって
  \[
  \dim(V\cap F^{i_{d+1-k}-1})\geq k
  \]
  である。

  逆に$V\in\mathcal{G}(d,n)$が$\dim(V\cap F^{i_{d+1-k}-1})\geq k$を満たしているとする。このとき$V$の基底$v_1,\cdots,v_d$として次の条件を満たすものがとれる。
  \begin{align*}
    &v_d\in V\cap F^{i_d-1}\\
    &v_{d-1}\in V\cap F^{i_{d-1}-1}\setminus F^{i_d-1}\\
    &\quad\vdots\\
    &v_1\in V\cap F^{i_1-1}\setminus F^{i_2-1}
  \end{align*}
  よって標準基底に関して成分表示すると
  \begin{align*}
    \begin{array}{ccr}
      v_d & = & c^{(d)}_{i_d}e_{i_d}+\cdots+c^{(1)}_{n}e_n\\
      v_{d-1} & = & c^{(d-1)}_{i_{d-1}}e_{i_{d-1}}+\cdots+c^{(d-1)}_{i_d-1}e_{i_d-1}+c^{(d-1)}_{i_d}e_{i_d}+\cdots+c^{(d-1)}_{n}e_n\\
          & \vdots & \\
      v_1 & = & c^{(1)}_{i_1}e_{i_1}+\cdots+c^{(1)}_{i_2-1}e_{i_2-1}+\cdots+c^{(1)}_{i_d}e_{i_d}+\cdots+c^{(1)}_{n}e_n
    \end{array}
  \end{align*}
  となるが、上の条件より
  \begin{align*}
    &c^{(d)}_{i_d},\dots,c^{(d)}_n\text{ は同時に$0$にならない}\\
    &c^{(d-1)}_{i_{d-1}},\cdots,c^{(d-1)}_{i_d-1}\text{ は同時に$0$にならない}\\
    &\quad\vdots\\
    &c^{(1)}_{i_1},\cdots,c^{(1)}_{i_2-1}\text{ は同時に$0$にならない}\\
  \end{align*}
  が成り立つ。各$k$において$0$でない$c^{(k)}_{i_{k}},\cdots,c^{(k)}_{n}$のうち最も左にあるものを$c^{(k)}_{j_k}$とおいて、$v_k$を$\frac{1}{c^{(k)}_{j_k}}v_k$で置き換えれば、
  \begin{align*}
    &v_d=e_{j_1}+f_1,\quad f_1\in\generated{e_{j_1+1},\cdots,e_n}=F^{j_1}\\
    &v_{d-1}=e_{j_2}+f_2,\quad f_2\in\generated{e_{j_2+1},\cdots,e_n}=F^{j_2}\\
    &\quad\vdots\\
    &v_d=e_{j_d}+f_d,\quad f_d\in\generated{e_{j_d+1},\cdots,e_n}=F^{j_d}
  \end{align*}
  の形にすることができる。これは$V=\generated{v_1,\cdots,v_d}$が$V\in\Omega_{j_1,\cdots,j_d}^\circ$であることに他ならないが、$j_k\geq i_k$であるので、$j_1,\cdots,j_d$の対応するYoung図形を$\mu$とすれば$\mu\supset\lambda$である。
\end{proof}



\begin{defin}
  $\complex^n$の部分空間の列
  \[
  \complex^n=F^0\supset F^1\supset \cdots\supset F^{n-1}\supset F^n=0,\qquad \dim F^k=n-k  
  \]
  を旗といい、$F^\bullet$と表す。とくに$e_1,\cdots,e_n$を標準基底として$F^k=\generated{e_{k+1},\cdots,e_n}$なる旗を標準旗といい$F_{st}^\bullet$と表す。$\proj^{n-1}$の線形部分多様体は$\complex^n$の部分空間と1対1に対応することを思い出せば、旗の各部分空間を射影化することで$\proj^{n-1}$の線形部分多様体の列を得る。このような$\proj^{n-1}$の部分集合の列も旗と呼ぶことにする。
\end{defin}

\begin{eg}
  $\complex^4$において旗
  \[
  \complex^4=F^0\supset F^1\supset F^2\supset F^3\supset F^4=0  
  \]
  の射影化は
  \[
  \proj^3\supset e_0\supset l_0\owns p_0\supset \varnothing  
  \]
  である。ここで$e_0$, $l_0$, $p_0$はそれぞれ$\proj^3$の平面, 直線, 点である。
\end{eg}

命題\ref{schubert_var}を一般化して次の定義を得る。

\begin{defin}
  $F^\bullet$を$\complex^n$の旗とする。$\lambda\in\mathcal{Y}_d(n)$に対して、
  \[
  \Omega_\lambda(F^\bullet)=\set{V\in\mathcal{G}(d,n)}{\dim(V\cap F^{\lambda_i+d-i})\geq i}
  \]
  を$F^\bullet$に付随するSchubert多様体という
\end{defin}


\begin{eg}\label{case_2,4}
$\mathcal{G}(2,4)$において、$F^\bullet$を任意の$\complex^4$の旗としてその射影化を$\proj^3\supset e_0\supset l_0\owns p_0\supset\varnothing$とする。このとき
\begin{align*}
  \ytableausetup{boxsize=0.5em}
  &\Omega_{\varnothing}=\mathcal{G}(2,4)\\
  &\Omega_{\ydiagram{1}}=\set{V\in\mathcal{G}(2,4)}{\dim(V\cap F^2)\geq 1}\approx\set{l\subset\proj^3:\text{直線}}{l\cap l_0\neq\varnothing}\\
  &\Omega_{\ydiagram{2}}=\set{V\in\mathcal{G}(2,4)}{\dim(V\cap F^3)\geq 1}\approx\set{l\subset\proj^3:\text{直線}}{l\owns p_0}\\
  &\Omega_{\ydiagram{1,1}}=\set{V\in\mathcal{G}(2,4)}{\dim(V\cap F^2)\geq 1,\:\dim(V\cap F^1)\geq 2}\approx\set{l\subset\proj^3:\text{直線}}{l\subset e_0}\\
  &\Omega_{\ydiagram{2,1}}=\set{V\in\mathcal{G}(2,4)}{\dim(V\cap F^3)\geq 1,\:\dim(V\cap F^1)\geq 2}\approx\set{l\subset\proj^3:\text{直線}}{ p_0\in l\subset e_0}\\
  &\Omega_{\ydiagram{2,2}}\approx\{l=l_0\}
  \ytableausetup{boxsize=normal}
\end{align*}
となり、Schubert多様体が幾何学的な線形部分多様体をパラメトライズしていることがわかる。

\end{eg}



\subsection{Schubert多様体の基本性質}
Schubert胞体・Schubert多様体の基本的な性質をいくつか示しておく。
まず、Schubert多様体がSchubert胞体の直和であることを示す。

\begin{prop}
  $\lambda\neq\mu$ならば、$\Omega_\lambda^\circ\cap\Omega_\mu^\circ=\varnothing$である。
\end{prop}

\begin{proof}
  $\lambda,\mu$に対応する整数列をそれぞれ$(i_1,\cdots,i_d)$, $(j_1,\cdots,j_d)$とおく。$V\in\Omega_\lambda^\circ$とすると、$V$の基底$v_1,\cdots,v_d$を
  \begin{align*}
    &v_1=e_{i_1}+f_1,\quad f_1\in\generated{e_{i_1+1},\cdots,e_n}=F^{i_1}\\
    &v_2=e_{i_2}+f_2,\quad f_2\in\generated{e_{i_2+1},\cdots,e_n}=F^{i_2}\\
    &\quad\vdots\\
    &v_d=e_{i_d}+f_d,\quad f_d\in\generated{e_{i_d+1},\cdots,e_n}=F^{i_d}
  \end{align*}
  となるように取れる。さらにもし$V\in\Omega_{\mu}^\circ$でもあるなら$V$の別の基底$w_1,\cdots,w_d$で
  \begin{align*}
    &w_1=e_{j_1}+s_1,\quad s_1\in\generated{e_{j_1+1},\cdots,e_n}=F^{j_1}\\
    &w_2=e_{j_2}+s_2,\quad s_2\in\generated{e_{j_2+1},\cdots,e_n}=F^{j_2}\\
    &\quad\vdots\\
    &w_d=e_{j_d}+s_d,\quad s_d\in\generated{e_{j_d+1},\cdots,e_n}=F^{j_d}
  \end{align*}
  となるものが存在する。ここで$j_k\neq\{i_1,\cdots,i_d\}$なる$j_k$に対して、$w_{k}\neq\generated{v_1,\cdots,v_d}$であることを示す。もし
  \[
  w_k=c_1v_1+\cdots+c_dv_d  
  \]
  となったとする。このとき
  \[
  e_{j_k}+s_k=c_1e_{i_1}+\cdots+c_de_{i_d}+c_1f_1+\cdots+c_df_d  
  \]
  だが、$i_1<\cdots<i_{t}<j_k<i_{t+1}<\cdots<i_d$として両辺$F^{j_k}$の剰余類を取れば
  \[
  \overline{e_{j_k}}=c_1\overline{e_{i_1}}+\cdots+c_t\overline{e_{i_t}}+c_1\overline{f_1}+\cdots+c_t\overline{f_t}  
  \]
  となる。$\overline{e_{i_1}}\in (F^{i_1-1}/F^{j_k})\setminus (F^{i_1}/F^{j_k})$かつ、それ以外のすべての元は$F^{i_1}/F^{j_k}$に含まれている。したがって$c_1=0$でなければならない。$e_{i_2}$に同様の議論をして$c_2=0$. 結局$c_t=0$までが言えるので、$\overline{e_{j_k}}=0$すなわち$e_{j_k}\in F^{j_k}$となるがこれは矛盾である。
\end{proof}

\begin{cor}
  次が成り立つ。
  \begin{enumerate}
    \item $\mathcal{G}(d,n)=\bigsqcup_{\lambda\in\mathcal{Y}_d(n)}\Omega_{\lambda}^\circ$
    \item $\Omega_{\lambda}(F^\bullet_{st})=\bigsqcup_{\mu\supset\lambda}\Omega_{\lambda}^\circ(F^\bullet_{st})$
  \end{enumerate}
\end{cor}





つぎに$\gl_n(\complex)$の作用について解説する。

$\gl_n(\complex)$は$\mathcal{G}(d,n)$に左からの積によって自然に作用する。$g:\mathcal{G}(d,n)\rightarrow\mathcal{G}(d,n)$は多項式写像であるので連続であり、$g^{-1}$がが逆写像を与えるので同相である。

旗$F^\bullet$と$g\in\gl_n(\complex)$に対して$gF^\bullet$を
\[
gF^\bullet:\complex^n=gF^0\supset gF^1\supset\cdots\supset gF^n=0
\]
によって定めれば$gF^\bullet$は新しい旗になる。逆に任意の旗$F^\bullet,E^\bullet$に対して、$F^k=\generated{v_{k+1},\cdots,v_n}$, $E^k=\generated{w_{k+1},\cdots,w_n}$となる$\complex^n$の基底$v_1,\cdots,v_n$, $w_1,\cdots,w_n$をとって変換行列$g$を考えれば、$gF^\bullet =E^\bullet$となる。言い換えれば$\gl_n(\complex)$は$\complex^n$の旗全体のなす集合に推移的に作用する。

$B\subset\gl_n(\complex)$を対角成分が$1$の下三角行列全体のなす部分群とし、$1\leq i_1<\cdots<i_d\leq n$に対して
\[
E_{i_1,\cdots,i_d}=\generated{e_{i_1},\cdots,e_{i_d}}\in\Omega_{i_1,\cdots,i_d}^\circ(F^\bullet_{st}) 
\]
とする。このとき
\[
Be_{i_k}=e_{i_k}+f_k,\quad f_k\in F^{i_k}  
\]
と書くことができるから、$\Omega_{i_1,\cdots,i_d}^\circ(F^\bullet_{st})=BE_{i_1,\cdots,i_d}$である。また$i_1,\cdots,i_d$に対応するYoung図形$\lambda$に対して
\[
E_\lambda=E_{i_1,\cdots,i_d}  
\]
とする。

\begin{prop}\label{flag_schubert}
  $F^\bullet$を旗、$g\in\gl_n(\complex)$とする。$g\Omega_\lambda(F^\bullet)=\Omega_{\lambda}(gF^\bullet)$である。  
\end{prop}

\begin{proof}
  \begin{align*}
    V\in\Omega_\lambda(gF^\bullet)
    &\Leftrightarrow \dim(V\cap gF^{\lambda_k+d-k})\geq k, \text{ for all $k$ }\\
    &\Leftrightarrow \dim(\inv{g}V\cap F^{\lambda_k+d-k})\geq k, \text{ for all $k$}\\
    &\Leftrightarrow \inv{g}V\in \Omega_{\lambda}(F^\bullet)\\
    &\Leftrightarrow V\in g\Omega_{\lambda}(F^\bullet)
  \end{align*}
\end{proof}

任意の旗$F^\bullet$に対して、$F^\bullet$はある$g\in\gl_n(\complex)$で$F^\bullet=gF_{st}^\bullet$と書けるが、
\[
\Omega_{\lambda}(F_{st}^\bullet)=\bigsqcup_{\mu\supset\lambda}\Omega_{\mu}^\circ  
\]
であったから、
\[
\Omega_{\lambda}(F)=\bigsqcup_{\mu\supset\lambda}g\Omega_{\mu}^\circ    
\]
そこで、$g\Omega_{\mu}^\circ$を旗$F^\bullet$に付随するSchubert胞体といい$\Omega_{\mu}^\circ(F^\bullet)$とかく。


このように一般のSchubert多様体は$\gl_n(\complex)$の作用によって得られる。

次にSchubert多様体が既約な代数多様体であることを示す。


\begin{prop}
  $\Omega_{\lambda}(F^\bullet)\subset \mathcal{G}(d,n)\subset\proj(\bigwedge^d\complex^n)$はZariski閉集合である。
\end{prop}

\begin{proof}
  命題\ref{flag_schubert}より$\Omega_{\lambda}(F^{\bullet}_{st})$に対して示せば十分である。
  また、命題\ref{zariski_vs_quotient}より、$\mathcal{M}(d,n)$によって誘導される位相に関して閉集合であることを示せばよい。$\Omega_{\lambda}(F^\bullet_{st})=\set{V\in\mathcal{G}(d,n)}{\dim V\cap F^{\lambda_i+d-i}\geq i}$において、線形写像$\eta_i(V)$を
  \[
  \eta_i(V):V\rightarrow \complex^n\rightarrow \complex^n/F^{\lambda_i+d-i}
  \]
  なる自然な写像とすれば、$\dim V\cap F^{\lambda_i+d-i}\geq i$は$\rank\eta_i(V)\geq d-i$と同値である。
  $\phi:\mathcal{M}(d,n)\rightarrow\mathcal{M}(d,n)/\gl_d(\complex)=\mathcal{G}(d,n)$を自然な写像とすれば
  \[
  \inv{\phi}(\Omega_{\lambda}(F^\bullet_{st}))=\set{A\in\mathcal{M}(d,n)}{\rank\eta_i([A])\geq d-i}  
  \]
  となる。したがって$\rank\eta_i([A])\geq d-i$が($A$の成分の多項式)$=0$の形で記述できることがわかればよい。$V=[A]$とする。$V$の基底を$A$の列ベクトル, $\complex^n$の基底を標準基底でとれば、
  \[
  V\rightarrow \complex^n
  \]
  の表現行列は$A$に他ならない。また、$\complex^n/F^{\lambda_i+d-i}$の基底として$e_1,\cdots,e_{\lambda_i+d-i}$がとれるから
  \[
  \complex^n\rightarrow \complex^n/F^{\lambda_i+d-i}  
  \]
  の表現行列は
  \[
  \pmat{E_{\lambda_i+d-i}&0},\qquad \text{ただし$E_{\lambda_i+d-i}$は$\lambda_i+d-i$次単位行列}  
  \]
  となる。したがって$\eta_i([A])$の表現行列は
  \[
  \pmat{E_{\lambda_i+d-i}&0}A=\pmat{A_{\lambda_i+d-i}&0}  
  \]
  となる。ただし$A_{\lambda_i+d-i}$は$A$の第$1$行から$\lambda_i+d-i$行までを取り出した小正方行列である。よって$\rank\eta_i(V)\geq d-i$は$\pmat{A_{\lambda_i+d-i}&0}$の$d-i+1$小行列式がすべて$0$になることと同値であるから、これは($A$の成分の多項式)$=0$の形である。
\end{proof}



\begin{prop}\label{schubert_closure}
  Schubert多様体$\Omega_{\lambda}(F^\bullet)$はSchubert胞体$\Omega_{\lambda}^\circ(F^\bullet)$の閉包である:
  \[
    \Omega_{\lambda}(F^\bullet)=\overline{\Omega_{\lambda}^\circ(F^\bullet)}
  \]
\end{prop}

\begin{proof}
  $\Omega_{\lambda}(F^{\bullet}_{st})$に対して示せば十分である。次の補題を用いる。
  \begin{lemm}
    $\lambda\subset\mu$ならば$\Omega_{\mu}^\circ(F^\bullet_{st})\subset\overline{\Omega_{\lambda}^\circ(F^\bullet_{st})}$
  \end{lemm}

  \begin{proof}
    $\lambda,\mu$に$I=(1\leq i_1<\cdots< i_d\leq n),J=(1\leq j_i<\cdots<j_d\leq n)$がそれぞれ対応しているとする。$i_k\leq j_k$ for all $k$である。次の操作を考える。
    \begin{equation}\label{operation}
    \text{$\alpha\notin I\setminus J$, $\beta\in J$なる$\alpha <\beta$をとり、$J$から$\beta$を取り除き$\alpha$を加える。}
    \end{equation}
    $J$に操作(\ref{operation})を有限回施すことで$I$を得ることができる。そこで、$J$が操作(\ref{operation})を1回施すことで$I$を得ることができる場合を考えればよい。実際、もしこの場合に証明できれば、$J$に$k$回操作(\ref{operation})を施したものを$J_k$とおけば、
    \[
    \Omega_{J}^\circ(F^\bullet_{st})
    \subset\overline{\Omega_{J_1}^\circ(F^\bullet_{st})}\subset\overline{\overline{\Omega_{J_2}^\circ(F^\bullet_{st})}}=\overline{\Omega_{J_2}^\circ(F^\bullet_{st})}  
    \subset\cdots
    \]
    となるから示せる。

    \[
    \text{例: }I=(1,3,6),\: J=(1,4,6)\qquad\rightarrow\qquad (\alpha=3,\beta=4)
    \]
    $E_J\in \overline{\Omega_{I}^\circ(F^\bullet_{st})}$を示せば
    $\Omega_{J}^\circ(F^\bullet_{st})
    =BE_J\subset B\overline{\Omega_{I}^\circ(F^\bullet_{st})}\subset \overline{\Omega_{I}^\circ(F^\bullet_{st})}$より主張が従う。$\phi:\proj^1\rightarrow \mathcal{G}(d,n)$を
    \[
    \phi([s:t])=\generated{\set{e_k}{k\in I\cap J}\cup\{se_{\alpha}+te_{\beta}\}}  
    \]
    によって定める。$\phi$は多項式写像であるので連続である。$\phi([1:0])=E_I$, $\phi([0:1])=E_J$であるから、$\phi$は$E_I$と$E_J$を結ぶ曲線だと思うことができる。しかも$\phi([1:t])\in \Omega_{I}^\circ(F^\bullet_{st})$である。よって
    \[
    E_J=\phi([0:1])\in\phi(\overline{\proj^1\setminus\{[0:1]\}})
    \subset\overline{\phi(\proj^1\setminus\{[0:1]\})}\subset\overline{\Omega_{I}^\circ(F^\bullet_{st})}
    \]
    となり示せた。
  \end{proof}

  命題\ref{schubert_closure}の証明に戻ろう。補題より
  \[
  \Omega_{\lambda}(F^\bullet_{st})
  =\bigsqcup_{\mu\supset\lambda}\Omega_{\mu}^\circ(F^\bullet_{st})  
  \subset
  \overline{\Omega_{\lambda}^\circ(F^\bullet_{st})}
  \]
  であり、$\Omega_{\lambda}^\circ(F^\bullet_{st})\subset\Omega_{\lambda}(F^\bullet_{st})$かつ$\Omega_{\lambda}(F^\bullet_{st})$は閉集合であるから、
  \[
    \overline{\Omega_{\lambda}^\circ(F^\bullet_{st})}\subset\Omega_{\lambda}(F^\bullet_{st})  
  \]
\end{proof}

\begin{cor}
  Schubert多様体は既約である。
\end{cor}

\begin{proof}
  一般に位相空間$X$の部分集合$A$が既約であるならその閉包$\overline{A}$も既約である。実際、$\overline{A}=Y_1\cup Y_2$となる閉集合$Y_1,Y_2$が存在したら、$A=(\overline{A}\cap Y_1)\cup(\overline{A}\cap Y_2)$となるから$A$は可約になる。

  Schubert胞体はアフィン空間と同相であるから既約であるので、その閉包であるSchubert多様体も既約である。
\end{proof}


一般に代数多様体$X$の部分多様体$Z$について、$\dim X-\dim Z$を$Z$の余次元といい$\codim Z$と書く。

\begin{fact}[\cite{enu_geo}]
  $\codim\Omega_{\lambda}(F^\bullet)=|\lambda|$である。
\end{fact}



\end{document} 


\newpage
\documentclass{ltjsreport}
\input{../../setting.tex}


\begin{document}

\section{表現環と対称関数環}

\subsection{続・対称多項式}
対称群の表現と対称多項式の間には深い関係がある。次節でそのことを解説するが、そのための準備として対称多項式に関してより詳しく解説する。以下正の整数$n$を固定し、$\Lambda^k_n$を$n$変数の$k$次斉次対称多項式のなす$\integer$加群とする。第1部の記号を復習すると、$n$行のYoung図形$\lambda$に対して
\[
m_\lambda=\sum_{\alpha\sim\lambda}x_1^{\alpha_1}\cdots x_n^{\alpha_n}  
\]
とし、に対して
\begin{align*}
  &e_k=m_{1^k}=\sum_{1\leq i_1<\cdots<i_k\leq n}x_{i_1}\cdots x_{i_k},\qquad(k=1,\cdots,n)\\
  &h_k=\sum_{\lambda\in\mathcal{P}_k}m_\lambda=\sum_{1\leq i_1\leq\cdots\leq i_k\leq n}x_{i_1}\cdots x_{i_k}\\
  &p_k=m_{(k)}=x_1^k+\cdots+x_n^k
\end{align*}
とするのであった。$e_k$や$h_k$に対しては、その母関数を考えることは有用である。すなわち
\begin{align}
  &E(t)=1+e_1t+e_2t^2+\cdots+e_nt^n=\prod_{i=1}^n(1+x_it) \label{gen_func_of_e}\\
  &H(t)=1-h_1t+h_2t^2+\cdots+(-1)^nh_nt^n+\cdots=\prod_{i=1}^n\frac{1}{1+x_it}
\end{align}
である。とくに$E(t)H(t)=1$であるので、$k=1,\cdots,n$のとき
\begin{equation}\label{e_to_h}
  e_k-h_1e_{k-1}+\cdots+(-1)^{k-1}e_1h_{k-1}+(-1)^kh_k=0   
\end{equation}
を得る。

\begin{prop}\label{det_formula}
  $k=1,\cdots,n$に対して
  \begin{align*}
    h_k=\vmat{
      e_1&e_2&e_3&\cdots&e_k\\
      1&e_1&e_2&\cdots&e_{k-1}\\
      0&1&e_1&\cdots&e_{k-2}\\
      \vdots&\vdots&\vdots&\ddots&\vdots&\\
      0&0&0&\cdots&e_1},\qquad
    e_k=\vmat{
      h_1&h_2&h_3&\cdots&h_k\\
      1&h_1&h_2&\cdots&h_{k-1}\\
      0&1&h_1&\cdots&h_{k-2}\\
      \vdots&\vdots&\vdots&\ddots&\vdots&\\
      0&0&0&\cdots&h_1
    }
  \end{align*}
\end{prop}

\begin{proof}
  まったく同様なので$h_k$の場合だけ示す。$e_1=h_1$であり、$k-1$までこの公式が成り立っていたとすると、
  \begin{align*}
    e_k&=h_1e_{k-1}-h_2e_{k-2}+\cdots+(-1)^{k-1}h_k \\
    &=h_1\vmat{
      h_1&h_2&\cdots&h_{k-1}\\
      1&h_1&\cdots&h_{k-2}\\
      \vdots&\vdots&\ddots&\vdots&\\
      0&0&\cdots&h_1
    }-h_2\vmat{
      h_1&\cdots&h_{k-1}\\
      \vdots&\ddots&\vdots&\\
      0&\cdots&h_1
    }+\cdots+(-1)^{k-1}h_k\\
    &=\vmat{
      h_1&h_2&h_3&\cdots&h_k\\
      1&h_1&h_2&\cdots&h_{k-1}\\
      0&1&h_1&\cdots&h_{k-2}\\
      \vdots&\vdots&\vdots&\ddots&\vdots&\\
      0&0&0&\cdots&h_1
    }
  \end{align*}
\end{proof}

次に、$\lambda\in\mathcal{P}_k$に対して
\begin{align*}
  &e_\lambda=e_{\lambda_1}\cdots e_{\lambda_n}\\
  &h_\lambda=h_{\lambda_1}\cdots h_{\lambda_n}\\
  &p_\lambda=p_{\lambda_1}\cdots p_{\lambda_n}
\end{align*}
とする。

\begin{prop}\label{various_basis}
  $\Lambda^k_n$の次の部分集合について
  \begin{enumerate}[(i)]
    \item $\{m_\lambda\}_\lambda$ ただし$\lambda$は大きさが$k$で$n$行
    \item $\{e_\lambda\}_\lambda$ ただし$\lambda$は大きさが$k$で$n$列
    \item $\{h_\lambda\}_\lambda$ ただし$\lambda$は大きさが$k$で$n$列
    \item $\{s_\lambda\}_\lambda$ ただし$\lambda$は大きさが$k$で$n$行
    \item $\{p_\lambda\}_\lambda$ ただし$\lambda$は大きさが$k$で$n$列
  \end{enumerate}
  (i)$\sim$(iv)は$\Lambda^k_n$の$\integer$上の基底をなし、(v)は$\quotient\otimes_{\integer}\Lambda^k_n$の基底をなす。
\end{prop}

\begin{proof}
  (i)は命題\ref{m_is_basis}の証明を斉次部分で考えればまったく同様である。(ii)については定理\ref{FT_of_sym}の証明において、任意の対称多項式$f$が
  \[
  e_1^{a_1}\cdots e_n^{a_n}
  \]
  で生成されていることを示したことからわかる。(iv)は\ref{schur}の証明を斉次部分で行えばよい。(iii)については、命題\ref{det_formula}より$\{h_\lambda\}$が$\{e_\lambda\}$を生成することがわかるが、ともに集合の濃度が等しいことから基底をなすことがわかる。(v)が基底をなすことを示そう。(v)が(iii)を生成することを示せばよい。
  \begin{align*}
    &\quad1+h_1t+h_2t^2+\cdots\\
    &=H(-t)\\
    &=\prod_{i=1}^n\frac{1}{1-x_it}\\
    &=\exp \left(\sum_{i=1}^n-\log(1-x_it)\right)\\
    &=\exp \left(\sum_{i=1}^n\sum_{r=1}^\infty\frac{x_i^rt^r}{r}\right)\\
    &=\exp \left(\sum_{r=1}^\infty\frac{p_r}{r}t^r\right)\\
    &=\prod_{r=1}^\infty\exp \left(\frac{p_r}{r}t^r\right)\\
    &=\prod_{r=1}^\infty \sum_{m_r=0}^\infty \frac{p_r^{m_r}}{m_r!\cdot r^{m_r}}t^{r\cdot m_r}\\
    &=
    \left(\sum_{m_1=0}^\infty \frac{p_1^{m_1}}{m_1!\cdot 1^{m_1}}t^{m_1}\right)\cdot
    \left(\sum_{m_2=0}^\infty \frac{p_2^{m_2}}{m_2!\cdot 2^{m_2}}t^{2 \cdot m_2}\right)\cdot
    \left(\sum_{m_3=0}^\infty \frac{p_3^{m_3}}{m_3!\cdot 3^{m_3}}t^{3 \cdot m_3}\right)\cdots
  \end{align*}
  となるから、Young図形$\lambda=(\lambda_1,\lambda_2,\cdots)$に対して
  \[
  z(\lambda)=\prod_{i}\lambda_i!\cdot i^{\lambda_i}  
  \]
  とおけば、最後の式は
  \begin{align*}
    &\quad\sum_{m_1,m_2,m_3,\cdots}\frac{p_1^{m_1}p_2^{m_2}p_3^{m_3}\cdots}{(m_1!\cdot 1^{m_1})(m_2!\cdot 2^{m_2})(m_3!\cdot 3^{m_3})\cdots}t^{m_1+2\cdot m_2+3\cdot m_3+\cdots}\\
    &=\sum_{\lambda}\frac{p_\lambda}{z(\lambda)}t^{|\lambda|}\\
    &=\sum_{k=0}^\infty \sum_{\lambda\in\mathcal{P}_{k}}\frac{p_\lambda}{z(\lambda)}t^k
  \end{align*}
  となる。よって
  \[
  h_k=\sum_{\lambda\in\mathcal{P}_{k}}\frac{p_\lambda}{z(\lambda)}
  \]
  が成り立つ。$n$列のYoung図形$\lambda$に対して
  \[
  h_\lambda=h_{\lambda_1}\cdots h_{\lambda_s},\qquad \lambda_i\leq n 
  \]
  とおくと、
  \[
  h_{\lambda_i}=\sum_{\mu_i\in\mathcal{P}_{\lambda_i}}\frac{p_{\mu_i}}{z(\mu_i)}  
  \]
  であるから
  \[
  h_\lambda=\sum_{\mu_1,\cdots,\mu_s}\frac{p_{\mu_1}\cdots p_{\mu_s}}{z(\mu_1)\cdots z(\mu_s)}  
  \]
  各$\mu_i$は$\lambda_i\leq n$の分割を与えているから、$p_{\mu_1}\cdots p_{\mu_s}$はたかだか$n$列のYoung図形に対応するべき和対称式である。よって(v)も基底を与える。
\end{proof}

証明中に現れた等式は重要なので再掲しておく。

\begin{prop}\label{p_to_h}
  正の整数$k,n$に対して
  \[
    h_k(x_1,\cdots,x_n)=\sum_{\lambda\in\mathcal{P}_{k}}\frac{p_\lambda(x_1,\cdots,x_n)}{z(\lambda)}
  \]
  が成り立つ。
\end{prop}

後に必要になる公式を用意しておく


\begin{lemm}[Cauchyの等式]\label{formal_power_series_relation}
  形式的べき級数$\prod_{i=1}^m\prod_{j=1}^n\frac{1}{1-x_iy_j}$は次と等しい。
  \begin{align*}
    &\text{(i):}\:\sum_{\lambda,\lambda_{n+1}=0}h_\lambda(x_1,\cdots,x_m)m_\lambda(y_1,\cdots,y_n),\qquad \text{ただし和は$n$行Young図形全体をわたる}\\
    &\text{(ii):}\:\sum_{\lambda}\frac{1}{z(\lambda)}p_\lambda(x_1,\cdots,x_m)p_\lambda(y_1,\cdots,y_n),\qquad \text{ただし和はすべてのYoung図形全体をわたる}\\
    &\text{(iii):}\:\sum_{\lambda}s_\lambda(x_1,\cdots,x_m)s_\lambda(y_1,\cdots,y_n),\qquad \text{ただし和はすべてのYoung図形全体をわたる}
  \end{align*}
\end{lemm}

\begin{proof}
  (i)を示す。
  \begin{align*}
    &\quad\prod_{i=1}^m\prod_{j=1}^n\frac{1}{1-x_iy_j}\\
    &=\prod_{j=1}^n H(y_j)\\
    &=\prod_{j=1}^n \left(
      \sum_{k_j=0}^\infty h_{k_j}(x_1,\cdots,x_m)y_j^{k_j}
      \right)\\
    &=\left(
      \sum_{k_1=0}^\infty h_{k_1}(x_1,\cdots,x_m)y_1^{k_1}
      \right)\cdot
      \left(
      \sum_{k_2=0}^\infty h_{k_2}(x_1,\cdots,x_m)y_2^{k_2}
      \right)
      \cdots
      \left(
      \sum_{k_n=0}^\infty h_{k_n}(x_1,\cdots,x_m)y_n^{k_n}
      \right)\\
    &=\sum_{k_1,\cdots,k_n}h_{k_1}\cdots h_{k_n}y_1^{k_1}  
      \cdots y_n^{k_n}\\
    &=\sum_{\lambda, \lambda_{n+1}=0}h_\lambda(x_1,\cdots,
      x_m)m_\lambda(y_1,\cdots,y_n)
  \end{align*}
  (ii)を示す。命題\ref{p_to_h}より
  \begin{align*}
  \prod_{i,j}\frac{1}{1-x_iy_j}
  &=1+h_1(\{x_iy_j\})+h_2(\{x_iy_j\})+\cdots\\
  &=1+\sum_{\lambda\in\mathcal{P}_1}\frac{1}{z(\lambda)}p_\lambda(\{x_iy_j\})+\sum_{\lambda\in\mathcal{P}_2}\frac{1}{z(\lambda)}p_\lambda(\{x_iy_j\})+\cdots\\
  &=\sum_{\lambda}\frac{1}{z(\lambda)}p_\lambda(\{x_iy_j\})\\
  &=\sum_{\lambda}\frac{1}{z(\lambda)}p_\lambda(x_1,\cdots,x_m)p_\lambda(y_1,\cdots,y_n)
  \end{align*}
  (iii)については、Robinson-Schensted-Knuth対応と呼ばれる対応を用いて証明される。詳細は付録を参照
\end{proof}











\subsection{表現環と対称関数環}

\begin{defin}
  可算無限個の変数をもつ形式的べき級数環$\integer[[x_1,x_2,\cdots]]$を考える。
  \[
  \mathfrak{S}=\set{\map{\sigma}{\natnum}{\natnum}}{\text{$f$は全単射で$f(n)\neq n$なる$n$が有限個}}  
  \]
  とする
  \footnote{
    $\mathfrak{S}$は対称群$\mathfrak{S}_n$と自然な包含$\iota:\mathfrak{S}_n\rightarrow\mathfrak{S}_{n+1}$のなす帰納系の帰納極限である。
  }。
  \[
  \Lambda=\set{f\in \integer[[x_1,x_2,\cdots]]}{\sigma f=f,\:(\text{for all }\sigma\in\mathfrak{S}),\:\text{$f$の単項式の次数は有界}}  
  \]
  $\Lambda$は$\integer[[x_1,x_2,\cdots]]$の部分環で対称関数環と呼ばれる。$\Lambda^k$を
  \[
  \Lambda^k=\set{f\in\Lambda}{\text{$f$の単項式の次数はすべて$k$}}  
  \]
  で定め、$\Lambda^k$の元を$k$次斉次対称関数という。
  \[
  \Lambda=\bigoplus_{k=0}^\infty\Lambda^k  
  \]
  より$\Lambda$は次数付き環の構造をもつ。
\end{defin}

ここで、$\Lambda$の定義において単項式の次数が有界であることを要請するのは自然である。実際、もし仮定しなければ$\Lambda$は$\Lambda^k$の直和にはならない。

\begin{eg}[単項対称関数]\label{monomial_sym_func}
  任意のYoung図形$\lambda=(\lambda_1,\cdots,\lambda_n)$に対して
  \[
  m_{\lambda}=\sum_{\alpha\sim\lambda}x_1^{\alpha_1}x_2^{\alpha_2}\cdots x_{n}^{\alpha_n}
  \]
  とする。ここで指数$\alpha$は、$\lambda$の置換になっているもの全体をわたる。すなわちある$\sigma\in\mathfrak{S}$が存在して$\alpha=\sigma\lambda$をみたすもの全体である。$m_\lambda$は対称関数である。対称多項式の場合と同様の議論で、$\Lambda^k$は$\{m_\lambda\}_{\lambda\in\mathcal{P}_k}$を基底に持つことがわかる。
\end{eg}

\begin{eg}[基本対称関数・完全対称関数]\label{elementary_func}
  \begin{align*}
  &e_k=m_{1^k}
  =\sum_{1\leq i_1<i_2<\cdots<i_k}x_{i_1}x_{i_2}\cdots x_{i_k}\\
  &h_k=\sum_{\lambda\in\mathcal{P}_k} m_\lambda
  =\sum_{1\leq i_1\leq i_2\leq \cdots\leq i_k}x_{i_1}x_{i_2}\cdots x_{i_k}
  \end{align*}
  をそれぞれ、基本対称関数, 完全対称関数という。また、任意のYoung図形$\lambda=(\lambda_1,\cdots,\lambda_n)$に対して
  \begin{align*}
    &e_\lambda=e_{\lambda_1}\cdots e_{\lambda_n}\\
    &h_\lambda=h_{\lambda_1}\cdots h_{\lambda_n}
  \end{align*}
  とする。
  \begin{align*}
    &e_1=x_1+x_2+x_3+\cdots\\
    &e_2=\sum_{i<j}x_ix_j
  \end{align*}
  である。
\end{eg}

\begin{eg}[べき和対称関数]\label{power_func}
  $(k)=(k,0,\cdots,0)$に対して
  \[
  p_{k}=m_{(k)}=x_1^k+x_2^k+\cdots
  \]
  とする。
\end{eg}

このように、対称関数はいままでみてきた対称多項式を自然に無限変数に拡張した概念であり、対称多項式で成り立っていた関係式が対称関数においても成立することが多い。このことは対称関数の$k$次斉次部分$\Lambda^k$が$k$次斉次対称多項式からの射影極限と考えることができることによる。$\Lambda_n^k$を$n$変数$k$次斉次対称多項式のなす$\integer$加群とする。$m\leq n$に対して線形写像$\map{\rho_{m,n}}{\Lambda^k_n}{\Lambda^k_m}$を
\[
\rho_{m,n}(f(x_1,\cdots,x_m,x_{m+1},\cdots,x_n))=f(x_1,\cdots,x_m,0,\cdots,0)  
\]
によって定める。ここで$\rho_{m,n}(f)$は実際に$m$変数の$k$次斉次対称多項式である\footnote{変数の置換と$0$を代入する操作が可換であることによる}。$l\leq m\leq n$に対して
\[
\rho_{l,m}\circ\rho_{m,n}=\rho_{l,n}  
\]
が成り立つから、$\{\Lambda^k_n,\rho_{m,n}\}$は射影系をなす。

\begin{prop}\label{sym_func_is_inverselimit}
  上の状況において、
  \[
    \Lambda^k=\mathop{\lim_{\longleftarrow}}\Lambda^k_n
  \]
  がなりたつ。
\end{prop}

\begin{proof}
  $\theta_n:\Lambda^k\rightarrow\Lambda^k_n$を$n+1$番目以降の変数を$0$にする写像とすれば、
  \[
  \rho_{m,n}\circ\theta_n=\theta_m  
  \]
  が成り立つから、射影極限の普遍性から線形写像
  \[
  \theta:\Lambda^k\rightarrow\mathop{\lim_{\longleftarrow}}\Lambda^k_n
  \]
  が誘導される。$\mathop{\lim_{\longleftarrow}}\Lambda^k_n$から$\Lambda^k$への写像$\varphi$は次のように定義する。$\mathop{\lim_{\longleftarrow}}\Lambda^k_n$の元$(f_n)_{n\in\integer_{>0}}$, ($f_n\in\Lambda^k_n$)に対して、$k$変数の部分に注目すると
  \[
  f_k=\sum_{\lambda\in\mathcal{P}_k}c_\lambda m_\lambda(x_1,\cdots,x_k)
  \]
  と一意的に表せるので
  \[
  \varphi((f_n)_{n\in\integer_{>0}})
  =\sum_{\lambda\in\mathcal{P}_k}c_\lambda m_\lambda
  \]
  と定める。ただし右辺の$m_\lambda$は例\ref{monomial_sym_func}の対称関数である。$\varphi$が$\theta$の逆写像であることを示そう。
  \begin{equation}\label{proj_of_monomial}
  \theta_n(m_\lambda)=
  \left\{\begin{array}{cc}
    m_\lambda(x_1,\cdots,x_n) & \text{if }\lambda_{n+1}=0\\
    0 & \text{otherwise}
  \end{array}\right.  
\end{equation}
  であるが、$\lambda$は$k$の分割であるので$n\geq k$において$\lambda_{n+1}=0$である。よって$n\geq k$ならば
  \begin{equation}\label{theta_n}
  \theta_n(\varphi((f_n)_{n\in\integer_{>0}}))
  =\sum_{\lambda\in\mathcal{P}_k}c_\lambda m_\lambda(x_1,\cdots,x_n)  
  \end{equation}
  が成り立つ。一方、
  \[
  f_n=\sum_{\lambda\in\mathcal{P}_k(n)}d_\lambda m_\lambda(x_1,\cdots,x_n)    
  \]
  とおくと$n\geq k$より$\mathcal{P}_k(n)=\mathcal{P}_k$だから
  \[
  f_n=\sum_{\lambda\in\mathcal{P}_k}d_\lambda m_\lambda(x_1,\cdots,x_n)    
  \]
  となる。よって$\rho_{k,n}(f_n)=f_k$と(\ref{theta_n})より
  \[
  f_n=\theta_n(\varphi((f_n)_{n\in\integer_{>0}}))
  \]
  次に$n<k$の場合、(\ref{proj_of_monomial})より
  \[
  \theta_n(\varphi((f_n)_{n\in\integer_{>0}}))  
  =\sum_{\lambda\in\mathcal{P}(k)}c_\lambda m_\lambda
  \]
  となるが、
  \[
  f_n=\sum_{\lambda\in\mathcal{P}_k}d_\lambda m_\lambda(x_1,\cdots,x_n)    
  \]
  とおけば$\rho_{n,k}(f_k)=f_n$より
  \[
  f_n=\theta_n(\varphi((f_n)_{n\in\integer_{>0}}))
  \]
  以上より
  \[
  \theta\circ\varphi=\text{id}
  \]
  がわかる。逆に任意の$f\in\Lambda^k$に対して
  \[
  f=\sum_{\lambda\in\mathcal{P}_k}c_\lambda m_\lambda  
  \]
  とおけば(\ref{proj_of_monomial})より
  \[
  \theta_k(f)=\sum_{\lambda\in\mathcal{P}_k}c_\lambda m_\lambda(x_1,\cdots,x_k)
  \]
  だから
  \[
  \varphi(\theta(f))=f  
  \]
  がわかる。
\end{proof}


\begin{notice}
  対称多項式環の射影極限をとっても対称関数環にはならないことに注意せよ。例えば対称多項式の列
  \[
    f=((1+x_1),\:(1+x_1)(1+x_2),\:(1+x_1)(1+x_2)(1+x_3),\:\cdots)
  \]
  を考えると、$f$は対称多項式の射影極限の元であるが、次数の有界性を満たさないので対称関数ではない。

  一方
  \[
  \integer[x_1,\cdots,x_n]^{\mathfrak{S}_n}=\bigoplus_{k=0}^\infty \Lambda^k_n  
  \]
  だから、対称関数環は射影極限が直和と可換でない例を与えている。
\end{notice}


\begin{eg}\label{monomial_sym_fun_as_lim}
  $n$行のYoung図形$\lambda$に対して
  \[
  \rho_{n,n+1}(m_\lambda(x_1,\cdots,x_n,x_{n+1}))=m_\lambda(x_1,\cdots,x_n)  
  \]
  が成り立つ。実際
  \begin{align*}
    \rho_{n,n+1}(m_\lambda(x_1,\cdots,x_n,x_{n+1}))
    &=\rho_{n,n+1}\left(
      \sum_{\alpha\sim\lambda}x_1^{\alpha_1}\cdots x_n^{\alpha_n}x_{n+1}^{\alpha_{n+1}}
      \right)\\
    &=\sum_{\substack{\alpha\sim\lambda \\ \alpha_{n+1}=0}}x_1^{\alpha_1}\cdots x_n^{\alpha_n}\\
    &=m_\lambda(x_1,\cdots,x_n)
  \end{align*}
  よって$k=|\lambda|$次対称多項式の列$(m_\lambda(x_1,\cdots,x_l))_{l\in\integer_{\geq n}}$は一つの対称関数を定めるが、これは単項対称関数$m_\lambda$に他ならない。
\end{eg}

\begin{eg}
  例\ref{monomial_sym_fun_as_lim}と命題\ref{various_basis}より$e_\lambda$, $h_\lambda$, $p_\lambda$,$s_\lambda$もすべて一つの対称関数を定める。$e_\lambda,h_\lambda,p_\lambda$の定める対称関数は、例\ref{elementary_func}と例\ref{power_func}に他ならない。また$s_\lambda$の定める対称関数はSchur関数という。
\end{eg}

命題\ref{various_basis}より、次が成り立つ。

\begin{prop}\label{various_symfunc_basis}
  $\Lambda^k$の次の部分集合について、
  \begin{enumerate}[(i)]
    \item $\{m_\lambda\}_{\lambda\in\mathcal{P}_k}$
    \item $\{e_\lambda\}_{\lambda\in\mathcal{P}_k}$
    \item $\{h_\lambda\}_{\lambda\in\mathcal{P}_k}$
    \item $\{s_\lambda\}_{\lambda\in\mathcal{P}_k}$
    \item $\{p_\lambda\}_{\lambda\in\mathcal{P}_k}$
  \end{enumerate}
  (i)$\sim$(iv)は$\Lambda^k$の$\integer$上の基底をなし、(v)は$\quotient\otimes_\integer{\Lambda^k}$上の基底をなす。特に$\lambda$の範囲をすべてのYoung図形全体に変えれば、これらは$\Lambda$または$\quotient\otimes_\integer{\Lambda}$の基底を与える。
\end{prop}

また、命題\ref{det_formula}や命題\ref{p_to_h}, 補題\ref{formal_power_series_relation}, Littlewood-Richardson規則などの関係式は、そのまま対称関数においても成立することがわかる。(変数の制限$\rho_{m,n}$は和や積と可換である)

\begin{prop}\label{various_relation}
  次の関係式が成り立つ\footnote{
    $e_k$の母関数$E(t)=\prod_{i=1}^\infty(1+x_it)$や$h_k$の母関数$H(t)=\prod_{i=1}^\infty\frac{1}{1-x_it}$を用いて補題\ref{formal_power_series_relation}と同じ計算で直接示してもよい。
  }
  \begin{align*}
    &\text{(i):}\:h_k=\vmat{
      e_1&e_2&e_3&\cdots&e_k\\
      1&e_1&e_2&\cdots&e_{k-1}\\
      0&1&e_1&\cdots&e_{k-2}\\
      \vdots&\vdots&\vdots&\ddots&\vdots&\\
      0&0&0&\cdots&e_1},\qquad
    e_k=\vmat{
      h_1&h_2&h_3&\cdots&h_k\\
      1&h_1&h_2&\cdots&h_{k-1}\\
      0&1&h_1&\cdots&h_{k-2}\\
      \vdots&\vdots&\vdots&\ddots&\vdots&\\
      0&0&0&\cdots&h_1
    }\\
    &\text{(ii):}\:h_k=\sum_{\lambda\in\mathcal{P}_k}\frac{p_\lambda}{z(\lambda)}\\
    &\text{(iii):}\:
    \sum_{\lambda}h_\lambda(x) m_\lambda(y)
    =\sum_{\lambda}\frac{1}{z(\lambda)}p_\lambda(x)p_\lambda(y)
    =\sum_{\lambda}s_\lambda(x)s_\lambda(y)
    =\prod_{i,j}\frac{1}{1-x_iy_j}\\
    &\text{(iv):}\:
    s_\lambda s_\mu=\sum_{\nu}\eta^{\nu}_{\lambda\mu}s_\nu,\qquad\text{ただし$\eta^{\nu}_{\lambda\mu}$はLittlewood-Richardson数}
  \end{align*}
\end{prop}

以下、$\Lambda$の係数を$\quotient$に拡大して考える。命題\ref{various_symfunc_basis}より、$\{s_\lambda\}$は$\Lambda$の基底をなすので、$\Lambda$に
\[
\generated{s_\lambda,s_\mu}=\delta_{\lambda\mu}  
\]
となるような内積を入れて考える。ただし$\delta_{\lambda\mu}$はKroneckerのデルタである。

\begin{prop}
  次が成り立つ
  \begin{enumerate}[(i)]
    \item $\generated{h_\lambda,m_\mu}=\delta_{\lambda\mu}$
    \item $\generated{p_\lambda,p_\mu}=\delta_{\lambda\mu}z(\lambda)$
  \end{enumerate}
\end{prop}

\begin{proof}
  $h_\lambda(x)=\sum_{\nu_1}a_{\lambda\nu_1}s_{\nu_1}(x)$, $m_\mu(y)=\sum_{\nu_2}b_{\mu\nu_2}s_{\nu_2}(y)$とおく。
  \begin{align*}
    h_\lambda(x)m_\mu(y)=\sum_{\nu_1,\nu_2}a_{\lambda\nu_1}b_{\mu\nu_2}s_{\nu_1}(x)s_{\nu_2}(y)
  \end{align*}
\end{proof}



次に対称群の表現全体から作られる環を導入する。

\begin{defin}
  $\mathfrak{S}_n$の表現の同値類全体で生成される自由Abel群を$\tilde{R_n}$とする。$D$を
  \[
  \set{[V\oplus W]-[V]-[W]\in R}{V,W\text{はそれぞれ$\mathfrak{S}_n$の表現}}  
  \]
  で生成される$\tilde{R_n}$の部分加群とし、$R_n=\tilde{R_n}/D$とする。$R_0=\integer$として$R=\bigoplus_{n=0}^\infty R_n$とおく。
  
  $\mathfrak{S}_n$, $\mathfrak{S}_m$の表現$V,W$に対して、
  \[
  [V]\circ [W]=[\ind_{\mathfrak{S}_n\times\mathfrak{S}_m}^{\mathfrak{S}_{n+m}}V\boxtimes W]
  \]
  と定める。ここで、$\mathfrak{S}_n\times\mathfrak{S}_m$は$\mathfrak{S}_n$の元を$n+1$から$n+m$を固定する置換と同一視し、$\mathfrak{S}_m$の元を$1$から$n$を固定する置換と同一視することで$\mathfrak{S}_n\times\mathfrak{S}_m$の部分群とみなしている。$\circ$を双線形に拡張することによって$R$は可換環の構造を持つ。$R$を対称群の表現環という。
\end{defin}

\begin{prop}
  $\circ$は実際に乗法を定め、$R$は次数付き可換環となる。
\end{prop}

\begin{proof}
  テンソル積と直和の可換性から、$\mathfrak{S}_n$の表現$V,V'$に対して
  \[
    \ind_{\mathfrak{S}_n\times\mathfrak{S}_m}^{\mathfrak{S}_{n+m}}(V\oplus V')\boxtimes W
    =(\ind_{\mathfrak{S}_n\times\mathfrak{S}_m}^{\mathfrak{S}_{n+m}}V\boxtimes W)
    \oplus
    (\ind_{\mathfrak{S}_n\times\mathfrak{S}_m}^{\mathfrak{S}_{n+m}}V'\boxtimes W)
  \]
  よって
  \[
  ([V]+[V'])\circ [W]=[V\oplus V']\circ [W]=[V]\circ[W]+[V']\circ[W]  
  \]
  より$\circ$は双線形である。$\circ$が可換であることは$\mathfrak{S}_n$の元と$\mathfrak{S}_m$の元が$\mathfrak{S}_{n+m}$において可換であることからわかる。

  乗法が結合的であることを示そう。すなわち
  \[
  ([V]\circ[W])\circ[U]=[V]\circ([W]\circ [U])  
  \]
  を示す。そのためには$\mathfrak{S}_n$, $\mathfrak{m}$, $\mathfrak{S}_l$それぞれの表現$V,W,U$に対して、2つの表現
  \[
  \ind_{\mathfrak{S}_{n+m}\times\mathfrak{S}_l}^{\mathfrak{S}_{n+m+l}}\left\{
    (\ind_{\mathfrak{S}_n\times\mathfrak{S}_m}^{\mathfrak{S_{n+m}}}V\boxtimes W)\boxtimes U
  \right\},\qquad
  \ind_{\mathfrak{S}_{n}\times\mathfrak{S}_{m+l}}^{\mathfrak{S}_{n+m+l}}\left\{
    V\boxtimes (\ind_{\mathfrak{S}_m\times\mathfrak{S}_l}^{\mathfrak{S}_{m+l}}W\boxtimes U)
  \right\}
  \]
  の指標が等しいことをみればよい。
  $
    \ind_{\mathfrak{S}_{n+m}\times\mathfrak{S}_l}^{\mathfrak{S}_{n+m+l}}\left\{
      (\ind_{\mathfrak{S}_n\times\mathfrak{S}_m}^{\mathfrak{S_{n+m}}}V\boxtimes W)\boxtimes U
    \right\} 
  $ 
  の指標を$\chi_A$とおき、$\ind_{\mathfrak{S}_n\times\mathfrak{S}_m}^{\mathfrak{S_{n+m}}}V\boxtimes W$の指標を$\chi_{A'}$とおく。誘導指標の公式(命題\ref{ind_char})より、
  \begin{align*}
    \chi_A(g)
    &=\frac{1}{(n+m)!\:l!}
    \sum_{\substack{\sigma\in\mathfrak{S}_{n+m+l} \\ \inv{\sigma}g\sigma\in\mathfrak{S}_{n+m}\times\mathfrak{S}_l}}
    \chi_{A'}((\inv{\sigma}g\sigma)|_{n+m})\chi_U((\inv{\sigma}g\sigma)|_l)
  \end{align*}
  ここで、$(\inv{\sigma}g\sigma)|_{n+m}$は$\inv{\sigma}g\sigma$の最初の$n+m$文字の置換への制限である。
  $g$と$\inv{\sigma}g\sigma$の置換の型は同じだから、$\inv{\sigma}g\sigma\in\mathfrak{S}_{n+m}\times\mathfrak{S}_l$と、$g\in\mathfrak{S}_{n+m}\times\mathfrak{S}_l$は同値である。よって、
  \begin{align*}
    \chi_A(g)=\left\{\begin{array}{lc}
      \frac{(n+m+l)!}{(n+m)!\:l!}\chi_{A'}(g|_{n+m})\chi_U(g|_l) & \text{if $g\in\mathfrak{S}_{n+m}\times\mathfrak{S}_l$}\\
      0 & \text{otherwise}
    \end{array}\right.
  \end{align*}
  となる。
  同様の議論で、
  \begin{align*}
  \chi_{A'}(g|_{n+m})
  &=\frac{1}{n!\:m!}
  \sum_{ \substack{t\in\mathfrak{S_{n+m}} \\ \inv{t}g|_{n+m}t\in\mathfrak{S}_n\times\mathfrak{S}_m} }
  \chi_V((\inv{t}gt)_n)\chi_W((\inv{t}gt)_m) \\
  &=\left\{\begin{array}{lc}
    \frac{(n+m)!}{n!\:m!}\chi_V(g|_n)\chi_W(g|_m) & \text{if $g\in\mathfrak{S}_n\times\mathfrak{S}_m\times\mathfrak{S}_l$}\\
    0 & \text{otherwise}
    \end{array}\right.
  \end{align*}
  となる。結局、
  \begin{equation}\label{char_A}
  \chi_A(g)=\left\{\begin{array}{lc}
    \frac{(n+m+l)!}{n!\:m!\:l!}\chi_V(g|_n)\chi_W(g|_m)\chi_U(g|_l) & \text{if $g\in\mathfrak{S}_n\times\mathfrak{S}_m\times\mathfrak{S}_l$}\\
    0 & \text{otherwise}
  \end{array}\right.  
\end{equation}
  で、$n,m,l$について対称な形となった。同様に
  $
  \ind_{\mathfrak{S}_{n}\times\mathfrak{S}_{m+l}}^{\mathfrak{S}_{n+m+l}}\left\{
    V\boxtimes (\ind_{\mathfrak{S}_m\times\mathfrak{S}_l}^{\mathfrak{S}_{m+l}}W\boxtimes U)
  \right\}
  $
  の指標を計算すると、(\ref{char_A})となることがわかる。
\end{proof}

$R$の係数を$\quotient$に拡大して考える。$[V],[W]\in R_n$に対して
\[
\generated{[V],[W]}=\generated{\chi_V,\chi_W}  
\]
と定義し、$\generated{,}$を$R$に双線形に拡張し内積を入れる。本節の主定理の一つが次の定理である。

\begin{theo}\label{rep_ring_and_func_ring}
  $\varphi:\Lambda\rightarrow R$を$\varphi(h_\lambda)=[M_\lambda]$によって定めると$\varphi$は内積を保つ環の同型
  であり、
  \[
  \varphi(s_\lambda)=[S_\lambda]
  \]
  が成り立つ。ただし$M_\lambda$は例\ref{ind_from_horizontal_perm}の誘導表現, $S_\lambda$は$\lambda$で定まる対称群の既約表現である。
\end{theo}

\begin{proof}
  $\{h_\lambda\}$は$\Lambda$の基底だから$\varphi$は$\quotient$線形写像として定義できることに注意する。また、Youngの規則(定理\ref{Young_rule_for_rep})より
  \[
  [M_\lambda]=[S_\lambda]+\sum_{\mu>\lambda}k_{\lambda\mu}[S_\mu]
  \]
  となるから、$R_n$の基底$\{[S_\lambda]\}$に関して$\{[M_\lambda]\}$を行列表示したとき、対角成分がすべて1の上三角行列になる。すなわち$\{[M_\lambda]\}$は$R_n$の基底となることがわかる。したがって$\varphi$は少なくとも線形同型であることがわかる。

  $\varphi$が環準同型であることを示す。$h_\lambda=h_{\lambda_1}\cdots h_{\lambda_r}$だから、
  \[
  [M_\lambda]=[M_{(\lambda_1)}]\circ\cdots\circ[M_{(\lambda_r)}]  
  \]
  となることを示せばよい。$n=|\lambda|$とおく。各$M_{(\lambda_i)}$について、$(\lambda_i)$は1行のYoung図形だから$M_{(\lambda_i)}$は$\mathfrak{S}_{\lambda_i}$の自明な表現である。よって$M_{(\lambda_i)}\boxtimes M_{(\lambda_j)}$は$\mathfrak{S}_{\lambda_i}\times\mathfrak{S}_{\lambda_j}$の自明な表現だから、
  \[
  [M_{(\lambda_i)}]\circ[M_{(\lambda_j)}]
  =[\ind_{\mathfrak{S}_{\lambda_i}\times\mathfrak{S}_{\lambda_j}}^{\mathfrak{S}_{\lambda_i+\lambda_j}}\mathbf{1}]
  =[M_{(\lambda_i,\lambda_j)}]
  \]
  積は可換かつ結合的だから
  \[
  [M_\lambda]=[M_{(\lambda_1)}]\circ\cdots\circ[M_{(\lambda_r)}]  
  \]
  
  次に$\varphi$が内積を保つことを示す。
\end{proof}



この定理によって対称群の表現に関する性質を、対称関数の知識を使って調べることができる。

\begin{cor}
  $\mathfrak{S}_n$の既約表現$S_\lambda$, $\mathfrak{S}_m$の既約表現$S_\mu$について
  \[
  \ind_{\mathfrak{S}_n\times\mathfrak{S}_m}^{\mathfrak{S}_{n+m}}S_\lambda\boxtimes S_\mu =\bigoplus_{\nu}S_\nu^{\oplus\eta^{\nu}_{\lambda\mu}} 
  \]
  が成り立つ。ただし$\eta^{nu}_{\lambda\mu}$はLittlewood-Richardson数である。
\end{cor}

\begin{cor}
  $k_{\lambda\mu}$をKostka数とする。
  \[
  M_\lambda=S_\lambda\oplus \left(\bigoplus_{\mu>\lambda}S_\mu^{\oplus k_{\lambda\mu}}\right)  
  \]
  が成り立つ。
\end{cor}

\begin{proof}
  系\ref{Young_rule_for_poly}, 定理\ref{Young_rule_for_rep}より
\end{proof}

\end{document}

\newpage
\documentclass{ltjsreport}
\input{../../setting.tex}


\begin{document}

\section{一般線形群の表現とSchur-Weyl双対性}

前節までで対称群の既約表現に関して解説してきたが、次に対称群と表現論的に関係の深い一般線形群の表現について解説する。とくに多項式表現と呼ばれる表現のクラスが、Schur-Weyl双対性を通して対称群の表現と密接にかかわりあっている。



\subsection{Lie群とLie代数}
Schur-Weyl双対性のために若干のLie群・Lie代数の知識を用いる。

\begin{defin}[Lie群]
  $G$を群であり複素多様体でもあるとする。$G$の演算$\cdot:G\times G\rightarrow G$, および逆元を取る写像$\inv{}:G\rightarrow G$がともに正則であるとき、$G$を(複素)Lie群という。Lie群の間の写像$f:G\rightarrow H$について、$f$が群準同型かつ正則であるとき$f$をLie群の準同型という。
\end{defin}

\begin{eg}
  $\complex$ベクトル空間$V$に対して一般線形群$\gl(V)$は行列の積に関してLie群である。実際、行列の積は成分の多項式であるし、逆行列は分母が$0$でない有理関数で表される。同様に$\text{SL}(V)$, $\text{U}(n)$, $\text{SU}(n)$もLie群である。
\end{eg}

\begin{defin}[Lie代数]
  $\mathfrak{g}$を$\complex$ベクトル空間とする。写像$[\:,\:]:\mathfrak{g}\times \mathfrak{g}\rightarrow \mathfrak{g}$が与えられており
  \begin{enumerate}[(i)]
    \item $[\:,\:]$は双線形
    \item $[x,x]=0$, (交代性)
    \item $[x,[y,z]]+[y,[z,x]]+[z,[x,y]]=0$, (Jacobiの恒等式)
  \end{enumerate}
  をみたすとき、$\mathfrak{g}$を(複素)Lie代数という。Lie代数の間の写像$f:\mathfrak{g}\rightarrow \mathfrak{h}$について、$f$が線形写像かつ$f([X,Y])=[f(X),f(Y)]$をみたすとき、$f$をLie代数の準同型という。
\end{defin}

\begin{eg}
  $\mathfrak{gl}(V)=\End(V)$とし、$X,Y\in\mathfrak{gl}(V)$に対して
  \[
  [X,Y]=XY-YX  
  \]
  とおくと$\mathfrak{gl}(V)$は複素Lie代数である。同様の演算で
  \begin{itemize}
    \item $\mathfrak{sl}(V)=\set{X\in\mathfrak{gl}(V)}{\tr(X)=0}$
    \item $\mathfrak{alt}(V)=\set{X\in\mathfrak{gl}(V)}{\transpose{X}=-X}$
  \end{itemize}
  などもLie代数である。
\end{eg}

\end{document}

\end{document}